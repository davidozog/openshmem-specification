\chapter{Interoperability with Other Programming Models}\label{sec:interoperability}

OpenSHMEM routines may be used in conjunction with the routines of other
communication libraries or parallel languages in the same program. This section
describes the interoperability with other programming models, including
clarification of undefined behaviors caused by mixed use of different models,
and advice to \openshmem library users and developers that may improve the portability
and performance of hybrid programs.


\section{MPI Interoperability}

\openshmem and \ac{MPI} are two commonly used parallel programming models for
distributed-memory systems. The user can choose to utilize both models in the same program
to efficiently and easily support various communication patterns.

A vendor may implement the \openshmem and \ac{MPI} libraries in different ways. For
instance, one may implement both \openshmem and \ac{MPI} as standalone libraries,
each of which allocates and initializes fully isolated communication
resources.
Another approach
is to implement both \openshmem and \ac{MPI} interfaces within the
same software system in order to share a communication resource when possible.

To improve interoperability and portability in \openshmem + \ac{MPI} hybrid
programming, we clarify the relevant semantics in the following subsections.


\subsection{Initialization}
In order to ensure that a hybrid program can be portably performed with different vendor
implementations, the \openshmem environment of the program must be initialized by
a call to \FUNC{shmem\_init} or \FUNC{shmem\_init\_thread} and be finalized by
a call to \FUNC{shmem\_finalize}; the \ac{MPI} environment of the program must be initialized
by a call to \FUNC{MPI\_Init} or \FUNC{MPI\_Init\_thread} and be finalized by a
call to \FUNC{MPI\_Finalize}.

\parimpnotes{
  Portable implementations of OpenSHMEM and \ac{MPI} must ensure that the initialization
  calls can be made in an arbitrary order within a program; the same rule also
  applies to the finalization calls. A software runtime that utilizes a shared
  communication resource for \openshmem and \ac{MPI} communication may maintain an
  internal reference counter in order to ensure that the shared resource is
  initialized only once and thus no shared resource is released until the last
  finalization call is made.
}


\subsection{Dynamic Process Creation}
\label{subsec:interoperability:mpmd}

\ac{MPI} defines a dynamic process model that allows creation of processes after
an \ac{MPI} application has started (e.g., by calling \FUNC{MPI\_Comm\_spawn}) and
connection to independent processes (e.g., through \FUNC{MPI\_Comm\_accept}
and \FUNC{MPI\_Comm\_connect}).
It provides a mechanism to establish communication
between the newly created processes and the existing \ac{MPI} application (see
\ac{MPI} standard version 3.1, Chapter 10).
Unlike \ac{MPI}, \openshmem starts all processes at once and requires all \acp{PE} to
collectively allocate and initialize resources (e.g., symmetric heap) used by
the \openshmem library before any other \openshmem routine may
be called. \openshmem does not support communication with dynamically created
or connected processes. In such a scenario, \ac{MPI} can be used to communicate
with these processes.


\subsection{Thread Safety}
\label{subsec:interoperability:thread}
Both \openshmem and \ac{MPI} define the interaction with user threads in a program
with routines that can be used for initializing and querying the thread
environment. A hybrid program may request different thread levels
at the initialization calls of \openshmem and \ac{MPI} environments; however, the
returned support level provided by the \openshmem or \ac{MPI} library might be different
from that returned in an non-hybrid program. For instance, the former
initialization call in a hybrid program may initialize a resource with the
requested thread level, but the supported level cannot be updated by a subsequent
initialization call if the underlying software runtime of \openshmem and \ac{MPI}
share the same internal communication resource.
The program should always check the \VAR{provided} thread level returned
at the corresponding initialization call or query the level of thread support
after initialization to portably ensure thread support in each communication
environment.

Both \openshmem and \ac{MPI} define similar thread levels, namely, \VAR{THREAD\_SINGLE},
\VAR{THREAD\_FUNNELED}, \VAR{THREAD\_SERIALIZED}, and \VAR{THREAD\_MULTIPLE}.
When requesting threading support in a hybrid program, however,
the following additional rules are applied if the implementations of \openshmem
and \ac{MPI} share the same internal communication resource.
It is strongly recommended to always follow these rules to ensure program
portability.

\begin{itemize}
    \item The \VAR{THREAD\_SINGLE} thread level requires a single-threaded program.
    Hence, a hybrid program should not request \VAR{THREAD\_SINGLE} at the initialization
    call of either \openshmem or \ac{MPI} but request a different thread level at the
    initialization call of the other model.

    \item The \VAR{THREAD\_FUNNELED} thread level allows only the main thread to
    make communication calls. A hybrid program using the \VAR{THREAD\_FUNNELED}
    thread level in both \openshmem and \ac{MPI} should ensure that the same main thread
    is used in both communication environments.

    \item The \VAR{THREAD\_SERIALIZED} thread level requires the program to ensure
    that communication calls are not made concurrently by multiple threads. If a
    hybrid program uses \VAR{THREAD\_SERIALIZED} in one communication environment
    and \VAR{THREAD\_SERIALIZED} or \VAR{THREAD\_FUNNELED} in the other one, it
    should also guarantee that the \openshmem and \ac{MPI} calls are not made concurrently
    from two distinct threads.
\end{itemize}

\subsection{Mapping Process Identification Numbers}
\label{subsec:interoperability:id}

Similar to the \ac{PE} number in \openshmem, \ac{MPI} defines rank as the
identification number of a process in a communicator. Both the \openshmem \ac{PE}
and the \ac{MPI} rank are unique integers assigned from zero to one less than the total
number of processes. In a hybrid program, the \openshmem
\ac{PE} number in \LibHandleRef{SHMEM\_TEAM\_WORLD}
and the \ac{MPI} rank in \VAR{MPI\_COMM\_WORLD} of a process can be equal.
This feature, however, may be provided by only some of the \openshmem and \ac{MPI}
implementations (e.g., if both environments share the same underlying process
manager) and is not portably guaranteed. A portable program should always
use the standard functions in each model, namely, \FUNC{shmem\_team\_my\_pe} or \FUNC{shmem\_my\_pe} in \openshmem 
and \FUNC{MPI\_Comm\_rank} in \ac{MPI}, to query the process identification numbers
in each communication environment and manage the mapping of identifiers in the
program when necessary.

\subsubsection*{Examples}
\label{subsubsec:interoperability:id:example}

\SourceExample{example_code/hybrid_mpi_mapping_id.c}{
  The following example demonstrates how to manage the mapping between
  \openshmem \ac{PE} numbers and \ac{MPI} ranks in
  \VAR{MPI\_COMM\_WORLD} in a hybrid \openshmem and \ac{MPI} program.
}


\SourceExample{example_code/hybrid_mpi_mapping_id_shmem_comm.c}{
  The following example demonstrates an alternative approach for
  managing the mapping of process identification numbers in a hybrid
  program. The program creates a new MPI communicator, named
  \VAR{shmem\_comm}, that contains all processes in
  \VAR{MPI\_COMM\_WORLD} and each process has the same \ac{MPI} rank
  number as its \openshmem \ac{PE} number.
}

\subsection{RMA Programming Models}
\label{subsec:interoperability:rma}

\openshmem and \ac{MPI} each define similar one-sided communication models;
however, a portable program should not assume interoperability between these
models.
For instance, \openshmem guarantees the atomicity only of concurrent \openshmem \ac{AMO} operations
that operate on symmetric data with the same datatype. Access to the same symmetric
object with \ac{MPI} atomic operations, such as an \FUNC{MPI\_Fetch\_and\_op}, may
result in an undefined result. A hybrid program should avoid situations where \ac{MPI} and
\openshmem one-sided operations perform concurrent accesses to the same memory
location; otherwise, the behavior is undefined.

\subsection{Communication Progress}
\label{subsec:interoperability:progress}

\openshmem promises the progression of communication both with and without
\openshmem calls and requires the software progress mechanism in the implementation
(e.g., a progress thread) when the hardware does not provide asynchronous communication
capabilities (see Section \ref{subsec:progress}).
In \ac{MPI}, however, a weak progress semantics is applied. That is,
an \ac{MPI} communication call is guaranteed only to complete in finite time. For
instance, an \FUNC{MPI\_Put} may be completed only when the remote process makes an \ac{MPI}
call that internally triggers the progress of \ac{MPI}, if the underlying hardware
does not support asynchronous communication. A hybrid program
should not assume that the \openshmem library also makes progress for \ac{MPI}.
It can explicitly manage the asynchronous communication of \ac{MPI} in
order to prevent any deadlock or performance degradation.
