\openshmem \emph{sessions} provide a mechanism for applications to inform the
\openshmem library of an upcoming sequence of communication routines that
exhibit suitable patterns for runtime optimizations.
A session is associated with a specific \openshmem communication context
(Section~\ref{sec:ctx}), and it indicates the beginning and ending of
communication phases on that context.
The \FUNC{shmem\_ctx\_session\_start} routine indicates the beginning of a session,
and the \FUNC{shmem\_ctx\_session\_stop} routine indicates the end of a session.
The \LibConstRef{SHMEM\_CTX\_SESSION\_*} options (Table~\ref{session_opts}) indicate
which patterns of \openshmem RMA and AMO routines will occur within a session.
These options serve only as \textit{hints} to the library; it is up to the
implementation whether or not to apply any optimizations within a session.
A session may be provided a configuration argument that specifies attributes
associated with the session. This configuration argument is of type
\CTYPE{shmem\_ctx\_session\_config\_t}, which is detailed further in
Section~\ref{subsec:shmem_team_config_t}.

Usage of the \openshmem session APIs on a particular context must comply with
the requirements of all options set on that context.
Starting and stopping \openshmem sessions should not affect the completion or
ordering semantics of any \openshmem routines in the program.
For these reasons, multi-threaded \openshmem programs may require additional
thread synchronization to ensure sessions hints are correctly applied to
shareable contexts.
Because sessions are associated with an \openshmem communication context,
routines not performed on a communication context (like collective routines)
are ineligible for session hints.

The \FUNC{shmem\_ctx\_session\_config\_t} object requires the \CONST{SIZE\_MAX}
macro defined in \HEADER{stdint.h} by \Cstd[99]~\S7.18.3 and
\Cstd[11]~\S7.20.3.
