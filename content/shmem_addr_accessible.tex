\apisummary{
    Determines whether an address is accessible via \openshmem data transfer
    routines from the specified target \ac{PE}.
}

\begin{apidefinition}

\begin{Csynopsis}
int @\FuncDecl{shmem\_addr\_accessible}@(const void *addr, int pe);
\end{Csynopsis}

\begin{apiarguments}
    \apiargument{IN}{addr}{Local address of data object to query.}
    \apiargument{IN}{pe}{Integer id of a target \ac{PE}.}
\end{apiarguments}

\apidescription{
    \FUNC{shmem\_addr\_accessible} is a query routine that indicates whether
    the address \VAR{addr} can be used to access the given data object on the
    specified target \ac{PE} via \openshmem routines.

    This routine verifies that the data object is symmetric and accessible with
    respect to a target \ac{PE} via \openshmem data transfer routines.  The
    specified address \VAR{addr} is the local address of the data object on the
    local \ac{PE}.
}

\apireturnvalues{
    The return value is \CONST{1} if the local address \VAR{addr} is also a symmetric
    address and the given data object is accessible via \openshmem routines on
    the specified target \ac{PE}; otherwise, it is \CONST{0}.
}

\apinotes{
    This routine may be particularly useful for hybrid programming with other
    communication libraries (such as \ac{MPI}) or parallel languages.  For
    example, when an \ac{MPI} job uses \ac{MPMD} mode, multiple executable
    \ac{MPI} programs may use \openshmem routines.  In such cases, static
    memory, such as a \Cstd global variable, is
    symmetric between processes running from the same executable file, but is
    not symmetric between processes running from different executable files.
    Data allocated from the symmetric heap (e.g., using \FUNC{shmem\_malloc})
    is symmetric across the same or different executable files.
}

\end{apidefinition}
