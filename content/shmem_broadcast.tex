\apisummary{
    Broadcasts a block of data from one \ac{PE} to one or more destination
    \acp{PE}.
}

\begin{apidefinition}

%% C11
\begin{C11synopsis}
int @\FuncDecl{shmem\_broadcast}@(shmem_team_t team, TYPE *dest, const TYPE *source, size_t nelems, int PE_root);
\end{C11synopsis}
where \TYPE{} is one of the standard \ac{RMA} types specified by Table \ref{stdrmatypes}.

%% C/C++
\begin{Csynopsis}
\end{Csynopsis}
\begin{CsynopsisCol}
int @\FuncDecl{shmem\_\FuncParam{TYPENAME}\_broadcast}@(shmem_team_t team, TYPE *dest, const TYPE *source, size_t nelems, int PE_root);
\end{CsynopsisCol}
where \TYPE{} is one of the standard \ac{RMA} types and has a corresponding \TYPENAME{} specified by Table \ref{stdrmatypes}.

\begin{CsynopsisCol}
int @\FuncDecl{shmem\_broadcastmem}@(shmem_team_t team, void *dest, const void *source, size_t nelems, int PE_root);
\end{CsynopsisCol}

\begin{DeprecateBlock}
\begin{CsynopsisCol}
void @\FuncDecl{shmem\_broadcast32}@(void *dest, const void *source, size_t nelems, int PE_root, int PE_start, int logPE_stride, int PE_size, long *pSync);
void @\FuncDecl{shmem\_broadcast64}@(void *dest, const void *source, size_t nelems, int PE_root, int PE_start, int logPE_stride, int PE_size, long *pSync);
\end{CsynopsisCol}
\end{DeprecateBlock}

\begin{apiarguments}

\apiargument{IN}{team}{The team over which to perform the operation.}%

\apiargument{OUT}{dest}{Symmetric address of destination data object.
      The type of \dest{} should match that implied in the SYNOPSIS section.}
\apiargument{IN}{source}{Symmetric address of the source data object.
      The type of \source{} should match that implied in the SYNOPSIS section.}
\apiargument{IN}{nelems}{
  The number of elements in \source{} and \dest{} arrays.
  For \FUNC{shmem\_broadcastmem}, elements are bytes;
  for \FUNC{shmem\_broadcast\{32,64\}}, elements are 4 or 8 bytes,
  respectively.
}
\apiargument{IN}{PE\_root}{Zero-based ordinal of the \ac{PE}, with respect to
    the calling PEs, from which the data is copied.}

\begin{DeprecateBlock}

\apiargument{IN}{PE\_start}{The lowest \ac{PE} number of the active set of
    \acp{PE}.}
\apiargument{IN}{logPE\_stride}{ The log (base 2) of the stride between
    consecutive \ac{PE} numbers in the active set.}
    \apiargument{IN}{PE\_size}{The number of \acp{PE} in the active set.}
\apiargument{IN}{pSync}{
    Symmetric address of a work array of size at least \CONST{SHMEM\_BCAST\_SYNC\_SIZE}.}
\end{DeprecateBlock}

\end{apiarguments}

\apidescription{
    \openshmem team-based broadcast routines are collective routines over a valid \openshmem team.
    They copy the \source{} data object on the \ac{PE} specified by
    \VAR{PE\_root} to the \dest{} data object on the \acp{PE}
    participating in the collective operation.
    The same \dest{} and \source{} data objects and the same value of
    \VAR{PE\_root} must be passed by all \acp{PE} participating in the
    collective operation.

    For team-based broadcasts:
    \begin{itemize}
    \item The \dest{} object is updated on all \acp{PE}.
    \item All \acp{PE} in the \VAR{team} argument must participate in
      the operation.
    \item Only \acp{PE} in the team  may call the routine.  If a
      \ac{PE} not in the team calls a team-based
      collective routine, the behavior is undefined.
    \item If \VAR{team} compares equal to \LibConstRef{SHMEM\_TEAM\_INVALID} or is
      otherwise invalid, the behavior is undefined.
    \item \ac{PE} numbering is relative to the team. The specified
      root \ac{PE} must be a valid \ac{PE} number for the team,
      between \CONST{0} and \VAR{N$-$1}, where \VAR{N} is the size of
      the team.
    \end{itemize}

    Before the local \ac{PE} calls a broadcast routine, the following conditions
    must be ensured, otherwise the behavior is undefined:
    \begin{itemize}
        \item The \dest{} array on all \acp{PE} in the team is ready to
            accept the result of the operation.
        \item The \source{} array at the local root \ac{PE} is ready to be
            read by any \ac{PE} in the team.
    \end{itemize}
    The application does not need to synchronize to ensure that the \source{}
    array is ready across all \acp{PE} prior to calling this routine.

    Upon return from a team-based broadcast routine, the following are true for the local
    \ac{PE}:
    \begin{itemize}
        \item The \dest{} data object is updated.
        \item The \source{} data object may be safely reused.
    \end{itemize}

\begin{DeprecateBlock}
    \openshmem active-set broadcast routines are collective routines over an active set.
    They copy the \source{} data object on the \ac{PE} specified by
    \VAR{PE\_root} to the \dest{} data object on the \acp{PE}
    participating in the collective operation.
    The same \dest{} and \source{} data objects and the same value of
    \VAR{PE\_root} must be passed by all \acp{PE} participating in the
    collective operation.

    For active-set-based broadcasts:
    \begin{itemize}
        \item The \VAR{dest} object is updated on all PEs other than the root PE.
        \item All \acp{PE} in the active set defined by the
        \VAR{PE\_start}, \VAR{logPE\_stride}, \VAR{PE\_size} triplet
        must participate in the operation.
       \item Only \acp{PE} in the active set may call the routine.  If a
       \ac{PE} not in the active set calls an active-set-based
      collective routine, the behavior is undefined.
      \item The values of arguments \VAR{PE\_root}, \VAR{PE\_start},
      \VAR{logPE\_stride}, and \VAR{PE\_size} must be the same value
      on all \acp{PE} in the active set.
      \item The value of \VAR{PE\_root} must be between \CONST{0} and
      \VAR{PE\_size $-$ 1}.
      \item The same \VAR{pSync} work array must be passed by all \acp{PE}
      in the active set.
    \end{itemize}

    Before any \ac{PE} calls a active-set-based broadcast routine, the following
    conditions must be ensured:
    \begin{itemize}
         \item The \dest{} array on all \acp{PE} participating in the broadcast
         is ready to accept the broadcast data.
         \item The \VAR{pSync} array on all \acp{PE} in the
         active set is not still in use from a prior call to an \openshmem
         collective routine.
    \end{itemize}
    Otherwise, the behavior is undefined.

    Upon return from an active-based broadcast routine, the following are true for the local
    \ac{PE}:
    \begin{itemize}
      \item If the current PE is not the root PE, the \dest{} data object is updated.
      \item The \source{} data object may be safely reused.
      \item The values in the \VAR{pSync} array are restored to the
        original values.
    \end{itemize}
\end{DeprecateBlock}
}


\apireturnvalues{
  For team-based broadcasts, zero on successful local completion; otherwise, nonzero.

\begin{DeprecateBlock}
  For active-set-based broadcasts, none.
\end{DeprecateBlock}

}

\apinotes{
    Team handle error checking and integer return codes are currently undefined.
    Implementations may define these behaviors as needed, but programs should
    ensure portability by doing their own checks for invalid team handles and for
    \LibConstRef{SHMEM\_TEAM\_INVALID}.
}

\begin{apiexamples}

\apicexample
    {In the following \Cstd[11] example, the call to \FUNC{shmem\_broadcast} copies \source{}
    on \ac{PE} $0$ to \dest{} on \acp{PE} $0\dots npes-1$.

    \CorCpp{} example:}
    {./example_code/shmem_broadcast_example.c}
    {}

\end{apiexamples}

\end{apidefinition}
