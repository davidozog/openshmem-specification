\apisummary{
    Start a communication session.
}

\begin{apidefinition}

\begin{Csynopsis}
void @\FuncDecl{shmem\_ctx\_session\_start}@(shmem_ctx_t ctx, long options, const shmem_ctx_session_config_t *config, long config_mask);
\end{Csynopsis}

\begin{apiarguments}
    \apiargument{IN}{ctx}{A context handle specifying the context associated
    with this session.}
    \apiargument{IN}{options}{The set of requested options from
    Table~\ref{session_opts} for this session.  Multiple options may be
    requested by combining them with a bitwise OR operation; otherwise,
    \CONST{0} can be given if no options are requested.}
    \apiargument{IN}{config}{
      A pointer to the configuration parameters for the session.}
    \apiargument{IN}{config\_mask}{
      The bitwise mask representing the set of configuration parameters to use
      from \VAR{config}.}
\end{apiarguments}

\apidescription{
    \FUNC{shmem\_ctx\_session\_start} is a non-collective routine that begins a
    session on communication context \VAR{ctx} with hints requested via
    \VAR{options}.
    Sessions on a communication context must be stopped with a call to
    \FUNC{shmem\_ctx\_session\_stop} on the same context.
    If a session is already started on a given context, another call to
    \FUNC{shmem\_ctx\_session\_start} on that same context combines new options
    via a bitwise OR operation. In such a case, unmasked member values in the
    \VAR{config} argument replace any existing configuration values that are
    already applied to the session.

    If \VAR{ctx} compares equal to \LibConstRef{SHMEM\_CTX\_INVALID} then
    \FUNC{shmem\_ctx\_session\_start} performs no action and returns immediately.

    No combination of \VAR{options} passed to \FUNC{shmem\_ctx\_session\_start}
    results in undefined behavior, but some combinations may be detrimental for
    performance; for example, when selecting an option that is not applicable
    to the session. It is the user's responsibility to determine which
    combination of \VAR{options} benefits the performance of the session.

    The \VAR{config} argument specifies session configuration parameters,
    which are described in Section~\ref{subsec:shmem_ctx_session_config_t}.

    The \VAR{config\_mask} argument is a bitwise mask representing the set of
    configuration parameters to use from \VAR{config}.
    A \VAR{config\_mask} value of \CONST{0} indicates that the session should
    be started with the default values for all configuration parameters.
    See Section~\ref{subsec:shmem_ctx_session_config_t} for field mask names and
    default configuration parameters.
}

\apireturnvalues{
    None.
}

\sessiontablebegin

\sessiontablerow{\LibConstRef{SHMEM\_CTX\_SESSION\_BATCH}}{
    A \textit{batch} is a series of calls to \openshmem routines that occur
    within a session on a communication context (i.e., after a call to
    \FUNC{shmem\_ctx\_session\_start} and before a corresponding call to
    \FUNC{shmem\_ctx\_session\_stop}), that might tolerate an increase in
    individual call latencies. Designating a batch may provide an opportunity
    to decrease the overall overhead typically involved with the \openshmem
    library implementing the series as individual RMA operations.  In other
    words, the performance of \openshmem programs that issue many consecutive
    and small-sized RMA routines might be improved by informing the library
    implementation ahead of time that it is free to delay transferring data
    in order to buffer, combine, and/or coalesce the issued \openshmem
    routines.  The specific mechanisms for improving performance using
    batching optimizations depend on the \openshmem library implementation.

    The \VAR{SHMEM\_CTX\_SESSION\_BATCH} hint indicates that a communication
    context will be used to issue a batch.  An example of a batch is an
    iterative loop of non-blocking RMA and/or AMO routines. A batch may
    include a memory ordering or collective operation, but such routines
    might require completions and/or synchronization that could degrade
    performance.

    Because sessions do not affect the completion or ordering semantics of any
    \openshmem routines in the program, routines such as non-blocking RMAs,
    non-blocking AMOs, non-blocking \OPR{put-with-signals}, blocking scalar
    \OPR{puts}, small blocking \OPR{puts}, and blocking non-fetching AMOs are
    viable candidates for batching.  Other routines, such as large blocking
    \OPR{puts}, all blocking \OPR{gets}, blocking fetching AMOs, and the
    memory ordering routines might require the library to enforce
    completions, reducing the potential benefit of batching.

    The \VAR{total\_ops} field of \VAR{config} indicates the expected maximum
    number of calls to \openshmem RMA routines within the session.
    See Section~\ref{subsec:shmem_ctx_session_config_t} for details
    about \VAR{shmem\_ctx\_session\_config\_t} parameters.
    } \hline

\sessiontablerow{\LibConstRef{SHMEM\_CTX\_SESSION\_SAME\_AMO}}{
    The \VAR{SHMEM\_CTX\_SESSION\_SAME\_AMO} hint indicates the session will
    contain a series of calls to AMO routines where each call performs the same
    atomic operation (e.g., using only the \textit{increment} operation) on the
    same datatype (e.g., only on objects of type \textit{int}).
    As a more specific example, this hint would apply to a session that
    includes \textit{only} calls to \FUNC{shmem\_int\_atomic\_inc}.
    However, this hint would not apply to a session that includes both calls to
    \FUNC{shmem\_int\_atomic\_inc} and \FUNC{shmem\_int\_atomic\_fetch},
    because the operation \textit{fetch} differs from \textit{increment}.
    (Similarly, this hint would not apply to a session that includes both calls to
    \FUNC{shmem\_int\_atomic\_inc} and \FUNC{shmem\_long\_atomic\_inc},
    because the datatype \textit{long} differs from \textit{int}.)
    This hint does not restrict the application from calling other RMA and/or
    signaling routines within the session.

    The \VAR{total\_ops} field of \VAR{config} indicates the expected maximum
    number of calls to \openshmem RMA routines within the session.
    See Section~\ref{subsec:shmem_ctx_session_config_t} for details about
    \VAR{shmem\_ctx\_session\_config\_t} parameters.
    } \hline

\sessiontableend

\apinotes{
    The \FUNC{shmem\_ctx\_session\_start} routine provides hints for improving
    performance, and \openshmem implementations are not required to apply any
    optimization.
    \FUNC{shmem\_ctx\_session\_start} is non-collective, so there is no implied
    synchronization.
    Blocking puts must be sufficiently small to benefit from batching, and the
    exact threshold for this benefit depends on the \openshmem implemenation
    and/or the application.
}

\end{apidefinition}
