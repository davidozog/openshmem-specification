\apisummary{
    A routine that allows any \ac{PE} to force termination of an entire program.
}

\begin{apidefinition}

\begin{C11synopsis}
_Noreturn void @\FuncDecl{shmem\_global\_exit}@(int status);
\end{C11synopsis}

\begin{Csynopsis}
void @\FuncDecl{shmem\_global\_exit}@(int status);
\end{Csynopsis}

\begin{apiarguments}
    \apiargument{IN}{status}{The exit status from the main program.}
\end{apiarguments}

\apidescription{
    \FUNC{shmem\_global\_exit} is a non-collective routine that allows any one
    \ac{PE} to force termination of an \openshmem program for all \acp{PE},
    passing an exit status to the execution environment. This routine terminates
    the entire program, not just the \openshmem portion.  When any \ac{PE} calls
    \FUNC{shmem\_global\_exit}, it results in the immediate notification to all
    \acp{PE} to terminate.  \FUNC{shmem\_global\_exit} flushes I/O and releases
    resources in accordance with \CorCpp language requirements for normal
    program termination. If more than one \ac{PE} calls
    \FUNC{shmem\_global\_exit}, then the exit status returned to the environment
    shall be one of the values passed to \FUNC{shmem\_global\_exit} as the
    status argument.  There is no return to the caller of
    \FUNC{shmem\_global\_exit}; control is returned from the \openshmem program
    to the execution environment for all \acp{PE}.
}

\apireturnvalues{
    None.
}


\apinotes{
    \FUNC{shmem\_global\_exit} may be used in situations where one or more
    \acp{PE} have determined that the program has completed and/or should
    terminate early.  Accordingly, the integer status argument can be used to
    pass any information about the nature of the exit; e.g., that the program
    encountered an error or found a solution.
    Since \FUNC{shmem\_global\_exit} is a non-collective
    routine, there is no implied synchronization, and all \acp{PE} must
    terminate regardless of their current execution state. While I/O must be
    flushed for standard language I/O calls from \CorCpp, it is
    implementation dependent as to how I/O done by other means (e.g., third
    party I/O libraries) are handled. Similarly, resources are released
    according to \CorCpp standard language requirements, but this may not
    include all resources allocated for the \openshmem program. However, a
    quality implementation will make a best effort to flush all I/O and clean
    up all resources.
}

\begin{apiexamples}

\apicexample
    {}
    {./example_code/shmem_global_exit_example.c}
    {}

\end{apiexamples}

\end{apidefinition}
