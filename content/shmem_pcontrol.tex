\apisummary{
  Allows the user to control profiling.
}

\begin{apidefinition}

\begin{Csynopsis}
void @\FuncDecl{shmem\_pcontrol}@(int level, ...);
\end{Csynopsis}

\begin{apiarguments}

  \apiargument{IN}{level}{The profiling level.}

\end{apiarguments}

\apidescription{
  \FUNC{shmem\_pcontrol} sets the profiling level and any other
  library defined effects through additional arguments. \openshmem libraries
  make no use of this routine and simply return immediately to the user code.
}

\apireturnvalues{
  None.
}

\apinotes{
  Since \openshmem has no control of the implementation of the profiling code,
  it is impossible to precisely specify the semantics that will be provided by
  calls to \FUNC{shmem\_pcontrol}. This vagueness extends to the number of
  arguments to the function and their datatypes. However, to provide some
  level of portability of user code to different profiling libraries, the
  following \VAR{level} values are recommended.

  \begin{itemize}
  \item \texttt{level <= 0} Profiling is disabled.
  \item \texttt{level == 1} Profiling is enabled at the default level of detail.
  \item \texttt{level == 2} Profiling is enabled and profile buffers are
  flushed if available.
  \item \texttt{level > 2} Profiling is enabled with profile library defined
  effects and additional arguments.
  \end{itemize}

  The default state after \FUNC{shmem\_init} is recommended to have profiling
  enabled at the default level of detail (\texttt{level == 1}). This allows users
  to link with a profiling library and to obtain profile output without
  having to modify the user-level source code.
}

\end{apidefinition}
