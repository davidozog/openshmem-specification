\apisummary {
  Performs inclusive or exclusive prefix sum operations
}

\begin{apidefinition}

%% C11
\begin{C11synopsis}
int @\FuncDecl{shmem\_sum\_inscan}@(shmem_team_t team, TYPE *dest, const TYPE *source, size_t nelems);
int @\FuncDecl{shmem\_sum\_exscan}@(shmem_team_t team, TYPE *dest, const TYPE *source, size_t nelems);
\end{C11synopsis}
where \TYPE{} is one of the integer, real, or complex types supported
for the SUM operation as specified by Table \ref{teamreducetypes}.

%% C/C++
\begin{Csynopsis}
int @\FuncDecl{shmem\_\FuncParam{TYPENAME}\_sum\_inscan}@(shmem_team_t team, TYPE *dest, const TYPE *source, size_t nelems);
int @\FuncDecl{shmem\_\FuncParam{TYPENAME}\_sum\_exscan}@(shmem_team_t team, TYPE *dest, const TYPE *source, size_t nelems);
\end{Csynopsis}
where \TYPE{} is one of the integer, real, or complex types supported
for the SUM operation and has a corresponding \TYPENAME{} as specified
by Table \ref{teamreducetypes}.

\begin{apiarguments}
  \apiargument{IN}{team}{
    The team over which to perform the operation.
  }
  \apiargument{OUT}{dest}{
    Symmetric address of an array, of length \VAR{nelems} elements,
    to receive the result of the scan operation.  The type of
    \dest{} should match that implied in the SYNOPSIS section.
  }
  \apiargument{IN}{source}{
    Symmetric address of an array, of length \VAR{nelems} elements,
    that contains one element for each separate scan operation.
    The type of \source{} should match that implied in the SYNOPSIS
    section.
  }
  \apiargument{IN}{nelems}{
    The number of elements in the \dest{} and \source{} arrays.
  }
\end{apiarguments}

\apidescription{

  The \FUNC{shmem\_sum\_inscan} and \FUNC{shmem\_sum\_exscan} routines
  are collective routines over an \openshmem team that compute one or
  more scan (or prefix sum) operations across symmetric arrays on
  multiple \acp{PE}.  The scan operations are performed with the SUM
  operator.

  The \VAR{nelems} argument determines the number of separate scan
  operations to perform.  The \source{} array on all \acp{PE}
  participating in the operation provides one element for each scan.
  The results of the scan operations are placed in the \dest{} array
  on all \acp{PE} participating in the scan.

  The \FUNC{shmem\_sum\_inscan} routine performs an inclusive scan
  operation, while the \FUNC{shmem\_sum\_exscan} routine performs an
  exclusive scan operation.

  For \FUNC{shmem\_sum\_inscan}, the value of the $j$-th element in
  the \VAR{dest} array on \ac{PE}~$i$ is defined as:
  \begin{equation*}
    \textrm{dest}_{i,j} = \displaystyle\sum_{k=0}^{i} \textrm{source}_{k,j}
  \end{equation*}

  For \FUNC{shmem\_sum\_exscan}, the value of the $j$-th element in
  the \VAR{dest} array on \ac{PE}~$i$ is defined as:
  \begin{equation*}
    \textrm{dest}_{i,j} =
    \begin{cases}
    \displaystyle\sum_{k=0}^{i-1} \textrm{source}_{k,j}, & \text{if} \; i \neq 0 \\
    0,  & \text{if} \; i = 0
    \end{cases}
  \end{equation*}


  The same \source{} and \dest{} arrays must be passed by all PEs that
  participate in the collective.
  The \source{} and \dest{} arguments must either be the same
  symmetric address, or two different symmetric addresses
  corresponding to buffers that do not overlap in memory.
  That is, they must be completely overlapping (sometimes referred to as an
  ``in place'' reduction) or completely disjoint.

  Team-based scan routines operate over all \acp{PE} in the provided
  team argument. All \acp{PE} in the provided team must participate in
  the scan operation.  If \VAR{team} compares equal to
  \LibConstRef{SHMEM\_TEAM\_INVALID} or is otherwise invalid, the
  behavior is undefined.

  Before the local \ac{PE} calls a scan routine, the following conditions must be
  ensured, otherwise the behavior is undefined:
  \begin{itemize}
      \item The \dest{} array on all \acp{PE} in the team is ready to accept
          the result of the operation.
      \item The \source{} array at the local \ac{PE} is ready to be read by
          any \ac{PE} in the team.
  \end{itemize}
  The application does not need to synchronize to ensure that the \source{}
  array is ready across all \acp{PE} prior to calling this routine.

  Upon return from a scan routine, the following are true for the
  local \ac{PE}: the \dest{} array is updated, and the \source{} array
  may be safely reused.

  When the \Cstd translation environment does not support complex
  types, an \openshmem implementation is not required to provide
  support for these complex-typed interfaces.
}

\apireturnvalues{
  Zero on successful local completion. Nonzero otherwise.
}

\begin{apiexamples}

  \apicexample{
    In the following \Cstd[11] example, the \FUNC{collect\_at}
    function gathers a variable amount of data from each \ac{PE} and
    concatenates it, in order, at the target \ac{PE} \VAR{who}.  Note
    that this routine is behaviorally similar to
    \FUNC{shmem\_collect}, except that this routine only gathers the
    data to a single \ac{PE}.
  }
  {./example_code/shmem_scan_example.c}
  {}

\end{apiexamples}

\end{apidefinition}
