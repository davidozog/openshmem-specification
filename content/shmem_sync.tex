\apisummary{
    Registers the arrival of a \ac{PE} at a synchronization point.
    This routine does not return until all other \acp{PE} in a given OpenSHMEM team
    arrive at this synchronization point.
\begin{DeprecateBlock}
    Registers the arrival of a \ac{PE} at a synchronization point.
    This routine does not return until all other \acp{PE} in a given OpenSHMEM active set arrive at this synchronization point.
\end{DeprecateBlock}
}

\begin{apidefinition}

\begin{C11synopsis}
int @\FuncDecl{shmem\_sync}@(shmem_team_t team);
\end{C11synopsis}

\begin{Csynopsis}
int @\FuncDecl{shmem\_team\_sync}@(shmem_team_t team);
\end{Csynopsis}

\begin{DeprecateBlock}
\begin{CsynopsisCol}
void @\FuncDecl{shmem\_sync}@(int PE_start, int logPE_stride, int PE_size, long *pSync);
\end{CsynopsisCol}
\end{DeprecateBlock}

\begin{apiarguments}

\apiargument{IN}{team}{The team over which to perform the operation.}%

\begin{DeprecateBlock}
\apiargument{IN}{PE\_start}{The lowest \ac{PE} number of the active set of
    \acp{PE}.}
\apiargument{IN}{logPE\_stride}{The log (base 2) of the stride between
    consecutive \ac{PE} numbers in the active set.}
\apiargument{IN}{PE\_size}{The number of \acp{PE} in the active set.}
\apiargument{IN}{pSync}{
  Symmetric address of a work array of size at least \CONST{SHMEM\_SYNC\_SIZE}.}
\end{DeprecateBlock}

\end{apiarguments}

\apidescription{
    \FUNC{shmem\_sync} is a collective synchronization routine over an
    existing \openshmem team.

    The routine registers the arrival of a \ac{PE} at a synchronization point in the program.
    This is a fast mechanism for synchronizing all \acp{PE} that participate in this
    collective call. The routine blocks the calling \ac{PE} until all \acp{PE} in the
    specified team have called \FUNC{shmem\_sync}. In a multithreaded \openshmem
    program, only the calling thread is blocked.

    Team-based sync routines operate over all \acp{PE} in the provided team argument. All
    \acp{PE} in the provided team must participate in the sync operation.
    If \VAR{team} compares equal to \LibConstRef{SHMEM\_TEAM\_INVALID} or is
    otherwise invalid, the behavior is undefined.

    In contrast with the \FUNC{shmem\_barrier} routine, \FUNC{shmem\_sync} only
    ensures completion and visibility of previously issued memory stores and does not ensure
    completion of remote memory updates issued via \openshmem routines.

\begin{DeprecateBlock}
    \FUNC{shmem\_sync} is a collective synchronization routine over an active set.

    The routine registers the arrival of a \ac{PE} at a synchronization point in the program.
    This is a fast mechanism for synchronizing all \acp{PE} that participate in this
    collective call. The routine blocks the calling \ac{PE} until all \acp{PE} in the
    active set have called \FUNC{shmem\_sync}. In a multithreaded \openshmem
    program, only the calling thread is blocked.

    Active-set-based sync routines operate over all \acp{PE} in the active set
    defined by the \VAR{PE\_start}, \VAR{logPE\_stride}, \VAR{PE\_size} triplet.

    As with all active set-based collective routines,
    each of these routines assumes
    that only \acp{PE} in the active set call the routine.  If a \ac{PE} not in
    the active set calls an active set-based collective routine,
    the behavior is undefined.

    The values of arguments \VAR{PE\_start}, \VAR{logPE\_stride}, and
    \VAR{PE\_size} must be equal on all \acp{PE} in the active set.  The same
    work array must be passed in \VAR{pSync} to all \acp{PE} in the active set.

    The same \VAR{pSync} array may be reused on consecutive calls to
    \FUNC{shmem\_sync} if the same active set is used.
\end{DeprecateBlock}


}

\apireturnvalues{
    Zero on successful local completion. Nonzero otherwise.
}

\apinotes{
    The \FUNC{shmem\_sync} routine can be used to portably ensure that
    memory access operations observe remote updates in the order enforced by the
    initiator \acp{PE}, provided that the initiator PE ensures completion of remote
    updates with a call to \FUNC{shmem\_quiet} prior to the call to the
    \FUNC{shmem\_sync} routine.
}

\begin{apiexamples}

\apicexample
    {The following \FUNC{shmem\_sync} example is
    for \Cstd[11] programs:}
    {./example_code/shmem_sync_example.c}
    {}

\end{apiexamples}

\end{apidefinition}
