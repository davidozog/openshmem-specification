\apisummary{
    Returns a local pointer to a symmetric data object on the specified target \ac{PE} in the specified team.
}

\begin{apidefinition}

\begin{Csynopsis}
void *@\FuncDecl{shmem\_team\_ptr}@(shmem_team_t team, const void *dest, int pe);
\end{Csynopsis}

\begin{apiarguments}
\apiargument{IN}{team}{A handle to the specified team.}
\apiargument{IN}{dest}{The symmetric address of the remotely accessible data
    object to be referenced.}
\apiargument{IN}{pe}{An integer that indicates the \ac{PE} number in the
    provided team on which \dest{} is to be accessed.}
\end{apiarguments}

\apidescription{
    \FUNC{shmem\_team\_ptr} returns an address that may be used to directly reference
    \dest{} on the specified target \ac{PE} in the specified team.  This address can be assigned to a
    pointer.  After that, ordinary loads and stores to \dest{} may be
    performed.  The address returned by \FUNC{shmem\_team\_ptr} is a local address to
    a remotely accessible data object.  Providing this address to an argument of
    an \openshmem routine that requires a symmetric address results in
    undefined behavior.

    The \FUNC{shmem\_team\_ptr} routine can provide efficient means to accomplish
    communication, for example when a sequence of reads and writes to a data
    object on a target \ac{PE} does not match the access pattern provided in an
    \openshmem data transfer routine like \FUNC{shmem\_put} or
    \FUNC{shmem\_iget}.
}

\apireturnvalues{
    A local pointer to the remotely accessible \dest{} data object is returned
    when it can be accessed using memory loads and stores.  Otherwise, a null
    pointer is returned.

    If \VAR{team} compares equal to \LibConstRef{SHMEM\_TEAM\_INVALID},
    then a null pointer is returned.
    If \VAR{team} is otherwise invalid, the behavior is undefined.
}

\apinotes{
    When calling \FUNC{shmem\_team\_ptr}, \dest{} is the address of the referenced
    symmetric data object on the calling \ac{PE}.
}

\end{apidefinition}
