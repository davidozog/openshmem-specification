\apisummary{
    Indicate whether any one variable within an array of variables on the local
    \ac{PE} meets its specified test condition.
}

\begin{apidefinition}

\begin{C11synopsis}
size_t @\FuncDecl{shmem\_test\_any\_vector}@(TYPE *ivars, size_t nelems, const int *status, int cmp,
    TYPE *cmp_values);
\end{C11synopsis}
where \TYPE{} is one of the point-to-point synchronization types specified by
Table \ref{p2psynctypes}.

\begin{Csynopsis}
size_t @\FuncDecl{shmem\_\FuncParam{TYPENAME}\_test\_any\_vector}@(TYPE *ivars, size_t nelems, const int *status,
    int cmp, TYPE *cmp_values);
\end{Csynopsis}
where \TYPE{} is one of the point-to-point synchronization types and has a
corresponding \TYPENAME{} specified by Table \ref{p2psynctypes}.

\begin{apiarguments}

  \apiargument{IN}{ivars}{Local address of an array of remotely accessible data
    objects.
    The type of \VAR{ivars} should match that implied in the SYNOPSIS section.}
  \apiargument{IN}{nelems}{The number of elements in the \VAR{ivars} array.}
  \apiargument{IN}{status}{Local address of an optional mask array of length \VAR{nelems}
    that indicates which elements in \VAR{ivars} are excluded from the test set.}
  \apiargument{IN}{cmp}{A comparison operator from Table~\ref{p2p-consts} that
    compares elements of \VAR{ivars} with elements of \VAR{cmp\_values}.}
  \apiargument{IN}{cmp\_values}{Local address of an array of length \VAR{nelems}
    containing values to be compared with the respective objects in \VAR{ivars}.
    The type of \VAR{cmp\_values} should match that implied in the SYNOPSIS section.}

\end{apiarguments}

\apidescription{
    The \FUNC{shmem\_test\_any\_vector} routine indicates whether any
    entry in the test set specified by \VAR{ivars} and \VAR{status} has
    satisfied the test condition at the calling \ac{PE}.  The \VAR{ivars}
    objects at the calling \ac{PE} may be updated by an \ac{AMO} performed by a
    thread located within the calling \ac{PE} or within another \ac{PE}.
    This routine does not
    block and returns \CONST{SIZE\_MAX} if no entries in \VAR{ivars} satisfied
    the test condition.  This routine compares each element of the
    \VAR{ivars} array in the test set with each respective value in
    \VAR{cmp\_values} according to the comparison operator \VAR{cmp} at the
    calling \ac{PE}.  The order in which these elements are tested is
    unspecified.  If an entry $i$ in \VAR{ivars} within the test set satisfies
    the test condition, a series of calls to
    \FUNC{shmem\_test\_any\_vector} must eventually return $i$.

    The optional \VAR{status} is a mask array of length \VAR{nelems} where each
    element corresponds to the respective element in \VAR{ivars} and indicates
    whether the element is excluded from the test set.  Elements of
    \VAR{status} set to 0 will be included in the test set, and elements set to
    1 will be ignored.  If all elements in \VAR{status} are set to 1 or
    \VAR{nelems} is 0, the test set is empty and this routine returns
    \CONST{SIZE\_MAX}.  If \VAR{status} is a null pointer, it is ignored and
    all elements in \VAR{ivars} are included in the test set.  The \VAR{ivars}
    and \VAR{status} arrays must not overlap in memory.

    Implementations must ensure that \FUNC{shmem\_test\_any\_vector} does not
    return an index before the update of the memory indicated by the
    corresponding \VAR{ivars} element is fully complete.
}

\apireturnvalues{
    \FUNC{shmem\_test\_any\_vector} returns the index of an element in the \VAR{ivars}
    array that satisfies the test condition. If the test set is empty or no
    conditions in the test set are satisfied, this routine returns \CONST{SIZE\_MAX}.
}

\end{apidefinition}
