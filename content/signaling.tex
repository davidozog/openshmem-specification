This section specifies the OpenSHMEM support for \OPR{put-with-signal},
nonblocking \OPR{put-with-signal}, and \OPR{signal-\{add, fetch, set\}} routines. The
put-with-signal routines provide a method for copying data from a contiguous
local data object to a data object on a specified \ac{PE} and subsequently
updating a remote flag to signal completion.
The signal-add and signal-set routines provide methods for updating
the signal object without the associated data transfer of a
put-with-signal operation.
The signal-fetch routine provides support for reading a local signal value.

\openshmem \OPR{put-with-signal} and \OPR{signal-\{add, set\}}
routines specified in this section have two
variants. In one of the variants, the context handle, \VAR{ctx}, is explicitly
passed as an argument. In this variant, the operation is performed on the
specified context. If the context handle \VAR{ctx} does not correspond to a
valid context, the behavior is undefined. In the other variant, the context
handle is not explicitly passed and thus, the operations are performed on the
default context.

\subsubsection{Atomicity Guarantees for Signaling Operations}
\label{subsec:signal_atomicity}
All signaling operations put-with-signal, nonblocking put-with-signal, and
signal-\{add, fetch, set\} are performed on a signal data object, a remotely accessible
symmetric object of type \VAR{uint64\_t}. A signal operator in the
put-with-signal routine is an \openshmem library constant that determines the
type of update to be performed as a signal on the signal data object.

All signaling operations on the signal data object complete as if performed
atomically with respect to the following:
\begin{itemize}
    \item other blocking or nonblocking variant of the put-with-signal routine
    that updates the signal data object using the same signal update operator;
    \item signal-add routine when the put-with-signal routine uses the
      \LibConstRef{SHMEM\_SIGNAL\_ADD} signal operator;
    \item signal-set routine when the put-with-signal routine uses the
      \LibConstRef{SHMEM\_SIGNAL\_SET} signal operator;
    \item signal-fetch routine that fetches the signal data object; and
    \item any point-to-point synchronization routine that accesses the signal
    data object.
\end{itemize}

\subsubsection{Available Signal Operators}
\label{subsec:signal_operator}

With the atomicity guarantees as described in
Section~\ref{subsec:signal_atomicity}, the following options can be used as a
signal operator.

    \apitablerow{\LibConstRef{SHMEM\_SIGNAL\_SET}}{An update to signal data
    object is an atomic set operation. It writes an unsigned 64-bit value as a
    signal into the signal data object on a remote \VAR{PE} as an atomic
    operation.}

    \apitablerow{\LibConstRef{SHMEM\_SIGNAL\_ADD}}{An update to signal data
    object is an atomic add operation. It adds an unsigned 64-bit value as a
    signal into the signal data object on a remote \VAR{PE} as an atomic
    operation.}
