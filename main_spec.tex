\documentclass[10pt]{book}

\input{utils/packages}

\makeindex

\newcommand{\minitab}[2][l]{\begin{tabular}{@{}#1@{}}#2\end{tabular}}

\definecolor{ListingBG}{rgb}{0.91,0.91,0.91}
\definecolor{shadecolor}{rgb}{0.92,0.92,0.92}

\hyphenation{Open-SHMEM}

\renewcommand{\chaptername}{Chapter}
\renewcommand{\appendixname}{Annex}

% Place some penalty for doing the break
% The penalty for a ``\gb'' should be greater than a \hyphenpenalty.
% \hyphenpenalty is 50 in plain.tex.
\def\gb{\penalty10000\hskip 0pt plus 8em\penalty4800\hskip 0pt plus-8em%
\penalty10000}

% This macro enables that all "_" (underscore) characters in the pfd
% file are searchable, and that cut&paste will copy the "_" as underscore.
% Without the following macro, the \_ is treated in searches and cut&paste
% as a " " (space character).
% This macro does not modify the behavior of _ in math or in verbatim
% environments. In verbatim environments, the "_" is always treated
% as a searchable character.
%
\DeclareRobustCommand{\_}{\texttt{\char`\_}}
%

\def\colorswapnt{\colorlet{saved}{.}\color{ForestGreen}}
\def\colorswapot{\colorlet{saved}{.}\color{red}}
\def\prevcolor{\color{saved}}

\newcommand{\newtext}[1]{\textcolor{ForestGreen}{#1}}
\newcommand{\oldtext}[1]{\textcolor{magenta}{\sout{#1}}}
\newcommand{\insertDocVersion}{1.5}
\newcommand{\openshmem}[1][]{%
  {Open\-SHMEM\ifthenelse{\equal{#1}{}}{}{~#1}}\xspace}
\newcommand{\HEADER}[1]{\textit{#1}}
\newcommand{\FUNC}[1]{\textit{#1}}
\newcommand{\CTYPE}[1]{\textit{#1}}
\newcommand{\VAR}[1]{\textit{#1}}
\newcommand{\ENVVAR}[1]{\textit{#1}}
\newcommand{\CONST}[1]{\textit{#1}}
\newcommand{\KEYWORD}[1]{\textit{#1}}
%%
\newcommand{\CorCpp}{\textit{C/C++}\xspace}
\newcommand{\Fortran}[1][]{%
  \textit{Fortran\ifthenelse{\equal{#1}{}}{}{~#1}}\xspace}
\newcommand{\Cstd}[1][]{%
  \textit{C\ifthenelse{\equal{#1}{}}{}{#1}}\xspace}
\newcommand{\Cpp}[1][]{%
  \textit{C++\ifthenelse{\equal{#1}{}}{}{#1}}\xspace}
%%
\newcommand{\TYPE}{\emph{TYPE}}
\newcommand{\TYPENAME}{\emph{TYPENAME}}
\newcommand{\SIZE}{\emph{SIZE}}

\newcommand{\source}{\textit{source}}
\newcommand{\dest}{\textit{dest}}
\newcommand{\PUT}{\textit{Put}}
\newcommand{\GET}{\textit{Get}}
\newcommand{\OPR}[1]{\textit{#1}}
\newcommand{\shmemprefix}{\textit{SHMEM\_}}
\newcommand{\shmemprefixLC}{\textit{shmem\_}}
\newcommand{\shmemprefixC}{\textit{\_SHMEM\_}}
\newcommand{\ith}{${\textit{i}^{\text{\tiny th}}}$}
\newcommand{\jth}{${\textit{j}^{\text{\tiny th}}}$}
\newcommand{\kth}{${\textit{k}^{\text{\tiny th}}}$}
\newcommand{\lth}{${\textit{l}^{\text{\tiny th}}}$}

%% Generate indexed reference.
\newcommand{\EnvVarIndex}[1]{\index{#1}}
\newcommand{\FuncIndex}[1]{\index{#1}}
\newcommand{\LibConstIndex}[1]{\index{#1}}
\newcommand{\LibHandleIndex}[1]{\index{#1}}
\newcommand{\TableIndex}[1]{\index{#1}\index{Tables!#1}}
%% Write text and generate reference.
\newcommand{\EnvVarRef}[1]{\ENVVAR{#1}\EnvVarIndex{#1}}
\newcommand{\FuncRef}[1]{\FUNC{#1}\FuncIndex{#1}}
\newcommand{\LibConstRef}[1]{\CONST{#1}\LibConstIndex{#1}}
\newcommand{\LibHandleRef}[1]{\CONST{#1}\LibHandleIndex{#1}}
\newcommand{\TableCaptionRef}[1]{\caption{#1}\TableIndex{#1}}
%% Specialized declaration/creation and generate reference.
\newcommand{\EnvVarDecl}[1]{\EnvVarRef{#1}}
\newcommand{\FuncDecl}[1]{{\ListingsCurrentStyle{#1}}\FuncIndex{#1}}
\newcommand{\FuncParam}[1]{<{\ListingsKeywordStyle{#1}}>}
\newcommand{\LibConstDecl}[2][\CorCpp]{%
  \parbox[t]{5cm}{~\\[-4pt] #1: \\\hspace*{8mm} \LibConstRef{#2} \\~}}
\newcommand{\LibHandleDecl}[2][\CorCpp]{%
  \parbox[t]{0pt}{~\\[-4pt] #1: \\\hspace*{8mm} \LibHandleRef{#2} \\~}}

\begin{acronym}
\acro{RMA}{\emph{Remote Memory Access}}
\acro{RMO}{\emph{Remote Memory Operation}}
\acro{AMO}{\emph{Atomic Memory Operation}}
\acro{PE}{\emph{Processing Element}}
\acrodefplural{PE}[PEs]{\emph{Processing Elements}}
\acro{PGAS}{\emph{Partitioned Global Address Space}}
\acro{API}{\emph{Application Programming Interface}}
\acro{MPI}{\emph{Message Passing Interface}}
\acro{SPMD}{\emph{Single Program Multiple Data}}
\acro{ANL}{Argonne National Labratory}
\acro{ARL}{Army Research Laboratory}
\acro{AMD}{Advanced Micro Devices}
\acro{MPMD}{\emph{Multiple Program Multiple Data}}
\acro{TCP}{\emph{Transmission Control Protocol}}
\acro{UH}{University of Houston}
\acro{UO}{University of Oregon}
\acro{ORNL}{Oak Ridge National Laboratory}
\acro{LANL}{Los Alamos National Laboratory}
\acro{ESSC}{Extreme Scale Systems Center}
\acro{OS}{Operating System}
\acro{OSSS}{Open Source Software Solutions}
\acro{SGI}{Silicon Graphics International}
\acro{DoD}{U.S. Department of Defense}
\acro{SBU}{Stonybrook University}
\acro{UTK}{University of Tenneesee at Knoxville}
\acro{HPE}{Hewlett Packard Enterprise}
\end{acronym}


% Grab current listings style for use in environment escape to LaTeX.
% https://tex.stackexchange.com/a/209644
\makeatletter
\newcommand\ListingsCurrentStyle{}
\lst@AddToHook{Output}{\global\let\ListingsCurrentStyle\lst@thestyle}
\lst@AddToHook{OutputOther}{\global\let\ListingsCurrentStyle\lst@thestyle}
\newcommand\ListingsKeywordStyle{}
\lst@AddToHook{Output}{\global\let\ListingsKeywordStyle\lst@keywordstyle}
\lst@AddToHook{OutputOther}{\global\let\ListingsKeywordStyle\lst@keywordstyle}
\makeatother

%
% This is used to put line numbers on plain pages.  Used in draft.tex
%
\makeatletter

\def\withlinenumbers{\relax
  \def\@evenfoot{\hbox to 0pt{\hss\LineNumberRuler\hskip 1.5pc}\hfil}\relax
  \def\@oddfoot{\hfil\hbox to 0pt{\hskip 1.5pc\LineNumberRuler\hss}}}

\def\LineNumberRuler{\vbox to 0pt{\vss\normalsize \baselineskip13.6pt
    \lineskip 1pt \normallineskip 1pt \def\baselinestretch{1}\relax
    \LNR{1}\LNR{2}\LNR{3}\LNR{4}\LNR{5}\LNR{6}\LNR{7}\LNR{8}\LNR{9}
    \LNR{10}\LNR{11}\LNR{12}\LNR{13}\LNR{14}
        \LNR{15}\LNR{16}\LNR{17}\LNR{18}\LNR{19}
    \LNR{20}\LNR{21}\LNR{22}\LNR{23}\LNR{24}
        \LNR{25}\LNR{26}\LNR{27}\LNR{28}\LNR{29}
    \LNR{30}\LNR{31}\LNR{32}\LNR{33}\LNR{34}\LNR{35}
        \LNR{36}\LNR{37}\LNR{38}\LNR{39}
    \LNR{40}\LNR{41}\LNR{42}\LNR{43}\LNR{44}
        \LNR{45}\LNR{46}\LNR{47}\LNR{48}
    \vskip 31pt}}
\def\LNR#1{\hbox to 1pc{\hfil\tiny#1\hfil}}

\def\ps@plainwithlinenumbers{\let\@mkboth\@gobbletwo
     \def\@oddhead{}
     \def\@oddfoot{\hfil\rm\thepage\hfil
       \hbox to 0pt{\hskip 1.5pc\LineNumberRuler\hss}}
     \def\@evenhead{}
     \def\@evenfoot{\hbox to 0pt{\hss
     \LineNumberRuler\hskip 1.5pc}\rm\hfil\thepage\hfil}}

    % Contents is done with \chapter*{Contents}, so we need to turn off the
    % line numbers in this case.  Easiest to look at def

\newwrite\chappages
\immediate\openout\chappages=chappage.txt
\def\writespace{ }

\def\incontents{0}
\newif\ifcontents
\contentsfalse
\def\chapter{\clearpage \ifcontents\else\thispagestyle{plainwithlinenumbers}\fi
        \write\chappages{Chapter \thechapter\writespace - \the\count0}
        \global\@topnum\z@ \@afterindentfalse \secdef\@chapter\@schapter}

\makeatother

%
% End this is used to put line numbers on plain pages.  Used in draft.tex
%

%
% Use Sans Serif font for sections, etc.
%
%
\makeatletter
\def\section{\@startsection {section}{1}{\z@}{-3.5ex plus -1ex minus
-.2ex}{2.3ex plus .2ex}{\Large\sf}}
\def\subsection{\@startsection{subsection}{2}{\z@}{-3.25ex plus -1ex minus
-.2ex}{1.5ex plus .2ex}{\large\sf}}
\def\subsubsection{\@startsection{subsubsection}{3}{\z@}{-3.25ex plus
-1ex minus -.2ex}{1.5ex plus .2ex}{\normalsize\sf\bf}}
\def\paragraph{\@startsection {paragraph}{4}{\z@}{3.25ex plus 1ex minus .2ex}
{-1em}{\normalsize\sf\bf}} % Indent after \paragraph
\makeatother
%
% End use Sans Serif font for sections, etc.  S. Otto
%


%
% This section is for example code listings
%
\definecolor{gray}{rgb}{0.92,0.92,0.92}

\lstset{ % set defaults for languages not otherwise defined
  breakatwhitespace=true,         % sets if automatic breaks should only happen at whitespace
  basicstyle=\ttfamily\footnotesize,
  breaklines=true,                 % sets automatic line breaking
  extendedchars=true,              % lets you use non-ASCII characters; for 8-bits
                                   % encodings only, does not work with UTF-8
  keepspaces=true,                 % keeps spaces in text, useful for keeping indentation of code
                                   % (possibly needs columns=flexible)
  morekeywords={*,...},            % if you want to add more keywords to the set
  showspaces=false,                % show spaces everywhere adding particular underscores;
                                   % it overrides 'showstringspaces'
  showstringspaces=false,          % underline spaces within strings only
  showtabs=false,                  % show tabs within strings adding particular underscores
}

\def\StandardListing {
  \lstset {
    breakatwhitespace=false,         % sets if automatic breaks should only happen at whitespace
    basicstyle=\ttfamily\footnotesize,
    breaklines=true,                 % sets automatic line breaking
    escapeinside={\%*}{*)},          % if you want to add LaTeX within your code
    extendedchars=true,              % lets you use non-ASCII characters; for 8-bits
                                     % encodings only, does not work with UTF-8
    keepspaces=true,                 % keeps spaces in text, useful for keeping
                                     % indentation of code (possibly needs columns=flexible)
    morekeywords={*,...},            % if you want to add more keywords to the set
    showspaces=false,                % show spaces everywhere adding particular underscores;
                                     % it overrides 'showstringspaces'
    showstringspaces=false,          % underline spaces within strings only
    showtabs=false,                  % show tabs within strings adding particular underscores
    backgroundcolor=\color{gray},
  }
}

\def\ProgramNumberedListing {
  \StandardListing
  \lstset {
    numbers=left,
    numberstyle=\footnotesize
  }
}

\newcommand{\numberedlisting}[2] {
  \ProgramNumberedListing
  \lstinputlisting[#1]{#2}
  \StandardListing
}

\newcommand{\outputlisting}[2] {
\begin{minipage}{\linewidth}
\vspace{0.1in}
  \lstinputlisting[#1]{#2}
  \StandardListing
\vspace{0.1in}
\end{minipage}
}

\lstdefinelanguage{OSH+C}[]{C}{
  classoffset=1,
  morekeywords={
    size_t, ptrdiff_t,
    SHMEM_BCAST_SYNC_SIZE, SHMEM_SYNC_VALUE,
    start_pes,
    my_pe, _my_pe, shmem_my_pe,
    num_pes, _num_pes, shmem_n_pes,
    shmem_int_p, shmem_short_p, shmem_long_p,
    shmem_int_put, shmem_short_put, shmem_long_put,
    shmem_barrier_all, shmem_barrier,
    shmalloc,  shfree, shrealloc,
    shmem_broadcast32, shmem_broadcast64,
    shmem_short_inc, shmem_int_inc, shmem_long_inc,
    shmem_short_add, shmem_int_add, shmem_long_add,
    shmem_short_finc, shmem_int_finc, shmem_long_finc,
    shmem_short_fadd, shmem_int_fadd, shmem_long_fadd,
    shmem_set_lock, shmem_test_lock, shmem_clear_lock,
    shmem_long_sum_to_all,
    shmem_complexd_sum_to_all
  },
  keywordstyle=\color{black}\textbf,
  classoffset=0,
  sensitive=true
}

\lstdefinelanguage{OSH2+C}[]{OSH+C}{
  classoffset=1,
  morekeywords={
    shmem_init,
    shmem_finalize,
    shmem_malloc,
    shmem_my_pe,
    shmem_error,
    shmem_global_exit,
  },
  keywordstyle=\color{black}\textbf,
  classoffset=0,
  sensitive=true
}

\lstdefinelanguage{OSH+F}[]{Fortran}{
  classoffset=1,
  morekeywords={
    SHMEM_BCAST_SYNC_SIZE, SHMEM_SYNC_VALUE,
    start_pes,
    my_pe, shmem_my_pe,
    num_pes, shmem_n_pes,
    shmem_int_p, shmem_short_p, shmem_long_p,
    shmem_int_put, shmem_short_put, shmem_long_put,
    shmem_barrier_all, shmem_barrier,
    shpalloc,  shpdeallc, shpclmove,
    shmem_broadcast32, shmem_broadcast64,
    shmem_broadcast4, shmem_broadcast8,
    shmem_short_inc, shmem_int_inc, shmem_long_inc,
    shmem_short_add, shmem_int_add, shmem_long_add,
    shmem_short_finc, shmem_int_finc, shmem_long_finc,
    shmem_short_fadd, shmem_int_fadd, shmem_long_fadd,
    shmem_set_lock, shmem_test_lock, shmem_clear_lock,
    shmem_long_sum_to_all,
  },
  keywordstyle=\color{black}\textbf,
  classoffset=0,
  sensitive=false
}

\lstdefinelanguage{OSH2+F}[]{OSH+F}{
  classoffset=1,
  morekeywords={
    shmem_init,
    shmem_finalize,
    shmem_malloc,
    shmem_my_pe,
    shmem_error,
    shmem_global_exit,
  },
  keywordstyle=\color{black}\textbf,
  classoffset=0,
  sensitive=true
}

%
% End this section is for example code listings
%

%
% Deprecation Helpers
%

\newcommand{\strikeline}[1][red]{{\color{#1}\raisebox{.5ex}{\rule{1em}{.4pt}}}}
\newcommand{\stretchline}[1][red]{\xrfill[.5ex]{.4pt}[#1]}
\newcommand{\DeprecationStart}[1][red]{{\color{#1} deprecation start} \mbox{}}
\newcommand{\DeprecationEnd}[1][red]{{\color{#1} deprecation end} \mbox{}}

\newcommand{\StartDeprecateBlock}{
  {\strikeline\mbox{} \DeprecationStart \stretchline\mbox{}}}
\newcommand{\EndDeprecateBlock}{%
  \mbox{}\stretchline\mbox{} \DeprecationEnd \strikeline}
\newenvironment{DeprecateBlock}{%
  \par \StartDeprecateBlock \par}{\par \EndDeprecateBlock \par}

\newcommand{\StartInlineDeprecate}{%
  \strikeline\mbox{} \DeprecationStart \strikeline \mbox{}}
\newcommand{\EndInlineDeprecate}{%
  \strikeline\mbox{} \DeprecationEnd \strikeline}
\newenvironment{DeprecateInline}{\StartInlineDeprecate}{\EndInlineDeprecate}

\newcommand{\deprecationstart}{%
  \color{red} \strikeline\mbox{} deprecation start \stretchline \mbox{}}
\newcommand{\deprecationend}{%
  \mbox{}\stretchline\mbox{} \color{red} deprecation end \strikeline}
\newenvironment{deprecate}{\deprecationstart \\}{\\ \deprecationend}

%
% Design feedback request helpers
%

\newcommand{\feedbackstart}{\color{RoyalBlue} \strikeline[RoyalBlue]
  design feedback requested \stretchline[RoyalBlue] \mbox{}}
\newcommand{\feedbackend}{\mbox{}\stretchline[RoyalBlue]\mbox{}}

\newenvironment{FeedbackRequest}{\feedbackstart \\}{\\ \feedbackend}

%
% Library API description template commands
%

\newcommand{\apisummary}[1]{
    #1
\hfill
}

\newenvironment{apidefinition}{
\begin{description}
\item[SYNOPSIS] \hfill \\ \\
\vspace{-2em}
}
{
\end{description} % ends \begin{description} from \apidescription
\end{description}
}

\lstnewenvironment{Cpp11synopsis}
{
  \textbf{C++11:}
  \lstset{language={C++}, backgroundcolor=\color{gray}, lineskip=2pt,
    escapechar=@,
  morekeywords={size_t, ptrdiff_t, TYPE, noreturn},
  aboveskip=0pt, belowskip=0pt}}{}

\lstnewenvironment{C11synopsis}
{
  \textbf{C11:}
  \lstset{language={C}, backgroundcolor=\color{gray}, lineskip=2pt,
    escapechar=@,
    morekeywords={size_t, ptrdiff_t, TYPE, _Noreturn, shmem_ctx_t,
      shmem_team_t, shmem_team_config_t, uint64_t},
  aboveskip=0pt, belowskip=0pt}}{}

\lstnewenvironment{CsynopsisCol}
{
  \lstset{language={C}, backgroundcolor=\color{gray}, lineskip=2pt,
    escapechar=@,
    morekeywords={size_t, ptrdiff_t, TYPE, TYPENAME, SIZE, shmem_ctx_t,
      shmem_team_t, shmem_team_config_t, uint64_t},
  aboveskip=0pt, belowskip=0pt}}{}


\lstnewenvironment{Csynopsis}
{
  \textbf{C/C++:}
  \lstset{language={C}, backgroundcolor=\color{gray}, lineskip=2pt,
    escapechar=@,
    morekeywords={size_t, ptrdiff_t, TYPE, TYPENAME, SIZE, shmem_ctx_t,
      shmem_team_t, shmem_team_config_t, uint64_t},
  aboveskip=0pt, belowskip=0pt}}{}

\lstnewenvironment{CsynopsisST}
{
  \textbf{C/C++:}
  \color{red}
  {\lstset{language={C}, backgroundcolor=\color{gray}, lineskip=2pt,
    escapechar=@,
    morekeywords={size_t, ptrdiff_t, TYPE, TYPENAME, SIZE, shmem_ctx_t,
      shmem_team_t, uint64_t},
    aboveskip=0pt, belowskip=0pt}}}{}

\lstnewenvironment{Fsynopsis}
{ \deprecationstart \\
  \textbf{FORTRAN:}
  \lstset{language={Fortran}, backgroundcolor=\color{gray}, lineskip=3pt,
    escapechar=@,
  deletekeywords=[2]{STATUS},
  deletekeywords=[3]{LOG}, aboveskip=0pt,
  belowskip=0pt}}
{ \deprecationend }

\newenvironment{apiarguments}{
\newcommand{\apiargument}[3]{
\begin{tabular}{p{2cm} p{2cm} p{10cm}}
\textbf{##1} & \textit{##2} & {##3} \\
\end{tabular}
}
\hfill
\item[DESCRIPTION] \hfill

\begin{description}
\item[Arguments] \hfill \\
}
{
\hfill
\end{description}
}

\newcommand{\apidescription}[1]{
\begin{description}
\vspace{-1em}
\item[API description] \hfill \\
    \begin{sloppypar}
    #1
    \end{sloppypar}
\hfill
}

\newcommand{\apidesctable}[4] {\hfill \\ #1 \\ \\
    \begin{tabular}{p{5cm} p{9cm}}
       \hline
       #2 & #3 \\
       \hline \tabularnewline
       \end{tabular}\\
        #4
}

\newcommand{\apireturnvalues}[1]{
\hfill
\item[Return Values] \hfill \\
    #1
\\
\hfill
}

\newcommand{\apitablerow}[2]{
 \begin{tabular}{p{5cm} p{9cm}}
 #1 & #2 \tabularnewline
  \end{tabular}\\
}

\newcommand{\apinotes}[1]{
\item[Notes] \hfill \\
    #1
\hfill \\
}

\newcommand{\apiimpnotes}[1]{
\begin{description}
\item[Note to implementors] \hfill \\
    #1
\hfill \\
\end{description}
}

\newenvironment{apiexamples}{
\newcommand{\apicexample}[3]{
    ##1
    \lstinputlisting[language={C}, tabsize=2,
      basicstyle=\ttfamily\footnotesize,
      morekeywords={size_t, ptrdiff_t, shmem_ctx_t,  _Thread_local, shmem_team_t, uint64_t}]{##2}
    ##3 }
\vspace{-2pt}
\item[EXAMPLES] \hfill \\
\vspace{-2pt}
}
{
}

\newcommand{\cexample}[2]{
    #1
    \lstinputlisting[language={C}, tabsize=2,
      basicstyle=\ttfamily\footnotesize,
      morekeywords={size_t, ptrdiff_t, shmem_ctx_t}]{#2}}
%
% End library API description template commands
%


\begin{document}

\hypersetup{pageanchor=true,citecolor=blue}

% Set header/footer for opening content
\pagestyle{fancy}
\fancyhead{}
\fancyhead[L]{\insertDocVersion}
\fancyhead[C]{--- DRAFT ---}
\SetWatermarkText{DRAFT}
\SetWatermarkScale{1}
\SetWatermarkLightness{.91}
\fancyfoot[C]{\thepage} %affects page numbering for the first pages,
                            %except the first ToC page

\pagenumbering{roman} %sets coverpage and toc page numbers to roman numerals

\thispagestyle{empty}
\begin{center}
\textbf{\Huge \openshmem}
\par
\end{center}

\begin{center}
\textbf{\LARGE Application Programming Interface}\\
\includegraphics[scale=0.65]{figures/OpenSHMEM_Pound}\\
\url{http://www.openshmem.org/}
\par
\end{center}

\begin{center}
Version \insertDocVersion
\par
\end{center}

\vspace{0.5in}
\begin{center}
\today
\end{center}

\vspace{0.5in}

\vfill{}

\section*{Development by}
\begin{itemize}
\item For a current list of contributors and collaborators please see\\
  \url{http://www.openshmem.org/site/Contributors/}
\item For a current list of OpenSHMEM implementations and tools, please see\\
  \url{http://openshmem.org/site/Links#impl/}

\end{itemize}

\pagebreak{}

\section*{Sponsored by}
\begin{itemize}
\item \ac{DoD}\\
  \url{http://www.defense.gov/ }
\item \ac{ORNL}\\
  \url{http://www.ornl.gov/}
\item \ac{LANL}\\
  \url{http://www.lanl.gov/}
\end{itemize}

\section*{Current Authors and Collaborators}
\begin{itemize}
\item Matthew Baker, \ac{ORNL}
\item Swen Boehm, \ac{ORNL}
\item Aurelien Bouteiller, \ac{UTK}
\item Barbara Chapman, \ac{SBU}
\item Bob Cernohous, \ac{HPE}
\item James Culhane, \ac{LANL}
\item Tony Curtis, \ac{SBU}
\item James Dinan, NVIDIA
\item Mike Dubman, Mellanox
\item Manjunath Gorentla Venkata, Mellanox
\item Anshuman Goswami, NVIDIA
\item Megan Grodowitz, Arm Inc.
\item Max Grossman, Georgia Tech
\item Khaled Hamidouche, \ac{AMD}
\item Jeff Hammond, Intel
\item Yossi Itigin, Mellanox
\item Bryant Lam, \ac{DoD}
\item Akhil Langer, NVIDIA
\item Jeff Kuehn, \ac{LANL}
\item Jens Manser, \ac{DoD}
\item Tiffany M. Mintz, \ac{ORNL}
\item David Ozog, Intel
\item Nicholas Park, \ac{DoD}
\item Steve Poole, \ac{OSSS}
\item Wendy Poole, \ac{OSSS}
\item Swaroop Pophale, \ac{ORNL}
\item Sreeram Potluri, NVIDIA
\item Howard Pritchard, \ac{LANL}
\item Md. Wasi-ur- Rahman, Intel
\item Naveen Ravichandrasekaran, \ac{HPE}
\item Michael Raymond, \ac{HPE}
\item James Ross, \ac{ARL}
\item Pavel Shamis, Arm Inc.
\item Sameer Shende, \ac{UO}
\item Min Si, \ac{ANL}
\item Lauren Smith, \ac{DoD}

\end{itemize}

\section*{Alumni Authors and Collaborators}
\begin{itemize}
\item Amrita Banerjee, \ac{UH}
\item Monika ten Bruggencate, Cray Inc.
\item Eduardo D'Azevedo, \ac{ORNL}
\item Karl Feind, Altair
\item Oscar Hernandez, \ac{ORNL}
\item David Knaak, Cray Inc.
\item Gregory Koenig, \ac{ORNL}
\item Graham Lopez, \ac{ORNL}
\item Ricardo Mauricio, \ac{UH}
\item Ram Nanjegowda, \ac{UH}
\item Aaron Welch, \ac{ORNL}

\end{itemize}

\date{\today}

\section*{Acknowledgments}
The \openshmem specification belongs to Open Source Software Solutions, Inc.
(OSSS), a non-profit organization, under an agreement with HPE. For a current list
of Contributors and Collaborators, please see
  \url{http://www.openshmem.org/site/Contributors/}.
We gratefully acknowledge support from
Oak Ridge National Laboratory's
Extreme Scale Systems Center and the continuing support of the Department of Defense.\\
\\
We would also like to acknowledge the contribution of the members of the
\openshmem mailing list for their ideas, discussions, suggestions, and
constructive criticism which has helped us improve this document.\\
\\
\openshmem[1.4] is dedicated to the memory of David Charles Knaak. David was a highly involved
colleague and contributor to the entire OpenSHMEM project. He will be missed.


\setcounter{tocdepth}{4}
\setcounter{secnumdepth}{4}
\tableofcontents

\mainmatter  % included for use of documenttype 'book'

% Set header/footer for main content
\pagestyle{fancy}   %replacing {headings} with {fancy} for customization
\fancyhf{}
\fancyhead[L]{\leftmark}
\fancyhead[R]{\thepage}
\renewcommand{\headrulewidth}{0pt}
\let\thesectionOrig\thesection % Used by backmatter to restore Annex numbering.
\renewcommand{\thesection}{\arabic{section}}

{ %using setlength to force standardized spacing, if needed
% this command is ended in backmatter.tex
%\setlength{\baselineskip}{3pt plus 3pt minus 3pt}

\setlength{\parskip}{3pt}




\section{The OpenSHMEM Effort}\label{subsec:openshmem_effort}
\openshmem is a \ac{PGAS} library interface specification. \openshmem aims to
provide a standard \ac{API} for SHMEM libraries to aid portability and
facilitate uniform predictable results of \openshmem programs by explicitly
stating the behavior and semantics of the \openshmem library calls. Through the
different versions, \openshmem will continue to address the requirements of the
\ac{PGAS} community.  As of this specification, many existing vendors support
\openshmem-compliant implementations and new vendors are developing
\openshmem library implementations to help the users write portable \openshmem
code. This ensures that programs can run on multiple platforms without having to
deal with subtle vendor-specific implementation differences. For more details on
the history of \openshmem please refer to the
\hyperref[sec:openshmem_history]{History of \openshmem} section.

The \openshmem\footnote{The \openshmem specification is owned by Open Source
Software Solutions Inc., a nonprofit organization, under an agreement with
\ac{HPE}.} effort is driven by the \ac{DoD} with continuous input from the \openshmem community.
To see all of the contributors and participants for the \openshmem \ac{API},
please see: \url{http://www.openshmem.org/site/Contributors}. In addition to the
specification, the effort includes a reference \openshmem
implementation, validation and verification suites, tools, a mailing list and
website infrastructure to support specification activities. For more information
please refer to: \url{http://www.openshmem.org/}.


\section{Programming Model Overview}\label{subsec:programming_model}
\openshmem implements \ac{PGAS} by defining remotely accessible data objects as
mechanisms to share information among \openshmem processes, or \acp{PE}, and
private data objects that are accessible by only the \ac{PE} itself. The \ac{API}
allows communication and synchronization operations on both private (local to
the \ac{PE} initiating the operation) and remotely accessible data objects. The key
feature of \openshmem is that data transfer operations are
\emph{one-sided} in nature. This means that a local \ac{PE} executing
a data transfer routine does not require the participation of the remote \ac{PE}
to complete the routine. This allows for overlap between communication and
computation to hide data transfer latencies, which makes  \openshmem ideal for
unstructured, small-to-medium-sized data communication patterns. The \openshmem
library has the potential to provide a low-latency, high-bandwidth
communication \ac{API} for use in highly parallelized scalable programs.

\openshmem's interfaces can be used to implement \ac{SPMD} style programs.
It provides interfaces to start the \openshmem \acp{PE} in parallel and
communication and synchronization interfaces to access remotely accessible data
objects across \acp{PE}. These interfaces can be leveraged to divide a problem
into multiple sub-problems that can be solved independently or with coordination
using the communication and synchronization interfaces.  The \openshmem
specification defines library calls, constants, variables, and language bindings
for \Cstd.
The \Cpp interface is currently the same as that
for \Cstd. Unlike Unified Parallel C, \Fortran[2008], Titanium, X10, and Chapel, which are all
PGAS languages, \openshmem relies on the user to use the library calls  to
implement the correct semantics of its programming model.

An overview of the \openshmem routines is described below:

\begin{enumerate}

\item \textbf{Library Setup and Query}
\begin{enumerate}
  \item \OPR{Initialization}: The \openshmem library environment is initialized,
   where the \acp{PE} are either single or multithreaded.
  \item \OPR{Query}: The local \ac{PE} may get the number of \acp{PE} running
      the same program and its unique integer identifier.
  \item \OPR{Accessibility}: The local \ac{PE} can find out if a remote \ac{PE} is
      executing the same binary, or if a particular symmetric data object can be
      accessed by a remote \ac{PE}, or may obtain a pointer to a symmetric data
      object on the specified remote \ac{PE} on shared memory systems.
\end{enumerate}

\item \textbf{Symmetric Data Object Management}
\begin{enumerate}
  \item \OPR{Allocation}: All executing \acp{PE} must collectively participate in the
      allocation of a symmetric data object with identical arguments.
  \item  \OPR{Deallocation}: All executing \acp{PE} must collectively participate in the
      deallocation of the same symmetric data object with identical arguments.
  \item  \OPR{Reallocation}: All executing \acp{PE} must collectively participate in the
      reallocation of the same symmetric data object with identical arguments.
\end{enumerate}

\item \textbf{Communication Management}
\begin{enumerate}
    \item \OPR{Contexts}: Contexts are containers for communication operations.
        Each context provides an environment where the operations performed on
        that context are ordered and completed independently of other operations
        performed by the application.
\end{enumerate}

\item \textbf{Team Management}
\begin{enumerate}
    \item \OPR{Teams}: Teams are \ac{PE} subsets created by grouping a set of
    \acp{PE}. Teams are involved in both collective and point-to-point
    communication operations. Collective communication operations are performed
    on all \acp{PE} in a valid team and point-to-point communication operations
    are performed between a local and remote \ac{PE} with team-based \ac{PE}
    numbering through team-based contexts.
\end{enumerate}

\item \textbf{\acf{RMA}}
\begin{enumerate}
    \item \PUT: The local \ac{PE} specifies the \source{} data object, private
        or symmetric, that is copied to the symmetric data object on the remote
        \ac{PE}.
  \item \GET: The local \ac{PE} specifies the symmetric data object on the remote
      \ac{PE} that is copied to a data object, private or symmetric, on the local
      \ac{PE}.
\end{enumerate}

\item \textbf{\acfp{AMO}}
\begin{enumerate}
  \item \OPR{Fetch}: The \ac{PE} initiating the fetch returns the value of the
      symmetric data object on the remote \ac{PE}.
  \item \OPR{Set}: The \ac{PE} initiating the set copies a new value to the
      symmetric data object on the remote \ac{PE}.
  \item \OPR{Swap}: The \ac{PE} initiating the swap copies a new value to the
      symmetric data object on the remote \ac{PE} and returns the old value.
  \item \OPR{Increment}: The \ac{PE} initiating the increment adds 1 to the
      symmetric data object on the remote \ac{PE}.
  \item \OPR{Add}: The \ac{PE} initiating the add specifies the value to be added
      to the symmetric data object on the remote \ac{PE}.
  \item \OPR{Bitwise Operations}: The \ac{PE} initiating the bitwise
      operation specifies the operand value to the bitwise operation to be
      performed on the symmetric data object on the remote \ac{PE}.
  \item \OPR{Compare and Swap}: The \ac{PE} initiating the compare and swap
      conditionally copies a new value to the symmetric data object on the
      remote \ac{PE} and returns the old value.
  \item \OPR{Fetch and Increment}: The \ac{PE} initiating the increment adds 1
      to the symmetric data object on the remote \ac{PE} and returns the old
      value.
  \item \OPR{Fetch and Add}: The \ac{PE} initiating the add specifies the value to
      be added to the symmetric data object on the remote \ac{PE} and returns
      the old value.
  \item \OPR{Fetch and Bitwise Operations}: The \ac{PE} initiating the bitwise
      operation specifies the operand value to the bitwise operation to be
      performed on the symmetric data object on the remote \ac{PE}
      and returns the old value.
\end{enumerate}

\item \textbf{Signaling Operations}
\begin{enumerate}
  \item \OPR{Put Signal}: The local \ac{PE} specifies the \source{} data object
      to be copied to the symmetric data object on the remote \ac{PE} and
      another symmetric data object on the remote \ac{PE} is subsequently
      updated to signal completion.
  \item \OPR{Signal Add}: The local \ac{PE} specifies a value to be added to
      the symmetric data object on the remote \ac{PE}.
  \item \OPR{Signal Set}: The local \ac{PE} specifies a value to be copied to
      the symmetric data object on the remote \ac{PE}.
  \item \OPR{Signal Fetch}: The local \ac{PE} returns the value of a local data
      object.
\end{enumerate}

\item \textbf{Session Management}
\begin{enumerate}
  \item \OPR{Sessions}: Sessions are a mechanism for the application to inform
      the implementation about an upcoming sequence of operations that exhibit
      a pattern that may be suitable for runtime optimization.
\end{enumerate}

\item \textbf{Synchronization and Ordering}
\begin{enumerate}
  \item \OPR{Fence}: The \ac{PE} calling fence ensures ordering of
  \PUT, \ac{AMO}, and memory store operations
  to symmetric data objects with respect to a specific
      destination \ac{PE}.
  \item \OPR{Quiet}: The \ac{PE} calling quiet ensures remote completion of remote access
      operations and stores to symmetric data objects.
  \item \OPR{Barrier}: All or some \acp{PE} collectively synchronize and ensure
      completion of all remote and local updates prior to any \ac{PE} returning
      from the call.
  \item \OPR{Wait and Test}: A \ac{PE} calling a point-to-point synchronization
      routine ensures the value of a local symmetric object meets a specified
      condition.  Wait operations block until the specified condition is
      met, whereas test operations return immediately and indicate whether or
      not the specified condition is met.
\end{enumerate}

\item \textbf{Collective Communication}
\begin{enumerate}
  \item \OPR{Broadcast}: The \VAR{root} \ac{PE} specifies a symmetric data
      object to be copied to a symmetric data object on one or more remote
      \acp{PE}.
  \item \OPR{Collection}: All \acp{PE} participating in the routine get the result
      of concatenated symmetric objects contributed by each of the \acp{PE} in
      another symmetric data object.
  \item \OPR{Reduction}: All \acp{PE} participating in the routine get the result
      of a binary operation over elements of the specified symmetric
      data object on another symmetric data object.
  \item \OPR{All-to-All}: All \acp{PE} participating in the routine exchange
      a fixed amount of contiguous or strided data with all other participating
      \acp{PE}.
  \item \OPR{Scan}: All \acp{PE} participating in the routine perform an
      inclusive or exclusive prefix sum over elements of the specified
      symmetric data object.
\end{enumerate}

\item \textbf{Mutual Exclusion}
\begin{enumerate}
  \item \OPR{Set Lock}: The \ac{PE} acquires exclusive access to the region
      bounded by the symmetric \VAR{lock} variable.
  \item \OPR{Test Lock}: The \ac{PE} tests the symmetric \VAR{lock} variable
      for availability.
  \item \OPR{Clear Lock}: The \ac{PE} which has previously acquired the
      \VAR{lock} releases it.
\end{enumerate}

\end{enumerate}


\section{Memory Model}\label{subsec:memory_model}
\begin{figure}[h]
\includegraphics[width=0.95\textwidth]{figures/mem_model}
\caption{\openshmem Memory Model}
\label{fig:mem_model}
\end{figure}
%
An \openshmem program consists of data objects that are private to each \ac{PE}
and data  objects that are remotely accessible by all \acp{PE}. Private data
objects are stored in the local memory of each \ac{PE} and can only be accessed
by the \ac{PE} itself; these data objects cannot be accessed by other \acp{PE}
via \openshmem routines. Private data objects follow the memory model of
\Cstd. Remotely accessible objects, however, can be accessed by
target \acp{PE} using \openshmem routines.  Remotely accessible data objects are
called \emph{Symmetric Data Objects}.  Each symmetric data object has a
corresponding object with the same name, type, and size on all \acp{PE} where that object is
accessible via the \openshmem \ac{API}\footnote{For efficiency reasons,
the same offset (from an arbitrary memory address) for symmetric data
objects might be used on all \acp{PE}. Further discussion about symmetric heap
layout and implementation efficiency can be found in Section~\ref{sec:memory_management}}.
(For the definition of what is accessible, see the
descriptions for \FUNC{shmem\_pe\_accessible} and \FUNC{shmem\_addr\_accessible}
in Sections~\ref{subsec:shmem_pe_accessible} and
\ref{subsec:shmem_addr_accessible}.) In \openshmem the following kinds of
data objects are symmetric:
%
\begin{itemize}
\item Global and static \Cstd and \Cpp variables. These data objects must
  not be defined in a dynamic shared object (DSO).
\item \Cstd and \Cpp data allocated by \openshmem memory management routines
  (Section~\ref{sec:memory_management})
\end{itemize}

\openshmem dynamic memory allocation routines (e.g.,
\FUNC{shmem\_malloc}) allow collective allocation of \emph{Symmetric Data
Objects} on a special memory region called the \emph{Symmetric Heap}. The
Symmetric Heap is created during the execution of a program at a memory location
determined by the implementation. The Symmetric Heap may reside in different
memory regions on different \acp{PE}.
Figure~\ref{fig:mem_model} shows an example \openshmem
memory layout, illustrating the location of remotely accessible symmetric
objects and private data objects.  As shown, symmetric data objects can be
located either in the symmetric heap or in the global/static memory section of
each \ac{PE}.

\subsection{Pointers to Symmetric Objects}\label{subsec:pointers_to_symmetric_objects}

Symmetric data objects are referenced in \openshmem operations through the
local pointer to the desired remotely accessible object.  The address contained
in this pointer is referred to as a {\em symmetric address}.  Every symmetric
address is also a {\em local address} that is valid for direct memory access;
however, not all local addresses are symmetric.  Manipulation of symmetric
addresses passed to \openshmem routines---including pointer arithmetic,
array indexing, and access of structure or union members---are permitted as long as
the resulting local pointer remains within the same symmetric allocation or
object.  Symmetric addresses are only valid at the \ac{PE} where they were
generated; using a symmetric address generated by a different \ac{PE} for
direct memory access or as an argument to an \openshmem routine results
in undefined behavior.

Symmetric addresses provided to typed and type-generic \openshmem interfaces
must be naturally aligned based on their type and any requirements of the
underlying architecture.  Symmetric addresses provided to fixed-size \openshmem
interfaces (e.g., \FUNC{shmem\_put32}) must also be aligned to the given
size.  Symmetric objects provided to fixed-size \openshmem interfaces
must have storage size equal to the bit-width of the given
operation\footnote{The bit-width of a byte is implementation-defined in \Cstd.  The
\CONST{CHAR\_BIT} constant in \HEADER{limits.h} can be used to portably
calculate the bit-width of a \Cstd object.}.  Because \CorCpp{} structures may
contain implementation-defined padding, the fixed-size interfaces should not be
used with \CorCpp{} structures.
The ``mem'' interfaces (e.g., \FUNC{shmem\_putmem}) have no alignment
requirements.

The \FUNC{shmem\_ptr} and \FUNC{shmem\_team\_ptr} routines allow the application to query a {\em local
address} to a remotely accessible data object at a specified \ac{PE}.  The
resulting pointer is valid for direct memory access; however, providing this
address as an argument of an \openshmem routine that requires a symmetric
address results in undefined behavior.

\subsection{Atomicity Guarantees}\label{subsec:amo_guarantees}

\openshmem contains a number of routines that perform atomic operations on
symmetric data objects, which are defined in Section~\ref{sec:amo}.
The atomic routines
guarantee that concurrent accesses by any of these routines to the same
location, using the same datatype (specified in Tables~\ref{stdamotypes} and
\ref{extamotypes}), and using communication contexts (see Section~\ref{sec:ctx})
in the same atomicity domain will be exclusive.
Exclusivity is also guaranteed when the target \ac{PE} performs a wait or test
operation on the same location and with the same datatype as one or more atomic
operations.

An \openshmem \emph{atomicity domain} is a set of communication
contexts whose associated teams (see Section~\ref{subsec:team}) are
all split by (possibly recursive) calls to a
\FUNC{shmem\_team\_split\_*} routine from a common predefined team.
\openshmem defines two such predefined teams, \LibHandleRef{SHMEM\_TEAM\_WORLD}
and \LibHandleRef{SHMEM\_TEAM\_SHARED} (see Section~\ref{subsec:library_handles}).%
\footnote{
  Although all \acp{PE} in \LibHandleRef{SHMEM\_TEAM\_SHARED} are also
  in \LibHandleRef{SHMEM\_TEAM\_WORLD}, and a \ac{PE}'s number can be
  translated from its \LibHandleRef{SHMEM\_TEAM\_SHARED} to
  \LibHandleRef{SHMEM\_TEAM\_WORLD}, the
  \LibHandleRef{SHMEM\_TEAM\_SHARED} team is defined as not having
  been created by a call to a \FUNC{shmem\_team\_split\_*} routine on
  \LibHandleRef{SHMEM\_TEAM\_WORLD}.
  Therefore, the two teams are distinct predefined teams forming
  separate atomicity domains.
}

\openshmem atomic operations do not guarantee exclusivity in the following
scenarios, all of which result in undefined behavior.
\begin{enumerate}
    \item \label{amo-scenario/1}
        When concurrent accesses to the same location are performed using
        \openshmem atomic operations using communication contexts in
        different atomicity domains.
    \item \label{amo-scenario/2}
        When concurrent accesses to the same location are performed using
        \openshmem atomic operations using different datatypes.
    \item \label{amo-scenario/3}
        When atomic and non-atomic \openshmem operations are used to access
        the same location concurrently.
    \item \label{amo-scenario/4}
        When \openshmem atomic operations and non-\openshmem operations (e.g.,
        load and store operations) are used to access the same location
        concurrently.
\end{enumerate}

\SourceExample{./example_code/amo_scenario_1.c}{
  The following \CorCpp example illustrates scenario 1.
  In this example, different atomicity domains are used to access
  the same location, resulting in undefined behavior.
  The undefined behavior can be resolved by using communication
  contexts in the same atomicity domain in all concurrent operations.
}

\SourceExample{./example_code/amo_scenario_2.c}{
  The following \CorCpp example illustrates scenario 2.  In this example,
  different datatypes are used to access the same location concurrently,
  resulting in undefined behavior.  The undefined behavior can be resolved by
  using the same datatype in all concurrent operations.  For example, the
  32-bit value can be left-shifted and a 64-bit atomic OR operation can be
  used.
}

\SourceExample{./example_code/amo_scenario_3.c}{
  The following \CorCpp example illustrates scenario 3.  In this example,
  atomic increment operations are concurrent with a non-atomic reduction
  operation, resulting in undefined behavior.  The undefined behavior can be
  resolved by inserting a barrier operation before the reduction.  The
  barrier ensures that all local and remote \acp{AMO} have completed before the
  reduction operation accesses $x$.
}

\SourceExample{./example_code/amo_scenario_4.c}{
  The following \CorCpp example illustrates scenario 4.  In this example, an
  \openshmem atomic increment operation is concurrent with a local increment
  operation, resulting in undefined behavior.  The undefined behavior can be
  resolved by replacing the local increment operation with an \openshmem
  atomic increment.
}



\section{Execution Model}\label{subsec:execution_model}
An \openshmem program consists of a set of \openshmem processes called
\acp{PE}.  While not required by \openshmem, in typical usage, \acp{PE} are
executed using a single program, multiple data (\ac{SPMD}) model.  \ac{SPMD}
requires each \ac{PE} to use the same executable; however, \acp{PE} are able to
follow divergent control paths.  \acp{PE} are often implemented using operating
system (\ac{OS}) processes and \acp{PE} are permitted to create additional
threads, when supported by the \openshmem library.

\ac{PE} execution is loosely coupled, relying on \openshmem operations to
communicate and synchronize among executing \acp{PE}.  The \openshmem phase in
a program begins with a call to the initialization routine \FUNC{shmem\_init}%
\footnote{\FUNC{start\_pes} has been deprecated as of \openshmem[1.2]}
or \FUNC{shmem\_init\_thread}, which must be performed before using any of the
other \openshmem library routines. 
An \openshmem program concludes its use of the \openshmem library when all \acp{PE} call
\FUNC{shmem\_finalize} or any \ac{PE} calls \FUNC{shmem\_global\_exit}.
During a call to \FUNC{shmem\_finalize}, the \openshmem library must
complete all pending communication and release all the resources associated to
the library using an implicit collective synchronization across \acp{PE}.
Calling any \openshmem routine before initialization or after
\FUNC{shmem\_finalize} leads to undefined behavior. After finalization, a
subsequent initialization call also leads to undefined behavior.

The \acp{PE} of the \openshmem program are identified by unique integers.  The
identifiers are integers assigned in a monotonically increasing manner from zero
to one less than the total number of \acp{PE}. \ac{PE} identifiers are used for
\openshmem calls (e.g. to specify \OPR{put} or \OPR{get} routines on symmetric
data objects, collective synchronization calls) or to dictate a control flow for
\acp{PE} using constructs of \Cstd. The identifiers are fixed for
the duration of the \openshmem phase of a program.

\subsection{Progress of OpenSHMEM Operations}\label{subsec:progress}

The \openshmem model assumes that computation and communication are naturally
overlapped. \openshmem programs are expected to exhibit progression of
communication both with and without \openshmem calls. Consider a \ac{PE} that is
engaged in a computation with no \openshmem calls. Other \acp{PE} should be able
to communicate (\OPR{put}, \OPR{get}, \OPR{atomic}, etc) and
complete communication operations with that computationally-bound \ac{PE}
without that \ac{PE} issuing any explicit \openshmem calls. One-sided \openshmem
communication calls involving that \ac{PE} should progress regardless of when
that \ac{PE} next engages in an \openshmem call.

\parimpnotes{
  An \openshmem implementation for hardware that does not provide
  asynchronous communication capabilities may require a software progress
  thread in order to process remotely-issued communication requests without
  explicit program calls to the \openshmem library.

  High performance implementations of \openshmem are expected to leverage
  hardware offload capabilities and provide asynchronous one-sided
  communication without software assistance.

  Implementations should avoid deferring the execution of one-sided
  operations until a synchronization point where data is known to be
  available. High-quality implementations should attempt asynchronous delivery
  whenever possible, for performance reasons. Additionally, the \openshmem
  community discourages releasing \openshmem implementations that do not
  provide asynchronous one-sided operations, as these have very limited
  performance value for \openshmem programs.
}

\subsection{Invoking OpenSHMEM Operations}

Pointer arguments to \openshmem routines that point to non-const data must not
overlap in memory with other arguments to the same \openshmem operation, with
the exception of in-place reductions as described in Section~\ref{subsec:shmem_reductions}.
Otherwise, the behavior is undefined.  Two arguments overlap in memory if any
of their data elements are contained in the same physical memory locations.
For example, consider an address $a$ returned by the \FUNC{shmem\_ptr} operation
for symmetric object $A$ on \ac{PE} $i$.  Providing the local address $a$ and
the symmetric address of object $A$ to an \openshmem operation targeting
\ac{PE} $i$ results in undefined behavior.

\openshmem routines with multiple symmetric object arguments do not require
these symmetric objects to be located within the same symmetric memory segment.
For example, objects located in the symmetric data segment and objects located
in the symmetric heap can be provided as arguments to the same \openshmem
operation.


\section{Language Bindings and Conformance}\label{subsec:bindings}
\input{content/language_bindings_and_conformance}

\section{Library Constants}\label{subsec:library_constants}
\TableIndex{Library Constants}
\TableIndex{Constants}

The \openshmem library provides a set of compile-time constants that may
be used to specify options to \ac{API} routines, provide implementation-specific
parameters, or return information about the implementation.
All constants that start with \CONST{\_SHMEM\_*} are deprecated,
but provided for backwards compatibility.

\begin{longtable}{|p{0.45\textwidth}|p{0.5\textwidth}|}
\hline
\textbf{Constant} & \textbf{Description}
\tabularnewline \hline
\endhead
%%
\LibConstDecl{SHMEM\_THREAD\_SINGLE} &
The \openshmem thread support level which specifies that the program
must not be multithreaded.
See Section~\ref{subsec:thread_support} for more detail about its use.
\tabularnewline \hline
%%
\LibConstDecl{SHMEM\_THREAD\_FUNNELED} &
The \openshmem thread support level which specifies that the program
may be multithreaded but must ensure that only the main thread invokes
the \openshmem interfaces.
See Section~\ref{subsec:thread_support} for more detail about its use.
\tabularnewline \hline
%%
\LibConstDecl{SHMEM\_THREAD\_SERIALIZED} &
The \openshmem thread support level which specifies that the program
may be multithreaded but must ensure that the \openshmem interfaces
are not invoked concurrently by multiple threads.
See Section~\ref{subsec:thread_support} for more detail about its use.
\tabularnewline \hline
%%
\LibConstDecl{SHMEM\_THREAD\_MULTIPLE} &
The \openshmem thread support level which specifies that the program
may be multithreaded and any thread may invoke the \openshmem interfaces.
See Section~\ref{subsec:thread_support} for more detail about its use.
\tabularnewline \hline
%%
\LibConstDecl{SHMEM\_TEAM\_NUM\_CONTEXTS} &
The bitwise flag which specifies that a team creation routine should use the
\VAR{num\_contexts} member of the provided
\CTYPE{shmem\_team\_config\_t} configuration parameter as a request.
See Sections~\ref{subsec:shmem_team_config_t} and
\ref{subsec:shmem_team_split_strided} for more detail about its use.
\tabularnewline \hline
%%
\LibConstDecl{SHMEM\_TEAM\_INVALID} &
A value corresponding to an invalid team.
This value can be used to initialize or update team handles to indicate
that they do not reference a valid team.
When managed in this way, applications can use an equality comparison
to test whether a given team handle references a valid team.
See Section~\ref{subsec:team} for more detail about its use.
\tabularnewline \hline
%%
\LibConstDecl{SHMEM\_CTX\_INVALID} &
A value corresponding to an invalid communication context.
This value can be used to initialize or update context handles to indicate
that they do not reference a valid context.
When managed in this way, applications can use an equality comparison
to test whether a given context handle references a valid context.
See Section~\ref{sec:ctx} for more detail about its use.
\tabularnewline \hline
%%
\LibConstDecl{SHMEM\_CTX\_SERIALIZED} &
The context creation option which specifies that the given context
is shareable but will not be used by multiple threads concurrently.
See Section~\ref{subsec:shmem_ctx_create} for more detail about its use.
\tabularnewline \hline
%%
\LibConstDecl{SHMEM\_CTX\_PRIVATE} &
The context creation option which specifies that the given context
will be used only by the thread that created it.
See Section~\ref{subsec:shmem_ctx_create} for more detail about its use.
\tabularnewline \hline
%%
\LibConstDecl{SHMEM\_CTX\_NOSTORE} &
The context creation option which specifies that quiet and fence operations
performed on the given context are not required to enforce completion and
ordering of memory store operations.
See Section~\ref{subsec:shmem_ctx_create} for more detail about its use.
\tabularnewline \hline
%%
\LibConstDecl{SHMEM\_CTX\_SESSION\_TOTAL\_OPS} &
The bitwise flag which specifies that a session start routine should use the
\VAR{total\_ops} member of the provided \CTYPE{shmem\_ctx\_session\_config\_t}
configuration parameter as a hint. See \ref{subsec:shmem_ctx_session_config_t}
for more detail about its use.
\tabularnewline \hline
%%
\LibConstDecl{SHMEM\_CTX\_SESSION\_BATCH} &
The session start option which specifies that operations in the given session
are latency tolerant and may be candidates for batching. See
\ref{subsec:shmem_ctx_session_start} for more detail about its use.
\tabularnewline \hline
%%
\LibConstDecl{SHMEM\_SIGNAL\_SET} &
An integer constant expression corresponding to the signal update set operation.
See Section~\ref{subsec:shmem_put_signal} and
Section~\ref{subsec:shmem_put_signal_nbi} for more detail about its use.
\tabularnewline \hline
%%
\LibConstDecl{SHMEM\_SIGNAL\_ADD} &
An integer constant expression corresponding to the signal update add operation.
See Section~\ref{subsec:shmem_put_signal} and
Section~\ref{subsec:shmem_put_signal_nbi} for more detail about its use.
\tabularnewline \hline
%%
\LibConstDecl{SHMEM\_MALLOC\_ATOMICS\_REMOTE} &
The hint to the memory allocation routine which specifies that the allocated
memory will be used for atomic variables. See Section \ref{subsec:shmmallochint}
for more detail about its use.
\tabularnewline \hline
%%
\LibConstDecl{SHMEM\_MALLOC\_SIGNAL\_REMOTE} &
The hint to the memory allocation routine which specifies that the allocated
memory will be used for signal variables. See Section
\ref{subsec:shmmallochint} for more detail about its use.
\tabularnewline \hline
%%
\begin{DeprecateBlock}
  \LibConstDecl{SHMEM\_SYNC\_VALUE}
  \LibConstDecl{\_SHMEM\_SYNC\_VALUE}
\end{DeprecateBlock}
&
The value used to initialize the elements of \VAR{pSync} arrays.
The value of this constant is implementation specific.
See Section~\ref{subsec:coll} for more detail about its use.
\tabularnewline \hline
%%
\begin{DeprecateBlock}
  \LibConstDecl{SHMEM\_SYNC\_SIZE}
\end{DeprecateBlock}
&
Length of a work array that can be used with any \openshmem collective
communication operation.
Work arrays sized for specific operations may consume less memory.
The value of this constant is implementation specific.
See Section~\ref{subsec:coll} for more detail about its use.
\tabularnewline \hline
%%
\begin{DeprecateBlock}
  \LibConstDecl{SHMEM\_BCAST\_SYNC\_SIZE}
  \LibConstDecl{\_SHMEM\_BCAST\_SYNC\_SIZE}
\end{DeprecateBlock}
&
Length of the \VAR{pSync} arrays needed for broadcast routines. The value
of this constant is implementation specific.
See Section~\ref{subsec:shmem_broadcast} for more detail about its use.
\tabularnewline \hline
%%
\begin{DeprecateBlock}
  \LibConstDecl{SHMEM\_REDUCE\_SYNC\_SIZE}
  \LibConstDecl{\_SHMEM\_REDUCE\_SYNC\_SIZE}
\end{DeprecateBlock}
&
Length of the work arrays needed for reduction routines.
The value of this constant is implementation specific.
See Section~\ref{subsec:shmem_reductions} for more detail about its use.
\tabularnewline \hline
%%
\begin{DeprecateBlock}
  \LibConstDecl{SHMEM\_BARRIER\_SYNC\_SIZE}
  \LibConstDecl{\_SHMEM\_BARRIER\_SYNC\_SIZE}
\end{DeprecateBlock}
&
Length of the work array needed for barrier routines.
The value of this constant is implementation specific.
See Section~\ref{subsec:shmem_barrier} for more detail about its use.

\tabularnewline \hline
%%
\begin{DeprecateBlock}
  \LibConstDecl{SHMEM\_COLLECT\_SYNC\_SIZE}
  \LibConstDecl{\_SHMEM\_COLLECT\_SYNC\_SIZE}
\end{DeprecateBlock}
&
Length of the work array needed for collect routines.
The value of this constant is implementation specific.
See Section~\ref{subsec:shmem_collect} for more detail about its use.
\tabularnewline \hline
%%
\begin{DeprecateBlock}
  \LibConstDecl{SHMEM\_ALLTOALL\_SYNC\_SIZE}
\end{DeprecateBlock}
&
Length of the work array needed for \FUNC{shmem\_alltoall} routines.
The value of this constant is implementation specific.
See Section~\ref{subsec:shmem_alltoall} for more detail about its use.
\tabularnewline \hline
%%
\begin{DeprecateBlock}
  \LibConstDecl{SHMEM\_ALLTOALLS\_SYNC\_SIZE}
\end{DeprecateBlock}
&
Length of the work array needed for \FUNC{shmem\_alltoalls} routines.
The value of this constant is implementation specific.
See Section~\ref{subsec:shmem_alltoalls} for more detail about its use.
\tabularnewline \hline
%%
\begin{DeprecateBlock}
  \LibConstDecl{SHMEM\_REDUCE\_MIN\_WRKDATA\_SIZE}
  \LibConstDecl{\_SHMEM\_REDUCE\_MIN\_WRKDATA\_SIZE}
\end{DeprecateBlock}
&
Minimum length of work arrays used in various collective routines.
\tabularnewline \hline
%%
\LibConstDecl{SHMEM\_MAJOR\_VERSION}
\begin{DeprecateBlock}
  \LibConstDecl{\_SHMEM\_MAJOR\_VERSION}
\end{DeprecateBlock}
&
Integer representing the major version of \openshmem Specification in use.
\tabularnewline \hline
%%
\LibConstDecl{SHMEM\_MINOR\_VERSION}
\begin{DeprecateBlock}
  \LibConstDecl{\_SHMEM\_MINOR\_VERSION}
\end{DeprecateBlock}
&
Integer representing the minor version of \openshmem Specification in use.
\tabularnewline \hline
%%
\LibConstDecl{SHMEM\_MAX\_NAME\_LEN}
\begin{DeprecateBlock}
  \LibConstDecl{\_SHMEM\_MAX\_NAME\_LEN}
\end{DeprecateBlock}
&
Integer representing the maximum length of \CONST{SHMEM\_VENDOR\_STRING}.
\tabularnewline \hline
%%
\LibConstDecl{SHMEM\_VENDOR\_STRING}
\begin{DeprecateBlock}
  \LibConstDecl{\_SHMEM\_VENDOR\_STRING}
\end{DeprecateBlock}
&
String representing vendor defined information of size at most
\CONST{SHMEM\_MAX\_NAME\_LEN}.
In \CorCpp{}, the string is terminated by a null character.
\tabularnewline \hline
%%
\LibConstDecl{SHMEM\_CMP\_EQ}
\begin{DeprecateBlock}
  \LibConstDecl{\_SHMEM\_CMP\_EQ}
\end{DeprecateBlock}
&
An integer constant expression corresponding to the
``equal to'' comparison operation.
See Section~\ref{subsec:p2p_intro} for more detail about its use.
\tabularnewline \hline
%%
\LibConstDecl{SHMEM\_CMP\_NE}
\begin{DeprecateBlock}
  \LibConstDecl{\_SHMEM\_CMP\_NE}
\end{DeprecateBlock}
&
An integer constant expression corresponding to the
``not equal to'' comparison operation.
See Section~\ref{subsec:p2p_intro} for more detail about its use.
\tabularnewline \hline
%%
\LibConstDecl{SHMEM\_CMP\_LT}
\begin{DeprecateBlock}
  \LibConstDecl{\_SHMEM\_CMP\_LT}
\end{DeprecateBlock}
&
An integer constant expression corresponding to the
``less than'' comparison operation.
See Section~\ref{subsec:p2p_intro} for more detail about its use.
\tabularnewline \hline
%%
\LibConstDecl{SHMEM\_CMP\_LE}
\begin{DeprecateBlock}
  \LibConstDecl{\_SHMEM\_CMP\_LE}
\end{DeprecateBlock}
&
An integer constant expression corresponding to the
``less than or equal to'' comparison operation.
See Section~\ref{subsec:p2p_intro} for more detail about its use.
\tabularnewline \hline
%%
\LibConstDecl{SHMEM\_CMP\_GT}
\begin{DeprecateBlock}
  \LibConstDecl{\_SHMEM\_CMP\_GT}
\end{DeprecateBlock}
&
An integer constant expression corresponding to the
``greater than'' comparison operation.
See Section~\ref{subsec:p2p_intro} for more detail about its use.
\tabularnewline \hline
%%
\LibConstDecl{SHMEM\_CMP\_GE}
\begin{DeprecateBlock}
  \LibConstDecl{\_SHMEM\_CMP\_GE}
\end{DeprecateBlock}
&
An integer constant expression corresponding to the
``greater than or equal to'' comparison operation.
See Section~\ref{subsec:p2p_intro} for more detail about its use.
\tabularnewline \hline
%%
\end{longtable}


\section{Library Handles}\label{subsec:library_handles}
\TableIndex{Library Handles}
\TableIndex{Handles}

The \openshmem library provides a set of predefined named constant handles.
All named constants can be used in initialization expressions or assignments,
but not necessarily in array declarations or as labels in \Cstd switch statements.
This implies named constants to be link-time but not necessarily compile-time
constants.

\begin{longtable}{|p{0.45\textwidth}|p{0.5\textwidth}|}
\hline
\textbf{Handle} & \textbf{Description}
\tabularnewline \hline
\endhead
%%
\LibHandleDecl{SHMEM\_TEAM\_WORLD} &
Handle of type \CTYPE{shmem\_team\_t} that corresponds to the
default team of all \acp{PE} in the \openshmem program.  All point-to-point
communication operations and collective synchronizations that do not specify a team
are performed on the default team.
See Section~\ref{subsec:team} for more detail about its use.
\tabularnewline \hline
%%
\LibHandleDecl{SHMEM\_TEAM\_SHARED} &
Handle of type \CTYPE{shmem\_team\_t} that corresponds to a team of \acp{PE}
that share a memory domain. When this handle is used by some \ac{PE},
it will refer to a team of one or more \acp{PE} that would return a non-null
pointer from \FUNC{shmem\_ptr} for symmetric objects on that \ac{PE},
and vice versa. This means that symmetric objects on each \ac{PE} are
directly load/store accessible by all \acp{PE} in the team.
See Section~\ref{subsec:team} for more detail about its use.
\tabularnewline \hline
%%
\LibHandleDecl{SHMEM\_CTX\_DEFAULT} &
Handle of type \CTYPE{shmem\_ctx\_t} that corresponds to the
default communication context.  All point-to-point communication operations
and synchronizations that do not specify a context are performed on the
default context.
See Section~\ref{sec:ctx} for more detail about its use.
\tabularnewline \hline
%%
\end{longtable}


\section{Environment Variables }\label{subsec:environment_variables}
\input{content/environment_variables}




\clearpage



\section{OpenSHMEM Library API}\label{sec:openshmem_library_api}

\subsection{Library Setup, Exit, and Query Routines}
The library setup and query interfaces that initialize and monitor the parallel
environment of the \acp{PE}.

\subsubsection{\textbf{SHMEM\_INIT}}\label{subsec:shmem_init}
\apisummary{
    A collective operation that allocates and initializes the resources used by
    the \openshmem library.
}

\begin{apidefinition}

\begin{Csynopsis}
void @\FuncDecl{shmem\_init}@(void);
\end{Csynopsis}

\begin{apiarguments}
    \apiargument{None.}{}{}
\end{apiarguments}

\apidescription{
    \FUNC{shmem\_init} allocates and initializes resources used by the \openshmem
    library. It is a collective operation that all \acp{PE} must call before any
    other \openshmem routine may be called, except \FUNC{shmem\_query\_initialized}
    which checks the current initialized state of the library. In the
    \openshmem program which it initialized, each call to \FUNC{shmem\_init} must
    be matched with a corresponding call to \FUNC{shmem\_finalize}.

    The \FUNC{shmem\_init} and \FUNC{shmem\_init\_thread} initialization
    routines may be called multiple times within an \openshmem program. A
    corresponding call to \FUNC{shmem\_finalize} must be made for each call to
    an \openshmem initialization routine. The \openshmem library must not be
    finalized until after the last call to \FUNC{shmem\_finalize} and may be
    re-initialized with a subsequent call to an initialization routine.

}

\apireturnvalues{
    None.
}

\begin{DeprecateBlock}
\apinotes{
    As of \openshmem[1.2], the use of \FUNC{start\_pes} has been
    deprecated and calls to it should be replaced with calls to \FUNC{shmem\_init}.
    While support for \FUNC{start\_pes} is still required in \openshmem libraries,
    users are encouraged to use \FUNC{shmem\_init}. An important difference between
    \FUNC{shmem\_init} and \FUNC{start\_pes} is that every call to
    \FUNC{shmem\_init} within a program must be matched with a call to \FUNC{shmem\_finalize}.
}
\end{DeprecateBlock}

\begin{apiexamples}

\apicexample
    {The following \FUNC{shmem\_init} example is for \Cstd[11] programs:}
    {example_code/shmem_init_example.c}
    {}

\end{apiexamples}

\end{apidefinition}


\subsubsection{\textbf{SHMEM\_MY\_PE}}\label{subsec:shmem_my_pe}
\input{content/shmem_my_pe}

\subsubsection{\textbf{SHMEM\_N\_PES}}\label{subsec:shmem_n_pes}
\apisummary{
    Returns the number of \acp{PE} running in a program.
}

\begin{apidefinition}

\begin{Csynopsis}
int @\FuncDecl{shmem\_n\_pes}@(void);
\end{Csynopsis}

\begin{apiarguments}
    \apiargument{None.}{}{}
\end{apiarguments}

\apidescription{
    The routine returns the number of \acp{PE} running in the program.
}

\apireturnvalues{
    Integer -  Number of \acp{PE} running in the \openshmem program.
}

\apinotes{
    As of \openshmem[1.2] the use of \FUNC{\_num\_pes} has been
    deprecated. Although \openshmem libraries are required to support the call,
    users are encouraged to use \FUNC{shmem\_n\_pes} instead.  The behavior and
    signature  of the routine \FUNC{shmem\_n\_pes} remains unchanged from the
    deprecated \FUNC{\_num\_pes} version.
}

\begin{apiexamples}

\apicexample
    {The following \FUNC{shmem\_my\_pe} and \FUNC{shmem\_n\_pes} example is for
    \CorCpp{} programs:}
    {./example_code/shmem_npes_example.c}
    {}

\end{apiexamples}

\end{apidefinition}


\subsubsection{\textbf{SHMEM\_FINALIZE}}\label{subsec:shmem_finalize}
\apisummary{
    A collective operation that releases all resources used by the \openshmem
    library.  This only terminates the \openshmem portion of a program, not the
    entire program.
}

\begin{apidefinition}

\begin{Csynopsis}
void @\FuncDecl{shmem\_finalize}@(void);
\end{Csynopsis}

\begin{apiarguments}
    \apiargument{None.}{}{}
\end{apiarguments}

\apidescription{
    \FUNC{shmem\_finalize} ends the \openshmem
    portion of a program previously initialized by \FUNC{shmem\_init} or \FUNC{shmem\_init\_thread}.
   This is a collective
    operation that requires all \acp{PE} to participate in the call.

    An \openshmem program may perform a series of matching
    initialization and finalization calls.
    The last call to \FUNC{shmem\_finalize} in this series
    releases all resources used by the \openshmem library. 
    This call destroys all teams created by the \openshmem program.
    As a result, all shareable contexts are destroyed.
    The user is
    responsible for destroying all contexts with the
    \CONST{SHMEM\_CTX\_PRIVATE} option enabled prior to this call;
    otherwise, the behavior is undefined.

    The last call to \FUNC{shmem\_finalize} performs an implicit global barrier
    to ensure that pending communications are completed and that no resources
    are released until all \acp{PE} have entered \FUNC{shmem\_finalize}. All
    other calls to \FUNC{shmem\_finalize} perform an operation semantically
    equivalent to \FUNC{shmem\_barrier\_all} and return without freeing any
    \openshmem resources.

    The last call to \FUNC{shmem\_finalize} causes the \openshmem library
    to enter an uninitialized state. No further \openshmem calls may be
    made until an \openshmem initialization routine is called.
    All processes
    that represent the \acp{PE} will still exist after the
    call to \FUNC{shmem\_finalize} returns, but they will no longer have access
    to resources that have been released.
}

\apireturnvalues{
    None.
}

\apinotes{
    The last call to \FUNC{shmem\_finalize} releases all resources used by the \openshmem library
    including the symmetric memory heap and pointers initiated by
    \FUNC{shmem\_ptr}. This collective operation requires all \acp{PE} to
    participate in the call, not just a subset of the \acp{PE}. The
    non-\openshmem portion of a program may continue after a call to
    \FUNC{shmem\_finalize} by all \acp{PE}.

    Calls to \FUNC{shmem\_finalize} that are not the last in a series of
    initialization and finalization calls do not free any \openshmem resources.
    Thus, teams, contexts, or symmetric memory allocations may be leaked until
    the final call to \FUNC{shmem\_finalize}. Applications that perform
    multiple initialization and finalization calls should free resources prior
    to calling \FUNC{shmem\_finalize} to avoid such leaks.
}

\begin{apiexamples}

\apicexample
    {The following finalize example is for \Cstd[11] programs:}
    {./example_code/shmem_finalize_example.c}
    {}

\end{apiexamples}

\end{apidefinition}


\subsubsection{\textbf{SHMEM\_GLOBAL\_EXIT}}\label{subsec:shmem_global_exit}
\apisummary{
    A routine that allows any \ac{PE} to force termination of an entire program.
}

\begin{apidefinition}

\begin{C11synopsis}
_Noreturn void @\FuncDecl{shmem\_global\_exit}@(int status);
\end{C11synopsis}

\begin{Csynopsis}
void @\FuncDecl{shmem\_global\_exit}@(int status);
\end{Csynopsis}

\begin{apiarguments}
    \apiargument{IN}{status}{The exit status from the main program.}
\end{apiarguments}

\apidescription{
    \FUNC{shmem\_global\_exit} is a non-collective routine that allows any one
    \ac{PE} to force termination of an \openshmem program for all \acp{PE},
    passing an exit status to the execution environment. This routine terminates
    the entire program, not just the \openshmem portion.  When any \ac{PE} calls
    \FUNC{shmem\_global\_exit}, it results in the immediate notification to all
    \acp{PE} to terminate.  \FUNC{shmem\_global\_exit} flushes I/O and releases
    resources in accordance with \CorCpp language requirements for normal
    program termination. If more than one \ac{PE} calls
    \FUNC{shmem\_global\_exit}, then the exit status returned to the environment
    shall be one of the values passed to \FUNC{shmem\_global\_exit} as the
    status argument.  There is no return to the caller of
    \FUNC{shmem\_global\_exit}; control is returned from the \openshmem program
    to the execution environment for all \acp{PE}.
}

\apireturnvalues{
    None.
}


\apinotes{
    \FUNC{shmem\_global\_exit} may be used in situations where one or more
    \acp{PE} have determined that the program has completed and/or should
    terminate early.  Accordingly, the integer status argument can be used to
    pass any information about the nature of the exit; e.g., that the program
    encountered an error or found a solution.
    Since \FUNC{shmem\_global\_exit} is a non-collective
    routine, there is no implied synchronization, and all \acp{PE} must
    terminate regardless of their current execution state. While I/O must be
    flushed for standard language I/O calls from \CorCpp, it is
    implementation dependent as to how I/O done by other means (e.g., third
    party I/O libraries) are handled. Similarly, resources are released
    according to \CorCpp standard language requirements, but this may not
    include all resources allocated for the \openshmem program. However, a
    quality implementation will make a best effort to flush all I/O and clean
    up all resources.
}

\begin{apiexamples}

\apicexample
    {}
    {./example_code/shmem_global_exit_example.c}
    {}

\end{apiexamples}

\end{apidefinition}


\subsubsection{\textbf{SHMEM\_PE\_ACCESSIBLE}}\label{subsec:shmem_pe_accessible}
\apisummary{
    Determines whether a \ac{PE} is accessible via \openshmem's data transfer
    routines.
}

\begin{apidefinition}

\begin{Csynopsis}
int @\FuncDecl{shmem\_pe\_accessible}@(int pe);
\end{Csynopsis}

\begin{apiarguments}
    \apiargument{IN}{pe}{Specific \ac{PE} to be checked for accessibility from
    the local \ac{PE}.}
\end{apiarguments}

\apidescription{
    \FUNC{shmem\_pe\_accessible} is a query routine that indicates whether a
    specified target \ac{PE} is accessible via \openshmem from the local \ac{PE}. The
    \FUNC{shmem\_pe\_accessible} routine returns a value indicating whether the remote
    \ac{PE} is a process running from the same executable file as the local
    \ac{PE}, thereby indicating whether full support for symmetric data objects,
    which may reside in either static memory or the symmetric heap, is available.
}

\apireturnvalues{
    The return value is 1 if the specified target \ac{PE} is a valid \ac{PE}
    for \openshmem routines; otherwise, it is 0.
}

\apinotes{
    This routine may be particularly useful for hybrid programming with other
    communication libraries (such as \ac{MPI}) or parallel languages.  For
    example, when an \ac{MPI} job uses \ac{MPMD} mode, multiple executable
    \ac{MPI} programs are executed as part of the same MPI job.  In such cases,
    \openshmem support may only be available between processes running from the
    same executable file.  In addition, some environments may allow a hybrid
    job to span multiple network partitions.  In such scenarios, \openshmem
    support may only be available between \acp{PE} within the same partition.
}

\end{apidefinition}


\subsubsection{\textbf{SHMEM\_ADDR\_ACCESSIBLE}}\label{subsec:shmem_addr_accessible}
\apisummary{
    Determines whether an address is accessible via \openshmem data transfer
    routines from the specified target \ac{PE}.
}

\begin{apidefinition}

\begin{Csynopsis}
int @\FuncDecl{shmem\_addr\_accessible}@(const void *addr, int pe);
\end{Csynopsis}

\begin{apiarguments}
    \apiargument{IN}{addr}{Local address of data object to query.}
    \apiargument{IN}{pe}{Integer id of a target \ac{PE}.}
\end{apiarguments}

\apidescription{
    \FUNC{shmem\_addr\_accessible} is a query routine that indicates whether
    the address \VAR{addr} can be used to access the given data object on the
    specified \ac{PE} via \openshmem routines.

    This routine verifies that the data object is symmetric and accessible with
    respect to a target \ac{PE} via \openshmem data transfer routines.  The
    specified address \VAR{addr} is the local address of the data object on the
    local \ac{PE}.
}

\apireturnvalues{
    The return value is \CONST{1} if the local address \VAR{addr} is also a symmetric
    address and the given data object is accessible via \openshmem routines on
    the specified target \ac{PE}; otherwise, it is \CONST{0}.
}

\apinotes{
    This routine may be particularly useful for hybrid programming with other
    communication libraries (such as \ac{MPI}) or parallel languages.  For
    example, when an \ac{MPI} job uses \ac{MPMD} mode, multiple executable
    \ac{MPI} programs may use \openshmem routines.  In such cases, static
    memory, such as a \Cstd global variable, is
    symmetric between processes running from the same executable file, but is
    not symmetric between processes running from different executable files.
    Data allocated from the symmetric heap (e.g., using \FUNC{shmem\_malloc})
    is symmetric across the same or different executable files.
}

\end{apidefinition}


\subsubsection{\textbf{SHMEM\_PTR}}\label{subsec:shmem_ptr}
\apisummary{
    Returns a local pointer to a symmetric data object on the specified \ac{PE} in the world team.
}

\begin{apidefinition}

\begin{Csynopsis}
void *@\FuncDecl{shmem\_ptr}@(const void *dest, int pe);
\end{Csynopsis}

\begin{apiarguments}
\apiargument{IN}{dest}{The symmetric address of the remotely accessible data
    object to be referenced.}
\apiargument{IN}{pe}{An integer that indicates the \ac{PE} number on which \dest{} is to
    be accessed.}
\end{apiarguments}

\apidescription{
    \FUNC{shmem\_ptr} returns an address that may be used to directly reference
    \dest{} on the specified \ac{PE} in the world team.  This address can be
    assigned to a pointer.  After that, ordinary loads and stores to \dest{} may
    be performed.  The address returned by \FUNC{shmem\_ptr} is a local address
    to a remotely accessible data object.  Providing this address to an argument
    of an \openshmem routine that requires a symmetric address results in
    undefined behavior.


    The \FUNC{shmem\_ptr} routine can provide efficient means to accomplish
    communication, for example when a sequence of reads and writes to a data
    object on a remote \ac{PE} does not match the access pattern provided in an
    \openshmem data transfer routine like \FUNC{shmem\_put} or
    \FUNC{shmem\_iget}.
}

\apireturnvalues{
    A local pointer to the remotely accessible \dest{} data object is returned
    when it can be accessed using memory loads and stores.  Otherwise, a null
    pointer is returned.
}

\apinotes{
    When calling \FUNC{shmem\_ptr}, \dest{} is the address of the referenced
    symmetric data object on the calling \ac{PE}.
}

\begin{apiexamples}

\apicexample
    {In the following \Cstd[11] example, \ac{PE} 0 uses the \FUNC{shmem\_ptr}
    routine to query a pointer and directly access the \VAR{dest} array on
    \ac{PE} 1:}
    {./example_code/shmem_ptr_example.c}
    {}

\end{apiexamples}

\end{apidefinition}


\subsubsection{\textbf{SHMEM\_INFO\_GET\_VERSION}}\label{subsec:shmem_info_get_version}
\apisummary{
    Returns the major and minor version of the library implementation.
}

\begin{apidefinition}

\begin{Csynopsis}
void @\FuncDecl{shmem\_info\_get\_version}@(int *major, int *minor);
\end{Csynopsis}

\begin{apiarguments}
    \apiargument{OUT}{major}{The major version of the \openshmem Specification in use.}
    \apiargument{OUT}{minor}{The minor version of the \openshmem Specification in use.}
\end{apiarguments}

\apidescription{
    This routine returns the major and minor version of the \openshmem Specification
    in use.  For a given library implementation, the major and minor version
    returned by these calls are consistent with the library constants
    \CONST{SHMEM\_MAJOR\_VERSION} and \CONST{SHMEM\_MINOR\_VERSION}.
}

\apireturnvalues{
    None.
}

\end{apidefinition}


\subsubsection{\textbf{SHMEM\_INFO\_GET\_NAME}}\label{subsec:shmem_info_get_name}
\apisummary{
    This routine returns the vendor defined name string that is consistent
    with the library constant \CONST{SHMEM\_VENDOR\_STRING}.
}

\begin{apidefinition}

\begin{Csynopsis}
void @\FuncDecl{shmem\_info\_get\_name}@(char *name);
\end{Csynopsis}

\begin{apiarguments}
    \apiargument{OUT}{name}{The vendor defined string.}
\end{apiarguments}

\apidescription{
    This routine returns the vendor defined name string of size defined by
    the library constant \CONST{SHMEM\_MAX\_NAME\_LEN}. The program calling
    this function provides the \VAR{name} memory buffer of at least size
    \CONST{SHMEM\_MAX\_NAME\_LEN}. The implementation copies the vendor defined
    string of size at most \CONST{SHMEM\_MAX\_NAME\_LEN} to \VAR{name}. In
    \CorCpp, the string is terminated by a null character. If the
    \VAR{name} memory buffer is provided with size less than
    \CONST{SHMEM\_MAX\_NAME\_LEN}, behavior is undefined. For a given library
    implementation, the vendor string returned is consistent with the library
    constant \CONST{SHMEM\_VENDOR\_STRING}.
}

\apireturnvalues{
    None.
}

\end{apidefinition}


\subsubsection{\textbf{START\_PES}}\label{subsec:start_pes}
\input{content/start_pes}

\subsection{Thread Support}
\label{subsec:thread_support}
This section specifies the interaction between the \openshmem interfaces and
user threads.  It also describes the routines that can be used for initializing and
querying the thread environment. There are four levels of threading defined by
the \openshmem specification.

\begin{description}
\item[\LibConstRef{SHMEM\_THREAD\_SINGLE}] \hfill \\
The \openshmem program must not be multithreaded.

\item[\LibConstRef{SHMEM\_THREAD\_FUNNELED}] \hfill \\
The \openshmem program may be multithreaded. However, the program must ensure
that only the main thread invokes the \openshmem interfaces. The main thread
is the thread that invokes either \FUNC{shmem\_init} or \FUNC{shmem\_init\_thread}.

\item[\LibConstRef{SHMEM\_THREAD\_SERIALIZED}] \hfill \\
The \openshmem program may be multithreaded. However, the program must ensure
that the \openshmem interfaces are not invoked concurrently by multiple threads.

\item[\LibConstRef{SHMEM\_THREAD\_MULTIPLE}] \hfill \\
The \openshmem program may be multithreaded and any thread may invoke the \openshmem
interfaces.
\end{description}

\sloppypar
\noindent The thread level constants must have increasing integer values; i.e.,
\CONST{SHMEM\_THREAD\_SINGLE} < \CONST{SHMEM\_THREAD\_FUNNELED} <
\CONST{SHMEM\_THREAD\_SERIALIZED} < \CONST{SHMEM\_THREAD\_MULTIPLE}.
The following semantics apply to the usage of these models:

\begin{enumerate}
\item
In the \CONST{SHMEM\_THREAD\_FUNNELED}, \CONST{SHMEM\_THREAD\_SERIALIZED}, and
\CONST{SHMEM\_THREAD\_MULTIPLE} thread levels, the \FUNC{shmem\_init\_thread} and
\FUNC{shmem\_finalize} calls must be invoked by the same thread.

\item
Any \openshmem operation initiated by a thread is considered an action of the
\ac{PE} as a whole. The symmetric heap and symmetric variables scope are not
impacted by multiple threads invoking the \openshmem interfaces.
Each \ac{PE} has a single symmetric data segment and symmetric heap that is shared by
all threads within that \ac{PE}.  For example, a thread invoking a memory allocation
routine such as \FUNC{shmem\_malloc} allocates memory that is accessible by
all threads of the \ac{PE}. The requirement that the same symmetric heap operations
must be executed by all \acp{PE} in the same order also applies in a threaded
environment. Similarly, the completion of collective operations is not impacted
by multiple threads. For example, \FUNC{shmem\_barrier\_all} is completed when
all \acp{PE} enter and exit the \FUNC{shmem\_barrier\_all} call, even though
only one thread in the \ac{PE} is participating in the collective call.

\item Blocking \openshmem calls will only block the calling thread, allowing
other threads, if available, to continue executing. The calling thread will
be blocked until the event on which it is waiting occurs. Once the blocking call is
completed, the thread is ready to continue execution. A blocked thread
will not prevent progress of other threads on the same \ac{PE} and will not
prevent them from executing other \openshmem calls when the thread level permits.
In addition, a blocked thread will not prevent the progress of \openshmem calls
performed on other \acp{PE}.

\item In the \CONST{SHMEM\_THREAD\_MULTIPLE} thread level, all \openshmem calls
are thread-safe. That is, any two concurrently running threads may make \openshmem calls.

\item In the \CONST{SHMEM\_THREAD\_SERIALIZED} and \CONST{SHMEM\_THREAD\_MULTIPLE} thread levels,
if multiple threads call collective routines, including the symmetric heap
management routines, it is the programmer's responsibility to ensure the
correct ordering of collective calls.

\end{enumerate}


\subsubsection{\textbf{SHMEM\_INIT\_THREAD}}
\label{subsec:shmem_init_thread}
\apisummary{
Initializes the \openshmem library, similar to \FUNC{shmem\_init}, and performs any
initialization required for supporting the provided thread level.
}

\begin{apidefinition}

\begin{Csynopsis}
int @\FuncDecl{shmem\_init\_thread}@(int requested, int *provided);
\end{Csynopsis}

\begin{apiarguments}
\apiargument{IN}{requested}{The thread level support requested by the user.}
\apiargument{OUT}{provided}{The thread level support provided by the \openshmem implementation.}
\end{apiarguments}

\apidescription{
\FUNC{shmem\_init\_thread} initializes the \openshmem library in the same way as 
\FUNC{shmem\_init}. In addition, \FUNC{shmem\_init\_thread} also performs 
the initialization required for supporting the provided thread level. 
The argument \VAR{requested} is used to specify the desired level of 
thread support. The argument \VAR{provided} returns the support level 
provided by the library. The allowed values for \VAR{provided} and 
\VAR{requested} are \CONST{SHMEM\_THREAD\_SINGLE}, \CONST{SHMEM\_THREAD\_FUNNELED},
\CONST{SHMEM\_THREAD\_SERIALIZED}, and \CONST{SHMEM\_THREAD\_MULTIPLE}.

The \FUNC{shmem\_init} and \FUNC{shmem\_init\_thread} initialization
routines may be called multiple times within an \openshmem program. A
corresponding call to \FUNC{shmem\_finalize} must be made for each call to
an \openshmem initialization routine. The \openshmem library must not be
finalized until after the last call to \FUNC{shmem\_finalize} and may be
re-initialized with a subsequent call to an initialization routine.

If the call to \FUNC{shmem\_init\_thread}
is unsuccessful in allocating and initializing resources for the 
\openshmem library, then the behavior of any subsequent call 
to the \openshmem library is undefined.


}

\apireturnvalues{
\FUNC{shmem\_init\_thread} returns 0 upon success; otherwise, it returns a
nonzero value.
}

\apinotes{
The \openshmem library can be initialized either by \FUNC{shmem\_init} 
or \FUNC{shmem\_init\_thread}. If the \openshmem library is initialized 
by \FUNC{shmem\_init}, the library implementation can choose to 
support any one of the defined thread levels.

The \openshmem library may not be able to change the level of threading support
provided after the first initialization call has been made.
}

\end{apidefinition}


\subsubsection{\textbf{SHMEM\_QUERY\_THREAD}}
\label{subsec:shmem_query_thread}
\apisummary{
Returns the level of thread support provided by the library.
}

\begin{apidefinition}

\begin{Csynopsis}
void @\FuncDecl{shmem\_query\_thread}@(int *provided);
\end{Csynopsis}

\begin{apiarguments}
\apiargument{OUT}{provided}{The thread level support provided by the \openshmem implementation.}
\end{apiarguments}

\apidescription{
The \FUNC{shmem\_query\_thread} call returns the level of thread support
currently being provided. The value returned will be same as was returned in \VAR{provided}
by a call to \FUNC{shmem\_init\_thread}, if the \openshmem library was
initialized by \FUNC{shmem\_init\_thread}. If the library was initialized by
\FUNC{shmem\_init}, the implementation can choose to provide any one of the defined
thread levels, and \FUNC{shmem\_query\_thread} returns this thread level.

This function may be called at any time, regardless of the thread safety
level of the \openshmem library.
}

\apireturnvalues{
None.
}

\end{apidefinition}



\subsection{Memory Management Routines}
\label{sec:memory_management}

\openshmem provides a set of \acp{API} for managing the symmetric heap. The
\acp{API} allow one to dynamically allocate, deallocate, reallocate and align
symmetric data objects in the symmetric heap.

\subsubsection{\textbf{SHMEM\_MALLOC, SHMEM\_FREE, SHMEM\_REALLOC, SHMEM\_ALIGN}}\label{subsec:shfree}
\apisummary{
  Collectively allocate symmetric memory.
}

\begin{apidefinition}

\begin{Csynopsis}
void *@\FuncDecl{shmem\_malloc}@(size_t size);
\end{Csynopsis}

\begin{apiarguments}
  \apiargument{IN}{size}{The size, in bytes, of a block to be
    allocated from the symmetric heap.}
\end{apiarguments}


\apidescription{
  The \FUNC{shmem\_malloc} routine is a collective operation on the
  world team and returns the symmetric address of a
  block of at least \VAR{size} bytes, which shall be suitably aligned
  so that it may be assigned to a pointer to any type of object.
  This space is allocated from the symmetric heap (in contrast to
  \FUNC{malloc}, which allocates from the private heap).
  When \VAR{size} is zero, the \FUNC{shmem\_malloc} routine performs
  no action and returns a null pointer; otherwise,
  \FUNC{shmem\_malloc} calls a procedure that is semantically equivalent 
  to \FUNC{shmem\_barrier\_all} on exit. This ensures that all \acp{PE} participate
  in the memory allocation, and that the memory on other \acp{PE} can be used as soon as the local
  \ac{PE} returns. 
  The value of the \VAR{size} argument must be identical on all
  \acp{PE}; otherwise, the behavior is undefined.
}

\apireturnvalues{
  The \FUNC{shmem\_malloc} routine returns the symmetric address of
  the allocated space; otherwise, it returns a null pointer.
}

\end{apidefinition}


\subsubsection{\textbf{SHMEM\_CALLOC}}\label{subsec:shmem_calloc}
\apisummary{
  Collectively allocate a zeroed block of symmetric memory.
}

\begin{apidefinition}

\begin{Csynopsis}
void *@\FuncDecl{shmem\_calloc}@(size_t count, size_t size);
\end{Csynopsis}

\begin{apiarguments}
  \apiargument{IN}{count}{The number of elements to allocate.}
  \apiargument{IN}{size}{The size in bytes of each element to allocate.}
\end{apiarguments}


\apidescription{
  The \FUNC{shmem\_calloc} routine is a collective operation
  on the world team that allocates a
  region of remotely-accessible
  memory for an array of \VAR{count} objects of \VAR{size} bytes each and
  returns a pointer to the lowest byte address of the allocated symmetric
  memory. The space is initialized to all bits zero.

  If the allocation succeeds, the pointer returned shall be suitably
  aligned so that it may be assigned to a pointer to any type of object.
  If the allocation does not succeed, or either \VAR{count} or \VAR{size} is
  \CONST{0}, the return value is a null pointer.

  The values for \VAR{count} and \VAR{size} shall each be equal across
  all \acp{PE} calling \FUNC{shmem\_calloc}; otherwise, the behavior is
  undefined.

  When \VAR{count} or \VAR{size} is \CONST{0}, the \FUNC{shmem\_calloc} routine
  returns without performing a barrier.  Otherwise, this
  routine calls a procedure that is semantically equivalent to
  \FUNC{shmem\_barrier\_all} on exit.
}

\apireturnvalues{
  The \FUNC{shmem\_calloc} routine returns a pointer to the lowest byte
  address of the allocated space; otherwise, it returns a null pointer.
}

\end{apidefinition}


\subsubsection{\textbf{SHPALLOC}}\label{subsec:shpalloc}
\input{content/shpalloc.tex}

\subsubsection{\textbf{SHPCLMOVE}}\label{subsec:shpclmove}
\input{content/shpclmove.tex}

\subsubsection{\textbf{SHPDEALLC}}\label{subsec:shpdeallc}
\input{content/shpdeallc.tex}


\subsection{Communication Management Routines}
\label{sec:ctx}
All \openshmem \ac{RMA}, \ac{AMO}, and memory ordering routines \oldtext{are} \newtext{must be}
performed on a \newtext{valid} communication context.  The communication context defines an
independent ordering and completion environment, allowing users to manage the
overlap of communication with computation and also to manage communication
operations performed by separate threads within a multithreaded \ac{PE}.  For
example, in single-threaded environments, contexts may be used to pipeline
communication and computation.  In multithreaded environments, contexts may
additionally provide thread isolation, eliminating overheads resulting from
thread interference.

\oldtext{A handle to the desired context}
\newtext{A specific communication context is referenced through a context handle, which} is
passed as an argument in the \Cstd \CTYPE{shmem\_ctx\_*} and type-generic \ac{API}
routines.  \ac{API} routines that do not accept a context \newtext{handle} argument operate on the
default context.  The default context can be used explicitly through the
\LibHandleRef{SHMEM\_CTX\_DEFAULT} handle.
Context handles are of type \CTYPE{shmem\_ctx\_t} and \newtext{may be used} \oldtext{are valid} for
language-level assignment and equality comparison.

\newtext{
The default context is valid for the duration of the \openshmem portion of
an application.
Contexts created by a successful call to \FUNC{shmem\_ctx\_create} remain
valid until they are destroyed.
A handle value that does not correspond to a valid context is considered
to be invalid, and its use in \ac{RMA} and \ac{AMO} routines results in
undefined behavior.
A context handle may be initialized to or assigned the value
\CONST{SHMEM\_CTX\_INVALID} to indicate that handle does not reference a
valid communication context.
When managed in this way, applications can use an equality comparison
to test whether a given context handle references a valid context.
}

\subsubsection{\textbf{SHMEM\_CTX\_CREATE}}
\label{subsec:shmem_ctx_create}
\apisummary{
    Create a communication context.
}

\begin{apidefinition}

\begin{Csynopsis}
int @\FuncDecl{shmem\_ctx\_create}@(long options, shmem_ctx_t *ctx);
\end{Csynopsis}

\begin{apiarguments}
    \apiargument{IN}{options}{The set of options requested for the given context.
        Multiple options may be requested by combining them with a bitwise
        OR operation; otherwise, \CONST{0} can be given if no options are
        requested.}
    \apiargument{OUT}{ctx}{A handle to the newly created context.}
\end{apiarguments}

\apidescription{
    The \FUNC{shmem\_ctx\_create} routine creates a new communication context
    and returns its handle through the \VAR{ctx} argument.  If the context was
    created successfully, a value of zero is returned
    and the context handle pointed to by \VAR{ctx} specifies a valid context;
    otherwise, a nonzero value is returned
    and the context handle pointed to by \VAR{ctx} is not modified.
    An unsuccessful context
    creation call is not treated as an error and the \openshmem library remains
    in a correct state.  The creation call can be reattempted with different
    options or after additional resources become available.

    A newly created communication context has a fixed association with the
    default team.
    All \openshmem routines that operate on this context will do so with
    respect to the associated \ac{PE} team.
    That is, all point-to-point routines operating on this context will use
    team-relative \ac{PE} numbering.

    By default, contexts are {\em shareable} and, when it is allowed by the
    threading model provided by the \openshmem library, they can be used concurrently by
    multiple threads within the PE where they were created.
    %
    The following options can be supplied during context creation to restrict
    this usage model and enable performance optimizations.  When using a given
    context, the application must comply with the requirements of all options
    set on that context; otherwise, the behavior is undefined.
    No options are enabled on the default context.

        \apitablerow{\LibConstRef{SHMEM\_CTX\_SERIALIZED}}{
            The given context is shareable; however, it will not be used by multiple threads
            concurrently.  When the \CONST{SHMEM\_CTX\_SERIALIZED} option is
            set, the user must ensure that operations involving the given
            context are serialized by the application.}

        \apitablerow{\LibConstRef{SHMEM\_CTX\_PRIVATE}}{
            The given context will be used only by the thread that created it.}

        \apitablerow{\LibConstRef{SHMEM\_CTX\_NOSTORE}}{
            Quiet and fence operations performed on the given context are not
            required to enforce completion and ordering of memory store
            operations.
            When ordering of store operations is needed, the application must
            perform a synchronization operation on a context without the
            \CONST{SHMEM\_CTX\_NOSTORE} option enabled.}

}

\apireturnvalues{
    Zero on success and nonzero otherwise.
}

\end{apidefinition}



\subsubsection{\textbf{SHMEM\_CTX\_DESTROY}}
\label{subsec:shmem_ctx_destroy}
\input{content/shmem_ctx_destroy.tex}


\subsection{Remote Memory Access Routines}\label{sec:rma}
The \ac{RMA} routines described in this section can be used to perform
reads from and writes to symmetric data objects. These operations
are one-sided, meaning that the \ac{PE} invoking an operation provides all
communication parameters and the targeted \ac{PE} is passive. A characteristic
of one-sided communication is that it decouples communication from
synchronization. One-sided communication mechanisms transfer data; however,
they do not synchronize the sender of the data with the receiver of the data.

\openshmem \ac{RMA} routines are performed on symmetric data objects.  The
initiator \ac{PE} of a call is designated as the \emph{origin} \ac{PE} and the
\ac{PE} targeted by an operation is designated as the \emph{destination} \ac{PE}.  The
\source{} and \dest{} designators refer to the data objects that an operation
reads from and writes to.  In the case of the remote update routine, \PUT{},
the origin \ac{PE} provides the \source{} data object and the destination
\ac{PE} provides the \dest{} data object. In the case of the remote read
routine, \GET{}, the origin \ac{PE} provides the \dest{} data object and the
destination \ac{PE} provides the \source{} data object.

The destination \ac{PE} is specified as an integer representing the \ac{PE} number.
This \ac{PE} number is relative to the team associated with the
communication context being used for the operation. If no context argument is passed to the routine,
then the routine operates on the default context, which implies that
the \ac{PE} number is relative to the world team.
If the \ac{PE} number passed to the routine is invalid, being negative
or greater than or equal to the size of the \openshmem team, then the behavior is undefined.

\openshmem \ac{RMA} routines specified in this section have two variants. In
one of the variants, the context handle, \VAR{ctx}, is explicitly passed as
an argument. In this variant, the operation is performed on the specified
context. If the context handle \VAR{ctx} does not correspond to a valid
context, the behavior is undefined. In the other variant, the context handle
is not explicitly passed and thus, the operations are performed on the
default context.

Where appropriate compiler support is available, \openshmem provides type-generic
one-sided communication interfaces via \Cstd[11] generic selection
(\Cstd[11]~\S6.5.1.1\footnote{Formally, the \Cstd[11] specification is ISO/IEC 9899:2011(E).})
for block, scalar, and block-strided put and get communication.
Such type-generic routines are supported for the ``standard \ac{RMA} types''
listed in Table \ref{stdrmatypes}.

The standard \ac{RMA} types include the exact-width integer types defined in
\HEADER{stdint.h} by \Cstd[99]%
\footnote{Formally, the \Cstd[99] specification is ISO/IEC~9899:1999(E).}%
~\S7.18.1.1 and \Cstd[11]~\S7.20.1.1. When the \Cstd translation environment
does not provide exact-width integer types with \HEADER{stdint.h}, an
\openshmem implementation is not required to provide support for these types.

\begin{table}[h]
  \begin{center}
    \begin{tabular}{|l|l|}
      \hline
      \TYPE              & \TYPENAME  \\ \hline
      float              & float      \\ \hline
      double             & double     \\ \hline
      long double        & longdouble \\ \hline
      char               & char       \\ \hline
      signed char        & schar      \\ \hline
      short              & short      \\ \hline
      int                & int        \\ \hline
      long               & long       \\ \hline
      long long          & longlong   \\ \hline
      unsigned char      & uchar      \\ \hline
      unsigned short     & ushort     \\ \hline
      unsigned int       & uint       \\ \hline
      unsigned long      & ulong      \\ \hline
      unsigned long long & ulonglong  \\ \hline
      int8\_t            & int8       \\ \hline
      int16\_t           & int16      \\ \hline
      int32\_t           & int32      \\ \hline
      int64\_t           & int64      \\ \hline
      uint8\_t           & uint8      \\ \hline
      uint16\_t          & uint16     \\ \hline
      uint32\_t          & uint32     \\ \hline
      uint64\_t          & uint64     \\ \hline
      size\_t            & size       \\ \hline
      ptrdiff\_t         & ptrdiff    \\ \hline
    \end{tabular}
    \TableCaptionRef{Standard \ac{RMA} Types and Names}
    \label{stdrmatypes}
  \end{center}
\end{table}


\subsubsection{\textbf{SHMEM\_PUT}}\label{subsec:shmem_put}
\apisummary{
    The  put routines  provide  a method for copying data from a contiguous local
    data object to a data object on a specified target \ac{PE}.
}

\begin{apidefinition}

\begin{C11synopsis}
void @\FuncDecl{shmem\_put}@(TYPE *dest, const TYPE *source, size_t nelems, int pe);
void @\FuncDecl{shmem\_put}@(shmem_ctx_t ctx, TYPE *dest, const TYPE *source, size_t nelems, int pe);
\end{C11synopsis}
where \TYPE{} is one of the standard \ac{RMA} types specified by Table \ref{stdrmatypes}.

\begin{Csynopsis}
void @\FuncDecl{shmem\_\FuncParam{TYPENAME}\_put}@(TYPE *dest, const TYPE *source, size_t nelems, int pe);
void @\FuncDecl{shmem\_ctx\_\FuncParam{TYPENAME}\_put}@(shmem_ctx_t ctx, TYPE *dest, const TYPE *source, size_t nelems, int pe);
\end{Csynopsis}
where \TYPE{} is one of the standard \ac{RMA} types and has a corresponding \TYPENAME{} specified by Table \ref{stdrmatypes}.

\begin{CsynopsisCol}
void @\FuncDecl{shmem\_put\FuncParam{SIZE}}@(void *dest, const void *source, size_t nelems, int pe);
void @\FuncDecl{shmem\_ctx\_put\FuncParam{SIZE}}@(shmem_ctx_t ctx, void *dest, const void *source, size_t nelems, int pe);
\end{CsynopsisCol}
where \SIZE{} is one of \CONST{8, 16, 32, 64, 128}.

\begin{CsynopsisCol}
void @\FuncDecl{shmem\_putmem}@(void *dest, const void *source, size_t nelems, int pe);
void @\FuncDecl{shmem\_ctx\_putmem}@(shmem_ctx_t ctx, void *dest, const void *source, size_t nelems, int pe);
\end{CsynopsisCol}

\begin{apiarguments}
    \apiargument{IN}{ctx}{A context handle specifying the context on which to perform the operation.
      When this argument is not provided, the operation is performed on
      the default context.}
    \apiargument{OUT}{dest}{Symmetric address of the destination data object.
      The type of \dest{} should match that implied in the SYNOPSIS section.}
    \apiargument{IN}{source}{Local address of the data object containing the data to be copied.
      The type of \source{} should match that implied in the SYNOPSIS section.}
    \apiargument{IN}{nelems}{Number of elements in the \VAR{dest} and \VAR{source} arrays.
      For \FUNC{shmem\_putmem} and \FUNC{shmem\_ctx\_putmem}, elements are bytes.}
    \apiargument{IN}{pe}{\ac{PE} number of the target \ac{PE} relative to the team associated
      with the given \VAR{ctx} when provided, or the default context otherwise.}
\end{apiarguments}

\apidescription{
    The routines return after the data has been copied out of the \source{} array
    on the local \ac{PE}.  The delivery of data words into the data object on the
    destination \ac{PE} may occur in any order.  Furthermore, two successive put
    routines may deliver data out of order unless a call to \FUNC{shmem\_fence} is
    introduced between the two calls.
 }

\apireturnvalues{
    None.
}

\begin{apiexamples}

\apicexample
    { The following \FUNC{shmem\_put} example is for \Cstd[11] programs:}
    {./example_code/shmem_put_example.c}
    {}
\end{apiexamples}

\end{apidefinition}


\subsubsection{\textbf{SHMEM\_P}}\label{subsec:shmem_p}
\apisummary{
    Copies one data item to a remote \ac{PE}.
}

\begin{apidefinition}

\begin{C11synopsis}
void @\FuncDecl{shmem\_p}@(TYPE *dest, TYPE value, int pe);
void @\FuncDecl{shmem\_p}@(shmem_ctx_t ctx, TYPE *dest, TYPE value, int pe);
\end{C11synopsis}
where \TYPE{} is one of the standard \ac{RMA} types specified by Table \ref{stdrmatypes}.

\begin{Csynopsis}
void @\FuncDecl{shmem\_\FuncParam{TYPENAME}\_p}@(TYPE *dest, TYPE value, int pe);
void @\FuncDecl{shmem\_ctx\_\FuncParam{TYPENAME}\_p}@(shmem_ctx_t ctx, TYPE *dest, TYPE value, int pe);
\end{Csynopsis}
where \TYPE{} is one of the standard \ac{RMA} types and has a corresponding \TYPENAME{} specified by Table \ref{stdrmatypes}.

\begin{apiarguments}
  \apiargument{IN}{ctx}{A context handle specifying the context on which to perform the operation.
    When this argument is not provided, the operation is performed on
    the default context.}
  \apiargument{OUT}{dest}{Symmetric address of the destination data object.
    The type of \dest{} should match that implied in the SYNOPSIS section.}
  \apiargument{IN}{value}{The value to be transferred to \VAR{dest}.
    The type of \VAR{value} should match that implied in the SYNOPSIS section.}
  \apiargument{IN}{pe}{\ac{PE} number of the remote \ac{PE} relative to the team associated
    with the given \VAR{ctx} when provided, or the default context otherwise.}
\end{apiarguments}

\apidescription{
    These routines provide a very low latency put capability for single elements of
    most basic types.

    As with \FUNC{shmem\_put}, these routines start the remote transfer and may
    return before the data is delivered to the remote \ac{PE}.  Use
    \FUNC{shmem\_quiet} to force completion of all remote \PUT{} transfers.
}

\apireturnvalues{
    None.
}

\begin{apiexamples}

    \apicexample
    {The following example uses \FUNC{shmem\_p} in a \Cstd[11] program.}
    {./example_code/shmem_p_example.c}
    {}

\end{apiexamples}

\end{apidefinition}


\subsubsection{\textbf{SHMEM\_IPUT}}\label{subsec:shmem_iput}
\apisummary{
    Copies strided data to a specified target \ac{PE}.
}

\begin{apidefinition}

\begin{C11synopsis}
void @\FuncDecl{shmem\_iput}@(TYPE *dest, const TYPE *source, ptrdiff_t dst, ptrdiff_t sst, size_t nelems, int pe);
void @\FuncDecl{shmem\_iput}@(shmem_ctx_t ctx, TYPE *dest, const TYPE *source, ptrdiff_t dst, ptrdiff_t sst, size_t nelems, int pe);
\end{C11synopsis}
where \TYPE{} is one of the standard \ac{RMA} types specified by Table \ref{stdrmatypes}.

\begin{Csynopsis}
void @\FuncDecl{shmem\_\FuncParam{TYPENAME}\_iput}@(TYPE *dest, const TYPE *source, ptrdiff_t dst, ptrdiff_t sst, size_t nelems, int pe);
void @\FuncDecl{shmem\_ctx\_\FuncParam{TYPENAME}\_iput}@(shmem_ctx_t ctx, TYPE *dest, const TYPE *source, ptrdiff_t dst, ptrdiff_t sst, size_t nelems, int pe);
\end{Csynopsis}
where \TYPE{} is one of the standard \ac{RMA} types and has a corresponding \TYPENAME{} specified by Table \ref{stdrmatypes}.

\begin{CsynopsisCol}
void @\FuncDecl{shmem\_iput\FuncParam{SIZE}}@(void *dest, const void *source, ptrdiff_t dst, ptrdiff_t sst, size_t nelems, int pe);
void @\FuncDecl{shmem\_ctx\_iput\FuncParam{SIZE}}@(shmem_ctx_t ctx, void *dest, const void *source, ptrdiff_t dst, ptrdiff_t sst, size_t nelems, int pe);
\end{CsynopsisCol}
where \SIZE{} is one of \CONST{8, 16, 32, 64, 128}.

\begin{apiarguments}
    \apiargument{IN}{ctx}{A context handle specifying the context on which to perform the operation.
        When this argument is not provided, the operation is performed on
        the default context.}
    \apiargument{OUT}{dest}{Symmetric address of the destination array data object.
        The type of \dest{} should match that implied in the SYNOPSIS section.}
    \apiargument{IN}{source}{Local address of the array containing the data to be copied.
        The type of \source{} should match that implied in the SYNOPSIS section.}
    \apiargument{IN}{dst}{The stride between consecutive elements of the \dest{}
        array. The stride must be greater than or equal to \CONST{1} and is
        scaled by the element size of the \dest{} array. A value of \CONST{1}
        indicates contiguous data.}
    \apiargument{IN}{sst}{The stride between consecutive elements of the \source{}
        array. The stride must be greater than or equal to \CONST{1} and is
        scaled by the element size of the \source{} array. A value of \CONST{1}
        indicates contiguous data.}
    \apiargument{IN}{nelems}{Number of elements in the \dest{} and \source{} arrays.}
    \apiargument{IN}{pe}{\ac{PE} number of the target \ac{PE} relative to the team associated
      with the given \VAR{ctx} when provided, or the default context otherwise.}
\end{apiarguments}


\apidescription{
    The \FUNC{shmem\_iput} routines provide a method  for  copying strided data
    elements (specified by \VAR{sst}) of an array from a \source{} array on the
    local \ac{PE} to locations specified by stride \VAR{dst} on a \dest{} array
    on specified target \ac{PE}. The routines return when the data has
    been copied out of the \VAR{source} array on the local \ac{PE} but not
    necessarily before the data has been delivered to the remote data object.
}

\apireturnvalues{
    None.
}

\begin{apiexamples}

\apicexample
    {Consider the following \FUNC{shmem\_iput} example for \Cstd[11] programs.}
    {./example_code/shmem_iput_example.c}
    {}
\end{apiexamples}

\end{apidefinition}


\subsubsection{\textbf{SHMEM\_GET}}\label{subsec:shmem_get}
\apisummary{
    Copies data from a specified \ac{PE}.
}

\begin{apidefinition}

\begin{C11synopsis}
void @\FuncDecl{shmem\_get}@(TYPE *dest, const TYPE *source, size_t nelems, int pe);
void @\FuncDecl{shmem\_get}@(shmem_ctx_t ctx, TYPE *dest, const TYPE *source, size_t nelems, int pe);
\end{C11synopsis}
where \TYPE{} is one of the standard \ac{RMA} types specified by Table \ref{stdrmatypes}.

\begin{Csynopsis}
void @\FuncDecl{shmem\_\FuncParam{TYPENAME}\_get}@(TYPE *dest, const TYPE *source, size_t nelems, int pe);
void @\FuncDecl{shmem\_ctx\_\FuncParam{TYPENAME}\_get}@(shmem_ctx_t ctx, TYPE *dest, const TYPE *source, size_t nelems, int pe);
\end{Csynopsis}
where \TYPE{} is one of the standard \ac{RMA} types and has a corresponding \TYPENAME{} specified by Table \ref{stdrmatypes}.

\begin{CsynopsisCol}
void @\FuncDecl{shmem\_get\FuncParam{SIZE}}@(void *dest, const void *source, size_t  nelems, int pe);
void @\FuncDecl{shmem\_ctx\_get\FuncParam{SIZE}}@(shmem_ctx_t ctx, void *dest, const void *source, size_t nelems, int pe);
\end{CsynopsisCol}
where \SIZE{} is one of \CONST{8, 16, 32, 64, 128}.

\begin{CsynopsisCol}
void @\FuncDecl{shmem\_getmem}@(void *dest, const void *source, size_t nelems, int pe);
void @\FuncDecl{shmem\_ctx\_getmem}@(shmem_ctx_t ctx, void *dest, const void *source, size_t nelems, int pe);
\end{CsynopsisCol}

\begin{apiarguments}
    \apiargument{IN}{ctx}{A context handle specifying the context on which to perform the operation.
        When this argument is not provided, the operation is performed on
        the default context.}
    \apiargument{OUT}{dest}{Local address of the data object to be updated.
      The type of \dest{} should match that implied in the SYNOPSIS section.}
    \apiargument{IN}{source}{Symmetric address of the source data object.
      The type of \source{} should match that implied in the SYNOPSIS section.}
    \apiargument{IN}{nelems}{Number of elements in the \dest{} and \source{} arrays.
      For \FUNC{shmem\_getmem} and \FUNC{shmem\_ctx\_getmem}, elements are bytes.}
    \apiargument{IN}{pe}{\ac{PE} number of the target \ac{PE} relative to the team associated
      with the given \VAR{ctx} when provided, or the default context otherwise.}
\end{apiarguments}

\apidescription{
   The get routines provide a method for copying a contiguous symmetric data
   object from a target \ac{PE} to a contiguous data object on the local
   \ac{PE}.  The routines return after the data has been delivered to the
   \dest{} array on the local \ac{PE}.
}

\apireturnvalues{
    None.
}

\apinotes{
    See Section \ref{subsec:memory_model} for a definition of the term
    remotely accessible.
}

\end{apidefinition}


\subsubsection{\textbf{SHMEM\_G}}\label{subsec:shmem_g}
\apisummary{
    Copies one data item from a remote \ac{PE}
}

\begin{apidefinition}

\begin{C11synopsis}
TYPE @\FuncDecl{shmem\_g}@(const TYPE *source, int pe);
TYPE @\FuncDecl{shmem\_g}@(shmem_ctx_t ctx, const TYPE *source, int pe);
\end{C11synopsis}
where \TYPE{} is one of the standard \ac{RMA} types specified by Table \ref{stdrmatypes}.

\begin{Csynopsis}
TYPE @\FuncDecl{shmem\_\FuncParam{TYPENAME}\_g}@(const TYPE *source, int pe);
TYPE @\FuncDecl{shmem\_ctx\_\FuncParam{TYPENAME}\_g}@(shmem_ctx_t ctx, const TYPE *source, int pe);
\end{Csynopsis}
where \TYPE{} is one of the standard \ac{RMA} types and has a corresponding \TYPENAME{} specified by Table \ref{stdrmatypes}.

\begin{apiarguments}
  \apiargument{IN}{ctx}{A context handle specifying the context on which to perform the operation.
    When this argument is not provided, the operation is performed on
    the default context.}
  \apiargument{IN}{source}{Symmetric address of the source data object.
    The type of \source{} should match that implied in the SYNOPSIS section.}
  \apiargument{IN}{pe}{The number of the remote \ac{PE} on which \VAR{source} resides.}
  \apiargument{IN}{pe}{\ac{PE} number of the remote \ac{PE} on which \VAR{source} resides 
    relative to the team associated with the given \VAR{ctx} when provided, or the 
    default context otherwise.}
\end{apiarguments}

\apidescription{
  These routines provide a very low latency get capability for single elements
  of most basic types.
}

\apireturnvalues{
    Returns a single element of type specified in the synopsis.
}

\begin{apiexamples}

\apicexample
    {The following \FUNC{shmem\_g} example is for \Cstd[11] programs:}
    {./example_code/shmem_g_example.c}
    {}
\end{apiexamples}

\end{apidefinition}


\subsubsection{\textbf{SHMEM\_IGET}}\label{subsec:shmem_iget}
\apisummary{
    Copies strided data from a specified \ac{PE}.
}

\begin{apidefinition}

\begin{C11synopsis}
void @\FuncDecl{shmem\_iget}@(TYPE *dest, const TYPE *source, ptrdiff_t dst, ptrdiff_t sst, size_t nelems, int pe);
void @\FuncDecl{shmem\_iget}@(shmem_ctx_t ctx, TYPE *dest, const TYPE *source, ptrdiff_t dst, ptrdiff_t sst, size_t nelems, int pe);
\end{C11synopsis}
where \TYPE{} is one of the standard \ac{RMA} types specified by Table \ref{stdrmatypes}.

\begin{Csynopsis}
void @\FuncDecl{shmem\_\FuncParam{TYPENAME}\_iget}@(TYPE *dest, const TYPE *source, ptrdiff_t dst, ptrdiff_t sst, size_t nelems, int pe);
void @\FuncDecl{shmem\_ctx\_\FuncParam{TYPENAME}\_iget}@(shmem_ctx_t ctx, TYPE *dest, const TYPE *source, ptrdiff_t dst, ptrdiff_t sst, size_t nelems, int pe);
\end{Csynopsis}
where \TYPE{} is one of the standard \ac{RMA} types and has a corresponding \TYPENAME{} specified by Table \ref{stdrmatypes}.

\begin{CsynopsisCol}
void @\FuncDecl{shmem\_iget\FuncParam{SIZE}}@(void *dest, const void *source, ptrdiff_t dst, ptrdiff_t sst, size_t  nelems, int pe);
void @\FuncDecl{shmem\_ctx\_iget\FuncParam{SIZE}}@(shmem_ctx_t ctx, void *dest, const void *source, ptrdiff_t dst, ptrdiff_t sst, size_t nelems, int pe);
\end{CsynopsisCol}
where \SIZE{} is one of \CONST{8, 16, 32, 64, 128}.

\begin{apiarguments}
    \apiargument{IN}{ctx}{A context handle specifying the context on which to perform the operation.
        When this argument is not provided, the operation is performed on
        the default context.}
    \apiargument{OUT}{dest}{Local address of the array to be updated.
        The type of \dest{} should match that implied in the SYNOPSIS section.}
    \apiargument{IN}{source}{Symmetric address of the source array data object.
        The type of \source{} should match that implied in the SYNOPSIS section.}
    \apiargument{IN}{dst}{The stride between consecutive elements of the \dest{}
        array.  The stride is scaled by the element size of the \dest{} array.
        A  value of \CONST{1} indicates contiguous data.}
    \apiargument{IN}{sst}{The stride between consecutive elements of the
        \source{} array.  The stride is scaled by the element size of the \source{}
        array.  A  value of \CONST{1} indicates contiguous data.}
    \apiargument{IN}{nelems}{Number of elements in the \dest{} and \source{}
        arrays.}
    \apiargument{IN}{pe}{\ac{PE} number of the remote \ac{PE}.}
\end{apiarguments}

\apidescription{
    The \FUNC{iget} routines provide a method for copying strided data elements from
    a symmetric array from a specified remote \ac{PE} to strided locations on a
    local array.  The routines return when the data has been copied into the local
    \VAR{dest} array.
    If the context handle \VAR{ctx} does not correspond to a valid context,
    the behavior is undefined.
}

\apireturnvalues{
    None.
}

\end{apidefinition}



\subsection{Non-blocking Remote Memory Access Routines}\label{sec:rma_nbi}

\subsubsection{\textbf{SHMEM\_PUT\_NBI}}\label{subsec:shmem_put_nbi}
\apisummary{
    The nonblocking put routines provide a method for copying data
    from a contiguous local data object to a data object on a specified \ac{PE}.
}

\begin{apidefinition}

\begin{C11synopsis}
void @\FuncDecl{shmem\_put\_nbi}@(TYPE *dest, const TYPE *source, size_t nelems, int pe);
void @\FuncDecl{shmem\_put\_nbi}@(shmem_ctx_t ctx, TYPE *dest, const TYPE *source, size_t nelems, int pe);
\end{C11synopsis}
where \TYPE{} is one of the standard \ac{RMA} types specified by Table \ref{stdrmatypes}.

\begin{Csynopsis}
void @\FuncDecl{shmem\_\FuncParam{TYPENAME}\_put\_nbi}@(TYPE *dest, const TYPE *source, size_t nelems, int pe);
void @\FuncDecl{shmem\_ctx\_\FuncParam{TYPENAME}\_put\_nbi}@(shmem_ctx_t ctx, TYPE *dest, const TYPE *source, size_t nelems, int pe);
\end{Csynopsis}
where \TYPE{} is one of the standard \ac{RMA} types and has a corresponding \TYPENAME{} specified by Table \ref{stdrmatypes}.

\begin{CsynopsisCol}
void @\FuncDecl{shmem\_put\FuncParam{SIZE}\_nbi}@(void *dest, const void *source, size_t nelems, int pe);
void @\FuncDecl{shmem\_ctx\_put\FuncParam{SIZE}\_nbi}@(shmem_ctx_t ctx, void *dest, const void *source, size_t nelems, int pe);
\end{CsynopsisCol}
where \SIZE{} is one of \CONST{8, 16, 32, 64, 128}.

\begin{CsynopsisCol}
void @\FuncDecl{shmem\_putmem\_nbi}@(void *dest, const void *source, size_t nelems, int pe);
void @\FuncDecl{shmem\_ctx\_putmem\_nbi}@(shmem_ctx_t ctx, void *dest, const void *source, size_t nelems, int pe);
\end{CsynopsisCol}

\begin{apiarguments}
  \apiargument{IN}{ctx}{A context handle specifying the context on which to perform the operation.
    When this argument is not provided, the operation is performed on
    the default context.}
  \apiargument{OUT}{dest}{Symmetric address of the destination data object.
    The type of \dest{} should match that implied in the SYNOPSIS section.}
  \apiargument{IN}{source}{Local address of the object containing the data to be copied.
    The type of \source{} should match that implied in the SYNOPSIS section.}
  \apiargument{IN}{nelems}{Number of elements in the \VAR{dest} and \VAR{source}
    arrays.}
    \apiargument{IN}{pe}{\ac{PE} number of the remote \ac{PE}.}
\end{apiarguments}

\apidescription{
    The routines return after initiating the operation.  The operation is considered
    complete after a subsequent call to \FUNC{shmem\_quiet}.
    At the completion of \FUNC{shmem\_quiet}, the data has been copied into the \dest{} array
    on the destination \ac{PE}.
    The delivery of data words into the data object on the
    destination \ac{PE} may occur in any order.
    Furthermore, two successive put
    routines may deliver data out of order unless a call to \FUNC{shmem\_fence} is
    introduced between the two calls.
    If the context handle \VAR{ctx} does not correspond to a valid context,
    the behavior is undefined.
 }

\apireturnvalues{
    None.
}

\end{apidefinition}


\subsubsection{\textbf{SHMEM\_GET\_NBI}}\label{subsec:shmem_get_nbi}
\apisummary{
    The nonblocking get routines provide a method for copying data from a
    contiguous remote data object on the specified \ac{PE} to the local data object.
}

\begin{apidefinition}

\begin{C11synopsis}
void @\FuncDecl{shmem\_get\_nbi}@(TYPE *dest, const TYPE *source, size_t nelems, int pe);
void @\FuncDecl{shmem\_get\_nbi}@(shmem_ctx_t ctx, TYPE *dest, const TYPE *source, size_t nelems, int pe);
\end{C11synopsis}
where \TYPE{} is one of the standard \ac{RMA} types specified by Table \ref{stdrmatypes}.

\begin{Csynopsis}
void @\FuncDecl{shmem\_\FuncParam{TYPENAME}\_get\_nbi}@(TYPE *dest, const TYPE *source, size_t nelems, int pe);
void @\FuncDecl{shmem\_ctx\_\FuncParam{TYPENAME}\_get\_nbi}@(shmem_ctx_t ctx, TYPE *dest, const TYPE *source, size_t nelems, int pe);
\end{Csynopsis}
where \TYPE{} is one of the standard \ac{RMA} types and has a corresponding \TYPENAME{} specified by Table \ref{stdrmatypes}.

\begin{CsynopsisCol}
void @\FuncDecl{shmem\_get\FuncParam{SIZE}\_nbi}@(void *dest, const void *source, size_t  nelems, int pe);
void @\FuncDecl{shmem\_ctx\_get\FuncParam{SIZE}\_nbi}@(shmem_ctx_t ctx, void *dest, const void *source, size_t  nelems, int pe);
\end{CsynopsisCol}
where \SIZE{} is one of \CONST{8, 16, 32, 64, 128}.

\begin{CsynopsisCol}
void @\FuncDecl{shmem\_getmem\_nbi}@(void *dest, const void *source, size_t nelems, int pe);
void @\FuncDecl{shmem\_ctx\_getmem\_nbi}@(shmem_ctx_t ctx, void *dest, const void *source, size_t nelems, int pe);
\end{CsynopsisCol}

\begin{apiarguments}
    \apiargument{IN}{ctx}{A context handle specifying the context on which to perform the operation.
        When this argument is not provided, the operation is performed on
        the default context.}
    \apiargument{OUT}{dest}{Local address of the data object to be updated.
        The type of \dest{} should match that implied in the SYNOPSIS section.}
    \apiargument{IN}{source}{Symmetric address of the source data object.
        The type of \source{} should match that implied in the SYNOPSIS section.}
    \apiargument{IN}{nelems}{Number of elements in the \dest{} and \source{}
        arrays. For \FUNC{shmem\_getmem\_nbi} and
        \FUNC{shmem\_ctx\_getmem\_nbi}, elements are bytes.}
    \apiargument{IN}{pe}{\ac{PE} number of the remote \ac{PE} relative to the team associated
      with the given \VAR{ctx} when provided, or the default context otherwise.}
\end{apiarguments}

\apidescription{
    The get routines provide a method for copying a contiguous symmetric data
   object from a different \ac{PE} to a contiguous data object on the local
   \ac{PE}.   The routines return after initiating the operation.  The operation is considered
    complete after a subsequent call to \FUNC{shmem\_quiet}.
    At the completion of \FUNC{shmem\_quiet}, the
    data has been delivered to the \dest{} array on the local \ac{PE}.
}

\apireturnvalues{
    None.
}

\apinotes{
    See Section \ref{subsec:memory_model} for a definition of the term
    remotely accessible.
}

\end{apidefinition}



\subsection{Atomic Memory Operations}\label{sec:amo}
An \ac{AMO} is a one-sided communication mechanism that combines memory read,
update, or write operations with atomicity guarantees described in Section~%
\ref{subsec:amo_guarantees}.  Similar to the \ac{RMA} routines, described in
Section~\ref{sec:rma}, the \acp{AMO} are performed only on symmetric objects.
\openshmem defines two types of \ac{AMO} routines:

\begin{itemize}

\item
  The \emph{fetching} routines return the original value of, and optionally
  update, the remote data object in a single atomic operation.  The routines
  return after the data has been fetched from the target \ac{PE} and delivered
  to the calling \ac{PE}.
  The data type of the returned value is the same as the type of
  the remote data object.

  The fetching routines include:
  \FUNC{shmem\_atomic\_\{fetch, compare\_swap, swap\}[\_nbi]} and
  \FUNC{shmem\_atomic\_fetch\_\{inc, add, and, or, xor\}[\_nbi]}.

\item
  The \emph{non-fetching} routines update the remote data object in a single
  atomic operation.  A call to a non-fetching atomic routine issues the atomic
  operation and may return before the operation executes on the target \ac{PE}.
  The \FUNC{shmem\_quiet}, \FUNC{shmem\_barrier}, or \FUNC{shmem\_barrier\_all}
  routines can be used to force completion for these non-fetching
  atomic routines.

  The non-fetching routines include:
  \FUNC{shmem\_atomic\_\{set, inc, add, and, or, xor\}[\_nbi]}.

\begin{DeprecateBlock}

Starting in \openshmem[1.4], all \ac{AMO} functions added "\_atomic\_" to the function 
name and deprecated the equivalent functions without "\_atomic\_" in the name.

\end{DeprecateBlock}

\end{itemize}

\openshmem \ac{AMO} routines specified in this section have two variants. In
one of the variants, the context handle, \VAR{ctx}, is explicitly passed as
an argument. In this variant, the operation is performed on the specified
context. If the context handle \VAR{ctx} does not correspond to a valid
context, the behavior is undefined. In the other variant, the context handle
is not explicitly passed and thus, the operations are performed on the
default context.

Where appropriate compiler support is available, \openshmem provides
type-generic \ac{AMO} interfaces via \Cstd[11] generic selection.
The type-generic support for the \ac{AMO} routines is as follows:

\begin{itemize}
\item \FUNC{shmem\_atomic\_\{compare\_swap, fetch\_inc, inc, fetch\_add, add\}[\_nbi]}
  support the ``standard \ac{AMO} types'' listed in Table~\ref{stdamotypes},
\item \FUNC{shmem\_atomic\_\{fetch, set, swap\}} support
  the ``extended \ac{AMO} types'' listed in Table~\ref{extamotypes}, and
\item \FUNC{shmem\_atomic\_\{fetch\_and, and, fetch\_or, or, fetch\_xor, xor\}[\_nbi]}
  support the ``bitwise \ac{AMO} types'' listed in Table~\ref{bitamotypes}.
\end{itemize}

The standard, extended, and bitwise \ac{AMO} types include some of the exact-width
integer types defined in \HEADER{stdint.h} by \Cstd[99]~\S7.18.1.1 and
\Cstd[11]~\S7.20.1.1. When the \Cstd translation environment
does not provide exact-width integer types with \HEADER{stdint.h}, an
\openshmem implementation is not required to provide support for these types.

\begin{table}[h]
  \begin{center}
    \begin{tabular}{|l|l|}
      \hline
      \TYPE              & \TYPENAME  \\ \hline
      int                & int        \\ \hline
      long               & long       \\ \hline
      long long          & longlong   \\ \hline
      unsigned int       & uint       \\ \hline
      unsigned long      & ulong      \\ \hline
      unsigned long long & ulonglong  \\ \hline
      int32\_t           & int32      \\ \hline
      int64\_t           & int64      \\ \hline
      uint32\_t          & uint32     \\ \hline
      uint64\_t          & uint64     \\ \hline
      size\_t            & size       \\ \hline
      ptrdiff\_t         & ptrdiff    \\ \hline
    \end{tabular}
    \TableCaptionRef{Standard \ac{AMO} Types and Names}
    \label{stdamotypes}
  \end{center}
\end{table}

\begin{table}[h]
  \begin{center}
    \begin{tabular}{|l|l|}
      \hline
      \TYPE              & \TYPENAME  \\ \hline
      float              & float      \\ \hline
      double             & double     \\ \hline
      int                & int        \\ \hline
      long               & long       \\ \hline
      long long          & longlong   \\ \hline
      unsigned int       & uint       \\ \hline
      unsigned long      & ulong      \\ \hline
      unsigned long long & ulonglong  \\ \hline
      int32\_t           & int32      \\ \hline
      int64\_t           & int64      \\ \hline
      uint32\_t          & uint32     \\ \hline
      uint64\_t          & uint64     \\ \hline
      size\_t            & size       \\ \hline
      ptrdiff\_t         & ptrdiff    \\ \hline
    \end{tabular}
    \TableCaptionRef{Extended \ac{AMO} Types and Names}
    \label{extamotypes}
  \end{center}
\end{table}

\begin{table}[h]
  \begin{center}
    \begin{tabular}{|l|l|}
      \hline
      \TYPE              & \TYPENAME  \\ \hline
      unsigned int       & uint       \\ \hline
      unsigned long      & ulong      \\ \hline
      unsigned long long & ulonglong  \\ \hline
      int32\_t           & int32      \\ \hline
      int64\_t           & int64      \\ \hline
      uint32\_t          & uint32     \\ \hline
      uint64\_t          & uint64     \\ \hline
    \end{tabular}
    \TableCaptionRef{Bitwise \ac{AMO} Types and Names}
    \label{bitamotypes}
  \end{center}
\end{table}


\subsubsection{\textbf{SHMEM\_ATOMIC\_FETCH}}
\label{subsec:shmem_atomic_fetch}
\apisummary{
    Atomically fetches the value of a remote data object.
}

\begin{apidefinition}

\begin{C11synopsis}
TYPE @\FuncDecl{shmem\_atomic\_fetch}@(const TYPE *source, int pe);
TYPE @\FuncDecl{shmem\_atomic\_fetch}@(shmem_ctx_t ctx, const TYPE *source, int pe);
\end{C11synopsis}
where \TYPE{} is one of the extended \ac{AMO} types specified by
Table~\ref{extamotypes}.

\begin{Csynopsis}
TYPE @\FuncDecl{shmem\_\FuncParam{TYPENAME}\_atomic\_fetch}@(const TYPE *source, int pe);
TYPE @\FuncDecl{shmem\_ctx\_\FuncParam{TYPENAME}\_atomic\_fetch}@(shmem_ctx_t ctx, const TYPE *source, int pe);
\end{Csynopsis}
where \TYPE{} is one of the extended \ac{AMO} types and has a corresponding
\TYPENAME{} specified by Table~\ref{extamotypes}.

\begin{DeprecateBlock}
\begin{C11synopsis}
TYPE @\FuncDecl{shmem\_fetch}@(const TYPE *source, int pe);
\end{C11synopsis}
where \TYPE{} is one of \{\CTYPE{float}, \CTYPE{double}, \CTYPE{int},
\CTYPE{long}, \CTYPE{long long}\}.

\begin{Csynopsis}
TYPE @\FuncDecl{shmem\_\FuncParam{TYPENAME}\_fetch}@(const TYPE *source, int pe);
\end{Csynopsis}
where \TYPE{} is one of \{\CTYPE{float}, \CTYPE{double}, \CTYPE{int},
\CTYPE{long}, \CTYPE{long long}\} and has a corresponding
\TYPENAME{} specified by Table~\ref{extamotypes}.
\end{DeprecateBlock}

\begin{apiarguments}

  \apiargument{IN}{ctx}{A context handle specifying the context on which to perform the operation.
    When this argument is not provided, the operation is performed on
    the default context.}
  \apiargument{IN}{source}{Symmetric address of the source data object.
    The type of \source{} should match that implied in the SYNOPSIS section.}
  \apiargument{IN}{pe}{An integer that indicates the \ac{PE} number from which
    \VAR{source} is to be fetched.}

\end{apiarguments}

\apidescription{
    \FUNC{shmem\_atomic\_fetch} performs an atomic fetch operation.
    It returns the contents of the \VAR{source} as an atomic operation.
    If the context handle \VAR{ctx} does not correspond to a valid context,
    the behavior is undefined.
}

\apireturnvalues{
    The contents at the \VAR{source} address on the remote \ac{PE}.
    The data type of the return value is the same as the type of
    the remote data object.
}

\end{apidefinition}


\subsubsection{\textbf{SHMEM\_ATOMIC\_SET}}
\label{subsec:shmem_atomic_set}
\apisummary{
    Atomically sets the value of a remote data object.
}

\begin{apidefinition}

\begin{C11synopsis}
void @\FuncDecl{shmem\_atomic\_set}@(TYPE *dest, TYPE value, int pe);
void @\FuncDecl{shmem\_atomic\_set}@(shmem_ctx_t ctx, TYPE *dest, TYPE value, int pe);
\end{C11synopsis}
where \TYPE{} is one of the extended \ac{AMO} types specified by
Table~\ref{extamotypes}.

\begin{Csynopsis}
void @\FuncDecl{shmem\_\FuncParam{TYPENAME}\_atomic\_set}@(TYPE *dest, TYPE value, int pe);
void @\FuncDecl{shmem\_ctx\_\FuncParam{TYPENAME}\_atomic\_set}@(shmem_ctx_t ctx, TYPE *dest, TYPE value, int pe);
\end{Csynopsis}
where \TYPE{} is one of the extended \ac{AMO} types and has a corresponding
\TYPENAME{} specified by Table~\ref{extamotypes}.

\begin{DeprecateBlock}
\begin{C11synopsis}
void @\FuncDecl{shmem\_set}@(TYPE *dest, TYPE value, int pe);
\end{C11synopsis}
where \TYPE{} is one of \{\CTYPE{float}, \CTYPE{double}, \CTYPE{int},
\CTYPE{long}, \CTYPE{long long}\}.

\begin{Csynopsis}
void @\FuncDecl{shmem\_\FuncParam{TYPENAME}\_set}@(TYPE *dest, TYPE value, int pe);
\end{Csynopsis}
where \TYPE{} is one of \{\CTYPE{float}, \CTYPE{double}, \CTYPE{int},
\CTYPE{long}, \CTYPE{long long}\} and has a corresponding
\TYPENAME{} specified by Table~\ref{extamotypes}.
\end{DeprecateBlock}

\begin{apiarguments}

\apiargument{IN}{ctx}{A context handle specifying the context on which to perform the operation.
    When this argument is not provided, the operation is performed on
    the default context.}
\apiargument{OUT}{dest}{Symmetric address of the destination data object.
    The type of \dest{} should match that implied in the SYNOPSIS section.}
\apiargument{IN}{value}{The operand to the atomic set operation.
    The type of \VAR{value} should match that implied in the SYNOPSIS section.}
\apiargument{IN}{pe}{\ac{PE} number of the target \ac{PE} relative to the team associated
    with the given \VAR{ctx} when provided, or the default context otherwise.}
\end{apiarguments}

\apidescription{
    \FUNC{shmem\_atomic\_set} performs an atomic set operation. It writes the
    \VAR{value} into \VAR{dest} on \VAR{pe} as an atomic operation.
}

\apireturnvalues{
    None.
}

\end{apidefinition}


\subsubsection{\textbf{SHMEM\_ATOMIC\_COMPARE\_SWAP}}
\label{subsec:shmem_atomic_compare_swap}
\apisummary{
    Performs an atomic conditional swap on a remote data object.
}

\begin{apidefinition}

\begin{C11synopsis}
TYPE @\FuncDecl{shmem\_atomic\_compare\_swap}@(TYPE *dest, TYPE cond, TYPE value, int pe);
TYPE @\FuncDecl{shmem\_atomic\_compare\_swap}@(shmem_ctx_t ctx, TYPE *dest, TYPE cond, TYPE value, int pe);
\end{C11synopsis}
where \TYPE{} is one of the standard \ac{AMO} types specified by
Table~\ref{stdamotypes}.

\begin{Csynopsis}
TYPE @\FuncDecl{shmem\_\FuncParam{TYPENAME}\_atomic\_compare\_swap}@(TYPE *dest, TYPE cond, TYPE value, int pe);
TYPE @\FuncDecl{shmem\_ctx\_\FuncParam{TYPENAME}\_atomic\_compare\_swap}@(shmem_ctx_t ctx, TYPE *dest, TYPE cond, TYPE value, int pe);
\end{Csynopsis}
where \TYPE{} is one of the standard \ac{AMO} types and has a corresponding
\TYPENAME{} specified by Table~\ref{stdamotypes}.

\begin{DeprecateBlock}
\begin{C11synopsis}
TYPE @\FuncDecl{shmem\_cswap}@(TYPE *dest, TYPE cond, TYPE value, int pe);
\end{C11synopsis}
where \TYPE{} is one of \{\CTYPE{int}, \CTYPE{long}, \CTYPE{long long}\}.

\begin{Csynopsis}
TYPE @\FuncDecl{shmem\_\FuncParam{TYPENAME}\_cswap}@(TYPE *dest, TYPE cond, TYPE value, int pe);
\end{Csynopsis}
where \TYPE{} is one of \{\CTYPE{int}, \CTYPE{long}, \CTYPE{long long}\}
and has a corresponding \TYPENAME{} specified by Table~\ref{stdamotypes}.
\end{DeprecateBlock}

\begin{apiarguments}
    \apiargument{IN}{ctx}{A context handle specifying the context on which to perform the operation.
        When this argument is not provided, the operation is performed on
        the default context.}
    \apiargument{OUT}{dest}{Symmetric address of the destination data object.
        The type of \dest{} should match that implied in the SYNOPSIS section.}
    \apiargument{IN}{cond}{\VAR{cond} is compared to the remote \VAR{dest}
        value. If \VAR{cond} and the remote \VAR{dest} are equal, then \VAR{value}
        is swapped into the remote \VAR{dest}; otherwise, the remote \VAR{dest} is
        unchanged.  In either case, the old value of the remote \VAR{dest} is
        returned as the routine return value. \VAR{cond} must be of the same data
        type as \VAR{dest}.}
    \apiargument{IN}{value}{The value to be atomically written to the target \ac{PE}.
        The type of \VAR{value} should match that implied in the SYNOPSIS section.}
    \apiargument{IN}{pe}{\ac{PE} number of the target \ac{PE} relative to the team associated
      with the given \VAR{ctx} when provided, or the default context otherwise.}
\end{apiarguments}

\apidescription{
    The conditional swap routines conditionally update a \VAR{dest} data object on
    the specified target \ac{PE} and return the prior contents of the data object in one
    atomic operation.
}

\apireturnvalues{
    The contents that had been in the \VAR{dest} data object on the remote
    \ac{PE} prior to the conditional swap. Data type is the same as the
    \VAR{dest} data type.
}

\begin{apiexamples}

\apicexample
    {The following call ensures that the first \ac{PE} to execute the
    conditional swap will successfully write its \ac{PE} number to
    \VAR{race\_winner} on \ac{PE} \CONST{0}.}
    {./example_code/shmem_atomic_compare_swap_example.c}
    {}

\end{apiexamples}

\end{apidefinition}


\subsubsection{\textbf{SHMEM\_ATOMIC\_SWAP}}
\label{subsec:shmem_atomic_swap}
\apisummary{
    Performs an atomic swap to a remote data object.
}

\begin{apidefinition}

\begin{C11synopsis}
TYPE @\FuncDecl{shmem\_atomic\_swap}@(TYPE *dest, TYPE value, int pe);
TYPE @\FuncDecl{shmem\_atomic\_swap}@(shmem_ctx_t ctx, TYPE *dest, TYPE value, int pe);
\end{C11synopsis}
where \TYPE{} is one of the extended \ac{AMO} types specified by Table \ref{extamotypes}.

\begin{Csynopsis}
TYPE @\FuncDecl{shmem\_\FuncParam{TYPENAME}\_atomic\_swap}@(TYPE *dest, TYPE value, int pe);
TYPE @\FuncDecl{shmem\_ctx\_\FuncParam{TYPENAME}\_atomic\_swap}@(shmem_ctx_t ctx, TYPE *dest, TYPE value, int pe);
\end{Csynopsis}
where \TYPE{} is one of the extended \ac{AMO} types and has a corresponding \TYPENAME{} specified by Table \ref{extamotypes}.

\begin{DeprecateBlock}
\begin{C11synopsis}
TYPE @\FuncDecl{shmem\_swap}@(TYPE *dest, TYPE value, int pe);
\end{C11synopsis}
where \TYPE{} is one of \{\CTYPE{float}, \CTYPE{double}, \CTYPE{int},
\CTYPE{long}, \CTYPE{long long}\}.

\begin{Csynopsis}
TYPE @\FuncDecl{shmem\_\FuncParam{TYPENAME}\_swap}@(TYPE *dest, TYPE value, int pe);
\end{Csynopsis}
where \TYPE{} is one of \{\CTYPE{float}, \CTYPE{double}, \CTYPE{int},
\CTYPE{long}, \CTYPE{long long}\} and has a corresponding
\TYPENAME{} specified by Table~\ref{extamotypes}.
\end{DeprecateBlock}

\begin{apiarguments}
  \apiargument{IN}{ctx}{A context handle specifying the context on which to perform the operation.
    When this argument is not provided, the operation is performed on
    the default context.}
  \apiargument{OUT}{dest}{Symmetric address of the destination data object.
    The type of \dest{} should match that implied in the SYNOPSIS section.}
  \apiargument{IN}{value}{The value to be atomically written to the remote \ac{PE}.
    The type of \VAR{value} should match that implied in the SYNOPSIS section.}
  \apiargument{IN}{pe}{ An integer that indicates the \ac{PE} number on which
    \dest{} is to be updated.}
\end{apiarguments}

\apidescription{
    \FUNC{shmem\_atomic\_swap} performs an atomic swap operation.
    It writes \VAR{value} into \dest{} on \ac{PE} and returns the previous
    contents of \dest{} as an atomic operation.
    If the context handle \VAR{ctx} does not correspond to a valid context,
    the behavior is undefined.
}

\apireturnvalues{
       The content that had been at the \dest{} address on the remote \ac{PE}
       prior to the swap is returned.
}

\begin{apiexamples}

\apicexample
    {The example below swaps values between odd numbered \acp{PE} and
    their right (modulo) neighbor and outputs the result of swap.}
    {./example_code/shmem_atomic_swap_example.c}
    {}

\end{apiexamples}

\end{apidefinition}


\subsubsection{\textbf{SHMEM\_ATOMIC\_FETCH\_INC}}
\label{subsec:shmem_atomic_fetch_inc}
\apisummary{
    Performs an atomic fetch-and-increment  operation on a remote data object.
}

\begin{apidefinition}

\begin{C11synopsis}
TYPE @\FuncDecl{shmem\_atomic\_fetch\_inc}@(TYPE *dest, int pe);
TYPE @\FuncDecl{shmem\_atomic\_fetch\_inc}@(shmem_ctx_t ctx, TYPE *dest, int pe);
\end{C11synopsis}
where \TYPE{} is one of the standard \ac{AMO} types specified by
Table~\ref{stdamotypes}.

\begin{Csynopsis}
TYPE @\FuncDecl{shmem\_\FuncParam{TYPENAME}\_atomic\_fetch\_inc}@(TYPE *dest, int pe);
TYPE @\FuncDecl{shmem\_ctx\_\FuncParam{TYPENAME}\_atomic\_fetch\_inc}@(shmem_ctx_t ctx, TYPE *dest, int pe);
\end{Csynopsis}
where \TYPE{} is one of the standard \ac{AMO} types and has a corresponding
\TYPENAME{} specified by Table~\ref{stdamotypes}.

\begin{DeprecateBlock}
\begin{C11synopsis}
TYPE @\FuncDecl{shmem\_finc}@(TYPE *dest, int pe);
\end{C11synopsis}
where \TYPE{} is one of \{\CTYPE{int}, \CTYPE{long}, \CTYPE{long long}\}.

\begin{Csynopsis}
TYPE @\FuncDecl{shmem\_\FuncParam{TYPENAME}\_finc}@(TYPE *dest, int pe);
\end{Csynopsis}
where \TYPE{} is one of \{\CTYPE{int}, \CTYPE{long}, \CTYPE{long long}\}
and has a corresponding \TYPENAME{} specified by Table~\ref{stdamotypes}.
\end{DeprecateBlock}

\begin{apiarguments}

\apiargument{IN}{ctx}{A context handle specifying the context on which to perform the operation.
    When this argument is not provided, the operation is performed on
    the default context.}
\apiargument{OUT}{dest}{Symmetric address of the destination data object.
    The type of \dest{} should match that implied in the SYNOPSIS section.}
\apiargument{IN}{pe}{An integer that indicates the \ac{PE} number on which
    \dest{} is to be updated.}

\end{apiarguments}


\apidescription{
   These routines perform a fetch-and-increment operation.  The \dest{} on
   \ac{PE} \VAR{pe} is increased by one and the routine returns the previous
   contents of \dest{} as an atomic operation.
   If the context handle \VAR{ctx} does not correspond to a valid context,
   the behavior is undefined.
}

\apireturnvalues{
    The contents that had been at the \dest{} address on the remote \ac{PE} prior to
    the increment.  The data type of the return value is the same as the \dest.
}

\begin{apiexamples}

\apicexample
    {The following \FUNC{shmem\_atomic\_fetch\_inc} example is for
    \Cstd[11] programs:}
    {./example_code/shmem_atomic_fetch_inc_example.c}
    {}

\end{apiexamples}

\end{apidefinition}


\subsubsection{\textbf{SHMEM\_ATOMIC\_INC}}
\label{subsec:shmem_atomic_inc}
\apisummary{
    Performs an atomic increment operation on a remote data object.
}

\begin{apidefinition}

\begin{C11synopsis}
void @\FuncDecl{shmem\_atomic\_inc}@(TYPE *dest, int pe);
void @\FuncDecl{shmem\_atomic\_inc}@(shmem_ctx_t ctx, TYPE *dest, int pe);
\end{C11synopsis}
where \TYPE{} is one of the standard \ac{AMO} types specified by
Table~\ref{stdamotypes}.

\begin{Csynopsis}
void @\FuncDecl{shmem\_\FuncParam{TYPENAME}\_atomic\_inc}@(TYPE *dest, int pe);
void @\FuncDecl{shmem\_ctx\_\FuncParam{TYPENAME}\_atomic\_inc}@(shmem_ctx_t ctx, TYPE *dest, int pe);
\end{Csynopsis}
where \TYPE{} is one of the standard \ac{AMO} types and has a corresponding
\TYPENAME{} specified by Table~\ref{stdamotypes}.

\begin{DeprecateBlock}
\begin{C11synopsis}
void @\FuncDecl{shmem\_inc}@(TYPE *dest, int pe);
\end{C11synopsis}
where \TYPE{} is one of \{\CTYPE{int}, \CTYPE{long}, \CTYPE{long long}\}.

\begin{Csynopsis}
void @\FuncDecl{shmem\_\FuncParam{TYPENAME}\_inc}@(TYPE *dest, int pe);
\end{Csynopsis}
where \TYPE{} is one of \{\CTYPE{int}, \CTYPE{long}, \CTYPE{long long}\}
and has a corresponding \TYPENAME{} specified by Table~\ref{stdamotypes}.
\end{DeprecateBlock}

\begin{apiarguments}

\apiargument{IN}{ctx}{A context handle specifying the context on which to perform the operation.
    When this argument is not provided, the operation is performed on
    the default context.}
\apiargument{OUT}{dest}{Symmetric address of the destination data object.
    The type of \dest{} should match that implied in the SYNOPSIS section.}
\apiargument{IN}{pe}{\ac{PE} number of the target \ac{PE} relative to the team associated
    with the given \VAR{ctx} when provided, or the default context otherwise.}
\end{apiarguments}

\apidescription{
    These  routines perform  an atomic increment operation on the \VAR{dest} data
    object on \ac{PE}.
}

\apireturnvalues{
    None.
}

\begin{apiexamples}

\apicexample
    { The following \FUNC{shmem\_atomic\_inc} example is for
    \Cstd[11] programs: }
    {./example_code/shmem_atomic_inc_example.c}
    {}

\end{apiexamples}

\end{apidefinition}


\subsubsection{\textbf{SHMEM\_ATOMIC\_FETCH\_ADD}}
\label{subsec:shmem_atomic_fetch_add}
\apisummary{
    Performs an atomic fetch-and-add operation on a remote data object.
}

\begin{apidefinition}

\begin{C11synopsis}
TYPE @\FuncDecl{shmem\_atomic\_fetch\_add}@(TYPE *dest, TYPE value, int pe);
TYPE @\FuncDecl{shmem\_atomic\_fetch\_add}@(shmem_ctx_t ctx, TYPE *dest, TYPE value, int pe);
\end{C11synopsis}
where \TYPE{} is one of the standard \ac{AMO} types specified by
Table~\ref{stdamotypes}.

\begin{Csynopsis}
TYPE @\FuncDecl{shmem\_\FuncParam{TYPENAME}\_atomic\_fetch\_add}@(TYPE *dest, TYPE value, int pe);
TYPE @\FuncDecl{shmem\_ctx\_\FuncParam{TYPENAME}\_atomic\_fetch\_add}@(shmem_ctx_t ctx, TYPE *dest, TYPE value, int pe);
\end{Csynopsis}
where \TYPE{} is one of the standard \ac{AMO} types and has a corresponding
\TYPENAME{} specified by Table~\ref{stdamotypes}.

\begin{DeprecateBlock}
\begin{C11synopsis}
TYPE @\FuncDecl{shmem\_fadd}@(TYPE *dest, TYPE value, int pe);
\end{C11synopsis}
where \TYPE{} is one of \{\CTYPE{int}, \CTYPE{long}, \CTYPE{long long}\}.

\begin{Csynopsis}
TYPE @\FuncDecl{shmem\_\FuncParam{TYPENAME}\_fadd}@(TYPE *dest, TYPE value, int pe);
\end{Csynopsis}
where \TYPE{} is one of \{\CTYPE{int}, \CTYPE{long}, \CTYPE{long long}\}
and has a corresponding \TYPENAME{} specified by Table~\ref{stdamotypes}.
\end{DeprecateBlock}

\begin{apiarguments}

\apiargument{IN}{ctx}{A context handle specifying the context on which to perform the operation.
    When this argument is not provided, the operation is performed on
    the default context.}
\apiargument{OUT}{dest}{Symmetric address of the destination data object.
    The type of \VAR{dest} should match that implied in the
    SYNOPSIS section.}
\apiargument{IN}{value}{The operand to the atomic fetch-and-add operation.  The
    type of \VAR{value} should match that implied in the SYNOPSIS section.}
\apiargument{IN}{pe}{An integer that indicates the \ac{PE} number on which
    \VAR{dest} is to be updated.}

\end{apiarguments}

\apidescription{
    \FUNC{shmem\_atomic\_fetch\_add} routines perform an atomic fetch-and-add operation.  An
    atomic fetch-and-add operation fetches the old \VAR{dest} and adds \VAR{value}
    to \VAR{dest} without the possibility of another atomic operation on the
    \VAR{dest} between the time of the fetch and the update.  These routines add
    \VAR{value} to \VAR{dest} on \VAR{pe} and return the previous contents of
    \VAR{dest} as an atomic operation.
    If the context handle \VAR{ctx} does not correspond to a valid context,
    the behavior is undefined.
}

\apireturnvalues{
    The contents that had been at the \VAR{dest} address on the remote \ac{PE}
    prior to the atomic addition operation.  The data type of the return value is
    the same as the \VAR{dest}.
}

\begin{apiexamples}

\apicexample
        {The following \FUNC{shmem\_atomic\_fetch\_add} example is for
        \Cstd[11] programs:}
        {./example_code/shmem_atomic_fetch_add_example.c}
        {}

\end{apiexamples}

\end{apidefinition}


\subsubsection{\textbf{SHMEM\_ATOMIC\_ADD}}
\label{subsec:shmem_atomic_add}
\apisummary{
    Performs an atomic add operation on a remote symmetric data object.
}

\begin{apidefinition}

\begin{C11synopsis}
void @\FuncDecl{shmem\_atomic\_add}@(TYPE *dest, TYPE value, int pe);
void @\FuncDecl{shmem\_atomic\_add}@(shmem_ctx_t ctx, TYPE *dest, TYPE value, int pe);
\end{C11synopsis}
where \TYPE{} is one of the standard \ac{AMO} types specified by
Table~\ref{stdamotypes}.

\begin{Csynopsis}
void @\FuncDecl{shmem\_\FuncParam{TYPENAME}\_atomic\_add}@(TYPE *dest, TYPE value, int pe);
void @\FuncDecl{shmem\_ctx\_\FuncParam{TYPENAME}\_atomic\_add}@(shmem_ctx_t ctx, TYPE *dest, TYPE value, int pe);
\end{Csynopsis}
where \TYPE{} is one of the standard \ac{AMO} types and has a corresponding
\TYPENAME{} specified by Table~\ref{stdamotypes}.

\begin{DeprecateBlock}
\begin{C11synopsis}
void @\FuncDecl{shmem\_add}@(TYPE *dest, TYPE value, int pe);
\end{C11synopsis}
where \TYPE{} is one of \{\CTYPE{int}, \CTYPE{long}, \CTYPE{long long}\}.

\begin{Csynopsis}
void @\FuncDecl{shmem\_\FuncParam{TYPENAME}\_add}@(TYPE *dest, TYPE value, int pe);
\end{Csynopsis}
where \TYPE{} is one of \{\CTYPE{int}, \CTYPE{long}, \CTYPE{long long}\}
and has a corresponding \TYPENAME{} specified by Table~\ref{stdamotypes}.
\end{DeprecateBlock}

\begin{apiarguments}
    \apiargument{IN}{ctx}{A context handle specifying the context on which to perform the operation.
        When this argument is not provided, the operation is performed on
        the default context.}
    \apiargument{OUT}{dest}{Symmetric address of the destination data object.
        The type of \dest{} should match that implied in the SYNOPSIS section.}
    \apiargument{IN}{value}{The operand to the atomic add operation.
        The type of \VAR{value} should match that implied in the SYNOPSIS section.}
    \apiargument{IN}{pe}{\ac{PE} number of the target \ac{PE} relative to the team associated
      with the given \VAR{ctx} when provided, or the default context otherwise.}
\end{apiarguments}

\apidescription{
    The \FUNC{shmem\_atomic\_add} routine performs an atomic add operation. It adds
    \VAR{value} to \dest{} on \ac{PE} \VAR{pe} and atomically updates the \dest{}
    without returning the value.
 }

\apireturnvalues{
    None.
}

\begin{apiexamples}

\apicexample
    {}
    {./example_code/shmem_atomic_add_example.c}
    {}

\end{apiexamples}

\end{apidefinition}


\subsubsection{\textbf{SHMEM\_ATOMIC\_FETCH\_AND}}
\label{subsec:shmem_atomic_fetch_and}
\apisummary{
  Atomically perform a fetching bitwise AND operation on a remote data object.
}

\begin{apidefinition}

\begin{C11synopsis}
TYPE @\FuncDecl{shmem\_atomic\_fetch\_and}@(TYPE *dest, TYPE value, int pe);
TYPE @\FuncDecl{shmem\_atomic\_fetch\_and}@(shmem_ctx_t ctx, TYPE *dest, TYPE value, int pe);
\end{C11synopsis}
where \TYPE{} is one of the bitwise \ac{AMO} types specified by
Table~\ref{bitamotypes}.

\begin{Csynopsis}
TYPE @\FuncDecl{shmem\_\FuncParam{TYPENAME}\_atomic\_fetch\_and}@(TYPE *dest, TYPE value, int pe);
TYPE @\FuncDecl{shmem\_ctx\_\FuncParam{TYPENAME}\_atomic\_fetch\_and}@(shmem_ctx_t ctx, TYPE *dest, TYPE value, int pe);
\end{Csynopsis}
where \TYPE{} is one of the bitwise \ac{AMO} types and has a corresponding
\TYPENAME{} specified by Table~\ref{bitamotypes}.

\begin{apiarguments}

  \apiargument{IN}{ctx}{A context handle specifying the context on which to perform the operation.
    When this argument is not provided, the operation is performed on
    the default context.}
  \apiargument{OUT}{dest}{Symmetric address of the destination data object.
    The type of \dest{} should match that implied in the SYNOPSIS section.}
  \apiargument{IN}{value}{The operand to the bitwise AND operation.
    The type of \VAR{value} should match that implied in the SYNOPSIS section.}
  \apiargument{IN}{pe}{\ac{PE} number of the remote \ac{PE} relative to the team associated
    with the given \VAR{ctx} when provided, or the default context otherwise.}
\end{apiarguments}

\apidescription{
  \FUNC{shmem\_atomic\_fetch\_and} atomically performs a fetching bitwise AND
  on the remotely accessible data object pointed to by \VAR{dest} at PE
  \VAR{pe} with the operand \VAR{value}.
}

\apireturnvalues{
    The value pointed to by \VAR{dest} on \ac{PE} \VAR{pe} immediately before the
  operation is performed.
}

\end{apidefinition}


\subsubsection{\textbf{SHMEM\_ATOMIC\_AND}}
\label{subsec:shmem_atomic_and}
\apisummary{
  Atomically perform a non-fetching bitwise AND operation on a
  remote data object.
}

\begin{apidefinition}

\begin{C11synopsis}
void @\FuncDecl{shmem\_atomic\_and}@(TYPE *dest, TYPE value, int pe);
void @\FuncDecl{shmem\_atomic\_and}@(shmem_ctx_t ctx, TYPE *dest, TYPE value, int pe);
\end{C11synopsis}
where \TYPE{} is one of the bitwise \ac{AMO} types specified by
Table~\ref{bitamotypes}.

\begin{Csynopsis}
void @\FuncDecl{shmem\_\FuncParam{TYPENAME}\_atomic\_and}@(TYPE *dest, TYPE value, int pe);
void @\FuncDecl{shmem\_ctx\_\FuncParam{TYPENAME}\_atomic\_and}@(shmem_ctx_t ctx, TYPE *dest, TYPE value, int pe);
\end{Csynopsis}
where \TYPE{} is one of the bitwise \ac{AMO} types and has a corresponding
\TYPENAME{} specified by Table~\ref{bitamotypes}.

\begin{apiarguments}

  \apiargument{IN}{ctx}{A context handle specifying the context on which to perform the operation.
    When this argument is not provided, the operation is performed on
    the default context.}
  \apiargument{OUT}{dest}{Symmetric address of the destination data object.
    The type of \dest{} should match that implied in the SYNOPSIS section.}
  \apiargument{IN}{value}{The operand to the bitwise AND operation.
    The type of \VAR{value} should match that implied in the SYNOPSIS section.}
  \apiargument{IN}{pe}{\ac{PE} number of the remote \ac{PE} relative to the team associated
    with the given \VAR{ctx} when provided, or the default context otherwise.}
\end{apiarguments}

\apidescription{
  \FUNC{shmem\_atomic\_and} atomically performs a non-fetching bitwise AND
  on the remotely accessible data object pointed to by \VAR{dest} at PE
  \VAR{pe} with the operand \VAR{value}.
}

\apireturnvalues{
  None.
}

\end{apidefinition}


\subsubsection{\textbf{SHMEM\_ATOMIC\_FETCH\_OR}}
\label{subsec:shmem_atomic_fetch_or}
\apisummary{
  Atomically perform a fetching bitwise OR operation on a remote data object.
}

\begin{apidefinition}

\begin{C11synopsis}
TYPE @\FuncDecl{shmem\_atomic\_fetch\_or}@(TYPE *dest, TYPE value, int pe);
TYPE @\FuncDecl{shmem\_atomic\_fetch\_or}@(shmem_ctx_t ctx, TYPE *dest, TYPE value, int pe);
\end{C11synopsis}
where \TYPE{} is one of the bitwise \ac{AMO} types specified by
Table~\ref{bitamotypes}.

\begin{Csynopsis}
TYPE @\FuncDecl{shmem\_\FuncParam{TYPENAME}\_atomic\_fetch\_or}@(TYPE *dest, TYPE value, int pe);
TYPE @\FuncDecl{shmem\_ctx\_\FuncParam{TYPENAME}\_atomic\_fetch\_or}@(shmem_ctx_t ctx, TYPE *dest, TYPE value, int pe);
\end{Csynopsis}
where \TYPE{} is one of the bitwise \ac{AMO} types and has a corresponding
\TYPENAME{} specified by Table~\ref{bitamotypes}.

\begin{apiarguments}

  \apiargument{IN}{ctx}{A context handle specifying the context on which to perform the operation.
    When this argument is not provided, the operation is performed on
    the default context.}
  \apiargument{OUT}{dest}{Symmetric address of the destination data object.
    The type of \dest{} should match that implied in the SYNOPSIS section.}
  \apiargument{IN}{value}{The operand to the bitwise OR operation.
    The type of \VAR{value} should match that implied in the SYNOPSIS section.}
  \apiargument{IN}{pe}{\ac{PE} number of the remote \ac{PE} relative to the team associated
    with the given \VAR{ctx} when provided, or the default context otherwise.}
\end{apiarguments}

\apidescription{
  \FUNC{shmem\_atomic\_fetch\_or} atomically performs a fetching bitwise OR
  on the remotely accessible data object pointed to by \VAR{dest} at PE
  \VAR{pe} with the operand \VAR{value}.
}

\apireturnvalues{
  The value pointed to by \VAR{dest} on PE \VAR{pe} immediately before the
  operation is performed.
}

\end{apidefinition}


\subsubsection{\textbf{SHMEM\_ATOMIC\_OR}}
\label{subsec:shmem_atomic_or}
\apisummary{
  Atomically perform a non-fetching bitwise OR operation on a
  remote data object.
}

\begin{apidefinition}

\begin{C11synopsis}
void @\FuncDecl{shmem\_atomic\_or}@(TYPE *dest, TYPE value, int pe);
void @\FuncDecl{shmem\_atomic\_or}@(shmem_ctx_t ctx, TYPE *dest, TYPE value, int pe);
\end{C11synopsis}
where \TYPE{} is one of the bitwise \ac{AMO} types specified by
Table~\ref{bitamotypes}.

\begin{Csynopsis}
void @\FuncDecl{shmem\_\FuncParam{TYPENAME}\_atomic\_or}@(TYPE *dest, TYPE value, int pe);
void @\FuncDecl{shmem\_ctx\_\FuncParam{TYPENAME}\_atomic\_or}@(shmem_ctx_t ctx, TYPE *dest, TYPE value, int pe);
\end{Csynopsis}
where \TYPE{} is one of the bitwise \ac{AMO} types and has a corresponding
\TYPENAME{} specified by Table~\ref{bitamotypes}.

\begin{apiarguments}

  \apiargument{IN}{ctx}{A context handle specifying the context on which to perform the operation.
    When this argument is not provided, the operation is performed on
    the default context.}
  \apiargument{OUT}{dest}{Symmetric address of the destination data object.
    The type of \dest{} should match that implied in the SYNOPSIS section.}
  \apiargument{IN}{value}{The operand to the bitwise OR operation.
    The type of \VAR{value} should match that implied in the SYNOPSIS section.}
  \apiargument{IN}{pe}{\ac{PE} number of the remote \ac{PE} relative to the team associated
    with the given \VAR{ctx} when provided, or the default context otherwise.}
\end{apiarguments}

\apidescription{
  \FUNC{shmem\_atomic\_or} atomically performs a non-fetching bitwise OR
  on the remotely accessible data object pointed to by \VAR{dest} at PE
  \VAR{pe} with the operand \VAR{value}.
}

\apireturnvalues{
  None.
}

\end{apidefinition}


\subsubsection{\textbf{SHMEM\_ATOMIC\_FETCH\_XOR}}
\label{subsec:shmem_atomic_fetch_xor}
\apisummary{
  Atomically perform a fetching bitwise exclusive OR (XOR) operation on a
  remote data object.
}

\begin{apidefinition}

\begin{C11synopsis}
TYPE @\FuncDecl{shmem\_atomic\_fetch\_xor}@(TYPE *dest, TYPE value, int pe);
TYPE @\FuncDecl{shmem\_atomic\_fetch\_xor}@(shmem_ctx_t ctx, TYPE *dest, TYPE value, int pe);
\end{C11synopsis}
where \TYPE{} is one of the bitwise \ac{AMO} types specified by
Table~\ref{bitamotypes}.

\begin{Csynopsis}
TYPE @\FuncDecl{shmem\_\FuncParam{TYPENAME}\_atomic\_fetch\_xor}@(TYPE *dest, TYPE value, int pe);
TYPE @\FuncDecl{shmem\_ctx\_\FuncParam{TYPENAME}\_atomic\_fetch\_xor}@(shmem_ctx_t ctx, TYPE *dest, TYPE value, int pe);
\end{Csynopsis}
where \TYPE{} is one of the bitwise \ac{AMO} types and has a corresponding
\TYPENAME{} specified by Table~\ref{bitamotypes}.

\begin{apiarguments}

  \apiargument{IN}{ctx}{A context handle specifying the context on which to perform the operation.
    When this argument is not provided, the operation is performed on
    the default context.}
  \apiargument{OUT}{dest}{Symmetric address of the destination data object.
    The type of \dest{} should match that implied in the SYNOPSIS section.}
  \apiargument{IN}{value}{The operand to the bitwise XOR operation.
    The type of \VAR{value} should match that implied in the SYNOPSIS section.}
  \apiargument{IN}{pe}{An integer value for the \ac{PE} on which \VAR{dest}
    is to be updated.}

\end{apiarguments}

\apidescription{
  \FUNC{shmem\_atomic\_fetch\_xor} atomically performs a fetching bitwise XOR
  on the remotely accessible data object pointed to by \VAR{dest} at PE
  \VAR{pe} with the operand \VAR{value}.
  If the context handle \VAR{ctx} does not correspond to a valid context,
  the behavior is undefined.
}

\apireturnvalues{
  The value pointed to by \VAR{dest} on PE \VAR{pe} immediately before the
  operation is performed.
}

\end{apidefinition}


\subsubsection{\textbf{SHMEM\_ATOMIC\_XOR}}
\label{subsec:shmem_atomic_xor}
\apisummary{
  Atomically perform a non-fetching bitwise exclusive OR (XOR) operation on a
  remote data object.
}

\begin{apidefinition}

\begin{C11synopsis}
void @\FuncDecl{shmem\_atomic\_xor}@(TYPE *dest, TYPE value, int pe);
void @\FuncDecl{shmem\_atomic\_xor}@(shmem_ctx_t ctx, TYPE *dest, TYPE value, int pe);
\end{C11synopsis}
where \TYPE{} is one of the bitwise \ac{AMO} types specified by
Table~\ref{bitamotypes}.

\begin{Csynopsis}
void @\FuncDecl{shmem\_\FuncParam{TYPENAME}\_atomic\_xor}@(TYPE *dest, TYPE value, int pe);
void @\FuncDecl{shmem\_ctx\_\FuncParam{TYPENAME}\_atomic\_xor}@(shmem_ctx_t ctx, TYPE *dest, TYPE value, int pe);
\end{Csynopsis}
where \TYPE{} is one of the bitwise \ac{AMO} types and has a corresponding
\TYPENAME{} specified by Table~\ref{bitamotypes}.

\begin{apiarguments}

  \apiargument{IN}{ctx}{A context handle specifying the context on which to perform the operation.
    When this argument is not provided, the operation is performed on
    the default context.}
  \apiargument{OUT}{dest}{Symmetric address of the destination data object.
    The type of \dest{} should match that implied in the SYNOPSIS section.}
  \apiargument{IN}{value}{The operand to the bitwise XOR operation.
    The type of \VAR{value} should match that implied in the SYNOPSIS section.}
  \apiargument{IN}{pe}{\ac{PE} number of the target \ac{PE} relative to the team associated
    with the given \VAR{ctx} when provided, or the default context otherwise.}
\end{apiarguments}

\apidescription{
  \FUNC{shmem\_atomic\_xor} atomically performs a non-fetching bitwise XOR
  on the remotely accessible data object pointed to by \VAR{dest} at PE
  \VAR{pe} with the operand \VAR{value}.
}

\apireturnvalues{
  None.
}

\end{apidefinition}






\subsection{Collective Routines}\label{subsec:coll}
\emph{Collective routines} are defined as \newtext{coordinated} communication or synchronization
operations \oldtext{on} \newtext{performed by} a group of \acp{PE} \oldtext{called an active set}.

{\color{Green}
\openshmem provides two types of collective routines:

\begin{enumerate}
\item Collective routines that operate on teams use a team handle parameter to determine
which \acp{PE} will participate in the routine, and use resources encapsulated by the team object
to perform operations. See Section~\ref{subsec:team} for details on team management.
These routines will be the standard for \openshmem moving forward.
\item Collective routines that operate on active sets use a set of parameters to determine
which \acp{PE} will participate and what resources are used to perform operations. These routines
are the legacy API for collectives which will be deprecated and phased out of
implementations moving forward.
\end{enumerate}

Collective routines with no team or active set parameters are deprecated,
and implicitly operate on the team consisting of all \acp{PE} in the computation,
\LibHandleRef{SHMEM\_TEAM\_WORLD}

The team-based collective routines are performed with respect to a valid
\openshmem team, which is specified by a team handle argument.
Team-based collective operations require all \acp{PE} in the team to call
the routine in order for the operation to complete. If an invalid team handle
or \LibConstRef{SHMEM\_TEAM\_NULL} is passed to a team-based collective
routine, the behavior is undefined.

Team objects encapsulate the per \ac{PE} system resources required to complete
team-based collective routines.
All \openshmem teams-based collective calls are blocking routines which may use those
system resources. On completion of a team-based collective call, the \ac{PE} may
immediately call another collective on that same team without any other intervening
synchronization across the team.

While \openshmem routines provide thread safety as per the requesting threading model
(see \ref{subsec:thread_support}), the teams object itself is not thread-safe. It is up
to the program to ensure that on a given \ac{PE}, there are no simutanously calls to routines
on a given team object, including all team based collective calls.

Collective operations are matched across a given team based on ordering. So for a given team,
collectives must occur in the same order across all PEs in a team.

The team-based collective routines defined in the \openshmem Specification are:

\begin{itemize}
\item \FUNC{shmem\_team\_sync}
\item \FUNC{shmem\_team\_broadcast}
\item \FUNC{shmem\_team\_collect}
\item \FUNC{shmem\_team\_fcollect}
\item Reductions for the following operations: AND, MAX, MIN, SUM, PROD, OR, XOR
\item \FUNC{shmem\_team\_alltoall}
\item \FUNC{shmem\_team\_alltoalls}
\end{itemize}

In addition, all team creation functions are collective operations. In addition to the ordering
and thread safety requirements described here, there are additional synchronization requirements
on team creation operations. See section \ref{subsec:team} for more details.

The deprecated function \FUNC{shmem\_sync\_all} is provided for backward compatibility to synchronize
all \acp{PE} in the computation. This should be replaced in applications by the equivalent
\FUNC{shmem\_sync(SHMEM\_TEAM\_WORLD)}.
}

\begin{DeprecateBlock}
The \newtext{active-set-based} collective routines require all \acp{PE}
in the active set to simultaneously call the
routine.  A \ac{PE} that is not in the active set calling the collective
routine results in undefined behavior.  \oldtext{All collective routines have an
active set as an input parameter except \FUNC{shmem\_barrier\_all} and
\FUNC{shmem\_sync\_all}. Both \FUNC{shmem\_barrier\_all} and
\FUNC{shmem\_sync\_all} must be called by all \acp{PE} of the \openshmem program.}

The active set is defined by the arguments \VAR{PE\_start}, \VAR{logPE\_stride},
and \VAR{PE\_size}.  \VAR{PE\_start} specifies the starting \ac{PE} number and
is the lowest numbered \ac{PE} in the active set.  The stride between successive
\acp{PE} in the active set is $2^{logPE\_stride}$ and \VAR{logPE\_stride} must
be greater than or equal to zero.  \VAR{PE\_size} specifies the number of
\acp{PE} in the active set and must be greater than zero.  The active set must
satisfy the requirement that its last member corresponds to a valid \ac{PE}
number, that is
$0 \le PE\_start + (PE\_size - 1) * 2^{logPE\_stride} < npes$.

All \acp{PE} participating in the \newtext{active-set-based} collective routine must provide the same
values for these arguments.  If any of these requirements are not met, the
behavior is undefined.

Another argument important to \newtext{active-set-based} collective routines is \VAR{pSync}, which is a
symmetric work array.  All \acp{PE} participating in an \newtext{active-set-based} collective must pass the
same \VAR{pSync} array.  On completion of \newtext{such} a collective call, the \VAR{pSync} is
restored to its original contents.  The user is permitted to reuse a \VAR{pSync}
array if all previous collective routines using the \VAR{pSync} array have been
completed by all participating \acp{PE}. One can use a synchronization
collective routine such as \FUNC{shmem\_barrier} to ensure completion of previous \newtext{active-set-based} collective
routines. The \FUNC{shmem\_barrier} and \FUNC{shmem\_sync} routines allow the same
\VAR{pSync} array to be used on consecutive calls as long as the \acp{PE}
in the active set do not change.

All collective routines defined in the Specification are blocking.  The
collective routines return on completion.  The \newtext{active-set-based} collective
routines defined in the \openshmem Specification are:

\begin{itemize}
\item \FUNC{shmem\_barrier\_all}
\item \FUNC{shmem\_barrier}
\item \FUNC{shmem\_sync\_all}
\item \FUNC{shmem\_sync}
\item \FUNC{shmem\_broadcast\{32, 64\}}
\item \FUNC{shmem\_collect\{32, 64\}}
\item \FUNC{shmem\_fcollect\{32, 64\}}
\item Reductions for the following operations: AND, MAX, MIN, SUM, PROD, OR, XOR
\item \FUNC{shmem\_alltoall\{32, 64\}}
\item \FUNC{shmem\_alltoalls\{32, 64\}}
\end{itemize}

{\color{Green}
The active-set-based \FUNC{shmem\_barrier} and routine has been deprecated and
no team-based barrier routines will be defined. In future, the behavior
previously provided by \FUNC{shmem\_barrier} should be realized by first calling
\FUNC{shmem\_ctx\_quiet} on any relevant communication contexts followed by a call
to \FUNC{shmem\_sync} by some \openshmem team.

Calls to \FUNC{shmem\_barrier\_all}
should be replaced with a call to quiet the default communication context followed
by a call to \FUNC{shmem\_sync} by \LibHandleRef{SHMEM\_TEAM\_WORLD}.
}
\end{DeprecateBlock}


\subsubsection{\textbf{SHMEM\_BARRIER\_ALL}}\label{subsec:shmem_barrier_all}
\apisummary{
    Registers the arrival of a \ac{PE} at a barrier and suspends \ac{PE} execution
    until all other \acp{PE} arrive at the barrier and all local
    \newtext{updates} and remote memory updates \newtext{on the default
    contexts} are completed.
}

\begin{apidefinition}

\begin{Csynopsis}
void shmem_barrier_all(void);
\end{Csynopsis}

\begin{Fsynopsis}
CALL SHMEM_BARRIER_ALL
\end{Fsynopsis}

\begin{apiarguments}

    \apiargument{None.}{}{} 

\end{apiarguments}

\apidescription{   
    The \FUNC{shmem\_barrier\_all} routine registers the arrival of a \ac{PE} at
    a barrier. Barriers are a fast mechanism for synchronizing all \acp{PE} at
    once.  This routine causes a \ac{PE} to suspend execution until all \acp{PE}
    have called \FUNC{shmem\_barrier\_all}.  This routine must be used with
    \acp{PE} started by \FUNC{shmem\_init}.

    Prior to synchronizing with other \acp{PE}, \FUNC{shmem\_barrier\_all}
    ensures completion of all previously issued memory stores and remote memory
    updates issued \newtext{on the default contexts} via \openshmem{} \acp{AMO} and
    \ac{RMA} routine calls such
    as \FUNC{shmem\_int\_add}, \FUNC{shmem\_put32}, 
    \FUNC{shmem\_put\_nbi}, and \FUNC{shmem\_get\_nbi}.
}

\apireturnvalues{
    None.
}

\apinotes{
    The \FUNC{shmem\_barrier\_all} routine can be used to
    portably ensure that memory access operations observe remote updates in the order
    enforced by initiator PEs.

    \newtext{The \FUNC{shmem\_barrier\_all} routine does not ensure completion
    of operations performed on contexts with the \CONST{SHMEM\_CTX\_PRIVATE}
    option enabled.  Calls to \FUNC{shmem\_ctx\_quiet} can be performed prior
    to calling the barrier routine to ensure completion of operations issued on
    private contexts.}

    \textcolor{red}{REWORDING SUGGESTION: The \FUNC{shmem\_barrier\_all}
    routine does not quiet private contexts.  A call to
    \FUNC{shmem\_ctx\_quiet} prior to \FUNC{shmem\_barrier\_all} ensures
    completion of operations issued on private contexts.}
}

\begin{apiexamples}

\apicexample
    { The following \FUNC{shmem\_barrier\_all} example is for C11 programs:}
    {./example_code/shmem_barrierall_example.c}
    {} 

\end{apiexamples}

\end{apidefinition}


\subsubsection{\textbf{SHMEM\_BARRIER}}\label{subsec:shmem_barrier}
\apisummary{
    Performs all operations described in the \FUNC{shmem\_barrier\_all} interface
    but with respect to a subset of \acp{PE} defined by the \activeset.
}

\begin{apidefinition}

\begin{Csynopsis}
void shmem_barrier(int PE_start, int logPE_stride, int PE_size, long *pSync);
\end{Csynopsis}

\begin{Fsynopsis}
INTEGER PE_start, logPE_stride, PE_size
INTEGER pSync(SHMEM_BARRIER_SYNC_SIZE)
CALL SHMEM_BARRIER(PE_start, logPE_stride, PE_size, pSync)
\end{Fsynopsis}

\begin{apiarguments}

\apiargument{IN}{PE\_start}{The lowest \ac{PE} number of the \activeset of \acp{PE}.
    \VAR{PE\_start} must be of type integer.  If you are using \Fortran, it must be
    a default integer value.}
\apiargument{IN}{logPE\_stride}{The log (base 2) of the stride between consecutive
    \ac{PE} numbers in the \activeset.  \VAR{logPE\_stride} must be of type integer.
    If you are using \Fortran, it must be a default integer value.}
\apiargument{IN}{PE\_size}{The number of  \acp{PE} in the \activeset.  \VAR{PE\_size}
    must be of type integer.  If you are  using  \Fortran, it must be a default
    integer value.}
\apiargument{IN}{pSync}{A symmetric work array. In \CorCpp, \VAR{pSync} must
    be of type long and size \CONST{SHMEM\_BARRIER\_SYNC\_SIZE}.  In \Fortran,
    \VAR{pSync} must be of type integer and size \CONST{SHMEM\_BARRIER\_SYNC\_SIZE}.
    If you are using \Fortran, it must  be a default  integer type.  Every element
    of this array must be initialized to \CONST{SHMEM\_SYNC\_VALUE} before any of
    the \acp{PE} in the \activeset enter \FUNC{shmem\_barrier} the first time.}

\end{apiarguments}

\apidescription{
    \FUNC{shmem\_barrier} is a collective synchronization routine over an
    \activeset.  Control returns from \FUNC{shmem\_barrier} after all \acp{PE} in
    the \activeset (specified by \VAR{PE\_start}, \VAR{logPE\_stride}, and
    \VAR{PE\_size}) have called \FUNC{shmem\_barrier}.
    
    As with all \openshmem collective routines, each of these routines assumes that
    only \acp{PE} in the \activeset call the routine.  If a \ac{PE} not  in  the
    \activeset calls an \openshmem collective routine, undefined behavior results.
    
    The values of arguments \VAR{PE\_start}, \VAR{logPE\_stride}, and \VAR{PE\_size}
    must be equal on all \acp{PE} in the \activeset.  The same work array must be
    passed in \VAR{pSync} to all \acp{PE} in the \activeset.
    
    \FUNC{shmem\_barrier} ensures that all previously issued stores and remote
    memory updates, including \acp{AMO} and \ac{RMA} operations, done by any of the
    \acp{PE} in the \activeset \newtext{on the default context} are complete before returning.
    
    The  same  \VAR{pSync} array may be reused on consecutive calls   to
    \FUNC{shmem\_barrier} if the same active \ac{PE} set is used.
}

\apireturnvalues{
    None.
}

\apinotes{
    If the \VAR{pSync} array is initialized at run time, be sure to use some type of
    synchronization, for example, a call to \FUNC{shmem\_barrier\_all}, before
    calling \FUNC{shmem\_barrier} for the first time.
    
    If  the \activeset  does not change, \FUNC{shmem\_barrier} can  be called
    repeatedly with the same \VAR{pSync} array.  No additional synchronization
    beyond that implied by \FUNC{shmem\_barrier} itself is necessary in this case.

    The \FUNC{shmem\_barrier} routine can be used to
    portably ensure that memory access operations observe remote updates in the order
    enforced by initiator PEs.

    \newtext{The \FUNC{shmem\_barrier} routine does not ensure completion
    of operations performed on contexts with the \CONST{SHMEM\_CTX\_PRIVATE}
    option enabled.  Calls to \FUNC{shmem\_ctx\_quiet} can be performed prior
    to calling the barrier routine to ensure completion of operations issued on
    private contexts.}

    \textcolor{red}{REWORDING SUGGESTION: The \FUNC{shmem\_barrier}
    routine does not quiet private contexts.  A call to
    \FUNC{shmem\_ctx\_quiet} prior to \FUNC{shmem\_barrier} ensures
    completion of operations issued on private contexts.}
}

\begin{apiexamples}

\apicexample
	{The following barrier example is for C11 programs:}
	{./example_code/shmem_barrier_example.c}
	{}

\end{apiexamples}

\end{apidefinition}


\subsubsection{\textbf{SHMEM\_SYNC\_ALL}}\label{subsec:shmem_sync_all}
\input{content/shmem_sync_all.tex}

\subsubsection{\textbf{SHMEM\_SYNC}}\label{subsec:shmem_sync}
\apisummary{
    Registers the arrival of a \ac{PE} at a synchronization point.
    This routine does not return until all other \acp{PE} in a given OpenSHMEM team
    arrive at this synchronization point.
\begin{DeprecateBlock}
    Registers the arrival of a \ac{PE} at a synchronization point.
    This routine does not return until all other \acp{PE} in a given OpenSHMEM active set arrive at this synchronization point.
\end{DeprecateBlock}
}

\begin{apidefinition}

\begin{C11synopsis}
int @\FuncDecl{shmem\_sync}@(shmem_team_t team);
\end{C11synopsis}

\begin{Csynopsis}
int @\FuncDecl{shmem\_team\_sync}@(shmem_team_t team);
\end{Csynopsis}

\begin{DeprecateBlock}
\begin{CsynopsisCol}
void @\FuncDecl{shmem\_sync}@(int PE_start, int logPE_stride, int PE_size, long *pSync);
\end{CsynopsisCol}
\end{DeprecateBlock}

\begin{apiarguments}

\apiargument{IN}{team}{The team over which to perform the operation.}%

\begin{DeprecateBlock}
\apiargument{IN}{PE\_start}{The lowest \ac{PE} number of the active set of
    \acp{PE}.}
\apiargument{IN}{logPE\_stride}{The log (base 2) of the stride between
    consecutive \ac{PE} numbers in the active set.}
\apiargument{IN}{PE\_size}{The number of \acp{PE} in the active set.}
\apiargument{IN}{pSync}{
  Symmetric address of a work array of size at least \CONST{SHMEM\_SYNC\_SIZE}.}
\end{DeprecateBlock}

\end{apiarguments}

\apidescription{
    \FUNC{shmem\_sync} is a collective synchronization routine over an
    existing \openshmem team.

    The routine registers the arrival of a \ac{PE} at a synchronization point in the program.
    This is a fast mechanism for synchronizing all \acp{PE} that participate in this
    collective call. The routine blocks the calling \ac{PE} until all \acp{PE} in the
    specified team have called \FUNC{shmem\_sync}. In a multithreaded \openshmem
    program, only the calling thread is blocked.

    Team-based sync routines operate over all \acp{PE} in the provided team argument. All
    \acp{PE} in the provided team must participate in the sync operation.
    If \VAR{team} compares equal to \LibConstRef{SHMEM\_TEAM\_INVALID} or is
    otherwise invalid, the behavior is undefined.

    In contrast with the \FUNC{shmem\_barrier} routine, \FUNC{shmem\_sync} only
    ensures completion and visibility of previously issued memory stores and does not ensure
    completion of remote memory updates issued via \openshmem routines.

\begin{DeprecateBlock}
    \FUNC{shmem\_sync} is a collective synchronization routine over an active set.

    The routine registers the arrival of a \ac{PE} at a synchronization point in the program.
    This is a fast mechanism for synchronizing all \acp{PE} that participate in this
    collective call. The routine blocks the calling \ac{PE} until all \acp{PE} in the
    active set have called \FUNC{shmem\_sync}. In a multithreaded \openshmem
    program, only the calling thread is blocked.

    Active-set-based sync routines operate over all \acp{PE} in the active set
    defined by the \VAR{PE\_start}, \VAR{logPE\_stride}, \VAR{PE\_size} triplet.

    As with all active set-based collective routines,
    each of these routines assumes
    that only \acp{PE} in the active set call the routine.  If a \ac{PE} not in
    the active set calls an active set-based collective routine,
    the behavior is undefined.

    The values of arguments \VAR{PE\_start}, \VAR{logPE\_stride}, and
    \VAR{PE\_size} must be equal on all \acp{PE} in the active set.  The same
    work array must be passed in \VAR{pSync} to all \acp{PE} in the active set.

    The same \VAR{pSync} array may be reused on consecutive calls to
    \FUNC{shmem\_sync} if the same active set is used.
\end{DeprecateBlock}


}

\apireturnvalues{
    Zero on successful local completion. Nonzero otherwise.
}

\apinotes{
    The \FUNC{shmem\_sync} routine can be used to portably ensure that
    memory access operations observe remote updates in the order enforced by the
    initiator \acp{PE}, provided that the initiator PE ensures completion of remote
    updates with a call to \FUNC{shmem\_quiet} prior to the call to the
    \FUNC{shmem\_sync} routine.
}

\begin{apiexamples}

\apicexample
    {The following \FUNC{shmem\_sync} example is
    for \Cstd[11] programs:}
    {./example_code/shmem_sync_example.c}
    {}

\end{apiexamples}

\end{apidefinition}


\subsubsection{\textbf{SHMEM\_BROADCAST}}\label{subsec:shmem_broadcast}
\apisummary{
    Broadcasts a block of data from one \ac{PE} to one or more destination
    \acp{PE}.
}

\begin{apidefinition}

%% C11
\begin{C11synopsis}
int @\FuncDecl{shmem\_broadcast}@(shmem_team_t team, TYPE *dest, const TYPE *source, size_t nelems, int PE_root);
\end{C11synopsis}
where \TYPE{} is one of the standard \ac{RMA} types specified by Table \ref{stdrmatypes}.

%% C/C++
\begin{Csynopsis}
\end{Csynopsis}
\begin{CsynopsisCol}
int @\FuncDecl{shmem\_\FuncParam{TYPENAME}\_broadcast}@(shmem_team_t team, TYPE *dest, const TYPE *source, size_t nelems, int PE_root);
\end{CsynopsisCol}
where \TYPE{} is one of the standard \ac{RMA} types and has a corresponding \TYPENAME{} specified by Table \ref{stdrmatypes}.

\begin{CsynopsisCol}
int @\FuncDecl{shmem\_broadcastmem}@(shmem_team_t team, void *dest, const void *source, size_t nelems, int PE_root);
\end{CsynopsisCol}

\begin{DeprecateBlock}
\begin{CsynopsisCol}
void @\FuncDecl{shmem\_broadcast32}@(void *dest, const void *source, size_t nelems, int PE_root, int PE_start, int logPE_stride, int PE_size, long *pSync);
void @\FuncDecl{shmem\_broadcast64}@(void *dest, const void *source, size_t nelems, int PE_root, int PE_start, int logPE_stride, int PE_size, long *pSync);
\end{CsynopsisCol}
\end{DeprecateBlock}

\begin{apiarguments}

\apiargument{IN}{team}{The team over which to perform the operation.}%

\apiargument{OUT}{dest}{Symmetric address of destination data object.
      The type of \dest{} should match that implied in the SYNOPSIS section.}
\apiargument{IN}{source}{Symmetric address of the source data object.
      The type of \source{} should match that implied in the SYNOPSIS section.}
\apiargument{IN}{nelems}{
  The number of elements in \source{} and \dest{} arrays.
  For \FUNC{shmem\_broadcastmem}, elements are bytes;
  for \FUNC{shmem\_broadcast\{32,64\}}, elements are 4 or 8 bytes,
  respectively.
}
\apiargument{IN}{PE\_root}{Zero-based ordinal of the \ac{PE}, with respect to
    the calling PEs, from which the data is copied.}

\begin{DeprecateBlock}

\apiargument{IN}{PE\_start}{The lowest \ac{PE} number of the active set of
    \acp{PE}.}
\apiargument{IN}{logPE\_stride}{ The log (base 2) of the stride between
    consecutive \ac{PE} numbers in the active set.}
    \apiargument{IN}{PE\_size}{The number of \acp{PE} in the active set.}
\apiargument{IN}{pSync}{
    Symmetric address of a work array of size at least \CONST{SHMEM\_BCAST\_SYNC\_SIZE}.}
\end{DeprecateBlock}

\end{apiarguments}

\apidescription{   
    \openshmem team-based broadcast routines are collective routines over a valid \openshmem team.
    They copy the \source{} data object on the \ac{PE} specified by
    \VAR{PE\_root} to the \dest{} data object on the \acp{PE}
    participating in the collective operation.
    The same \dest{} and \source{} data objects and the same value of
    \VAR{PE\_root} must be passed by all \acp{PE} participating in the
    collective operation.

    For team-based broadcasts:
    \begin{itemize}
    \item The \dest{} object is updated on all \acp{PE}.
    \item All \acp{PE} in the \VAR{team} argument must participate in
      the operation.
    \item Only \acp{PE} in the team  may call the routine.  If a
      \ac{PE} not in the team calls a team-based
      collective routine, the behavior is undefined.
    \item If \VAR{team} compares equal to \LibConstRef{SHMEM\_TEAM\_INVALID} or is
      otherwise invalid, the behavior is undefined.
    \item \ac{PE} numbering is relative to the team. The specified
      root \ac{PE} must be a valid \ac{PE} number for the team,
      between \CONST{0} and \VAR{N$-$1}, where \VAR{N} is the size of
      the team.
    \end{itemize}

    Before any \ac{PE} calls a broadcast routine, the following
    conditions must be ensured:
    \begin{itemize}
    \item The \dest{} array on all \acp{PE} participating in the broadcast
      is ready to accept the broadcast data.
	\end{itemize}
    Otherwise, the behavior is undefined. 

	Upon return from a team-based broadcast routine, the following are true for the local
    \ac{PE}:
    \begin{itemize}
    \item The \dest{} data object is updated.
    \item The \source{} data object may be safely reused.
	\end{itemize}
    
\begin{DeprecateBlock}
    \openshmem active-set broadcast routines are collective routines over an active set. 
    They copy the \source{} data object on the \ac{PE} specified by
    \VAR{PE\_root} to the \dest{} data object on the \acp{PE}
    participating in the collective operation.
    The same \dest{} and \source{} data objects and the same value of
    \VAR{PE\_root} must be passed by all \acp{PE} participating in the
    collective operation.

    For active-set-based broadcasts:
    \begin{itemize}
	\item The \VAR{dest} object is updated on all PEs other than the root PE.
    \item All \acp{PE} in the active set defined by the
      \VAR{PE\_start}, \VAR{logPE\_stride}, \VAR{PE\_size} triplet
      must participate in the operation.
    \item Only \acp{PE} in the active set may call the routine.  If a
      \ac{PE} not in the active set calls an active-set-based
      collective routine, the behavior is undefined.
    \item The values of arguments \VAR{PE\_root}, \VAR{PE\_start},
      \VAR{logPE\_stride}, and \VAR{PE\_size} must be the same value
      on all \acp{PE} in the active set.
    \item The value of \VAR{PE\_root} must be between \CONST{0} and
      \VAR{PE\_size $-$ 1}.
    \item The same \VAR{pSync} work array must be passed by all \acp{PE}
      in the active set.
    \end{itemize}

    Before any \ac{PE} calls a active-set-based broadcast routine, the following
    conditions must be ensured:
    \begin{itemize}
    \item The \dest{} array on all \acp{PE} participating in the broadcast
      is ready to accept the broadcast data.
    \item The \VAR{pSync} array on all \acp{PE} in the
      active set is not still in use from a prior call to an \openshmem
      collective routine.
    \end{itemize}
    Otherwise, the behavior is undefined.

	Upon return from a active-based broadcast routine, the following are true for the local
    \ac{PE}:
    \begin{itemize}
      \item If the current PE is not the root PE, the \dest{} data object is updated.
      \item The \source{} data object may be safely reused.
      \item The values in the \VAR{pSync} array are restored to the
        original values.
    \end{itemize}
\end{DeprecateBlock}
}


\apireturnvalues{
  For team-based broadcasts, zero on successful local completion; otherwise, nonzero.

\begin{DeprecateBlock}
  For active-set-based broadcasts, none.
\end{DeprecateBlock}

}

\apinotes{
    Team handle error checking and integer return codes are currently undefined.
    Implementations may define these behaviors as needed, but programs should
    ensure portability by doing their own checks for invalid team handles and for
    \LibConstRef{SHMEM\_TEAM\_INVALID}.
}

\begin{apiexamples}

\apicexample
    {In the following \Cstd[11] example, the call to \FUNC{shmem\_broadcast} copies \source{}
    on \ac{PE} $0$ to \dest{} on \acp{PE} $0\dots npes-1$.

    \CorCpp{} example:}
    {./example_code/shmem_broadcast_example.c}
    {}

\end{apiexamples}

\end{apidefinition}


\subsubsection{\textbf{SHMEM\_COLLECT, SHMEM\_FCOLLECT}}\label{subsec:shmem_collect}
\apisummary{
    Concatenates blocks of data from multiple \acp{PE} to an array in every
    \ac{PE} participating in the collective routine.
}

\begin{apidefinition}

%% C11
\begin{C11synopsis}
int @\FuncDecl{shmem\_collect}@(shmem_team_t team, TYPE *dest, const TYPE *source, size_t nelems);
int @\FuncDecl{shmem\_fcollect}@(shmem_team_t team, TYPE *dest, const TYPE *source, size_t nelems);
\end{C11synopsis}
where \TYPE{} is one of the standard \ac{RMA} types specified by Table \ref{stdrmatypes}.

\begin{Csynopsis}
\end{Csynopsis}
\begin{CsynopsisCol}
int @\FuncDecl{shmem\_\FuncParam{TYPENAME}\_collect}@(shmem_team_t team, TYPE *dest, const TYPE *source, size_t nelems);
int @\FuncDecl{shmem\_\FuncParam{TYPENAME}\_fcollect}@(shmem_team_t team, TYPE *dest, const TYPE *source, size_t nelems);
\end{CsynopsisCol}
where \TYPE{} is one of the standard \ac{RMA} types and has a corresponding \TYPENAME{} specified by Table \ref{stdrmatypes}.

\begin{CsynopsisCol}
int @\FuncDecl{shmem\_collectmem}@(shmem_team_t team, void *dest, const void *source, size_t nelems);
int @\FuncDecl{shmem\_fcollectmem}@(shmem_team_t team, void *dest, const void *source, size_t nelems);
\end{CsynopsisCol}

\begin{DeprecateBlock}
\begin{CsynopsisCol}
void @\FuncDecl{shmem\_collect32}@(void *dest, const void *source, size_t nelems, int PE_start, int logPE_stride, int PE_size, long *pSync);
void @\FuncDecl{shmem\_collect64}@(void *dest, const void *source, size_t nelems, int PE_start, int logPE_stride, int PE_size, long *pSync);
void @\FuncDecl{shmem\_fcollect32}@(void *dest, const void *source, size_t nelems, int PE_start, int logPE_stride, int PE_size, long *pSync);
void @\FuncDecl{shmem\_fcollect64}@(void *dest, const void *source, size_t nelems, int PE_start, int logPE_stride, int PE_size, long *pSync);
\end{CsynopsisCol}
\end{DeprecateBlock}

\begin{apiarguments}

\apiargument{IN}{team}{A valid \openshmem team handle.}%

\apiargument{OUT}{dest}{Symmetric address of an array large enough
    to accept the concatenation of the \source{} arrays on all participating \acp{PE}.
    The type of \dest{} should match that implied in the SYNOPSIS section.}
\apiargument{IN}{source}{Symmetric address of the source data object.
    The type of \source{} should match that implied in the SYNOPSIS section.}
\apiargument{IN}{nelems}{
  The number of elements in \source{} array.
  For \FUNC{shmem\_[f]collectmem}, elements are bytes;
  for \FUNC{shmem\_[f]collect\{32,64\}}, elements are 4 or 8 bytes,
  respectively.
}

\begin{DeprecateBlock}
\apiargument{IN}{PE\_start}{The lowest \ac{PE} number of the active set of
    \acp{PE}.}
\apiargument{IN}{logPE\_stride}{The log (base \CONST{2}) of the stride between
    consecutive \ac{PE} numbers in the active set.}
    \apiargument{IN}{PE\_size}{The number of \acp{PE} in the active set.}
\apiargument{IN}{pSync}{
    Symmetric address of a work array of size at least \CONST{SHMEM\_COLLECT\_SYNC\_SIZE}.}
\end{DeprecateBlock}

\end{apiarguments}

\apidescription{
    \openshmem \FUNC{collect} and \FUNC{fcollect} routines perform a collective
    operation to concatenate \VAR{nelems}
    data items from the \source{} array into the
    \dest{} array, over an \openshmem team in processor number order. 
	The resultant \dest{} array contains the contribution from
    \acp{PE} as follows:
    
	\begin{itemize}
    \item For a team, the data from \ac{PE} number \CONST{0} in the team is first, then the
    contribution from \ac{PE} \CONST{1} in the team, and so on.
    \end{itemize}
    
    The collected result is written to the \dest{} array for all \acp{PE}
    that participate in the operation. The same \dest{} and \source{}
    arrays must be passed by all \acp{PE} that participate in the operation.
    
    The \FUNC{fcollect} routines require that \VAR{nelems} be the same value in all
    participating \acp{PE}, while the \FUNC{collect} routines allow \VAR{nelems} to
    vary from \ac{PE} to \ac{PE}.

    Team-based collect routines operate over all \acp{PE} in the provided team argument. All
    \acp{PE} in the provided team must participate in the operation.
    If \VAR{team} compares equal to \LibConstRef{SHMEM\_TEAM\_INVALID} or is
    otherwise invalid, the behavior is undefined.

\begin{DeprecateBlock}
    \openshmem \FUNC{collect} and \FUNC{fcollect} routines perform a collective
    operation to concatenate \VAR{nelems}
    data items from the \source{} array into the
    \dest{} array, over an \openshmem active set
    in processor number order. The resultant \dest{} array contains the contribution from
	\acp{PE} as follows:
    \begin{itemize}
   		 \item For an active set, the data from \ac{PE} \VAR{PE\_start} is first, then the
   		 contribution from \ac{PE} \VAR{PE\_start} + \VAR{PE\_stride} second, and so on.
   	\end{itemize}

    The collected result is written to the \dest{} array for all \acp{PE}
    that participate in the operation. The same \dest{} and \source{}
    arrays must be passed by all \acp{PE} that participate in the operation.
    
    The \FUNC{fcollect} routines require that \VAR{nelems} be the same value in all
    participating \acp{PE}, while the \FUNC{collect} routines allow \VAR{nelems} to
    vary from \ac{PE} to \ac{PE}.

    Active-set-based collective routines operate over all \acp{PE} in the active set
    defined by the \VAR{PE\_start}, \VAR{logPE\_stride}, \VAR{PE\_size} triplet.
    As with all active-set-based collective routines,
    each of these routines assumes that
    only \acp{PE} in the active set call the routine. If a \ac{PE} not in the
    active set and calls this collective routine, the behavior is undefined.
    
    The values of arguments \VAR{PE\_start}, \VAR{logPE\_stride}, and \VAR{PE\_size}
    must be the same value on all \acp{PE} in the active set. The same
    \VAR{pSync} work array must be passed by all \acp{PE} in the active set.
    
    Upon return from a collective routine, the following are true for the local
    \ac{PE}:
    \begin{itemize}
    \item The \dest{} array is updated and the \source{} array may be safely reused. 
    \item For active-set-based collective routines, the values in the \VAR{pSync} array are
    restored to the original values.
    \end{itemize}
\end{DeprecateBlock}
}

\apireturnvalues{
    Zero on successful local completion. Nonzero otherwise.
}

\apinotes{
\begin{DeprecateBlock}
    The collective routines operate on active \ac{PE} sets that have a
    non-power-of-two \VAR{PE\_size} with some performance degradation.  They operate
    with no performance degradation when \VAR{nelems} is a non-power-of-two value.
\end{DeprecateBlock}
    The collective routines that operate on teams containing a
    non-power-of-two of PEs do so with some performance degradation. They operate
    with no performance degradation when \VAR{nelems} is a non-power-of-two value.

}

\begin{apiexamples}

\apicexample
    {The following \FUNC{shmem\_collect} example is for \CorCpp{} programs:}
    {./example_code/shmem_collect_example.c}
    {}

\end{apiexamples}

\end{apidefinition}


\subsubsection{\textbf{SHMEM\_REDUCTIONS}}\label{subsec:shmem_reductions}
\apisummary{
    The following functions perform reduction operations across all
    \acp{PE} in a set of \acp{PE}.
}

\begin{apidefinition}


\begin{table}[h]
  \begin{center}
{\color{Green}
    \begin{tabular}{|l|l|l|l|l|}
      \hline
      \TYPE              & \TYPENAME  & \multicolumn{3}{c|}{Operations Supporting \TYPE}\\ \hline
      short              & short      & AND, OR, XOR & MAX, MIN & SUM, PROD \\ \hline
      int                & int        & AND, OR, XOR & MAX, MIN & SUM, PROD \\ \hline
      long               & long       & AND, OR, XOR & MAX, MIN & SUM, PROD \\ \hline
      long long          & longlong   & AND, OR, XOR & MAX, MIN & SUM, PROD \\ \hline
      float              & float      & & MAX, MIN & SUM, PROD \\ \hline
      double             & double     & & MAX, MIN & SUM, PROD \\ \hline
      long double        & longdouble & & MAX, MIN & SUM, PROD \\ \hline
      double \_Complex   & complexd   & & & SUM, PROD \\ \hline
      float  \_Complex   & complexf   & & & SUM, PROD \\ \hline
    \end{tabular}
    \TableCaptionRef{Reduction Types, Names and Supporting Operations}
    \label{reducetypes}
}
  \end{center} 
\end{table}


\paragraph{AND}
Performs a bitwise AND reduction across a set of \acp{PE}.\newline

%% C11
{\color{Green}
\begin{C11synopsis}
int @\FuncDecl{shmem\_and\_to\_all}@(TYPE *dest, const TYPE *source, int nreduce, shmem_team_t team);
\end{C11synopsis}
where \TYPE{} is one of the integer types supported for the AND operation as specified by Table \ref{reducetypes}.
}

%% C/C++
\begin{Csynopsis}
\end{Csynopsis}
{\color{Green}
\begin{CsynopsisCol}
int @\FuncDecl{shmem\_team\_\FuncParam{TYPENAME}\_and\_to\_all}@(TYPE *dest, const TYPE *source, int nreduce, shmem_team_t team);
\end{CsynopsisCol}
}
\begin{DeprecateBlock}
\begin{CsynopsisCol}
void @\FuncDecl{shmem\_\FuncParam{TYPENAME}\_and\_to\_all}@(TYPE *dest, const TYPE *source, int nreduce, int PE_start, int logPE_stride, int PE_size, short *pWrk, long *pSync);
\end{CsynopsisCol}
\end{DeprecateBlock}
\newtext{where \TYPE{} is one of the integer types supported for the AND operation and has a corresponding \TYPENAME{} as specified by Table \ref{reducetypes}.}

%% Fortran
\begin{Fsynopsis}
CALL @\FuncDecl{SHMEM\_INT4\_AND\_TO\_ALL}@(dest, source, nreduce, PE_start, logPE_stride, PE_size, pWrk, pSync)
CALL @\FuncDecl{SHMEM\_INT8\_AND\_TO\_ALL}@(dest, source, nreduce, PE_start, logPE_stride, PE_size, pWrk, pSync)
\end{Fsynopsis}
%%

\paragraph{OR}
Performs a bitwise OR reduction across a set of \acp{PE}.\newline

%% C11
{\color{Green}
\begin{C11synopsis}
int @\FuncDecl{shmem\_or\_to\_all}@(TYPE *dest, const TYPE *source, int nreduce, shmem_team_t team);
\end{C11synopsis}
where \TYPE{} is one of the integer types supported for the OR operation as specified by Table \ref{reducetypes}.
}

%% C/C++
\begin{Csynopsis}
\end{Csynopsis}
{\color{Green}
\begin{CsynopsisCol}
int @\FuncDecl{shmem\_team\_\FuncParam{TYPENAME}\_or\_to\_all}@(TYPE *dest, const TYPE *source, int nreduce, shmem_team_t team);
\end{CsynopsisCol}
}
\begin{DeprecateBlock}
\begin{CsynopsisCol}
void @\FuncDecl{shmem\_\FuncParam{TYPENAME}\_or\_to\_all}@(TYPE *dest, const TYPE *source, int nreduce, int PE_start, int logPE_stride, int PE_size, short *pWrk, long *pSync);
\end{CsynopsisCol}
\end{DeprecateBlock}
\newtext{where \TYPE{} is one of the integer types supported for the OR operation and has a corresponding \TYPENAME{} as specified by Table \ref{reducetypes}.}

%% Fortran
\begin{Fsynopsis}
CALL @\FuncDecl{SHMEM\_INT4\_OR\_TO\_ALL}@(dest, source, nreduce, PE_start, logPE_stride, PE_size, pWrk, pSync)
CALL @\FuncDecl{SHMEM\_INT8\_OR\_TO\_ALL}@(dest, source, nreduce, PE_start, logPE_stride, PE_size, pWrk, pSync)
\end{Fsynopsis}
%%

\paragraph{XOR}
Performs a bitwise exclusive OR (XOR) reduction across a set of \acp{PE}.\newline

%% C11
{\color{Green}
\begin{C11synopsis}
int @\FuncDecl{shmem\_xor\_to\_all}@(TYPE *dest, const TYPE *source, int nreduce, shmem_team_t team);
\end{C11synopsis}
where \TYPE{} is one of the integer types supported for the XOR operation as specified by Table \ref{reducetypes}.
}

%% C/C++
\begin{Csynopsis}
\end{Csynopsis}
{\color{Green}
\begin{CsynopsisCol}
int @\FuncDecl{shmem\_team\_\FuncParam{TYPENAME}\_xor\_to\_all}@(TYPE *dest, const TYPE *source, int nreduce, shmem_team_t team);
\end{CsynopsisCol}
}
\begin{DeprecateBlock}
\begin{CsynopsisCol}
void @\FuncDecl{shmem\_\FuncParam{TYPENAME}\_xor\_to\_all}@(TYPE *dest, const TYPE *source, int nreduce, int PE_start, int logPE_stride, int PE_size, short *pWrk, long *pSync);
\end{CsynopsisCol}
\end{DeprecateBlock}
\newtext{where \TYPE{} is one of the integer types supported for the XOR operation and has a corresponding \TYPENAME{} as specified by Table \ref{reducetypes}.}

%% Fortran
\begin{Fsynopsis}
CALL @\FuncDecl{SHMEM\_INT4\_XOR\_TO\_ALL}@(dest, source, nreduce, PE_start, logPE_stride, PE_size, pWrk, pSync)
CALL @\FuncDecl{SHMEM\_INT8\_XOR\_TO\_ALL}@(dest, source, nreduce, PE_start, logPE_stride, PE_size, pWrk, pSync)
\end{Fsynopsis}
%%

\paragraph{MAX}
Performs a maximum-value reduction across a set of \acp{PE}.\newline

%% C11
{\color{Green}
\begin{C11synopsis}
int @\FuncDecl{shmem\_max\_to\_all}@(TYPE *dest, const TYPE *source, int nreduce, shmem_team_t team);
\end{C11synopsis}
where \TYPE{} is one of the integer or real types supported for the MAX operation as specified by Table \ref{reducetypes}.
}

%% C/C++
\begin{Csynopsis}
\end{Csynopsis}
{\color{Green}
\begin{CsynopsisCol}
int @\FuncDecl{shmem\_team\_\FuncParam{TYPENAME}\_max\_to\_all}@(TYPE *dest, const TYPE *source, int nreduce, shmem_team_t team);
\end{CsynopsisCol}
}
\begin{DeprecateBlock}
\begin{CsynopsisCol}
void @\FuncDecl{shmem\_\FuncParam{TYPENAME}\_max\_to\_all}@(TYPE *dest, const TYPE *source, int nreduce, int PE_start, int logPE_stride, int PE_size, short *pWrk, long *pSync);
\end{CsynopsisCol}
\end{DeprecateBlock}
\newtext{where \TYPE{} is one of the integer or real types supported for the MAX operation and has a corresponding \TYPENAME{} as specified by Table \ref{reducetypes}.}

%% Fortran
\begin{Fsynopsis}
CALL @\FuncDecl{SHMEM\_INT4\_MAX\_TO\_ALL}@(dest, source, nreduce, PE_start, logPE_stride, PE_size, pWrk, pSync)
CALL @\FuncDecl{SHMEM\_INT8\_MAX\_TO\_ALL}@(dest, source, nreduce, PE_start, logPE_stride, PE_size, pWrk, pSync)
CALL @\FuncDecl{SHMEM\_REAL4\_MAX\_TO\_ALL}@(dest, source, nreduce, PE_start, logPE_stride, PE_size, pWrk, pSync)
CALL @\FuncDecl{SHMEM\_REAL8\_MAX\_TO\_ALL}@(dest, source, nreduce, PE_start, logPE_stride, PE_size, pWrk, pSync)
CALL @\FuncDecl{SHMEM\_REAL16\_MAX\_TO\_ALL}@(dest, source, nreduce, PE_start, logPE_stride, PE_size, pWrk, pSync)
\end{Fsynopsis}

\paragraph{MIN}
Performs a minimum-value reduction across a set of \acp{PE}.\newline

%% C11
{\color{Green}
\begin{C11synopsis}
int @\FuncDecl{shmem\_min\_to\_all}@(TYPE *dest, const TYPE *source, int nreduce, shmem_team_t team);
\end{C11synopsis}
where \TYPE{} is one of the integer or real types supported for the MIN operation as specified by Table \ref{reducetypes}.
}

%% C/C++
\begin{Csynopsis}
\end{Csynopsis}
{\color{Green}
\begin{CsynopsisCol}
int @\FuncDecl{shmem\_team\_\FuncParam{TYPENAME}\_min\_to\_all}@(TYPE *dest, const TYPE *source, int nreduce, shmem_team_t team);
\end{CsynopsisCol}
}
\begin{DeprecateBlock}
\begin{CsynopsisCol}
void @\FuncDecl{shmem\_\FuncParam{TYPENAME}\_min\_to\_all}@(TYPE *dest, const TYPE *source, int nreduce, int PE_start, int logPE_stride, int PE_size, short *pWrk, long *pSync);
\end{CsynopsisCol}
\end{DeprecateBlock}
\newtext{where \TYPE{} is one of the integer or real types supported for the MIN operation and has a corresponding \TYPENAME{} as specified by Table \ref{reducetypes}.}

%% Fortran
\begin{Fsynopsis}
CALL @\FuncDecl{SHMEM\_INT4\_MIN\_TO\_ALL}@(dest, source, nreduce, PE_start, logPE_stride, PE_size, pWrk, pSync)
CALL @\FuncDecl{SHMEM\_INT8\_MIN\_TO\_ALL}@(dest, source, nreduce, PE_start, logPE_stride, PE_size, pWrk, pSync)
CALL @\FuncDecl{SHMEM\_REAL4\_MIN\_TO\_ALL}@(dest, source, nreduce, PE_start, logPE_stride, PE_size, pWrk, pSync)
CALL @\FuncDecl{SHMEM\_REAL8\_MIN\_TO\_ALL}@(dest, source, nreduce, PE_start, logPE_stride, PE_size, pWrk, pSync)
CALL @\FuncDecl{SHMEM\_REAL16\_MIN\_TO\_ALL}@(dest, source, nreduce, PE_start, logPE_stride, PE_size, pWrk, pSync)
\end{Fsynopsis}

\paragraph{SUM}
Performs a sum reduction across a set of \acp{PE}.\newline

%% C11
{\color{Green}
\begin{C11synopsis}
int @\FuncDecl{shmem\_sum\_to\_all}@(TYPE *dest, const TYPE *source, int nreduce, shmem_team_t team);
\end{C11synopsis}
where \TYPE{} is one of the integer, real, or complex types supported for the SUM operation as specified by Table \ref{reducetypes}.
}

%% C/C++
\begin{Csynopsis}
\end{Csynopsis}
{\color{Green}
\begin{CsynopsisCol}
int @\FuncDecl{shmem\_team\_\FuncParam{TYPENAME}\_sum\_to\_all}@(TYPE *dest, const TYPE *source, int nreduce, shmem_team_t team);
\end{CsynopsisCol}
}
\begin{DeprecateBlock}
\begin{CsynopsisCol}
void @\FuncDecl{shmem\_\FuncParam{TYPENAME}\_sum\_to\_all}@(TYPE *dest, const TYPE *source, int nreduce, int PE_start, int logPE_stride, int PE_size, short *pWrk, long *pSync);
\end{CsynopsisCol}
\end{DeprecateBlock}
\newtext{where \TYPE{} is one of the integer, real, or complex types supported for the SUM operation and has a corresponding \TYPENAME{} as specified by Table \ref{reducetypes}.}

%% Fortran
\begin{Fsynopsis}
CALL @\FuncDecl{SHMEM\_COMP4\_SUM\_TO\_ALL}@(dest, source, nreduce, PE_start, logPE_stride, PE_size, pWrk, pSync)
CALL @\FuncDecl{SHMEM\_COMP8\_SUM\_TO\_ALL}@(dest, source, nreduce, PE_start, logPE_stride, PE_size, pWrk, pSync)
CALL @\FuncDecl{SHMEM\_INT4\_SUM\_TO\_ALL}@(dest, source, nreduce, PE_start, logPE_stride, PE_size, pWrk, pSync)
CALL @\FuncDecl{SHMEM\_INT8\_SUM\_TO\_ALL}@(dest, source, nreduce, PE_start, logPE_stride, PE_size, pWrk, pSync)
CALL @\FuncDecl{SHMEM\_REAL4\_SUM\_TO\_ALL}@(dest, source, nreduce, PE_start, logPE_stride, PE_size, pWrk, pSync)
CALL @\FuncDecl{SHMEM\_REAL8\_SUM\_TO\_ALL}@(dest, source, nreduce, PE_start, logPE_stride, PE_size, pWrk, pSync)
CALL @\FuncDecl{SHMEM\_REAL16\_SUM\_TO\_ALL}@(dest, source, nreduce, PE_start, logPE_stride, PE_size, pWrk, pSync)
\end{Fsynopsis}

\paragraph{PROD}
Performs a product reduction across a set of \acp{PE}.\newline

%% C11
{\color{Green}
\begin{C11synopsis}
int @\FuncDecl{shmem\_prod\_to\_all}@(TYPE *dest, const TYPE *source, int nreduce, shmem_team_t team);
\end{C11synopsis}
where \TYPE{} is one of the integer, real, or complex types supported for the PROD operation as specified by Table \ref{reducetypes}.
}

%% C/C++
\begin{Csynopsis}
\end{Csynopsis}
{\color{Green}
\begin{CsynopsisCol}
int @\FuncDecl{shmem\_team\_\FuncParam{TYPENAME}\_prod\_to\_all}@(TYPE *dest, const TYPE *source, int nreduce, shmem_team_t team);
\end{CsynopsisCol}
}
\begin{DeprecateBlock}
\begin{CsynopsisCol}
void @\FuncDecl{shmem\_\FuncParam{TYPENAME}\_prod\_to\_all}@(TYPE *dest, const TYPE *source, int nreduce, int PE_start, int logPE_stride, int PE_size, short *pWrk, long *pSync);
\end{CsynopsisCol}
\end{DeprecateBlock}
\newtext{where \TYPE{} is one of the integer, real, or complex types supported for the PROD operation and has a corresponding \TYPENAME{} as specified by Table \ref{reducetypes}.}

%% Fortran
\begin{Fsynopsis}
CALL @\FuncDecl{SHMEM\_COMP4\_PROD\_TO\_ALL}@(dest, source, nreduce, PE_start, logPE_stride, PE_size, pWrk, pSync)
CALL @\FuncDecl{SHMEM\_COMP8\_PROD\_TO\_ALL}@(dest, source, nreduce, PE_start, logPE_stride, PE_size, pWrk, pSync)
CALL @\FuncDecl{SHMEM\_INT4\_PROD\_TO\_ALL}@(dest, source, nreduce, PE_start, logPE_stride, PE_size, pWrk, pSync)
CALL @\FuncDecl{SHMEM\_INT8\_PROD\_TO\_ALL}@(dest, source, nreduce, PE_start, logPE_stride, PE_size, pWrk, pSync)
CALL @\FuncDecl{SHMEM\_REAL4\_PROD\_TO\_ALL}@(dest, source, nreduce, PE_start, logPE_stride, PE_size, pWrk, pSync)
CALL @\FuncDecl{SHMEM\_REAL8\_PROD\_TO\_ALL}@(dest, source, nreduce, PE_start, logPE_stride, PE_size, pWrk, pSync)
CALL @\FuncDecl{SHMEM\_REAL16\_PROD\_TO\_ALL}@(dest, source, nreduce, PE_start, logPE_stride, PE_size, pWrk, pSync)
\end{Fsynopsis}



\begin{apiarguments}

\apiargument{OUT}{dest}{A symmetric array, of length \VAR{nreduce} elements, to
    receive the result of the reduction routines.  The data type of \dest{} varies
    with the version of the reduction routine being called.  When calling from
    \CorCpp, refer to the SYNOPSIS section for data type information.}
\apiargument{IN}{source}{ A symmetric array, of length \VAR{nreduce} elements, that
    contains one element for each separate reduction routine.  The \source{}
    argument must have the same data type as \dest.}
\apiargument{IN}{nreduce}{The number of elements in the \dest{} and \source{}
    arrays.  \VAR{nreduce} must be of type integer.  When using \Fortran, it
    must be a default integer value.}

\newtext{%
\apiargument{IN}{team}{The team over which to perform the operation.}%
}

\begin{DeprecateBlock}
\apiargument{IN}{PE\_start}{The lowest \ac{PE} number of the active set of
    \acp{PE}.  \VAR{PE\_start} must be of type integer.  When using \Fortran,
    it must be a default integer value.}
\apiargument{IN}{logPE\_stride}{The log (base 2) of the stride between consecutive
    \ac{PE} numbers in the active set.  \VAR{logPE\_stride} must be of type integer.
    When using \Fortran, it must be a default integer value.}
\apiargument{IN}{PE\_size}{The number of \acp{PE} in the active set.
    \VAR{PE\_size} must be of type integer.  When using \Fortran, it must be a
    default integer value.}
\apiargument{IN}{pWrk}{
    A symmetric work array of size at least
    max(\VAR{nreduce}/2 + 1, \CONST{SHMEM\_REDUCE\_MIN\_WRKDATA\_SIZE})
    elements.}
\apiargument{IN}{pSync}{
    A symmetric work array of size \CONST{SHMEM\_REDUCE\_SYNC\_SIZE}.
    In \CorCpp, \VAR{pSync} must be an array of elements of type \CTYPE{long}.
    In \Fortran, \VAR{pSync} must be an array of elements of default integer type.
    Every element of this array must be initialized with the value
    \CONST{SHMEM\_SYNC\_VALUE} before any of the \acp{PE} in the active set
    enter the reduction routine.}
\end{DeprecateBlock}

\end{apiarguments}

\apidescription{
    \openshmem reduction routines \newtext{are collective routines over an active set or
    existing \openshmem team that} compute one or more reductions across symmetric
    arrays on multiple \acp{PE}.  A reduction performs an associative binary routine
    across a set of values.

    The \VAR{nreduce} argument determines the number of separate reductions to
    perform.  The \source{} array on all \acp{PE} \newtext{participating in the reduction}
    \oldtext{in the active set} %%
    provides one element for each reduction.  The results of the reductions are placed in the
    \dest{} array on all \acp{PE} \newtext{participating in the reduction.}
    \oldtext{in the active set.} %%
    
    The \source{} and \dest{} arrays may be the same array, but they may not be
    overlapping arrays. The same \dest{} and \source{} arrays
    must be passed to all \acp{PE} \newtext{participating in the reduction.}
    \oldtext{in the active set.} %%

{\color{Green}
    Team-based reduction routines operate over all \acp{PE} in the provided team argument. All
    \acp{PE} in the provided team must participate in the reduction. If an invalid team handle
    or \LibConstRef{SHMEM\_TEAM\_NULL} is passed to this routine, the behavior is undefined.

    Active-set-based sync routines operate over all \acp{PE} in the active set
    defined by the \VAR{PE\_start}, \VAR{logPE\_stride}, \VAR{PE\_size} triplet.
}

    As with all \oldtext{\openshmem} \newtext{active set-based} collective routines,
    each of these routines assumes
    that only \acp{PE} in the active set call the routine.  If a \ac{PE} not in
    the active set calls an \oldtext{\openshmem} \newtext{active set-based} collective routine,
    the behavior is undefined.

    The values of arguments \VAR{nreduce}, \VAR{PE\_start}, \VAR{logPE\_stride},
    and \VAR{PE\_size} must be equal on all \acp{PE} in the active set.
    The same \VAR{pWrk} and \VAR{pSync} work arrays must be passed to all
    \acp{PE} in the active set.

    Before any \ac{PE} calls a reduction routine, the following conditions must be ensured:
    \begin{itemize}
    \item The \dest{} array on all \acp{PE} \newtext{participating in the reduction}
      \oldtext{in the active set} %%
      is ready to accept the results of the \OPR{reduction}.
    \item \newtext{If using active-set-based routines,} the
      \VAR{pWrk} and \VAR{pSync} arrays on all \acp{PE} in the
      active set are not still in use from a prior call to a collective
      \openshmem routine.
    \end{itemize}
    Otherwise, the behavior is undefined.
    
    Upon return from a reduction routine, the following are true for the local
    \ac{PE}:
    \begin{itemize}
    \item The \dest{} array is updated and the \source{} array may be safely reused.
    \item \newtext{If using active-set-based routines,}
    the values in the \VAR{pSync} array are restored to the original values.
    \end{itemize}

    The complex-typed interfaces are only provided for sum and product reductions.
    When the \Cstd translation environment does not support complex types
    \footnote{That is, under \Cstd language standards prior to \Cstd[99] or under \Cstd[11]
    when \CONST{\_\_STDC\_NO\_COMPLEX\_\_} is defined to 1}, an \openshmem
    implementation is not required to provide support for these
    complex-typed interfaces.
}



%\deprecationstart
\apidesctable{
    When calling from \Fortran, the \dest{} date types are as follows:
}{Routine}{Data type}
    \apitablerow{shmem\_int8\_and\_to\_all}{Integer, with an element size of 8 bytes.}
    \apitablerow{shmem\_int4\_and\_to\_all}{Integer, with an element size of 4 bytes.}
    \apitablerow{shmem\_comp8\_max\_to\_all}{Complex, with an element size equal to two 8-byte real values.}
    \apitablerow{shmem\_int4\_max\_to\_all}{Integer, with an element size of 4 bytes.}
    \apitablerow{shmem\_int8\_max\_to\_all}{Integer, with an element size of 8 bytes.}
    \apitablerow{shmem\_real4\_max\_to\_all}{Real, with an element size of 4 bytes.}
    \apitablerow{shmem\_real16\_max\_to\_all}{Real, with an element size of 16 bytes.}
    \apitablerow{shmem\_int4\_min\_to\_all}{Integer, with an element size of 4 bytes.}
    \apitablerow{shmem\_int8\_min\_to\_all}{Integer, with an element size of 8 bytes.}
    \apitablerow{shmem\_real4\_min\_to\_all}{Real, with an element size of 4 bytes.}
    \apitablerow{shmem\_real8\_min\_to\_all}{Real, with an element size of 8 bytes.}
    \apitablerow{shmem\_real16\_min\_to\_all}{Real,with an element size of 16 bytes.}
    \apitablerow{shmem\_comp4\_sum\_to\_all}{Complex, with an element size equal to two 4-byte real values.}
    \apitablerow{shmem\_comp8\_sum\_to\_all}{Complex, with an element size equal to two 8-byte real values.}
    \apitablerow{shmem\_int4\_sum\_to\_all}{Integer, with an element size of 4 bytes.}
    \apitablerow{shmem\_int8\_sum\_to\_all}{Integer, with an element size of 8 bytes..}
    \apitablerow{shmem\_real4\_sum\_to\_all}{Real, with an element size of 4 bytes.}
    \apitablerow{shmem\_real8\_sum\_to\_all}{Real, with an element size of 8 bytes.}
    \apitablerow{shmem\_real16\_sum\_to\_all}{Real, with an element size of 16 bytes.}
    \apitablerow{shmem\_comp4\_prod\_to\_all}{Complex, with an element size equal to two 4-byte real values.}
    \apitablerow{shmem\_comp8\_prod\_to\_all}{Complex, with an element size equal to two 8-byte real values.}
    \apitablerow{shmem\_int4\_prod\_to\_all}{Integer, with an element size of 4 bytes.}
    \apitablerow{shmem\_int8\_prod\_to\_all}{Integer, with an element size of 8 bytes.}
    \apitablerow{shmem\_real4\_prod\_to\_all}{Real, with an element size of 4 bytes.}
    \apitablerow{shmem\_real8\_prod\_to\_all}{Real, with an element size of 8 bytes.}
    \apitablerow{shmem\_real16\_prod\_to\_all}{Real, with an element size of 16 bytes.}
    \apitablerow{shmem\_int8\_or\_to\_all}{Integer, with an element size of 8 bytes.}
    \apitablerow{shmem\_int4\_or\_to\_all}{Integer, with an element size of 4 bytes.}
    \apitablerow{shmem\_int8\_xor\_to\_all}{Integer, with an element size of 8 bytes.}
    \apitablerow{shmem\_int4\_xor\_to\_all}{Integer, with an element size of 4 bytes.}

%\deprecationend


\apireturnvalues{
  \newtext{Zero on successful local completion. Nonzero otherwise.}
}

\apinotes{
\newtext{%
    There are no specifically defined error codes for this routine.
    See section \ref{subsec:error_handling} for expected error checking and
    return code behavior specific to implementations. For portable
    error checking and debugging behavior, programs should do their own checks
    for invalid team handles or \LibConstRef{SHMEM\_TEAM\_NULL}
    }

    All \openshmem reduction routines reset the values in \VAR{pSync} before they
    return, so a particular \VAR{pSync} buffer need only be initialized the first
    time it is used. The user must ensure that the \VAR{pSync} array is not being updated on any \ac{PE}
    in the active set while any of the \acp{PE} participate in processing of an
    \openshmem reduction routine. Be careful to avoid the following situations: If
    the \VAR{pSync} array is initialized at run time, some type of synchronization
    is needed to ensure that all \acp{PE} in the working set have initialized
    \VAR{pSync} before any of them enter an \openshmem routine called with the
    \VAR{pSync} synchronization array. A \VAR{pSync} or \VAR{pWrk} array can be
    reused in a subsequent reduction routine call only if none of the \acp{PE} in
    the active set are still processing a prior reduction routine call that used
    the same \VAR{pSync} or \VAR{pWrk} arrays. In general, this can be assured only
    by doing some type of synchronization.
}

\begin{apiexamples}

\apicexample
    { The following \Cstd[11] example performs a \OPR{SUM} reduction of the
    integer array \VAR{values} across all PEs:}
    {./example_code/shmem_sum_example.c}
    {}

\apifexample
    {This \Fortran reduction example statically initializes the \VAR{pSync} array
    and finds the logical \OPR{AND} of the integer variable \VAR{FOO} across all
    even \acp{PE}.}
    {./example_code/shmem_and_example.f90}
    {}

\apifexample
    {This \Fortran example statically initializes the \VAR{pSync} array and finds
    the \OPR{maximum} value of real variable \VAR{FOO} across all even \acp{PE}.}
    {./example_code/shmem_max_example.f90}
    {}

\apifexample
    { This \Fortran example statically initializes the \VAR{pSync} array and finds
    the \OPR{minimum} value of real variable \VAR{FOO} across all the even
    \acp{PE}.}
    {./example_code/shmem_min_example.f90}
    {}

\apifexample
    {This \Fortran example statically initializes the \VAR{pSync} array and finds
    the \OPR{sum} of the real variable \VAR{FOO} across all even \acp{PE}.}
    {./example_code/shmem_sum_example.f90}
    {}

\apifexample
    {This \Fortran example statically initializes the \VAR{pSync} array and finds
    the \OPR{product} of the real variable \VAR{FOO} across all the even \acp{PE}.}
    {./example_code/shmem_prod_example.f90}
    {}

\apifexample
    {This \Fortran example statically initializes the \VAR{pSync} array and finds
    the logical \OPR{OR} of the integer variable \VAR{FOO} across all even
    \acp{PE}.}
    {./example_code/shmem_or_example.f90}
    {}

\apifexample
    {This \Fortran example statically initializes the \VAR{pSync} array and
    computes the exclusive \OPR{XOR} of variable \VAR{FOO} across all even
    \acp{PE}.}
    {./example_code/shmem_xor_example.f90}
    {}

\end{apiexamples}

\end{apidefinition}


\subsubsection{\textbf{SHMEM\_ALLTOALL}}\label{subsec:shmem_alltoall}
\apisummary{
  Exchanges a fixed amount of contiguous data blocks between all pairs
  of \acp{PE} participating in the collective routine.
}

\begin{apidefinition}

%% C11
\begin{C11synopsis}
int @\FuncDecl{shmem\_alltoall}@(shmem_team_t team, TYPE *dest, const TYPE *source, size_t nelems);
\end{C11synopsis}
where \TYPE{} is one of the standard \ac{RMA} types specified by Table \ref{stdrmatypes}.

\begin{Csynopsis}
\end{Csynopsis}
\begin{CsynopsisCol}
int @\FuncDecl{shmem\_\FuncParam{TYPENAME}\_alltoall}@(shmem_team_t team, TYPE *dest, const TYPE *source, size_t nelems);
\end{CsynopsisCol}
where \TYPE{} is one of the standard \ac{RMA} types and has a corresponding \TYPENAME{} specified by Table \ref{stdrmatypes}.

\begin{CsynopsisCol}
int @\FuncDecl{shmem\_alltoallmem}@(shmem_team_t team, void *dest, const void *source, size_t nelems);
\end{CsynopsisCol}

\begin{DeprecateBlock}
\begin{CsynopsisCol}
void @\FuncDecl{shmem\_alltoall32}@(void *dest, const void *source, size_t nelems, int PE_start, int logPE_stride, int PE_size, long *pSync);
void @\FuncDecl{shmem\_alltoall64}@(void *dest, const void *source, size_t nelems, int PE_start, int logPE_stride, int PE_size, long *pSync);
\end{CsynopsisCol}
\end{DeprecateBlock}

\begin{apiarguments}

\apiargument{IN}{team}{A valid \openshmem team handle to a team.}%

\apiargument{OUT}{dest}{Symmetric address of a data object large enough to receive
    the combined total of \VAR{nelems} elements from each \ac{PE} in the
    participating \acp{PE}.
    The type of \dest{} should match that implied in the SYNOPSIS section.}
\apiargument{IN}{source}{Symmetric address of a data object that contains \VAR{nelems}
    elements of data for each \ac{PE} in the participating \acp{PE}, ordered according to
    destination \ac{PE}.
    The type of \source{} should match that implied in the SYNOPSIS section.}
\apiargument{IN}{nelems}{
    The number of elements to exchange for each \ac{PE}.
    For \FUNC{shmem\_alltoallmem}, elements are bytes;
    for \FUNC{shmem\_alltoall\{32,64\}}, elements are 4 or 8 bytes,
    respectively.
}

\begin{DeprecateBlock}
\apiargument{IN}{PE\_start}{The lowest \ac{PE} number of the active set of
    \acp{PE}.}
\apiargument{IN}{logPE\_stride}{The log (base 2) of the stride between
    consecutive \ac{PE} numbers in the active set.}
    \apiargument{IN}{PE\_size}{The number of \acp{PE} in the active set.}
\apiargument{IN}{pSync}{
    Symmetric address of a work array of size at least \CONST{SHMEM\_ALLTOALL\_SYNC\_SIZE}.}
\end{DeprecateBlock}

\end{apiarguments}

\apidescription{
    The \FUNC{shmem\_alltoall} routines are collective routines. Each \ac{PE}
    participating in the operation exchanges \VAR{nelems} data elements
    with all other \acp{PE} participating in the operation.
    The size of a data element is:
    \begin{itemize}
    \item 32 bits for \FUNC{shmem\_alltoall32}
    \item 64 bits for \FUNC{shmem\_alltoall64}
    \item 8 bits for \FUNC{shmem\_alltoallmem}
    \item \FUNC{sizeof}(\TYPE{}) for alltoall routines taking typed \VAR{source} and \VAR{dest}
    \end{itemize}

    The data being sent and received are
    stored in a contiguous symmetric data object. The total size of each \ac{PE}'s
    \VAR{source} object and \VAR{dest} object is \VAR{nelems} times the size of
    an element
    times \VAR{N}, where \VAR{N} equals the number of \acp{PE} participating
    in the operation.
    The \VAR{source} object contains \VAR{N} blocks of data
    (where the size of each block is defined by \VAR{nelems}) and each block of data
    is sent to a different \ac{PE}.

    The same \dest{} and \source{}
    arrays, and same value for nelems
    must be passed by all \acp{PE} that participate in the collective.

    Given a \ac{PE} \VAR{i} that is the \kth \ac{PE}
    participating in the operation and a \ac{PE}
    \VAR{j} that is the \lth \ac{PE}
    participating in the operation, \ac{PE} \VAR{i} sends the \lth block of its \VAR{source} object to
    the \kth block of
    the \VAR{dest} object of \ac{PE} \VAR{j}.

    Team-based collect routines operate over all \acp{PE} in the provided team
    argument. All \acp{PE} in the provided team must participate in the collective.
    If \VAR{team} compares equal to \LibConstRef{SHMEM\_TEAM\_INVALID} or is
    otherwise invalid, the behavior is undefined.

    Before any \ac{PE} calls a \FUNC{shmem\_alltoall} routine, the following
    conditions must be ensured, otherwise the behavior is undefined:
    \begin{itemize}
        \item The \dest{} array on all \acp{PE} in the team is ready to
            accept the result of the operation.
        \item The \source{} buffer at the local \ac{PE} is ready to be
            read by any \ac{PE} in the team.
    \end{itemize}
    The application does not need to synchronize to ensure that the \source{}
    buffer is ready across all \acp{PE} prior to calling this routine.

    Upon return from a \FUNC{shmem\_alltoall} routine, the following is true for
    the local PE:
    \begin{itemize}
        \item Its \VAR{dest} symmetric data object is completely updated and the 
	data has been copied out of the source data object.
    \end{itemize}

\begin{DeprecateBlock}
    Active-set-based collective routines operate over all \acp{PE} in the active set
    defined by the \VAR{PE\_start}, \VAR{logPE\_stride}, \VAR{PE\_size} triplet.

    As with all active-set-based collective routines,
    this routine assumes that only \acp{PE} in the active set call the routine.
    If a \ac{PE} not in the active set calls an
    active-set-based collective routine,
    the behavior is undefined.

    The values of arguments \VAR{PE\_start}, \VAR{logPE\_stride},
    and \VAR{PE\_size} must be equal on all \acp{PE} in the active set. The same
    \VAR{pSync} work
    array must be passed to all \acp{PE} in the active set.

    Before any \ac{PE} calls a \FUNC{shmem\_alltoall} routine,
    the following conditions must be ensured:

    \begin{itemize}
    	\item The \VAR{dest} data object on all \acp{PE} in the active set is
    	ready to accept the \FUNC{shmem\_alltoall} data.
    	\item For active-set-based routines, the \VAR{pSync} array
    	on all \acp{PE} in the active set is not still in use from a prior call
    	to a \FUNC{shmem\_alltoall} routine.
    \end{itemize}

    Otherwise, the behavior is undefined.

    Upon return from a \FUNC{shmem\_alltoall} routine, the following is true for
    the local PE:
    \begin{itemize}
    	\item Its \VAR{dest} symmetric data object is completely updated and the 
	data has been copied out of the source data object.
    	\item For active-set-based routines,
    	the values in the \VAR{pSync} array are restored to the original values.
    \end{itemize}
\end{DeprecateBlock}
}

\apireturnvalues{
    Zero on successful local completion. Nonzero otherwise.
}

\begin{apiexamples}

\apicexample
    {This \CorCpp{} example shows a \FUNC{shmem\_int64\_alltoall} on two 64-bit integers among all
    \acp{PE}.}
    {./example_code/shmem_alltoall_example.c}
    {}

\end{apiexamples}

\end{apidefinition}



\subsubsection{\textbf{SHMEM\_ALLTOALLS}}\label{subsec:shmem_alltoalls}
\apisummary{
    shmem\_alltoalls is a collective routine where each \ac{PE} exchanges a fixed amount of strided data with all other \acp{PE} \oldtext{in the active set} \newtext{participating in the collective}.
}

\begin{apidefinition}

%% C11
{\color{Green}
\begin{C11synopsis}
int @\FuncDecl{shmem\_alltoalls}@(TYPE *dest, const TYPE *source, ptrdiff_t dst, ptrdiff_t sst, size_t nelems, shmem_team_t team);
\end{C11synopsis}
where \TYPE{} is one of the standard \ac{RMA} types specified by Table
\ref{stdrmatypes}.
}

{\color{Green}
\begin{Csynopsis}
int @\FuncDecl{shmem\_team\_\FuncParam{TYPENAME}\_alltoalls}@(TYPE *dest, const TYPE *source, ptrdiff_t dst, ptrdiff_t sst, size_t nelems, shmem_team_t team);
\end{Csynopsis}
where \TYPE{} is one of the standard \ac{RMA} types and has a corresponding
\TYPENAME{} specified by Table \ref{stdrmatypes}.
}

\begin{DeprecateBlock}
\begin{CsynopsisCol}
void @\FuncDecl{shmem\_alltoalls32}@(void *dest, const void *source, ptrdiff_t dst, ptrdiff_t sst, size_t nelems, int PE_start, int logPE_stride, int PE_size, long *pSync);
void @\FuncDecl{shmem\_alltoalls64}@(void *dest, const void *source, ptrdiff_t dst, ptrdiff_t sst, size_t nelems, int PE_start, int logPE_stride, int PE_size, long *pSync);
\end{CsynopsisCol}
\end{DeprecateBlock}

\begin{Fsynopsis}
INTEGER pSync(SHMEM_ALLTOALLS_SYNC_SIZE)
INTEGER dst, sst, PE_start, logPE_stride, PE_size
INTEGER nelems 
CALL @\FuncDecl{SHMEM\_ALLTOALLS32}@(dest, source, dst, sst, nelems, PE_start, logPE_stride, PE_size, pSync)
CALL @\FuncDecl{SHMEM\_ALLTOALLS64}@(dest, source, dst, sst, nelems, PE_start, logPE_stride, PE_size, pSync)
\end{Fsynopsis}

\begin{apiarguments}

\apiargument{OUT}{dest}{A symmetric data object large enough to receive 
    the combined total of \VAR{nelems} elements from each \ac{PE} in the
    active set.}
\apiargument{IN}{source}{A symmetric data object that contains \VAR{nelems} 
    elements of data for each \ac{PE} in the active set, ordered according to 
    destination \ac{PE}.}
\apiargument{IN}{dst}{The stride between consecutive elements of the \dest{}
    data object.  The stride is scaled by the element size.  A
    value of \CONST{1} indicates contiguous data.  \VAR{dst} must be of type
    \CTYPE{ptrdiff\_t}.  When using \Fortran, it must be a default integer
    value.}
\apiargument{IN}{sst}{The  stride between consecutive elements of the
    \source{} data object.  The stride is scaled by the element size.
    A value of \CONST{1} indicates contiguous data.  \VAR{sst} must be
    of type \CTYPE{ptrdiff\_t}.  When using \Fortran, it must be a
    default integer value.}
\apiargument{IN}{nelems}{The number of elements to exchange for each \ac{PE}.
    \VAR{nelems} must be of type size\_t for \CorCpp.  When using
    \Fortran, it must be a default integer value.}
\newtext{%
\apiargument{IN}{team}{A valid \openshmem team handle.}
}

\begin{DeprecateBlock}
\apiargument{IN}{PE\_start}{The lowest \ac{PE} number of the active set of
    \acp{PE}.  \VAR{PE\_start} must be of type integer.  When using \Fortran,
    it must be a default integer value.}
\apiargument{IN}{logPE\_stride}{The log (base 2) of the stride between
    consecutive \ac{PE} numbers in the active set.  \VAR{logPE\_stride} must be of
    type integer.  When using \Fortran, it must be a default integer value.}
\apiargument{IN}{PE\_size}{The number of \acp{PE} in the active set.
    \VAR{PE\_size} must be of type integer.  When using \Fortran, it must
    be a default integer value.}
\apiargument{IN}{pSync}{
    A symmetric work array of size \CONST{SHMEM\_ALLTOALLS\_SYNC\_SIZE}.
    In \CorCpp, \VAR{pSync} must be an array of elements of type \CTYPE{long}.
    In \Fortran, \VAR{pSync} must be an array of elements of default integer type.
    Every element of this array must be initialized with the value
    \CONST{SHMEM\_SYNC\_VALUE} before any of the \acp{PE} in the active set
    enter the routine.}
\end{DeprecateBlock}

\end{apiarguments}

\apidescription{
    The \FUNC{shmem\_alltoalls} routines are collective routines.
    \newtext{These routines are equivalent in functionality to the corresponding
    \FUNC{shmem\_alltoall} routines except that they add explicit stride values
    for accessing the source and destination data arrays, whereas the array
    access in \FUNC{shmem\_alltoall} is always with a stride of \CONST{1}.}

    Each \ac{PE} \oldtext{in the active set} \newtext{participating in the operation}
    exchanges \VAR{nelems} strided data elements \oldtext{of size
    32 bits (for \FUNC{shmem\_alltoalls32}) or 64 bits (for \FUNC{shmem\_alltoalls64})}
    with all other \acp{PE} \oldtext{in the set} \newtext{participating in the operation}.
    Both strides, \VAR{dst} and \VAR{sst}, must be greater
    than or equal to \CONST{1}.

    \newtext{The same \dest{} and \source{} arrays and same values for values of
    arguments \VAR{dst}, \VAR{sst}, \VAR{nelems} must be passed by all \acp{PE}
    that participate in the collective.}
    
    Given a \ac{PE} \VAR{i} that is the \kth \ac{PE} \oldtext{in the active set}
    \newtext{participating in the operation} and a \ac{PE}
    \VAR{j} that is the \lth \ac{PE} \oldtext{in the active set}
    \newtext{participating in the operation}
    \ac{PE} \VAR{i} sends the \VAR{sst}*\lth block of the \VAR{source} data object to
    the \VAR{dst}*\kth block of the \VAR{dest} data object on
    \ac{PE} \VAR{j}.

{\color{Green}
    See the description of \FUNC{shmem\_alltoall} in section
    \ref{subsec:shmem_alltoall} for:
    \begin{itemize}
    \item Rules for \ac{PE} participation in the collective routine.
    \item The pre- and post-conditions for symmetric objects.
    \end{itemize}
}
    
} 


\apireturnvalues{
   \newtext{Zero on successful local completion. Nonzero otherwise.}
}

\apinotes{
    \newtext{See notes for \FUNC{shmem\_alltoall} in section \ref{subsec:shmem_alltoall}}.
}

\begin{apiexamples}

\apicexample
    {This \Cstd[11] example shows a \FUNC{shmem\_alltoalls} on two 64-bit integers among
    all \acp{PE}.}
    {./example_code/shmem_alltoalls_example.c}
    {}

\end{apiexamples}

\end{apidefinition}






\subsection{Point-To-Point Synchronization Routines}\label{subsec:p2p_intro}
The following section discusses \openshmem \acp{API} that provide a mechanism
for synchronization between two \acp{PE} based on the value of a symmetric data
object.
The point-to-point synchronization routines can be used to portably ensure
that memory access operations observe remote updates in the order enforced by
the initiator \ac{PE} using the \OPR{put-with-signal}, \FUNC{shmem\_fence} and
\FUNC{shmem\_quiet} routines.

Where appropriate compiler support is available, \openshmem provides
type-generic point-to-point synchronization interfaces via \Cstd[11] generic
selection. Such type-generic routines are supported for the
``standard \ac{AMO} types'' identified in
Table~\ref{stdamotypes}.

The standard \ac{AMO} types include some of the exact-width
integer types defined in \HEADER{stdint.h} by \Cstd[99]~\S7.18.1.1 and
\Cstd[11]~\S7.20.1.1. When the \Cstd translation environment
does not provide exact-width integer types with \HEADER{stdint.h}, an
\openshmem implementation is not required to provide support for these types.
The \FUNC{shmem\_test\_any} and \FUNC{shmem\_wait\_until\_any} routines
require the \CONST{SIZE\_MAX} macro defined in \HEADER{stdint.h} by
\Cstd[99]~\S7.18.3 and \Cstd[11]~\S7.20.3.

\begin{table}[h]
\begin{DeprecateBlock}
  \begin{center}
    \begin{tabular}{|l|l|}
      \hline
      \TYPE              & \TYPENAME  \\ \hline
      short              & short      \\ \hline
      int                & int        \\ \hline
      long               & long       \\ \hline
      long long          & longlong   \\ \hline
      unsigned short     & ushort     \\ \hline
      unsigned int       & uint       \\ \hline
      unsigned long      & ulong      \\ \hline
      unsigned long long & ulonglong  \\ \hline
      int32\_t           & int32      \\ \hline
      int64\_t           & int64      \\ \hline
      uint32\_t          & uint32     \\ \hline
      uint64\_t          & uint64     \\ \hline
      size\_t            & size       \\ \hline
      ptrdiff\_t         & ptrdiff    \\ \hline
    \end{tabular}
    \TableCaptionRef{Point-to-Point Synchronization Types and Names}
    \label{p2psynctypes}
  \end{center}
\end{DeprecateBlock}
\end{table}

The point-to-point synchronization interface provides named constants whose
values are integer constant expressions that specify the comparison operators
used by \openshmem synchronization routines.
The constant names and associated operations are
presented in Table~\ref{p2p-consts}.

\begin{table}[h]
  \begin{center}
    \begin{tabular}{ll}
      \hline
      Constant Name                 & Comparison               \\ \hline
      \LibConstRef{SHMEM\_CMP\_EQ}  & Equal                    \\
      \LibConstRef{SHMEM\_CMP\_NE}  & Not equal                \\
      \LibConstRef{SHMEM\_CMP\_GT}  & Greater than             \\
      \LibConstRef{SHMEM\_CMP\_GE}  & Greater than or equal to \\
      \LibConstRef{SHMEM\_CMP\_LT}  & Less than                \\
      \LibConstRef{SHMEM\_CMP\_LE}  & Less than or equal to    \\ \hline
    \end{tabular}
    \TableCaptionRef{Point-to-Point Comparison Constants}
    \label{p2p-consts}
  \end{center}
\end{table}


\subsubsection{\textbf{SHMEM\_WAIT\_UNTIL}}\label{subsec:shmem_wait_until}
\input{content/shmem_wait_until.tex}

\subsubsection{\textbf{SHMEM\_WAIT\_UNTIL\_ALL}}\label{subsec:shmem_wait_until_all}
\apisummary{
    Wait on an array of variables on the local \ac{PE} until all variables meet the specified wait condition.
}

\begin{apidefinition}

\begin{C11synopsis}
void @\FuncDecl{shmem\_wait\_until\_all}@(TYPE *ivars, size_t nelems, const int *status, int cmp,
    TYPE cmp_value);
\end{C11synopsis}
where \TYPE{} is one of the point-to-point synchronization types specified by
Table \ref{p2psynctypes}.

\begin{Csynopsis}
void @\FuncDecl{shmem\_\FuncParam{TYPENAME}\_wait\_until\_all}@(TYPE *ivars, size_t nelems, const int *status, int cmp, TYPE cmp_value);
\end{Csynopsis}
where \TYPE{} is one of the point-to-point synchronization types and has a
corresponding \TYPENAME{} specified by Table~\ref{p2psynctypes}.

\begin{apiarguments}

  \apiargument{IN}{ivars}{Local address of an array of remotely accessible data
    objects.
    The type of \VAR{ivars} should match that implied in the SYNOPSIS section.}
  \apiargument{IN}{nelems}{The number of elements in the \VAR{ivars} array.}
  \apiargument{IN}{status}{Local address of an optional mask array of length \VAR{nelems}
    that indicates which elements in \VAR{ivars} are excluded from the wait set.}
  \apiargument{IN}{cmp}{A comparison operator from Table~\ref{p2p-consts}
    that compares elements of \VAR{ivars} with \VAR{cmp\_value}.}
  \apiargument{IN}{cmp\_value}{The value to be compared with the objects
    pointed to by \VAR{ivars}.
    The type of \VAR{cmp\_value} should match that implied in the SYNOPSIS section.}

\end{apiarguments}

\apidescription{ 
    The \FUNC{shmem\_wait\_until\_all} routine waits until all entries in the
    wait set specified by \VAR{ivars} and \VAR{status} have satisfied the wait condition at the
    calling \ac{PE}.  The \VAR{ivars} objects at the calling \ac{PE} may be
    updated by an \ac{AMO} performed by a thread located within the calling
    \ac{PE} or within another \ac{PE}.
    If \VAR{nelems} is 0, the wait set is empty and this routine returns immediately.
    This routine compares each element of the \VAR{ivars} array in the
    wait set with the value \VAR{cmp\_value} according to the comparison
    operator \VAR{cmp} at the calling \ac{PE}.
    This routine is semantically similar to
    \FUNC{shmem\_wait\_until} in Section~\ref{subsec:shmem_wait_until}, but
    adds support for point-to-point synchronization involving an array of
    symmetric data objects.

    The optional \VAR{status} is a mask array of length \VAR{nelems} where each
    element corresponds to the respective element in \VAR{ivars} and indicates
    whether the element is excluded from the wait set.  Elements of
    \VAR{status} set to 0 will be included in the wait set, and elements set to
    1 will be ignored.  If all elements in \VAR{status} are set to 1 or
    \VAR{nelems} is 0, the wait set is empty and this routine returns
    immediately.  If \VAR{status} is a null pointer, it is ignored and
    all elements in \VAR{ivars} are included in the wait set.  The \VAR{ivars}
    and \VAR{status} arrays must not overlap in memory.

    Implementations must ensure that \FUNC{shmem\_wait\_until\_all} does not
    return before the update of the memory indicated by \VAR{ivars} is fully
    complete.
}


\apireturnvalues{
    None.
}


\begin{apiexamples}
  \apicexample
      {The following \Cstd[11] example demonstrates the use of
      \FUNC{shmem\_wait\_until\_all} to implement a simple linear barrier
      synchronization.}
      {./example_code/shmem_wait_until_all.c}
      {}

\end{apiexamples}

\end{apidefinition}


\subsubsection{\textbf{SHMEM\_WAIT\_UNTIL\_ANY}}\label{subsec:shmem_wait_until_any}
\apisummary{
    Wait on an array of variables on the local \ac{PE} until any one variable meets the specified wait condition.
}

\begin{apidefinition}

\begin{C11synopsis}
size_t @\FuncDecl{shmem\_wait\_until\_any}@(TYPE *ivars, size_t nelems, const int *status, int cmp,
    TYPE cmp_value);
\end{C11synopsis}
where \TYPE{} is one of the point-to-point synchronization types specified by
Table \ref{p2psynctypes}.

\begin{Csynopsis}
size_t @\FuncDecl{shmem\_\FuncParam{TYPENAME}\_wait\_until\_any}@(TYPE *ivars, size_t nelems, const int *status,
    int cmp, TYPE cmp_value);
\end{Csynopsis}
where \TYPE{} is one of the point-to-point synchronization types and has a
corresponding \TYPENAME{} specified by Table~\ref{p2psynctypes}.

\begin{apiarguments}

  \apiargument{IN}{ivars}{Local address of an array of remotely accessible data
    objects.
    The type of \VAR{ivars} should match that implied in the SYNOPSIS section.}
  \apiargument{IN}{nelems}{The number of elements in the \VAR{ivars} array.}
  \apiargument{IN}{status}{Local address of an optional mask array of length \VAR{nelems}
    that indicates which elements in \VAR{ivars} are excluded from the wait set.}
  \apiargument{IN}{cmp}{A comparison operator from Table~\ref{p2p-consts}
    that compares elements of \VAR{ivars} with \VAR{cmp\_value}.}
  \apiargument{IN}{cmp\_value}{The value to be compared with the objects
    pointed to by \VAR{ivars}.
    The type of \VAR{cmp\_value} should match that implied in the SYNOPSIS section.}

\end{apiarguments}

\apidescription{ 
    The \FUNC{shmem\_wait\_until\_any} routine waits until any one entry in the
    wait set specified by \VAR{ivars} and \VAR{status} satisfies the wait
    condition at the calling \ac{PE}.  The \VAR{ivars} objects at the calling
    \ac{PE} may be updated by an \ac{AMO} performed by a thread located within
    the calling \ac{PE} or within another \ac{PE}.
    This routine compares each element of the \VAR{ivars} array in the
    wait set with the value \VAR{cmp\_value} according to the comparison
    operator \VAR{cmp} at the calling \ac{PE}.
    The order in which these elements are
    waited upon is unspecified.  If an entry $i$ in \VAR{ivars} within the wait
    set satisfies the wait condition, a series of calls to
    \FUNC{shmem\_wait\_until\_any} must eventually return $i$.

    The optional \VAR{status} is a mask array of length \VAR{nelems} where each
    element corresponds to the respective element in \VAR{ivars} and indicates
    whether the element is excluded from the wait set.  Elements of
    \VAR{status} set to 0 will be included in the wait set, and elements set to
    1 will be ignored.  If all elements in \VAR{status} are set to 1 or
    \VAR{nelems} is 0, the wait set is empty and this routine returns
    \CONST{SIZE\_MAX}.  If
    \VAR{status} is a null pointer, it is ignored and all elements in
    \VAR{ivars} are included in the wait set.  The \VAR{ivars} and \VAR{status}
    arrays must not overlap in memory.

    Implementations must ensure that \FUNC{shmem\_wait\_until\_any} does not
    return before the update of the memory indicated by \VAR{ivars} is fully
    complete.
}

\apireturnvalues{
    \FUNC{shmem\_wait\_until\_any} returns the index of an element in the
    \VAR{ivars} array that satisfies the wait condition. If the wait set is
    empty, this routine returns \CONST{SIZE\_MAX}.
}

\begin{apiexamples}
  \apicexample
      {The following \Cstd[11] example demonstrates the use of
      \FUNC{shmem\_wait\_until\_any} to process a simple all-to-all transfer
      of $N$ data elements via a sum reduction.}
      {./example_code/shmem_wait_until_any_all2all_sum.c}
      {}

\end{apiexamples}

\end{apidefinition}



\subsubsection{\textbf{SHMEM\_WAIT\_UNTIL\_SOME}}\label{subsec:shmem_wait_until_some}
\apisummary{
    Wait on some number of variables on the local \ac{PE} to change to a
    specified value.
}

\begin{apidefinition}

\begin{C11synopsis}
size_t @\FuncDecl{shmem\_wait\_until\_some}@(TYPE *ivars, size_t nelems, _Bool *status,
    int cmp, TYPE cmp_value);
\end{C11synopsis}
where \TYPE{} is one of the point-to-point synchronization types specified by
Table \ref{p2psynctypes}.

\begin{Csynopsis}
size_t @\FuncDecl{shmem\_\FuncParam{TYPENAME}\_wait\_until\_some}@(TYPE *ivars, size_t nelems, _Bool *status,
    int cmp, TYPE cmp_value);
\end{Csynopsis}
where \TYPE{} is one of the point-to-point synchronization types and has a
corresponding \TYPENAME{} specified by Table~\ref{p2psynctypes}.

\begin{apiarguments}

  \apiargument{OUT}{ivars}{A pointer to an array of remotely accessible data
    objects. The type of \VAR{ivars} should match that implied in the SYNOPSIS
    section.} 
  \apiargument{IN}{nelems}{The number of elements in the \VAR{ivars} array to
    be compared with \VAR{cmp\_value}.}
  \apiargument{INOUT}{status}{A mask array of length \VAR{nelems}, where each
    element corresponds to the respective element in \VAR{ivars} and represents
    the status of each implied condition.  On input, elements of \VAR{status}
    set to 0 will be waited upon, and elements set to 1 will be ignored.  On
    output, each element of \VAR{status} equal to 1 corresponds to a satisfied
    condition in the respective element of \VAR{ivars} (or an ignored
    condition, if initialized to 1 on input).}
    \apiargument{IN}{cmp}{A comparison operator from Table~\ref{p2p-consts}
    that compares elements of \VAR{ivars} with \VAR{cmp\_value}.}
  \apiargument{IN}{cmp\_value}{The value to be compared with the objects
    pointed to by \VAR{ivars}.  The type of \VAR{cmp\_value} should match that
    implied in the SYNOPSIS section.}

\end{apiarguments}

\apidescription{ 
    The \FUNC{shmem\_wait\_until\_some} routine waits for some specified
    symmetric object(s) in the \VAR{ivars} array to change to the value
    \VAR{cmp\_value} according to the numeric comparison operator \VAR{cmp}.
    This routine can be used for point-to-point direct synchronization
    involving multiple data objects, potentially avoiding the overhead
    associated with multiple individual calls to \FUNC{shmem\_wait\_until}.

    On input, the \VAR{status} array indicates the respective elements of
    \VAR{ivars} for which to wait until the condition implied by \VAR{cmp} and
    \VAR{cmp\_value} is satisfied.  For each element of \VAR{status}
    initialized to 0, \FUNC{shmem\_wait\_until\_some} sets that value to 1 if
    the respective \VAR{ivars} element has changed to satisfy the implied
    condition. Elements in \VAR{status} initialized to 1 are ignored and not
    checked for the implied condition.  If one or more elements of
    \VAR{status} are intialized to 0, \FUNC{shmem\_wait\_until\_some} does not
    return until a \ac{PE} changes one of the corresponding elements in
    \VAR{ivars} to satisfy the implied condition.  On output, the \VAR{status}
    array indicates which respective elements in \VAR{ivars} have satisfied the
    implied condition or have been ignored.  \FUNC{shmem\_wait\_until\_some}
    returns the total number of the newly satisfied conditions excluding the
    ignored elements.
}


\apireturnvalues{
    \FUNC{shmem\_wait\_until\_some} returns the number of elements in
    \VAR{ivars} that satisfied the condition implied by \VAR{cmp} and
    \VAR{cmp\_value}, excluding the number of ignored elements indicated by the
    initial values in \VAR{status}.
}

\apinotes{
    None.
}

\apiimpnotes{
    Implementations must ensure that \FUNC{shmem\_wait\_until\_some} does not
    return before the update of the memory indicated by \VAR{ivars} is fully
    complete.  Partial updates to the memory must not cause
    \FUNC{shmem\_wait\_until\_some} to return.
}


\begin{apiexamples}

\apicexample
{The following \CorCpp{} example demonstrates the use of 
\FUNC{shmem\_wait\_until\_some} to process a simple all-to-all transfer of N data elements via a sum reduction.}
{./example_code/shmem_wait_until_some_all2all_sum.c}
{}

\end{apiexamples}

\end{apidefinition}


\subsubsection{\textbf{SHMEM\_TEST}}\label{subsec:shmem_test}
\apisummary{
    Indicate whether a variable on the local \ac{PE} meets the specified condition.
}

\begin{apidefinition}

\begin{C11synopsis}
int @\FuncDecl{shmem\_test}@(TYPE *ivar, int cmp, TYPE cmp_value);
\end{C11synopsis}
where \TYPE{} is one of the point-to-point synchronization types specified by
Table \ref{p2psynctypes}.

\begin{Csynopsis}
int @\FuncDecl{shmem\_\FuncParam{TYPENAME}\_test}@(TYPE *ivar, int cmp, TYPE cmp_value);
\end{Csynopsis}
where \TYPE{} is one of the point-to-point synchronization types and has a
corresponding \TYPENAME{} specified by Table \ref{p2psynctypes}.

\begin{apiarguments}

  \apiargument{IN}{ivar}{Local address of a remotely accessible data object.
    The type of \VAR{ivar} should match that implied in the SYNOPSIS section.}
  \apiargument{IN}{cmp}{The comparison operator that compares \VAR{ivar} with
    \VAR{cmp\_value}.}
  \apiargument{IN}{cmp\_value}{The value against which the object pointed to
    by \VAR{ivar} will be compared.
    The type of \VAR{cmp\_value} should match that implied in the SYNOPSIS section.}

\end{apiarguments}

\apidescription{
  \FUNC{shmem\_test} tests the numeric comparison of the symmetric object
  pointed to by \VAR{ivar} with the value \VAR{cmp\_value} according to the
  comparison operator \VAR{cmp}.  The \VAR{ivar} object at the
  calling \ac{PE} may be updated by an \ac{AMO} performed by a thread located
  within the calling \ac{PE} or within another \ac{PE}.

  Implementations must ensure that \FUNC{shmem\_test} does not return 1 before
  the update of the memory indicated by \VAR{ivar} is fully complete.
}

\apireturnvalues{
  \FUNC{shmem\_test} returns 1 if the comparison of the symmetric object
  pointed to by \VAR{ivar} with the value \VAR{cmp\_value} according to the
  comparison operator \VAR{cmp} evaluates to true; otherwise, it returns 0.
}

\begin{apiexamples}
  \apicexample
      {The following example demonstrates the use of \FUNC{shmem\_test} to
        wait on an array of symmetric objects and return the index of an
        element that satisfies the specified condition.}
      {./example_code/shmem_test_example1.c}
      {}
\end{apiexamples}

\end{apidefinition}


\subsubsection{\textbf{SHMEM\_TEST\_ALL}}\label{subsec:shmem_test_all}
\apisummary{
  Test whether all variables within an array of variables on the local \ac{PE} meet a specified test condition.
}

\begin{apidefinition}

\begin{C11synopsis}
int @\FuncDecl{shmem\_test\_all}@(TYPE * ivars, size_t nelems, int cmp, TYPE cmp_value);
\end{C11synopsis}
where \TYPE{} is one of the point-to-point synchronization types specified by
Table \ref{p2psynctypes}.

\begin{Csynopsis}
int @\FuncDecl{shmem\_\FuncParam{TYPENAME}\_test\_all}@(TYPE * ivars, size_t nelems, int cmp, TYPE cmp_value);
\end{Csynopsis}
where \TYPE{} is one of the point-to-point synchronization types and has a
corresponding \TYPENAME{} specified by Table \ref{p2psynctypes}.

\begin{apiarguments}

  \apiargument{OUT}{ivars}{A pointer to an array of remotely accessible data
    objects.}
  \apiargument{IN}{nelems}{The number of elements in the \VAR{ivars} array.}
  \apiargument{IN}{cmp}{A comparison operator from Table~\ref{p2p-consts}
    that compares elements of \VAR{ivars} with \VAR{cmp\_value}.}
  \apiargument{IN}{cmp\_value}{The value to be compared with the objects
    pointed to by \VAR{ivars}.}

\end{apiarguments}

\apidescription{
    The \FUNC{shmem\_test\_all} routine behaves similarly to
    \FUNC{shmem\_wait\_until\_all}, but it does not block and returns zero if
    no conditions are satisfied.  This routine tests the numeric comparison of
    each of the \VAR{nelems} elements in the \VAR{ivars} array with the value
    \VAR{cmp\_value} according to the comparison operator \VAR{cmp} at the
    calling PE.  The order in which these elements are tested is unspecified.
}

\apireturnvalues{
    \FUNC{shmem\_test\_all} returns 1 if all variables in \VAR{ivars} satisfy the test condition, otherwise this routine returns 0.
}

\apinotes{
  None.
}

%\begin{apiexamples}
%  \apicexample
%      {The following \CorCpp{} example demonstrates the use of
%      \FUNC{shmem\_test\_all} to process a simple all-to-all transfer of N
%      data elements via a sum reduction, while potentially overlapping
%      communication with computation.}
%      {./example_code/shmem_test_some_example.c}
%      {}
%\end{apiexamples}

\end{apidefinition}


\subsubsection{\textbf{SHMEM\_TEST\_ANY}}\label{subsec:shmem_test_any}
\apisummary{
  Indicate whether any one variable within an array of variables on the local \ac{PE} meets a specified test condition.
}

\begin{apidefinition}

\begin{C11synopsis}
size_t @\FuncDecl{shmem\_test\_any}@(TYPE *ivars, size_t nelems, const int *status, int cmp,
    TYPE cmp_value);
\end{C11synopsis}
where \TYPE{} is one of the point-to-point synchronization types specified by
Table \ref{p2psynctypes}.

\begin{Csynopsis}
size_t @\FuncDecl{shmem\_\FuncParam{TYPENAME}\_test\_any}@(TYPE *ivars, size_t nelems, const int *status, int cmp,
    TYPE cmp_value);
\end{Csynopsis}
where \TYPE{} is one of the point-to-point synchronization types and has a
corresponding \TYPENAME{} specified by Table \ref{p2psynctypes}.

\begin{apiarguments}

  \apiargument{IN}{ivars}{Local address of an array of remotely accessible data
    objects.
    The type of \VAR{ivars} should match that implied in the SYNOPSIS section.}
  \apiargument{IN}{nelems}{The number of elements in the \VAR{ivars} array.}
  \apiargument{IN}{status}{Local address of an optional mask array of length \VAR{nelems}
    that indicates which elements in \VAR{ivars} are excluded from the test set.}
  \apiargument{IN}{cmp}{A comparison operator from Table~\ref{p2p-consts}
    that compares elements of \VAR{ivars} with \VAR{cmp\_value}.}
  \apiargument{IN}{cmp\_value}{The value to be compared with the objects
    pointed to by \VAR{ivars}.
    The type of \VAR{cmp\_value} should match that implied in the SYNOPSIS section.}

\end{apiarguments}

\apidescription{
    The \FUNC{shmem\_test\_any} routine indicates whether any entry in the
    test set specified by \VAR{ivars} and \VAR{status} has satisfied the test
    condition at the calling \ac{PE}.  The \VAR{ivars} objects at the calling
    \ac{PE} may be updated by an \ac{AMO} performed by a thread located within
    the calling \ac{PE} or within another \ac{PE}.
    This routine does not block and returns \CONST{SIZE\_MAX} if
    no entries in \VAR{ivars} satisfied the test condition.
    This routine compares each element of the \VAR{ivars} array in the
    test set with the value \VAR{cmp\_value} according to the comparison
    operator \VAR{cmp} at the calling \ac{PE}.
    The order in which these elements are tested is
    unspecified.  If an entry $i$ in \VAR{ivars} within the test set satisfies
    the test condition, a series of calls to \FUNC{shmem\_test\_any} must
    eventually return $i$.

    The optional \VAR{status} is a mask array of length \VAR{nelems} where each element
    corresponds to the respective element in \VAR{ivars} and indicates whether
    the element is excluded from the test set.  Elements of
    \VAR{status} set to 0 will be included in the test set, and elements set to 1 will be ignored.  If all
    elements in \VAR{status} are set to 1 or \VAR{nelems} is 0, the test set is
    empty and this routine returns \CONST{SIZE\_MAX}.  If \VAR{status} is a
    null pointer, it is ignored and all
    elements in \VAR{ivars} are included in the test set.  The \VAR{ivars} and
    \VAR{status} arrays must not overlap in memory.

    Implementations must ensure that \FUNC{shmem\_test\_any} does not return an
    index before the update of the memory indicated by the corresponding
    \VAR{ivars} element is fully complete.
}

\apireturnvalues{
    \FUNC{shmem\_test\_any} returns the index of an element in the \VAR{ivars}
    array that satisfies the test condition. If the test set is empty or no
    conditions in the test set are satisfied, this routine returns \CONST{SIZE\_MAX}.
}

\begin{apiexamples}
  \apicexample
      {The following \Cstd[11] example demonstrates the use of
      \FUNC{shmem\_test\_any} to implement a simple linear barrier
      synchronization while potentially overlapping communication with
      computation.}
      {./example_code/shmem_test_any_example.c}
      {}
\end{apiexamples}

\end{apidefinition}


\subsubsection{\textbf{SHMEM\_TEST\_SOME}}\label{subsec:shmem_test_some}
\apisummary{
  Indicate whether at least one variable within an array of variables on the local \ac{PE} meets a specified test condition.
}

\begin{apidefinition}

\begin{C11synopsis}
size_t @\FuncDecl{shmem\_test\_some}@(TYPE *ivars, size_t nelems, size_t *indices, const int *status,
    int cmp, TYPE cmp_value);
\end{C11synopsis}
where \TYPE{} is one of the point-to-point synchronization types specified by
Table \ref{p2psynctypes}.

\begin{Csynopsis}
size_t @\FuncDecl{shmem\_\FuncParam{TYPENAME}\_test\_some}@(TYPE *ivars, size_t nelems, size_t *indices,
    const int *status, int cmp, TYPE cmp_value);
\end{Csynopsis}
where \TYPE{} is one of the point-to-point synchronization types and has a
corresponding \TYPENAME{} specified by Table \ref{p2psynctypes}.

\begin{apiarguments}

  \apiargument{IN}{ivars}{Local address of an array of remotely accessible data
    objects.
    The type of \VAR{ivars} should match that implied in the SYNOPSIS section.}
  \apiargument{IN}{nelems}{The number of elements in the \VAR{ivars} array.}
  \apiargument{OUT}{indices}{Local address of an array of indices of length at least
    \VAR{nelems} into \VAR{ivars} that satisfied the test condition.}
  \apiargument{IN}{status}{Local address of an optional mask array of length \VAR{nelems}
    that indicates which elements in \VAR{ivars} are excluded from the test set.}
  \apiargument{IN}{cmp}{A comparison operator from Table~\ref{p2p-consts}
    that compares elements of \VAR{ivars} with \VAR{cmp\_value}.}
  \apiargument{IN}{cmp\_value}{The value to be compared with the objects
    pointed to by \VAR{ivars}.
    The type of \VAR{cmp\_value} should match that implied in the SYNOPSIS section.}

\end{apiarguments}

\apidescription{
    The \FUNC{shmem\_test\_some} routine indicates whether at least one entry
    in the test set specified by \VAR{ivars} and \VAR{status} satisfies the
    test condition at the calling \ac{PE}.  The \VAR{ivars} objects at the
    calling \ac{PE} may be updated by an \ac{AMO} performed by a thread located
    within the calling \ac{PE} or within another \ac{PE}.
    This routine does not block and returns zero if
    no entries in \VAR{ivars} satisfied the test condition.
    This routine compares each element of the \VAR{ivars} array in the
    test set with the value \VAR{cmp\_value} according to the comparison
    operator \VAR{cmp} at the calling \ac{PE}.
    This routine tests all elements of \VAR{ivars} in the
    test set at least once, and the order in which the elements are tested is
    unspecified.  If an entry $i$ in \VAR{ivars} within the test set satisfies
    the test condition, a series of calls to \FUNC{shmem\_test\_some} must
    eventually return $i$.

    Upon return, the \VAR{indices} array contains the indices of the elements
    in the test set that satisfied the test condition during the call to
    \FUNC{shmem\_test\_some}.  The return value of \FUNC{shmem\_test\_some} is
    equal to the total number of these satisfied elements.  If the return value
    is $N$, then the first $N$ elements of the \VAR{indices} array contain
    those unique indices that satisfied the test condition.
    These first $N$ elements of \VAR{indices} may be unordered with respect to
    the corresponding indices of \VAR{ivars}.
    The array pointed
    to by \VAR{indices} must be at least \VAR{nelems} long.
    If an entry $i$ in \VAR{ivars} within the test set satisfies the test
    condition, a series of calls to \FUNC{shmem\_test\_some} must eventually
    include $i$ in the \VAR{indices} array.

    The optional \VAR{status} is a mask array of length \VAR{nelems} where each element
    corresponds to the respective element in \VAR{ivars} and indicates whether
    the element is excluded from the test set.  Elements of \VAR{status} set to
    0 will be included in the test set, and elements set to 1 will be ignored.  If all
    elements in \VAR{status} are set to 1 or \VAR{nelems} is 0, the test set is
    empty and this routine returns 0.  If \VAR{status} is a null pointer, it is ignored and all
    elements in \VAR{ivars} are included in the test set.  The \VAR{ivars},
    \VAR{indices}, and \VAR{status} arrays must not overlap in memory.

    Implementations must ensure that \FUNC{shmem\_test\_some} does not return
    indices before the updates of the memory indicated by the corresponding
    \VAR{ivars} elements are fully complete.
}

\apireturnvalues{
    \FUNC{shmem\_test\_some} returns the number of indices returned in
    the \VAR{indices} array. If the test set is empty, this routine returns 0.
}

\begin{apiexamples}
  \apicexample
      {The following \Cstd[11] example demonstrates the use of
      \FUNC{shmem\_test\_some} to process a simple all-to-all transfer of N
      data elements via a sum reduction, while potentially overlapping
      communication with computation.  This pattern is similar to the
      \FUNC{shmem\_test\_any} example above, but each while loop iteration may
      process more than one data item.}
      {./example_code/shmem_test_some_example.c}
      {}
\end{apiexamples}

\end{apidefinition}





\subsection{Memory Ordering Routines}\label{subsec:memory_order}
The following section discusses \openshmem \acp{API} that provide mechanisms to
ensure ordering and/or delivery of \OPR{Put}, \ac{AMO}, memory store,
and non-blocking \PUT{} and \GET{} routines to symmetric data objects.

\subsubsection{\textbf{SHMEM\_FENCE}}\label{subsec:shmem_fence}
\apisummary{
    Assures ordering of delivery of \PUT{}, \acp{AMO}, and memory store routines
    to symmetric data objects.
}

\begin{apidefinition}

\begin{Csynopsis}
void shmem_fence(void);
void shmem_ctx_fence(shmem_ctx_t ctx);
\end{Csynopsis}

\begin{Fsynopsis}
CALL SHMEM_FENCE
\end{Fsynopsis}

\begin{apiarguments}
    \newtext{
    \apiargument{IN}{ctx}{Context on which to perform the operation.  When none
    is provided, \CONST{SHMEM\_CTX\_DEFAULT} is used.}
    }
\end{apiarguments}

\apidescription{
    This routine assures ordering of delivery of \PUT{}, \acp{AMO}, and memory store
    routines to symmetric data objects.  All \PUT{}, \acp{AMO}, and memory store
    routines to symmetric data objects issued to a particular remote \ac{PE}
    \newtext{on the given context} prior
    to the call to \FUNC{shmem\_fence} are guaranteed to be delivered before any
    subsequent \PUT{}, \acp{AMO}, and memory store routines to symmetric data
    objects to the same \ac{PE}. \FUNC{shmem\_fence} guarantees order of delivery,
    not completion.
}

\apireturnvalues{
    None.
}

\apinotes{
    \FUNC{shmem\_fence} only provides per-\ac{PE} ordering guarantees and does not
    guarantee completion of delivery.  
    \FUNC{shmem\_fence} also does not have an effect on the ordering between memory 
    accesses issued by the target PE. \FUNC{shmem\_wait}, \FUNC{shmem\_wait\_until},
    \FUNC{shmem\_barrier}, \FUNC{shmem\_barrier\_all} routines can be called by the target PE to guarantee 
    ordering of its memory accesses.
    There is a subtle difference between
    \FUNC{shmem\_fence} and \FUNC{shmem\_quiet}, in that, \FUNC{shmem\_quiet}
    guarantees completion of \PUT{}, \acp{AMO}, and memory store routines to
    symmetric data objects which makes the updates visible to all other
    \acp{PE}. 
    
    The \FUNC{shmem\_quiet} routine should be called if completion of PUT{},
    \acp{AMO}, and memory store routines to symmetric data objects is desired
    when multiple remote \acp{PE} are involved.

    \textcolor{red}{MANJU SUGGESTED EXPLAINING ORDERING ISOLATION HERE: In
    multithreaded environments, \FUNC{shmem\_fence} can provide thread
    ordering isolation between different contexts.}

}

\begin{apiexamples}

\apicexample
    {The following \FUNC{shmem\_fence} example is for C11 programs: }
    {./example_code/shmem_fence_example.c}
    {\VAR{Put1} will be ordered to be delivered before \VAR{put3} and \VAR{put2}
    will be ordered to be delivered before \VAR{put4}.}

\end{apiexamples}

\end{apidefinition}


\subsubsection{\textbf{SHMEM\_QUIET}}\label{subsec:shmem_quiet}
\apisummary{
    Waits for completion of outstanding operations on symmetric data objects
    issued by a \ac{PE}.
}

\begin{apidefinition}

\begin{Csynopsis}
void @\FuncDecl{shmem\_quiet}@(void);
void @\FuncDecl{shmem\_ctx\_quiet}@(shmem_ctx_t ctx);
\end{Csynopsis}

\begin{apiarguments}
    \apiargument{IN}{ctx}{A context handle specifying the context on which to
    perform the operation. When this argument is not provided, the operation is
    performed on the default context.}
\end{apiarguments}

\apidescription{
    The \FUNC{shmem\_quiet} routine ensures completion of all operations
    on symmetric data objects issued by the calling \ac{PE} on the given context.
    Table~\ref{mem-order} lists the operations for which the \FUNC{shmem\_quiet}
    routine ensures completion. All operations on symmetric data objects are
    guaranteed to be complete and visible to all \acp{PE} when
    \FUNC{shmem\_quiet} returns. If \VAR{ctx} has the value
    \CONST{SHMEM\_CTX\_INVALID}, no operation is performed.
}


\apireturnvalues{
    None.
}

\apinotes{
    \FUNC{shmem\_quiet} is most useful as a way of ensuring completion of
    several operations on symmetric data objects initiated by the calling
    \ac{PE}. For example, one might use \FUNC{shmem\_quiet} to await delivery
    of a block of data before issuing another \PUT{} or nonblocking
    \PUT{} routine, which sets a completion flag on another \ac{PE}.
    \FUNC{shmem\_quiet} is not usually needed if
    \FUNC{shmem\_barrier\_all} or \FUNC{shmem\_barrier} are called.  The barrier
    routines wait for the completion of outstanding operations to
    symmetric data objects on all \acp{PE}.

    In an \openshmem program with multithreaded \acp{PE}, it is the
    user's responsibility to ensure ordering between operations issued by the
    threads in a \ac{PE} that target symmetric memory and calls by threads in
    that \ac{PE} to \FUNC{shmem\_quiet}. The \FUNC{shmem\_quiet} routine can
    enforce memory store ordering only for the calling thread. Thus, to ensure
    ordering for memory stores performed by a thread that is not the thread
    calling \FUNC{shmem\_quiet}, the update must be made visible to the calling
    thread according to the rules of the memory model associated with the
    threading environment.

    A call to \FUNC{shmem\_quiet} by a thread completes the operations posted
    prior to calling \FUNC{shmem\_quiet}. If the user intends to also complete
    operations issued by a thread that is not the thread calling
    \FUNC{shmem\_quiet}, the user must ensure that the operations are performed
    prior to the call to \FUNC{shmem\_quiet}. This may require the use of a
    synchronization operation provided by the threading package. For example,
    when using POSIX Threads, the user may call the
    \FUNC{pthread\_barrier\_wait} routine to ensure that all threads have issued
    operations before a thread calls \FUNC{shmem\_quiet}.

    \FUNC{shmem\_quiet} does not have an effect on the ordering between memory
    accesses issued by the target \ac{PE}. \FUNC{shmem\_wait\_until},
    \FUNC{shmem\_test}, \FUNC{shmem\_barrier}, \FUNC{shmem\_barrier\_all} routines
    can be called by the target \ac{PE} to guarantee ordering of its memory accesses.
}

\begin{apiexamples}

\apicexample
    {The following example uses \FUNC{shmem\_quiet} in a \Cstd[11] program: }
    {./example_code/shmem_quiet_example.c}
    {\VAR{Put1} and \VAR{put2} will be completed and visible before \VAR{put3}
    and \VAR{put4}.}
\end{apiexamples}

\end{apidefinition}


\subsubsection{Synchronization and Communication Ordering in OpenSHMEM}
When using the \openshmem \ac{API}, synchronization, ordering, and completion of
communication become critical. The updates via \PUT{} routines, \acp{AMO}, stores, and
nonblocking \PUT{} and \GET{} routines on symmetric data cannot be guaranteed until some form of
synchronization or ordering is introduced in the user's program. The table below
gives the different synchronization and ordering choices, and the situations
where they may be useful.\\

\begin{tabular}{p{0.2\textwidth} | p{0.7\textwidth}}
\hline
\textbf{\openshmem  \ac{API}} & \centering \textbf{Working of \openshmem \ac{API}} \tabularnewline
\hline
\hline
{Point-to-point synchronization}\\
\FUNC{shmem\_wait\_until}
&
\raisebox{-\totalheight}{\includegraphics[width=\linewidth]{figures/wait}}
\end{tabular}

\begin{tabular}{p{0.2\textwidth} | p{0.7\textwidth}}
{}
&
Waits for a symmetric variable to be updated by a target \ac{PE}. Should be
used when computation on the local \ac{PE} cannot proceed without the value that
the target \ac{PE} is to update. \tabularnewline
\hline
\end{tabular}

\begin{tabular}{p{0.2\textwidth} | p{0.7\textwidth}}

{Ordering puts issued by a local \ac{PE}} \\
\FUNC{shmem\_fence}
&
\raisebox{-\totalheight}{\includegraphics[width=\linewidth]{figures/fence}}
\end{tabular}

\begin{tabular}{p{0.2\textwidth} | p{0.7\textwidth}}
{}
&
All \PUT{}, \ac{AMO}, store, and nonblocking \PUT{} routines on symmetric data issued to
same \ac{PE}  are guaranteed to be delivered  before Puts (to the same \ac{PE})
issued after the \FUNC{fence} call. \tabularnewline
\hline
\end{tabular}

\begin{tabular}{p{0.2\textwidth} | p{0.7\textwidth}}
\hline
\textbf{\openshmem  \ac{API}} & \centering \textbf{Working of \openshmem \ac{API}} \tabularnewline
\hline
\hline
{Ordering puts issued by all \ac{PE} }\\
\FUNC{shmem\_quiet}
&
\raisebox{-\totalheight}{\includegraphics[width=\linewidth]{figures/quiet}}
\end{tabular}

\begin{tabular}{p{0.2\textwidth} | p{0.7\textwidth}}
{}
&
{All \PUT{}, \ac{AMO}, store, and nonblocking \PUT{} and \GET{} routines on symmetric data issued by a
local \ac{PE} to all  target \acp{PE} are guaranteed to be completed and visible
once quiet returns. This routine should be used when all remote writes issued by
a local \ac{PE} need to be visible  to all other \acp{PE} before the local
\ac{PE} proceeds. } \tabularnewline
\hline
\end{tabular}

\begin{tabular}{p{0.2\textwidth} | p{0.7\textwidth}}
\hline
\textbf{\openshmem  \ac{API}} & \centering \textbf{Working of \openshmem \ac{API}} \tabularnewline
\hline
\hline
{Collective synchronization over all \acp{PE}} \\
 \FUNC{shmem\_barrier\_all}
&
\raisebox{-\totalheight}{\includegraphics[width=\linewidth]{figures/barrierall}}
\end{tabular}

\begin{tabular}{p{0.2\textwidth} | p{0.7\textwidth}}
{}
&
{All local and remote memory operations issued by all \acp{PE} are guaranteed to
be completed before any \ac{PE} returns from the call. Additionally no \ac{PE}
shall return from the barrier until all \acp{PE} have entered the same
\FUNC{shmem\_barrier\_all} call. This routine should be used when
synchronization as well as completion of all stores and remote memory updates
via \openshmem is required over all \acp{PE}. } \tabularnewline
\hline
\end{tabular}
\clearpage

\begin{tabular}{p{0.2\textwidth} | p{0.7\textwidth}}
Collective synchronization over \\
\FUNC{shmem\_team\_sync}
&
\raisebox{-\totalheight}{\includegraphics[width=\linewidth]{figures/sync}}
\end{tabular}

\begin{tabular}{p{0.2\textwidth} | p{0.7\textwidth}}
{}
&
{\FUNC{shmem\_team\_sync} guarantees that no \ac{PE} shall return from the
synchronization routine until all \acp{PE} in the team have entered the same
\FUNC{shmem\_team\_sync} call. It does not guarantee completion of
local and remote memory operations issued by \acp{PE} within the team.
To do so, \FUNC{shmem\_quiet} should be called on the desired context(s) by all
\acp{PE} within the team before the \FUNC{shmem\_team\_sync} call to
guarantee the completion of the associated stores and remote memory updates via \openshmem.}
\tabularnewline
\hline
\end{tabular}
\clearpage







\subsection{Distributed Locking Routines}
The following section discusses \openshmem locks as a mechanism to provide
mutual exclusion. Three routines are available for distributed locking,
\textit{set, test} and \textit{clear}.

\subsubsection{\textbf{SHMEM\_LOCK}}\label{subsec:shmem_lock}
\input{content/shmem_lock.tex}





\subsection{Cache Management}
All of these routines are deprecated and are provided for backwards
compatibility.  Implementations must include all items in this section, and the
routines should function properly and may notify the user about deprecation of
their use.

\subsubsection{\textbf{SHMEM\_CACHE}}\label{subsec:shmem_cache}
\input{content/shmem_cache.tex}

\clearpage
\clearpage %%%%%%%%%%%%%%%%%%%%%%%%%%%%%%%%%%%%%%%%%%%%%%%%%%%%%%%%%%%%

\appendix

%defining pagestyle for annex
\pagestyle{fancy}
\fancyhf{}
\fancyhead[L]{\leftmark}
\fancyhead[R]{\thepage}
\renewcommand{\headrulewidth}{0pt}
\renewcommand{\thesection}{\thesectionOrig}




\chapter{Writing OpenSHMEM Programs}
\section*{Incorporating OpenSHMEM into Programs}\label{sec:writing_programs}

The following section describes how to write a ``Hello World" \openshmem program.
To write a ``Hello World" \openshmem program, the user must:

\begin{itemize}
\item Include the header file \HEADER{shmem.h} for \Cstd.
\item Add the initialization call \hyperref[subsec:shmem_init]{\FUNC{shmem\_init}}.
\item Use \openshmem calls to query the local \ac{PE} number
    (\hyperref[subsec:shmem_my_pe]{\FUNC{shmem\_my\_pe}}) and the total number
    of \acp{PE} (\hyperref[subsec:shmem_n_pes]{\FUNC{shmem\_n\_pes}}).
\item Add the finalization call \hyperref[subsec:shmem_finalize]{\FUNC{shmem\_finalize}}.
\end{itemize}

In \openshmem, the order in which lines appear in the output is not
deterministic because \acp{PE} execute asynchronously in parallel.

\SourceExample{example_code/hello-openshmem.c}{
  \label{openshmem-hello}
  ``Hello World'' example program in \Cstd
}

\ProgramOutput{example_code/hello-openshmem-c.output}{
  Possible ordering of expected output with 4 \acp{PE} from the
  program in Example~\ref{openshmem-hello}
}

\clearpage %%%%%%%%%%%%%%%%%%%%%%%%%%%%%%%%%%%%%%%%%%%%%%%%%%%%%%%%%%%%

Example~\ref{openshmem-hello-symmetric} shows a more complex
\openshmem program that illustrates the use of symmetric data objects.
Note the declaration of the \VAR{static short dest} array and its use as the
remote destination in \hyperref[subsec:shmem_put]{\FUNC{shmem\_put}}.

The \KEYWORD{static} keyword makes the \VAR{dest} array symmetric on all \acp{PE}.
Each \ac{PE} is able to transfer data to a remote \dest{} array by simply
specifying to an OpenSHMEM routine such as \hyperref[subsec:shmem_put]{\FUNC{shmem\_put}}
the local address of the symmetric data object that will receive the data.
This local address resolution aids programmability because the address of the
\dest{} need not be exchanged with the active side (\ac{PE} \CONST{0}) prior to
the \acf{RMA} routine.

Conversely, the declaration of the \VAR{short source} array is asymmetric
(local only).
The \source{} object does not need to be symmetric because \PUT{} handles the
references to the \VAR{source} array only on the active (local) side.

\SourceExample{example_code/writing_shmem_example.c}{
  \label{openshmem-hello-symmetric}
  Example program with symmetric data objects
}

\ProgramOutput{example_code/writing_shmem_example.output}{
  Possible ordering of expected output with 4~\acp{PE} from the
  program in Example~\ref{openshmem-hello-symmetric}
}

\chapter{Compiling and Running Programs}\label{sec:compiling}
The \openshmem Specification does not specify how
\openshmem programs are compiled, linked, and run. This section shows some
examples of how wrapper programs are utilized in the \openshmem Reference
Implementation to compile and launch programs.

\section{Compilation}
\subsection*{Programs written in \Cstd}

The \openshmem Reference Implementation provides a wrapper program, named
\textbf{oshcc}, to aid in the compilation of \Cstd programs.
The wrapper may be called as follows:

\begin{lstlisting}[]
oshcc <compiler options> -o myprogram myprogram.c
\end{lstlisting}
Where the $\langle\mbox{compiler options}\rangle$ are options understood by the
underlying \Cstd compiler called by \textbf{oshcc}.


\subsection*{Programs written in \Cpp}

The \openshmem Reference Implementation provides a wrapper program, named
\textbf{oshc++}, to aid in the compilation of \Cpp programs.
The wrapper may be called as follows:

\begin{lstlisting}[]
oshc++ <compiler options> -o myprogram myprogram.cpp
\end{lstlisting}
Where the $\langle\mbox{compiler options}\rangle$ are options understood by the
underlying \Cpp compiler called by \textbf{oshc++}.


\section{Running Programs}

The \openshmem Reference Implementation provides a wrapper program, named
\textbf{oshrun}, to launch \openshmem programs.
The wrapper may be called as follows:

\begin{lstlisting}[]
oshrun <runner options> -np <#> <program> <program arguments>
\end{lstlisting}
The arguments for \textbf{oshrun} are:

\begin{tabular}{p{0.3\textwidth}p{0.6\textwidth}}
$\langle\mbox{runner options}\rangle$ & {Options passed to the underlying launcher.}\tabularnewline
-np $\langle\mbox{\#}\rangle$ & {The number of \acp{PE} to be used in the execution.}\tabularnewline
$\langle\mbox{program}\rangle$ & {The program executable to be launched.}\tabularnewline
$\langle\mbox{program arguments}\rangle$ & {Flags and other parameters to pass to the program.}\tabularnewline
\end{tabular}




\chapter{Undefined Behavior in OpenSHMEM}\label{sec:undefined}

The \openshmem Specification formalizes the expected behavior of
its library routines.  In cases where routines are improperly used
or the input is not in accordance with the Specification, the behavior
is undefined.

\begin{longtable}{|>{\raggedright}p{0.3\textwidth}|>{\raggedright}p{0.6\textwidth}|}
\hline
\textbf{Inappropriate Usage} & \textbf{Undefined Behavior}\tabularnewline
\hline
\endhead
Uninitialized library & If the \openshmem library is not initialized,
calls to \openshmem routines that do not initialize the \openshmem library have undefined
behavior.  For example, an implementation may try to continue or may abort
immediately upon an \openshmem call into the uninitialized library.
\tabularnewline
\hline
Specifying invalid \ac{PE} numbers & For \openshmem routines that accept a
\ac{PE} number as an argument, if the \ac{PE} number is invalid for the
team associated with the operation (either implicitly or explicitly), the
behavior is undefined.  An invalid \ac{PE} number includes those that are
negative or greater than or equal to the size of the associated team.
\tabularnewline
\hline
Use of non-symmetric variables & Some routines require remotely accessible
variables to perform their function.  For example, an \openshmem libray may detect a \PUT{} to a non-symmetric variable
and choose to abort the program.  
However, another implementation may choose to continue execution with or without a warning.
\tabularnewline
\hline
Non-symmetric allocation of symmetric memory & The symmetric memory management routines are
collectives. For example, all \acp{PE} in the program must call
\FUNC{shmem\_malloc} with the same \VAR{size} argument.  Program behavior after a
mismatched \FUNC{shmem\_malloc} call is undefined.\tabularnewline
\hline
Use of null pointers with nonzero \VAR{len} specified & In any \openshmem routine
that takes a pointer and \VAR{len} describing the number of elements in that
pointer, a null pointer may not be given unless the corresponding \VAR{len} is also
specified as zero. Otherwise, the resulting behavior is undefined.
The following cases summarize this behavior:
\begin{itemize}
    \item \VAR{len} is 0, pointer is null: supported.
    \item \VAR{len} is not 0, pointer is null: undefined behavior.
    \item \VAR{len} is 0, pointer is non-null: supported.
    \item \VAR{len} is not 0, pointer is non-null: supported.
\end{itemize}
\tabularnewline
\hline
Multithreaded use of a team in concurrent team-based collectives &
Team-based collective operations are not thread-safe on the same \VAR{team}
object.
Concurrent collective operations on the same team from multiple threads may result in undefined
behavior.
For example, it is undefined behavior for one thread to call a team-implicit
collective which implicitly operates on the world team (e.g.,
\FUNC{shmem\_barrier\_all}) and another thread to concurrently call a
team-based collective (e.g., \FUNC{shmem\_broadcastmem}) on the same world team
object, \LibHandleRef{SHMEM\_TEAM\_WORLD}. \tabularnewline
\hline
Destroying a team with unfreed private contexts & Before destroying a given
team, the user is responsible for destroying all contexts created from that team
with the \LibConstRef{SHMEM\_CTX\_PRIVATE} option enabled; otherwise, the
behavior is undefined.\tabularnewline
\hline
\end{longtable}


\chapter{Interoperability with Other Programming Models}\label{sec:interoperability}

\openshmem routines may be used in conjunction with the routines of other
communication libraries or parallel languages in the same program. This section
describes the interoperability with other programming models, including
clarification of undefined behaviors caused by mixed use of different models,
and advice to \openshmem library users and developers that may improve the portability
and performance of hybrid programs.


\section{MPI Interoperability}

\openshmem and \ac{MPI} are two commonly used parallel programming models for
distributed-memory systems. The user can choose to utilize both models in the same program
to efficiently and easily support various communication patterns.

A vendor may implement the \openshmem and \ac{MPI} libraries in different ways. For
instance, one may implement both \openshmem and \ac{MPI} as standalone libraries,
each of which allocates and initializes fully isolated communication
resources.
Another approach
is to implement both \openshmem and \ac{MPI} interfaces within the
same software system in order to share a communication resource when possible.

To improve interoperability and portability in \openshmem + \ac{MPI} hybrid
programming, we clarify the relevant semantics in the following subsections.


\subsection{Initialization}
In order to ensure that a hybrid program can be portably performed with different vendor
implementations, the \openshmem environment of the program must be initialized by
a call to \FUNC{shmem\_init} or \FUNC{shmem\_init\_thread} and be finalized by
a call to \FUNC{shmem\_finalize}; the \ac{MPI} environment of the program must be initialized
by a call to \FUNC{MPI\_Init} or \FUNC{MPI\_Init\_thread} and be finalized by a
call to \FUNC{MPI\_Finalize}.

\parimpnotes{
  Portable implementations of \openshmem and \ac{MPI} must ensure that the initialization
  calls can be made in an arbitrary order within a program; the same rule also
  applies to the finalization calls. A software runtime that utilizes a shared
  communication resource for \openshmem and \ac{MPI} communication may maintain an
  internal reference counter in order to ensure that the shared resource is
  initialized only once and thus no shared resource is released until the last
  finalization call is made.
}


\subsection{Dynamic Process Creation}
\label{subsec:interoperability:mpmd}

\ac{MPI} defines a dynamic process model that allows creation of processes after
an \ac{MPI} application has started (e.g., by calling \FUNC{MPI\_Comm\_spawn}) and
connection to independent processes (e.g., through \FUNC{MPI\_Comm\_accept}
and \FUNC{MPI\_Comm\_connect}).
It provides a mechanism to establish communication
between the newly created processes and the existing \ac{MPI} application (see
\ac{MPI} standard version 3.1, Chapter 10).
Unlike \ac{MPI}, \openshmem starts all processes at once and requires all \acp{PE} to
collectively allocate and initialize resources (e.g., symmetric heap) used by
the \openshmem library before any other \openshmem routine may
be called. \openshmem does not support communication with dynamically created
or connected processes. In such a scenario, \ac{MPI} can be used to communicate
with these processes.


\subsection{Thread Safety}
\label{subsec:interoperability:thread}
Both \openshmem and \ac{MPI} define the interaction with user threads in a program
with routines that can be used for initializing and querying the thread
environment. A hybrid program may request different thread levels
at the initialization calls of \openshmem and \ac{MPI} environments; however, the
returned support level provided by the \openshmem or \ac{MPI} library might be different
from that returned in an non-hybrid program. For instance, the former
initialization call in a hybrid program may initialize a resource with the
requested thread level, but the supported level cannot be updated by a subsequent
initialization call if the underlying software runtime of \openshmem and \ac{MPI}
share the same internal communication resource.
The program should always check the \VAR{provided} thread level returned
at the corresponding initialization call or query the level of thread support
after initialization to portably ensure thread support in each communication
environment.

Both \openshmem and \ac{MPI} define similar thread levels, namely, \VAR{THREAD\_SINGLE},
\VAR{THREAD\_FUNNELED}, \VAR{THREAD\_SERIALIZED}, and \VAR{THREAD\_MULTIPLE}.
When requesting threading support in a hybrid program, however,
the following additional rules are applied if the implementations of \openshmem
and \ac{MPI} share the same internal communication resource.
It is strongly recommended to always follow these rules to ensure program
portability.

\begin{itemize}
    \item The \VAR{THREAD\_SINGLE} thread level requires a single-threaded program.
    Hence, a hybrid program should not request \VAR{THREAD\_SINGLE} at the initialization
    call of either \openshmem or \ac{MPI} but request a different thread level at the
    initialization call of the other model.

    \item The \VAR{THREAD\_FUNNELED} thread level allows only the main thread to
    make communication calls. A hybrid program using the \VAR{THREAD\_FUNNELED}
    thread level in both \openshmem and \ac{MPI} should ensure that the same main thread
    is used in both communication environments.

    \item The \VAR{THREAD\_SERIALIZED} thread level requires the program to ensure
    that communication calls are not made concurrently by multiple threads. If a
    hybrid program uses \VAR{THREAD\_SERIALIZED} in one communication environment
    and \VAR{THREAD\_SERIALIZED} or \VAR{THREAD\_FUNNELED} in the other one, it
    should also guarantee that the \openshmem and \ac{MPI} calls are not made concurrently
    from two distinct threads.
\end{itemize}

\subsection{Mapping Process Identification Numbers}
\label{subsec:interoperability:id}

Similar to the \ac{PE} number in \openshmem, \ac{MPI} defines rank as the
identification number of a process in a communicator. Both the \openshmem \ac{PE}
and the \ac{MPI} rank are unique integers assigned from zero to one less than the total
number of processes. In a hybrid program, the \openshmem
\ac{PE} number in \LibHandleRef{SHMEM\_TEAM\_WORLD}
and the \ac{MPI} rank in \VAR{MPI\_COMM\_WORLD} of a process can be equal.
This feature, however, may be provided by only some of the \openshmem and \ac{MPI}
implementations (e.g., if both environments share the same underlying process
manager) and is not portably guaranteed. A portable program should always
use the standard functions in each model, namely, \FUNC{shmem\_team\_my\_pe} or \FUNC{shmem\_my\_pe} in \openshmem 
and \FUNC{MPI\_Comm\_rank} in \ac{MPI}, to query the process identification numbers
in each communication environment and manage the mapping of identifiers in the
program when necessary.

\subsubsection*{Examples}
\label{subsubsec:interoperability:id:example}

\SourceExample{example_code/hybrid_mpi_mapping_id.c}{
  The following example demonstrates how to manage the mapping between
  \openshmem \ac{PE} numbers and \ac{MPI} ranks in
  \VAR{MPI\_COMM\_WORLD} in a hybrid \openshmem and \ac{MPI} program.
}


\SourceExample{example_code/hybrid_mpi_mapping_id_shmem_comm.c}{
  The following example demonstrates an alternative approach for
  managing the mapping of process identification numbers in a hybrid
  program. The program creates a new MPI communicator, named
  \VAR{shmem\_comm}, that contains all processes in
  \VAR{MPI\_COMM\_WORLD} and each process has the same \ac{MPI} rank
  number as its \openshmem \ac{PE} number.
}

\subsection{RMA Programming Models}
\label{subsec:interoperability:rma}

\openshmem and \ac{MPI} each define similar one-sided communication models;
however, a portable program should not assume interoperability between these
models.
For instance, \openshmem guarantees the atomicity only of concurrent \openshmem \ac{AMO} operations
that operate on symmetric data with the same datatype. Access to the same symmetric
object with \ac{MPI} atomic operations, such as an \FUNC{MPI\_Fetch\_and\_op}, may
result in an undefined result. A hybrid program should avoid situations where \ac{MPI} and
\openshmem one-sided operations perform concurrent accesses to the same memory
location; otherwise, the behavior is undefined.

\subsection{Communication Progress}
\label{subsec:interoperability:progress}

\openshmem promises the progression of communication both with and without
\openshmem calls and requires the software progress mechanism in the implementation
(e.g., a progress thread) when the hardware does not provide asynchronous communication
capabilities (see Section \ref{subsec:progress}).
In \ac{MPI}, however, a weak progress semantics is applied. That is,
an \ac{MPI} communication call is guaranteed only to complete in finite time. For
instance, an \FUNC{MPI\_Put} may be completed only when the remote process makes an \ac{MPI}
call that internally triggers the progress of \ac{MPI}, if the underlying hardware
does not support asynchronous communication. A hybrid program
should not assume that the \openshmem library also makes progress for \ac{MPI}.
It can explicitly manage the asynchronous communication of \ac{MPI} in
order to prevent any deadlock or performance degradation.


\chapter{History of OpenSHMEM}\label{sec:openshmem_history}

SHMEM has a long history as a parallel-programming model and has been
extensively used on a number of products since 1993, including the Cray T3D,
Cray X1E, Cray XT3 and XT4, \ac{SGI} Origin, \ac{SGI} Altix, Quadrics-based
clusters, and InfiniBand-based clusters.

\begin{itemize}
\item SHMEM Timeline
  \begin{itemize}
  \item Cray SHMEM
    \begin{itemize}
    \item SHMEM first introduced by Cray Research, Inc.\ in 1993 for Cray T3D
    \item Cray was acquired by \ac{SGI} in 1996
    \item Cray was acquired by Tera in 2000 (MTA)
    \item Platforms: Cray T3D, T3E, C90, J90, SV1, SV2, X1, X2, XE, XMT, XT
    \item \ac{HPE} acquired Cray in 2019
    \end{itemize}
  \item \ac{SGI} SHMEM
    \begin{itemize}
    \item \ac{SGI} acquired Cray Research, Inc.\ and SHMEM was integrated into
      \ac{SGI}'s Message Passing Toolkit (MPT)
    \item \ac{SGI} currently owns the rights to SHMEM and \openshmem
    \item Platforms: Origin, Altix 4700, Altix XE, ICE, UV
    \item \ac{SGI} was acquired by Rackable Systems in 2009
    \item \ac{SGI} and \ac{OSSS} signed a
      SHMEM trademark licensing agreement in 2010
    \item \ac{HPE} acquired \ac{SGI} in 2016
    \end{itemize}
  \end{itemize}
\end{itemize}

A listing of \openshmem implementations can be found on
\url{http://www.openshmem.org/}.








\chapter{Deprecated \acs{API}}\label{sec:dep}

\section{Overview}\label{dep:overview}
\TableIndex{Deprecated \acs{API}}
For the \openshmem Specification, deprecation is the process of identifying
\ac{API} that is supported but no longer recommended for use by users.
The deprecated \ac{API} \textbf{must} be supported until clearly
indicated as otherwise by the Specification.
This chapter records the \ac{API} or functionality that have been deprecated, the
version of the \openshmem Specification that effected the deprecation, and the
most recent version of the \openshmem Specification in which the feature was
supported before removal.

\begin{center}
\scriptsize
\begin{longtable}{|l|c|c|l|}
    \hline
    \textbf{Deprecated \ac{API}}
    & \textbf{Deprecated Since}
    & \textbf{Last Version Supported}
    & \textbf{Replaced By} \\
    \hline
    \endhead
    %% Deprecated in 1.1
    Header Directory: \hyperref[dep:mpp_header]{\HEADER{mpp}} & 1.1 & Current & (none) \\ \hline
    %% Deprecated in 1.2
    \CorCpp: \hyperref[subsec:start_pes]{\FuncRef{start\_pes}} & 1.2 & Current & \hyperref[subsec:shmem_init]{\FUNC{shmem\_init}} \\ \hline
    \Fortran: \hyperref[subsec:start_pes]{\FuncRef{START\_PES}} & 1.2 & 1.4 & \hyperref[subsec:shmem_init]{\FUNC{SHMEM\_INIT}} \\ \hline
    \hyperref[subsec:start_pes]{Implicit finalization} & 1.2 & Current & \hyperref[subsec:shmem_finalize]{\FUNC{shmem\_finalize}} \\ \hline
    \CorCpp: \FuncRef{\_my\_pe} & 1.2 & Current & \hyperref[subsec:shmem_my_pe]{\FUNC{shmem\_my\_pe}} \\ \hline
    \CorCpp: \FuncRef{\_num\_pes} & 1.2 & Current & \hyperref[subsec:shmem_n_pes]{\FUNC{shmem\_n\_pes}} \\ \hline
    \Fortran: \FuncRef{MY\_PE} & 1.2 & 1.4 & \hyperref[subsec:shmem_my_pe]{\FUNC{SHMEM\_MY\_PE}} \\ \hline
    \Fortran: \FuncRef{NUM\_PES} & 1.2 & 1.4 & \hyperref[subsec:shmem_n_pes]{\FUNC{SHMEM\_N\_PES}} \\ \hline
    \CorCpp: \FuncRef{shmalloc} & 1.2 & Current & \hyperref[sec:memory_management]{\FUNC{shmem\_malloc}} \\ \hline
    \CorCpp: \FuncRef{shfree} & 1.2 & Current & \hyperref[sec:memory_management]{\FUNC{shmem\_free}} \\ \hline
    \CorCpp: \FuncRef{shrealloc} & 1.2 & Current & \hyperref[sec:memory_management]{\FUNC{shmem\_realloc}} \\ \hline
    \CorCpp: \FuncRef{shmemalign} & 1.2 & Current & \hyperref[sec:memory_management]{\FUNC{shmem\_align}} \\ \hline
    \Fortran: \FuncRef{SHMEM\_PUT} & 1.2 & 1.4 & \hyperref[subsec:shmem_put]{\FUNC{SHMEM\_PUT8} or \FUNC{SHMEM\_PUT64}} \\ \hline
    %% Deprecated in 1.3
    \minitab{
        \CorCpp: \FuncRef{shmem\_clear\_cache\_inv}
        \\ \CorCpp: \FuncRef{shmem\_clear\_cache\_line\_inv}
        \\ \CorCpp: \FuncRef{shmem\_set\_cache\_inv}
        \\ \CorCpp: \FuncRef{shmem\_set\_cache\_line\_inv}
        \\ \CorCpp: \FuncRef{shmem\_udcflush}
        \\ \CorCpp: \FuncRef{shmem\_udcflush\_line}
        } & 1.3 & 1.4 & (none) \\ \hline
    \minitab{
        \Fortran: \FuncRef{SHMEM\_CLEAR\_CACHE\_INV}
        %% Note: At the time of deprecation, the Fortran API did not specify
        %% SHMEM_CLEAR_CACHE_LINE_INV. While this omission is certainly an error,
        %% Fortran was removed in 1.5 so the omission was never corrected.
        \\ \Fortran: \FuncRef{SHMEM\_SET\_CACHE\_INV}
        \\ \Fortran: \FuncRef{SHMEM\_SET\_CACHE\_LINE\_INV}
        \\ \Fortran: \FuncRef{SHMEM\_UDCFLUSH}
        \\ \Fortran: \FuncRef{SHMEM\_UDCFLUSH\_LINE}
        } & 1.3 & 1.4 & (none) \\ \hline
    \LibConstRef{\_SHMEM\_SYNC\_VALUE}         & 1.3 & Current & \hyperref[subsec:library_constants]{\CONST{SHMEM\_SYNC\_VALUE}} \\ \hline
    \LibConstRef{\_SHMEM\_BARRIER\_SYNC\_SIZE} & 1.3 & Current & \hyperref[subsec:library_constants]{\CONST{SHMEM\_BARRIER\_SYNC\_SIZE}} \\ \hline
    \LibConstRef{\_SHMEM\_BCAST\_SYNC\_SIZE}   & 1.3 & Current & \hyperref[subsec:library_constants]{\CONST{SHMEM\_BCAST\_SYNC\_SIZE}} \\ \hline
    \LibConstRef{\_SHMEM\_COLLECT\_SYNC\_SIZE} & 1.3 & Current & \hyperref[subsec:library_constants]{\CONST{SHMEM\_COLLECT\_SYNC\_SIZE}} \\ \hline
    \LibConstRef{\_SHMEM\_REDUCE\_SYNC\_SIZE}  & 1.3 & Current & \hyperref[subsec:library_constants]{\CONST{SHMEM\_REDUCE\_SYNC\_SIZE}} \\ \hline
    \LibConstRef{\_SHMEM\_REDUCE\_MIN\_WRKDATA\_SIZE} & 1.3 & Current & \hyperref[subsec:library_constants]{\CONST{SHMEM\_REDUCE\_MIN\_WRKDATA\_SIZE}} \\ \hline
    \LibConstRef{\_SHMEM\_MAJOR\_VERSION} & 1.3 & Current & \hyperref[subsec:library_constants]{\CONST{SHMEM\_MAJOR\_VERSION}} \\ \hline
    \LibConstRef{\_SHMEM\_MINOR\_VERSION} & 1.3 & Current & \hyperref[subsec:library_constants]{\CONST{SHMEM\_MINOR\_VERSION}} \\ \hline
    \LibConstRef{\_SHMEM\_MAX\_NAME\_LEN} & 1.3 & Current & \hyperref[subsec:library_constants]{\CONST{SHMEM\_MAX\_NAME\_LEN}} \\ \hline
    \LibConstRef{\_SHMEM\_VENDOR\_STRING} & 1.3 & Current & \hyperref[subsec:library_constants]{\CONST{SHMEM\_VENDOR\_STRING}} \\ \hline
    \LibConstRef{\_SHMEM\_CMP\_EQ} & 1.3 & Current & \hyperref[subsec:library_constants]{\CONST{SHMEM\_CMP\_EQ}} \\ \hline
    \LibConstRef{\_SHMEM\_CMP\_NE} & 1.3 & Current & \hyperref[subsec:library_constants]{\CONST{SHMEM\_CMP\_NE}} \\ \hline
    \LibConstRef{\_SHMEM\_CMP\_LT} & 1.3 & Current & \hyperref[subsec:library_constants]{\CONST{SHMEM\_CMP\_LT}} \\ \hline
    \LibConstRef{\_SHMEM\_CMP\_LE} & 1.3 & Current & \hyperref[subsec:library_constants]{\CONST{SHMEM\_CMP\_LE}} \\ \hline
    \LibConstRef{\_SHMEM\_CMP\_GT} & 1.3 & Current & \hyperref[subsec:library_constants]{\CONST{SHMEM\_CMP\_GT}} \\ \hline
    \LibConstRef{\_SHMEM\_CMP\_GE} & 1.3 & Current & \hyperref[subsec:library_constants]{\CONST{SHMEM\_CMP\_GE}} \\ \hline
    %% Deprecated in 1.4
    \EnvVarRef{SMA\_VERSION}         & 1.4 & Current & \hyperref[subsec:environment_variables]{\ENVVAR{SHMEM\_VERSION}} \\ \hline
    \EnvVarRef{SMA\_INFO}            & 1.4 & Current & \hyperref[subsec:environment_variables]{\ENVVAR{SHMEM\_INFO}} \\ \hline
    \EnvVarRef{SMA\_SYMMETRIC\_SIZE} & 1.4 & Current & \hyperref[subsec:environment_variables]{\ENVVAR{SHMEM\_SYMMETRIC\_SIZE}} \\ \hline
    \EnvVarRef{SMA\_DEBUG}           & 1.4 & Current & \hyperref[subsec:environment_variables]{\ENVVAR{SHMEM\_DEBUG}} \\ \hline
    \minitab{\CorCpp: \FuncRef{shmem\_wait}
        \\ \CorCpp: \FuncRef{shmem\_\FuncParam{TYPENAME}\_wait}}
        & 1.4 & Current & See \textbf{Notes} for \hyperref[subsec:shmem_wait_until]{\FUNC{shmem\_wait\_until}} \\ \hline
    \CorCpp: \FuncRef{shmem\_wait\_until} & 1.4 & Current
        & \Cstd[11]: \hyperref[subsec:shmem_wait_until]{\FUNC{shmem\_wait\_until}}, \CorCpp: \hyperref[subsec:shmem_wait_until]{\FUNC{shmem\_long\_wait\_until}} \\ \hline
    \minitab{\Cstd[11]: \FuncRef{shmem\_fetch}
        \\ \CorCpp: \FuncRef{shmem\_\FuncParam{TYPENAME}\_fetch}}
        & 1.4 & Current & \hyperref[subsec:shmem_atomic_fetch]{\FUNC{shmem\_atomic\_fetch}} \\ \hline
    \minitab{\Cstd[11]: \FuncRef{shmem\_set}
        \\ \CorCpp: \FuncRef{shmem\_\FuncParam{TYPENAME}\_set}}
        & 1.4 & Current & \hyperref[subsec:shmem_atomic_set]{\FUNC{shmem\_atomic\_set}} \\ \hline
    \minitab{\Cstd[11]: \FuncRef{shmem\_cswap}
        \\ \CorCpp: \FuncRef{shmem\_\FuncParam{TYPENAME}\_cswap}}
        & 1.4 & Current & \hyperref[subsec:shmem_atomic_compare_swap]{\FUNC{shmem\_atomic\_compare\_swap}} \\ \hline
    \minitab{\Cstd[11]: \FuncRef{shmem\_swap}
        \\ \CorCpp: \FuncRef{shmem\_\FuncParam{TYPENAME}\_swap}}
        & 1.4 & Current & \hyperref[subsec:shmem_atomic_swap]{\FUNC{shmem\_atomic\_swap}} \\ \hline
    \minitab{\Cstd[11]: \FuncRef{shmem\_finc}
        \\ \CorCpp: \FuncRef{shmem\_\FuncParam{TYPENAME}\_finc}}
        & 1.4 & Current & \hyperref[subsec:shmem_atomic_fetch_inc]{\FUNC{shmem\_atomic\_fetch\_inc}} \\ \hline
    \minitab{\Cstd[11]: \FuncRef{shmem\_inc}
        \\ \CorCpp: \FuncRef{shmem\_\FuncParam{TYPENAME}\_inc}}
        & 1.4 & Current & \hyperref[subsec:shmem_atomic_inc]{\FUNC{shmem\_atomic\_inc}} \\ \hline
    \minitab{\Cstd[11]: \FuncRef{shmem\_fadd}
        \\ \CorCpp: \FuncRef{shmem\_\FuncParam{TYPENAME}\_fadd}}
        & 1.4 & Current & \hyperref[subsec:shmem_atomic_fetch_add]{\FUNC{shmem\_atomic\_fetch\_add}} \\ \hline
    \minitab{\Cstd[11]: \FuncRef{shmem\_add}
        \\ \CorCpp: \FuncRef{shmem\_\FuncParam{TYPENAME}\_add}}
        & 1.4 & Current & \hyperref[subsec:shmem_atomic_add]{\FUNC{shmem\_atomic\_add}} \\ \hline
    Entire \Fortran \ac{API} & 1.4 & 1.4 & \openshmem \Cstd \ac{API} through \Fortran--\Cstd interoperability \\ \hline
    %% Deprecated in 1.5
    \minitab{
        \LibConstRef{SHMEM\_SYNC\_VALUE}
        \\ \LibConstRef{SHMEM\_SYNC\_SIZE}
        \\ \LibConstRef{SHMEM\_BARRIER\_SYNC\_SIZE}
        \\ \LibConstRef{SHMEM\_ALLTOALL\_SYNC\_SIZE}
        \\ \LibConstRef{SHMEM\_ALLTOALLS\_SYNC\_SIZE}
        \\ \LibConstRef{SHMEM\_BCAST\_SYNC\_SIZE}
        \\ \LibConstRef{SHMEM\_COLLECT\_SYNC\_SIZE}
        \\ \LibConstRef{SHMEM\_REDUCE\_SYNC\_SIZE}
        \\ \LibConstRef{SHMEM\_REDUCE\_MIN\_WRKDATA\_SIZE}
    } & 1.5 & Current & Team-based collectives, \minitab{Section~\ref{subsec:team_collectives}}. \\ \hline
    \CorCpp: Active-set-based \FuncRef{shmem\_sync}
        & 1.5 & Current & Team-based \hyperref[subsec:shmem_sync]{\FUNC{shmem\_sync}} \\ \hline
    \CorCpp: \FuncRef{shmem\_alltoall\{32, 64\}} & 1.5 & Current &
    \hyperref[subsec:shmem_alltoall]{\FUNC{shmem\_alltoall}} \\ \hline
    \CorCpp: \FuncRef{shmem\_alltoalls\{32, 64\}} & 1.5 & Current &
    \hyperref[subsec:shmem_alltoalls]{\FUNC{shmem\_alltoalls}} \\ \hline
    \CorCpp: \FuncRef{shmem\_broadcast\{32, 64\}} & 1.5 & Current &
    \hyperref[subsec:shmem_broadcast]{\FUNC{shmem\_broadcast}} \\ \hline
    \CorCpp: \FuncRef{shmem\_collect\{32, 64\}} & 1.5 & Current &
    \hyperref[subsec:shmem_collect]{\FUNC{shmem\_collect}} \\ \hline
    \CorCpp: \FuncRef{shmem\_fcollect\{32, 64\}} & 1.5 & Current &
    \hyperref[subsec:shmem_collect]{\FUNC{shmem\_fcollect}} \\ \hline
    \CorCpp: \FuncRef{shmem\_\FuncParam{TYPENAME}\_and\_to\_all}
        & 1.5 & Current & \hyperref[subsec:shmem_and_reduce]{\FUNC{shmem\_and\_reduce}} \\ \hline
    \CorCpp: \FuncRef{shmem\_\FuncParam{TYPENAME}\_or\_to\_all}
        & 1.5 & Current & \hyperref[subsec:shmem_or_reduce]{\FUNC{shmem\_or\_reduce}} \\ \hline
    \CorCpp: \FuncRef{shmem\_\FuncParam{TYPENAME}\_xor\_to\_all}
        & 1.5 & Current & \hyperref[subsec:shmem_xor_reduce]{\FUNC{shmem\_xor\_reduce}} \\ \hline
    \CorCpp: \FuncRef{shmem\_\FuncParam{TYPENAME}\_max\_to\_all}
        & 1.5 & Current & \hyperref[subsec:shmem_max_reduce]{\FUNC{shmem\_max\_reduce}} \\ \hline
    \CorCpp: \FuncRef{shmem\_\FuncParam{TYPENAME}\_min\_to\_all}
        & 1.5 & Current & \hyperref[subsec:shmem_min_reduce]{\FUNC{shmem\_min\_reduce}} \\ \hline
    \CorCpp: \FuncRef{shmem\_\FuncParam{TYPENAME}\_sum\_to\_all}
        & 1.5 & Current & \hyperref[subsec:shmem_sum_reduce]{\FUNC{shmem\_sum\_reduce}} \\ \hline
    \CorCpp: \FuncRef{shmem\_\FuncParam{TYPENAME}\_prod\_to\_all}
        & 1.5 & Current & \hyperref[subsec:shmem_prod_reduce]{\FUNC{shmem\_prod\_reduce}} \\ \hline
    \CorCpp: \hyperref[subsec:shmem_barrier]{\FuncRef{shmem\_barrier}}
        & 1.5 & Current & \hyperref[subsec:shmem_quiet]{\FuncRef{shmem\_quiet}} + \hyperref[subsec:shmem_sync]{\FuncRef{shmem\_sync}} \\ \hline
    \minitab{\Cstd[11]: \FuncRef{shmem\_wait\_until(\CTYPE{short} ...)}
        \\ \CorCpp: \FuncRef{shmem\_short\_wait\_until}}
        & 1.5 & Current & (none) \\ \hline
    \minitab{\Cstd[11]: \FuncRef{shmem\_wait\_until(\CTYPE{unsigned short} ...)}
        \\ \CorCpp: \FuncRef{shmem\_ushort\_wait\_until}}
        & 1.5 & Current & (none) \\ \hline
    \minitab{\Cstd[11]: \FuncRef{shmem\_test(\CTYPE{short} ...)}
        \\ \CorCpp: \FuncRef{shmem\_short\_test}}
        & 1.5 & Current & (none) \\ \hline
    \minitab{\Cstd[11]: \FuncRef{shmem\_test(\CTYPE{unsigned short} ...)}
        \\ \CorCpp: \FuncRef{shmem\_ushort\_test}}
        & 1.5 & Current & (none) \\ \hline
    \minitab{Table~\ref{p2psynctypes}: point-to-point synchronization types}
        & 1.5 & Current & Table~\ref{stdamotypes}: standard \ac{AMO} types \\ \hline
    %% Deprecated in 1.6
    %% Deprecated in 1.7
    %% Notes
    %% - If a hyperref spans more than one line vertically, the clickable box
    %%   will also span more than one line. To prevent this, wrap the hyperref
    %%   in a minitab. Example in 1.5: "Team-based collectives".
    \end{longtable}
\end{center}

\section{Deprecation Rationale}\label{dep:rationale}

\subsection{Header Directory: \HEADER{mpp}}
\label{dep:mpp_header}
In addition to the default system header paths, \openshmem implementations
must provide all \openshmem-specified header files from the \HEADER{mpp}
header directory such that these headers can be referenced in \CorCpp as
\begin{lstlisting}[language=]
#include <mpp/shmem.h>
#include <mpp/shmemx.h>
\end{lstlisting}
and in \Fortran as
\begin{lstlisting}[language=]
include 'mpp/shmem.fh'
include 'mpp/shmemx.fh'
\end{lstlisting}
for backwards compatibility with \ac{SGI} SHMEM.

\subsection{\CorCpp: \FUNC{start\_pes}}
\label{dep:start_pes}
The \CorCpp routine \FUNC{start\_pes} includes an unnecessary initialization
argument that is remnant of historical \emph{SHMEM} implementations and no
longer reflects the requirements of modern \openshmem implementations.
Furthermore, the naming of \FUNC{start\_pes} does not include the standardized
\shmemprefixLC{} naming prefix. This routine has been deprecated and
\openshmem users are encouraged to use \FUNC{shmem\_init} instead.

\subsection{Implicit Finalization}
Implicit finalization was deprecated and replaced with explicit finalization using the
\FUNC{shmem\_finalize} routine.  Explicit finalization improves portability and
also improves interoperability with profiling and debugging tools.

\subsection{\CorCpp: \FUNC{\_my\_pe}, \FUNC{\_num\_pes}, \FUNC{shmalloc},
    \FUNC{shfree}, \FUNC{shrealloc}, \FUNC{shmemalign}}
\label{dep:func_not_shmemunder}
The \CorCpp routines \FUNC{\_my\_pe}, \FUNC{\_num\_pes}, \FUNC{shmalloc},
\FUNC{shfree}, \FUNC{shrealloc}, and \FUNC{shmemalign} were deprecated in order
to normalize the \openshmem \ac{API} to use \shmemprefixLC{} as the standard
prefix for all routines.

\subsection{\textit{Fortran}: \FUNC{START\_PES}, \FUNC{MY\_PE}, \FUNC{NUM\_PES}} %% WARNING: Issue #66.
The \Fortran routines \FUNC{START\_PES}, \FUNC{MY\_PE}, and \FUNC{NUM\_PES}
were deprecated in order to minimize the \ac{API} differences from the deprecation
of \CorCpp routines \FUNC{start\_pes}, \FUNC{\_my\_pe}, and \FUNC{\_num\_pes}.

\subsection{\textit{Fortran}: \FUNC{SHMEM\_PUT}} %% WARNING: Issue #66.
\label{dep:fortran_shmem_put}
The \Fortran routine \FUNC{SHMEM\_PUT} is defined only for the \Fortran
\ac{API} and is semantically identical to \Fortran routines
\FUNC{SHMEM\_PUT8} and \FUNC{SHMEM\_PUT64}.  Since \FUNC{SHMEM\_PUT8} and
\FUNC{SHMEM\_PUT64} have defined equivalents in the \CorCpp interface,
\FUNC{SHMEM\_PUT} is ambiguous and has been deprecated.

\subsection{SHMEM\_CACHE}
\label{dep:shmem_cache}
The \FUNC{SHMEM\_CACHE} \ac{API}
\begin{center}
\begin{tabular}{ll}
    \CorCpp: & \Fortran: \\
    \FUNC{shmem\_clear\_cache\_inv}     & \FUNC{SHMEM\_CLEAR\_CACHE\_INV} \\
    \FUNC{shmem\_set\_cache\_inv}       & \FUNC{SHMEM\_SET\_CACHE\_INV} \\
    \FUNC{shmem\_set\_cache\_line\_inv} & \FUNC{SHMEM\_SET\_CACHE\_LINE\_INV} \\
    \FUNC{shmem\_udcflush}              & \FUNC{SHMEM\_UDCFLUSH} \\
    \FUNC{shmem\_udcflush\_line}        & \FUNC{SHMEM\_UDCFLUSH\_LINE} \\
    \FUNC{shmem\_clear\_cache\_line\_inv} \\
\end{tabular}
\end{center}
was originally implemented for systems with cache-management instructions.
This \ac{API} has largely gone unused on cache-coherent system architectures.
\FUNC{SHMEM\_CACHE} has been deprecated.

\subsection{\CONST{\_SHMEM\_*} Library Constants}
\label{dep:libconst_undershmem}
The library constants
\begin{center}
\begin{tabular}{ll}
    \CONST{\_SHMEM\_SYNC\_VALUE}         & \CONST{\_SHMEM\_MAX\_NAME\_LEN} \\
    \CONST{\_SHMEM\_BARRIER\_SYNC\_SIZE} & \CONST{\_SHMEM\_VENDOR\_STRING} \\
    \CONST{\_SHMEM\_BCAST\_SYNC\_SIZE}   & \CONST{\_SHMEM\_CMP\_EQ} \\
    \CONST{\_SHMEM\_COLLECT\_SYNC\_SIZE} & \CONST{\_SHMEM\_CMP\_NE} \\
    \CONST{\_SHMEM\_REDUCE\_SYNC\_SIZE}  & \CONST{\_SHMEM\_CMP\_LT} \\
    \CONST{\_SHMEM\_REDUCE\_MIN\_WRKDATA\_SIZE} & \CONST{\_SHMEM\_CMP\_LE} \\
    \CONST{\_SHMEM\_MAJOR\_VERSION}      & \CONST{\_SHMEM\_CMP\_GT} \\
    \CONST{\_SHMEM\_MINOR\_VERSION}      & \CONST{\_SHMEM\_CMP\_GE} \\
\end{tabular}
\end{center}
do not adhere to the \Cstd standard's reserved identifiers and the \Cpp
standard's reserved names.  These constants were deprecated and replaced
with corresponding constants of prefix \shmemprefix{} that adhere to \CorCpp{}
and \Fortran naming conventions.

\subsection{\ENVVAR{SMA\_*} Environment Variables}
\label{dep:envvar_sma}
The environment variables \ENVVAR{SMA\_VERSION}, \ENVVAR{SMA\_INFO},
\ENVVAR{SMA\_SYMMETRIC\_SIZE}, and \ENVVAR{SMA\_DEBUG}
were deprecated in order to normalize the \openshmem \ac{API} to use
\shmemprefix{} as the standard prefix for all environment variables.

\subsection{\CorCpp: \FUNC{shmem\_wait}}
\label{dep:shmem_wait}
The \CorCpp interface for \FUNC{shmem\_wait} and \FUNC{shmem\_\FuncParam{TYPENAME}\_wait}
was identified as unintuitive with respect to
the comparison operation it performed.  As \FUNC{shmem\_wait} can be trivially
replaced by \FUNC{shmem\_wait\_until} where \VAR{cmp} is
\CONST{SHMEM\_CMP\_NE}, the \FUNC{shmem\_wait} interface was deprecated in
favor of \FUNC{shmem\_wait\_until}, which makes the comparison operation
explicit and better communicates the developer's intent.

\subsection{\CorCpp: \FUNC{shmem\_wait\_until}}
\label{dep:long_typed_shmem_wait_until}
The \CTYPE{long}-typed \CorCpp routine \FUNC{shmem\_wait\_until} was deprecated
in favor of the \Cstd[11] type-generic interface of the same name or the
explicitly typed \CorCpp routine \FUNC{shmem\_long\_wait\_until}.

\subsection{\textit{C11} and \CorCpp: \FUNC{shmem\_fetch}, \FUNC{shmem\_set}, %% Issue #66.
    \FUNC{shmem\_cswap}, \FUNC{shmem\_swap}, \FUNC{shmem\_finc},
    \FUNC{shmem\_inc}, \FUNC{shmem\_fadd}, \FUNC{shmem\_add}}
\label{dep:amo_not_shmem_atomic}
The \Cstd[11] and \CorCpp interfaces for
\begin{center}
\begin{tabular}{ll}
    \Cstd[11]: & \CorCpp: \\
    \FUNC{shmem\_fetch} & \FUNC{shmem\_\FuncParam{TYPENAME}\_fetch} \\
    \FUNC{shmem\_set}   & \FUNC{shmem\_\FuncParam{TYPENAME}\_set}   \\
    \FUNC{shmem\_cswap} & \FUNC{shmem\_\FuncParam{TYPENAME}\_cswap} \\
    \FUNC{shmem\_swap}  & \FUNC{shmem\_\FuncParam{TYPENAME}\_swap}  \\
    \FUNC{shmem\_finc}  & \FUNC{shmem\_\FuncParam{TYPENAME}\_finc}  \\
    \FUNC{shmem\_inc}   & \FUNC{shmem\_\FuncParam{TYPENAME}\_inc}   \\
    \FUNC{shmem\_fadd}  & \FUNC{shmem\_\FuncParam{TYPENAME}\_fadd}  \\
    \FUNC{shmem\_add}   & \FUNC{shmem\_\FuncParam{TYPENAME}\_add}   \\
\end{tabular}
\end{center}
were deprecated and replaced with
similarly named interfaces within the \FUNC{shmem\_atomic\_*} namespace
in order to more clearly identify these calls as performing atomic operations.
In addition, the abbreviated names ``cswap'', ``finc'', and ``fadd'' were
expanded for clarity to ``compare\_swap'', ``fetch\_inc'', and ``fetch\_add''.

\subsection{\textit{Fortran} API} %% WARNING: Issue #66.
\label{dep:fortran}
The entire \openshmem \Fortran \ac{API} was deprecated in \openshmem[1.4] and
removed in \openshmem[1.5] because of a general lack of
use and a lack of conformance with legacy \Fortran standards. In lieu of an
extensive update of the \Fortran \ac{API}, \Fortran users are encouraged to
leverage the \openshmem Specification's \Cstd \ac{API} through the
\Fortran--\Cstd interoperability initially standardized by \Fortran[2003]%
\footnote{Formally, \Fortran[2003] is known as ISO/IEC~1539-1:2004(E).}.


\subsection{Active-set-based library constants and collectives}
\label{dep:active_set_libconst_and_collectives}
With the addition of \openshmem teams, Section~\ref{subsec:team}, the previous
method for performing collective
operations has been superseded by a more readable, flexible method for
organizing and communicating between groups of \acp{PE}. All collective routines
which previously indicated subgroups of \acp{PE} with a list of
parameters to describe the subgroup composition (active set) should be phased
out in favor of using collective operations with a team parameter.

The library constants
\begin{center}
\begin{tabular}{ll}
    \LibConstRef{SHMEM\_SYNC\_VALUE}            & \LibConstRef{SHMEM\_BCAST\_SYNC\_SIZE} \\
    \LibConstRef{SHMEM\_SYNC\_SIZE}             & \LibConstRef{SHMEM\_COLLECT\_SYNC\_SIZE} \\
    \LibConstRef{SHMEM\_BARRIER\_SYNC\_SIZE}    & \LibConstRef{SHMEM\_REDUCE\_SYNC\_SIZE} \\
    \LibConstRef{SHMEM\_ALLTOALL\_SYNC\_SIZE}   & \LibConstRef{SHMEM\_REDUCE\_MIN\_WRKDATA\_SIZE} \\
    \LibConstRef{SHMEM\_ALLTOALLS\_SYNC\_SIZE} \\
\end{tabular}
\end{center}
were deprecated as these constants pertain only to active-set-based collectives.

The \CorCpp active-set-based \FuncRef{shmem\_sync} routine was deprecated and
replaced with the team-based \Cstd[11] \FuncRef{shmem\_sync} or \CorCpp
\FuncRef{shmem\_team\_sync} routine.

The fixed-sized versions of the active-set-based routines
\begin{center}
\begin{tabular}{ll}
    \FuncRef{shmem\_alltoall32} & \FuncRef{shmem\_alltoall64} \\
    \FuncRef{shmem\_alltoalls32} & \FuncRef{shmem\_alltoalls64} \\
    \FuncRef{shmem\_broadcast32} & \FuncRef{shmem\_broadcast64} \\
    \FuncRef{shmem\_collect32} & \FuncRef{shmem\_collect64} \\
    \FuncRef{shmem\_fcollect32} & \FuncRef{shmem\_fcollect64} \\
\end{tabular}
\end{center}
were deprecated. Instead, all team-based collective routines use standard
\Cstd types with the option to use generic \Cstd[11] functions for more portable
and maintainable implementations.

The active-set-based reduction routines
\begin{center}
\begin{tabular}{ll}
    \FuncRef{shmem\_\FuncParam{TYPENAME}\_and\_to\_all} & \FuncRef{shmem\_\FuncParam{TYPENAME}\_max\_to\_all} \\
    \FuncRef{shmem\_\FuncParam{TYPENAME}\_or\_to\_all}  & \FuncRef{shmem\_\FuncParam{TYPENAME}\_min\_to\_all} \\
    \FuncRef{shmem\_\FuncParam{TYPENAME}\_xor\_to\_all} & \FuncRef{shmem\_\FuncParam{TYPENAME}\_sum\_to\_all} \\
                                                        & \FuncRef{shmem\_\FuncParam{TYPENAME}\_prod\_to\_all} \\
\end{tabular}
\end{center}
were deprecated and replaced with team-based reduction routines.


\subsection{\CorCpp: \FUNC{shmem\_barrier}}
\label{dep:shmem_barrier}
Each \openshmem team might
be associated with some number of communication contexts. The \FUNC{shmem\_barrier}
function implies that the default context is quiesced after synchronizing
some active set of \acp{PE}. Since teams may have some number of contexts associated
with the team, it becomes less clear which context would be the ``default'' context
for that particular team. Rather than continue to support \FUNC{shmem\_barrier}
for active-sets or teams, programs should use a call to \FUNC{shmem\_quiet}
followed by a call to \FUNC{shmem\_sync} in order to explicitly
indicate which context to quiesce.

\subsection{\textit{C11} and \CorCpp: \CTYPE{short} and \CTYPE{unsigned short} variants of \FUNC{shmem\_wait\_until} and \FUNC{shmem\_test}}
\label{dep:short_ushort_typed_shmem_wait_until_and_test}
The \CTYPE{short} and \CTYPE{unsigned short} type \CorCpp and \textit{C11}
routines for \FUNC{shmem\_wait\_until} and \FUNC{shmem\_test} were deprecated
because point-to-point synchronization routines are only compatible with
\acp{AMO} (as of \openshmem 1.5), and there is no corresponding \ac{AMO} for
\CTYPE{short} and \CTYPE{unsigned short}.

\subsection{Table~\ref{p2psynctypes}: point-to-point synchronization types}
\label{dep:p2p_sync_types}
As of \openshmem 1.5, the point-to-point synchronization routines are only
compatible with \acp{AMO}, so their interfaces are defined via the
standard \ac{AMO} types in Table~\ref{stdamotypes}.


%%
%% For section headings, use \textit{Fortran} instead of \Fortran.
%% Do not use commands that take optional arguments (e.g., \Fortran, \Cstd)
%% because they do not render at all in the PDF bookmarks.
%%
%% See Issue #66 for details.
%%


\chapter{Changes to this Document}\label{sec:changelog}

\section{Version 1.6}
\label{changelog:v1.6}
Major changes in \openshmem[1.6] include the addition of the new
\FUNC{shmem\_team\_ptr}, \FUNC{shmem\_ibget}, and \FUNC{shmem\_ibput}
functions.

The following list describes the specific changes in \openshmem[1.6]:
\begin{enumerate}
%
\item Added an inclusive (\FUNC{shmem\_sum\_inscan}) and exclusive 
(\FUNC{shmem\_sum\_exscan}) collective summation operation. 
\ChangelogRef{subsec:shmem_scan}
%
\item Added support for initialization and finalization routines to be called
    multiple times, and added an initialization status query API
    \FUNC{shmem\_query\_initialized}.
\ChangelogRef{subsec:shmem_init, subsec:shmem_finalize, subsec:shmem_query_initialized}%
%
\item Added interleaved block transfer APIs \FUNC{shmem\_ibget} and
    \FUNC{shmem\_ibput}.
\ChangelogRef{subsec:shmem_ibget, subsec:shmem_ibput}%
%
\item Added \FUNC{shmem\_signal\_add} and \FUNC{shmem\_signal\_set} to
  update a remote flag without associated data transfer of a put-with-signal operation.
\ChangelogRef{subsec:shmem_signal_add, subsec:shmem_signal_set}%
%
\item Added a team-based pointer query routine:
  \FUNC{shmem\_team\_ptr}.
\ChangelogRef{subsec:shmem_team_ptr}%
%
\item Added the session routines, \FUNC{shmem\_ctx\_session\_start} and
    \FUNC{shmem\_ctx\_session\_stop}, which allow users to pass hints to the
    \openshmem library to apply runtime optimizations.
\ChangelogRef{subsec:sessions}%
\item Added fine grained completion routine: \FUNC{shmem\_pe\_quiet}.
\ChangelogRef{subsec:shmem_pe_quiet}%
%
\item Split the listings for the \FUNC{shmem\_\{malloc, free, realloc, align\}}
  functions from a single entry in \openshmem[1.5] into separate entries.
\ChangelogRef{subsec:shmem_malloc, subsec:shmem_free, subsec:shmem_realloc,
  subsec:shmem_align}%
%
\item Clarified that the \FUNC{shmem\_\{malloc, free, realloc, align,
    malloc\_with\_hints, calloc\}} functions are collective operations on
    the world team.
\ChangelogRef{subsec:shmem_malloc, subsec:shmem_free, subsec:shmem_realloc,
  subsec:shmem_align, subsec:shmmallochint, subsec:shmem_calloc}%
%
\item Clarified that \FUNC{shmem\_team\_get\_config} returns the current
    configuration values, which may differ from the values assigned at the
        time of the team's creation.
\ChangelogRef{subsec:shmem_team_get_config}
%
\item Clarified the behavior of \FUNC{shmem\_team\_get\_config} when the
    \VAR{config\_mask} is 0 and/or the \VAR{config} argument is a null pointer.
\ChangelogRef{subsec:shmem_team_get_config}
%
\item Clarified the behavior of \FUNC{shmem\_team\_split\_strided} when the
    stride argument is 0 or negative.
\ChangelogRef{subsec:shmem_team_split_strided}
%
\item Added a new Errata Section~\ref{sec:errata} that indicates errors or ambiguities in the
    \openshmem specification and the version that required correction or clarification.
\ChangelogRef{sec:errata}
%
\item Removed \openshmem[1.5] Table 9, which was an incomplete duplicate of
    \openshmem[1.5] Table 10, and clarified the types, names, and supporting
    operations for team-based reductions. \label{changelog:reduction_table}
\ChangelogRef{teamreducetypes}%
%
\item Clarified that \VAR{source} and \VAR{dest} arrays must be the same
    across \acp{PE} in \openshmem reductions \label{changelog:reduction_args}
\ChangelogRef{subsec:shmem_reductions}
%
\item Clarified that \OPR{Fence} operations only guarantee ordering for
    operations that are performed on the same context. \label{changelog:fence_ctx}
\ChangelogRef{subsec:shmem_fence}%
%
\item Clarified that \FUNC{shmem\_test\_all} and \FUNC{shmem\_test\_all\_vector}
    routines return 1 when the test set is empty. \label{changelog:test_all}
\ChangelogRef{subsec:shmem_test_all,subsec:shmem_test_all_vector}%
%
\item Clarified that \FUNC{shmem\_team\_split\_strided} and
    \FUNC{shmem\_team\_split\_strided} return a nonzero value when the parent
        team compares equal to \LibConstRef{SHMEM\_TEAM\_INVALID}. \label{changelog:split_strided_2d}
\ChangelogRef{subsec:shmem_team_split_strided, subsec:shmem_team_split_2d}%
%
\item Corrected the level argument's recommended value in API notes for
    \FUNC{shmem\_pcontrol} to indicate that the value should be greater than
    2 to enable profiling with profile library defined effects and
    additional arguments. \label{changelog:pcontrol}
\ChangelogRef{subsec:shmem_pcontrol}
%
\item Corrected the requirements regarding associative and commutative binary
    operations used in \openshmem reductions.
    \label{changelog:reduction_associativity}
\ChangelogRef{subsec:shmem_reductions}
%

\end{enumerate}

\section{Version 1.5}
Major changes in \openshmem[1.5] include the addition of new team-based
collective functions, \OPR{put-with-signal} functions, nonblocking \ac{AMO}
functions, multiple-element point-to-point synchronization and vector
comparison functions, a \FUNC{shmem\_malloc\_with\_hints} function, a profiling
interface, and the removal of the entire \Fortran \ac{API}.

The following list describes the specific changes in \openshmem[1.5]:
\begin{enumerate}
%
\item Removed \FUNC{SHMEM\_CACHE}.
\ChangelogRef{dep:shmem_cache}%
%
\item Deprecated \CTYPE{short} and \CTYPE{unsigned short} variants for
\FUNC{shmem\_wait\_until} and \FUNC{shmem\_test}.
\ChangelogRef{
  subsec:shmem_wait_until,
  subsec:shmem_test,
  dep:short_ushort_typed_shmem_wait_until_and_test}%
%
\item Added \FUNC{shmem\_malloc\_with\_hints} interface and corresponding hints
\CONST{SHMEM\_MALLOC\_ATOMICS\_REMOTE} and \CONST{SHMEM\_MALLOC\_SIGNAL\_REMOTE}.
\ChangelogRef{subsec:shmmallochint, subsec:library_constants}%
%
\item Specified that team-based broadcast operations update the \VAR{dest} object on
all \acp{PE}, including the root \ac{PE}.
\ChangelogRef{subsec:shmem_broadcast}%
%
\item Deprecated active-set-based library constants and collective functions.
\ChangelogRef{
  subsec:library_constants,
  subsec:coll,
  dep:active_set_libconst_and_collectives,
  dep:shmem_barrier}%
%
\item Added team management functions:
  \FUNC{shmem\_team\_my\_pe},
  \FUNC{shmem\_team\_n\_pes},
  \FUNC{shmem\_team\_get\_config}, \\
  \FUNC{shmem\_team\_translate\_pe},
  \FUNC{shmem\_team\_split\_strided},
  \FUNC{shmem\_team\_split\_2d}, and
  \FUNC{shmem\_team\_destroy}.
\ChangelogRef{
  subsec:shmem_team_my_pe,
  subsec:shmem_team_n_pes,
  subsec:shmem_team_get_config,
  subsec:shmem_team_translate_pe,
  subsec:shmem_team_split_strided,
  subsec:shmem_team_split_2d,
  subsec:shmem_team_destroy}%
%
\item Added team-based communication-management functions:
  \FUNC{shmem\_team\_create\_ctx} and \\
  \FUNC{shmem\_ctx\_get\_team}.
\ChangelogRef{
  subsec:shmem_team_create_ctx,
  subsec:shmem_ctx_get_team}%
%
\item Added team-based collective functions: \FUNC{shmem\_sync},
  \FUNC{shmem\_alltoall[mem]}, \FUNC{shmem\_alltoalls[mem]}, \\
  \FUNC{shmem\_broadcast[mem]}, \FUNC{shmem\_collect[mem]}, \FUNC{shmem\_fcollect[mem]}, and \\
  \FUNC{shmem\_\{and, or, xor, max, min, sum, prod\}\_reduce}.
\ChangelogRef{
  subsec:shmem_sync,
  subsec:shmem_alltoall,
  subsec:shmem_alltoalls,
  subsec:shmem_broadcast,
  subsec:shmem_collect,
  subsec:shmem_reductions}%
%
\item Clarified interoperability of \openshmem with other programming models.
\ChangelogRef{sec:interoperability}%
%
\item Clarified restrictions on using pointers to symmetric objects.
\ChangelogRef{
  subsec:pointers_to_symmetric_objects,
  subsec:invoking_openshmem_operations}%
%
\item Added support for nonblocking \ac{AMO} functions.
\ChangelogRef{subsec:amo_nbi}%
%
\item Added support for blocking \OPR{put-with-signal} functions.
\ChangelogRef{subsec:shmem_put_signal}%
%
\item Added support for nonblocking \OPR{put-with-signal} functions.
\ChangelogRef{subsec:shmem_put_signal_nbi}%
%
\item Deprecated point-to-point synchronization types and names.
\ChangelogRef{p2psynctypes, dep:p2p_sync_types}%
%
\item Clarified that point-to-point synchronization routines preserve the
  atomicity of OpenSHMEM \acp{AMO}.
\ChangelogRef{subsec:amo_guarantees}%
%
\item Clarified that symmetric variables used as \VAR{ivar} arguments to
  point-to-point synchronization routines must be updated using OpenSHMEM
  \acp{AMO}.
\ChangelogRef{subsec:p2p_intro}%
%
\item Removed the entire \openshmem \Fortran \ac{API}.
\ChangelogRef{dep:fortran}%
%
\item Added support for multipliers in \VAR{SHMEM\_SYMMETRIC\_SIZE}
environment variables.
\ChangelogRef{subsec:environment_variables}%
%
\item Added support for a multiple-element point-to-point synchronization \ac{API} with
  the functions: \FUNC{shmem\_wait\_until\_all}, \FUNC{shmem\_wait\_until\_any},
  \FUNC{shmem\_wait\_until\_some}, \FUNC{shmem\_test\_all},
  \FUNC{shmem\_test\_any}, and \FUNC{shmem\_test\_some}.
\ChangelogRef{
  subsec:shmem_wait_until_all,
  subsec:shmem_wait_until_any,
  subsec:shmem_wait_until_some,
  subsec:shmem_test_all,
  subsec:shmem_test_any,
  subsec:shmem_test_some}%
%
\item Added support for vectorized comparison values in the multiple-element
  point-to-point synchronization \ac{API} with the functions:
  \FUNC{shmem\_wait\_until\_all\_vector}, \FUNC{shmem\_wait\_until\_any\_vector},
  \FUNC{shmem\_wait\_until\_some\_vector},
  \FUNC{shmem\_test\_all\_vector}, \FUNC{shmem\_test\_any\_vector}, and
  \FUNC{shmem\_test\_some\_vector}.
\ChangelogRef{
  subsec:shmem_wait_until_all_vector,
  subsec:shmem_wait_until_any_vector,
  subsec:shmem_wait_until_some_vector,
  subsec:shmem_test_all_vector,
  subsec:shmem_test_any_vector,
  subsec:shmem_test_some_vector}%
%
\item Added \openshmem profiling interface.
\ChangelogRef{sec:openshmem_profiling_interface}%
%
\item Specified the validity of communication contexts, added the constant
  \CONST{SHMEM\_CTX\_INVALID}, and clarified the behavior of
  \FUNC{shmem\_ctx\_*} routines on invalid contexts.
\ChangelogRef{sec:ctx}%
%
\item Clarified \ac{PE} active set requirements.
\ChangelogRef{subsec:coll}%
%
\item Clarified that when the \VAR{size} argument is zero, symmetric heap
    allocation routines perform no action and return a null pointer; that
    symmetric heap management routines that perform no action do not perform a
    barrier; and that the \VAR{alignment} argument to \FUNC{shmem\_align} must
    be power of two multiple of \CONST{sizeof(void*)}.
\ChangelogRef{subsec:shmem_malloc, subsec:shmem_align, subsec:shmem_realloc}%
%
\item Clarified that the \openshmem lock \ac{API} provides a non-reentrant mutex and
    that \FUNC{shmem\_clear\_lock} performs a quiet operation on the default
    context.
\ChangelogRef{subsec:shmem_lock}%
%
\item Clarified the atomicity guarantees of the \openshmem memory model.
\ChangelogRef{subsec:amo_guarantees}%
%
\end{enumerate}

\section{Version 1.4}
Major changes in \openshmem[1.4] include
multithreading support,
\emph{contexts} for communication management,
\FUNC{shmem\_sync},
\FUNC{shmem\_calloc},
expanded type support,
a new namespace for atomic operations,
atomic bitwise operations,
\FUNC{shmem\_test} for nonblocking point-to-point synchronization,
and \Cstd[11] type-generic interfaces for point-to-point synchronization.

The following list describes the specific changes in \openshmem[1.4]:
\begin{enumerate}
%
\item New communication management \ac{API}, including \FUNC{shmem\_ctx\_create};
    \FUNC{shmem\_ctx\_destroy}; and additional \ac{RMA}, \ac{AMO}, and memory ordering
    routines that accept \CTYPE{shmem\_ctx\_t} arguments.
\ChangelogRef{sec:ctx}%
%
\item New \ac{API} \FUNC{shmem\_sync\_all} and \FUNC{shmem\_sync} to provide \ac{PE}
    synchronization without completing pending communication operations.
\ChangelogRef{subsec:shmem_sync_all, subsec:shmem_sync}%
%
\item Clarified that the \openshmem extensions header files are required, even when empty.
\ChangelogRef{subsec:bindings}%
%
\item Clarified that the \FUNC{SHMEM\_GET64} and \FUNC{SHMEM\_GET64\_NBI}
    routines are included in the \Fortran language bindings.
\ChangelogRef{subsec:shmem_get, subsec:shmem_get_nbi}%
%
\item Clarified that \FUNC{shmem\_init} must be matched with a call to
    \FUNC{shmem\_finalize}.
\ChangelogRef{subsec:shmem_init, subsec:shmem_finalize}%
%
\item Added the \CONST{SHMEM\_SYNC\_SIZE} constant.
\ChangelogRef{subsec:library_constants}%
%
\item Added type-generic interfaces for \FUNC{shmem\_wait\_until}.
\ChangelogRef{subsec:shmem_wait_until}%
%
\item Removed the \VAR{volatile} qualifiers from the \VAR{ivar} arguments to
\FUNC{shmem\_wait} routines and the \VAR{lock} arguments in the lock \ac{API}.
\emph{Rationale: Volatile qualifiers were added to several \ac{API} routines in
\openshmem[1.3]; however, they were later found to be unnecessary.}
\ChangelogRef{subsec:shmem_wait_until, subsec:shmem_lock}%
%
\item Deprecated the \VAR{SMA\_}* environment variables and added equivalent
\VAR{SHMEM\_}* environment variables.
\ChangelogRef{subsec:environment_variables, dep:envvar_sma}%
%
\item Added the \Cstd[11] \CTYPE{\_Noreturn} function specifier to
\FUNC{shmem\_global\_exit}.
\ChangelogRef{subsec:shmem_global_exit}%
%
\item Clarified ordering semantics of memory ordering, point-to-point synchronization, and collective
synchronization routines.
%
\item Clarified deprecation overview and added deprecation rationale.
\ChangelogRef{dep:overview, dep:rationale}%
%
\item Deprecated header directory \HEADER{mpp}.
\ChangelogRef{dep:mpp_header}%
%
\item Deprecated the \FUNC{shmem\_wait} functions and the \CTYPE{long}-typed \CorCpp \FUNC{shmem\_wait\_until} function.
\ChangelogRef{subsec:shmem_wait_until, dep:shmem_wait, dep:long_typed_shmem_wait_until}%
%
\item Added the \FUNC{shmem\_test} functions.
\ChangelogRef{subsec:p2p_intro}%
%
\item Added the \FUNC{shmem\_calloc} function.
\ChangelogRef{subsec:shmem_calloc}%
%
\item Introduced the thread safe semantics that define the interaction between
    \openshmem routines and user threads.
\ChangelogRef{subsec:thread_support}%
%
\item Added the new routine \FUNC{shmem\_init\_thread} to initialize the
    \openshmem library with one of the defined thread levels.
\ChangelogRef{subsec:shmem_init_thread}%
%
\item Added the new routine \FUNC{shmem\_query\_thread} to query the thread
    level provided by the \openshmem implementation.
\ChangelogRef{subsec:shmem_query_thread}%
%
\item Clarified the semantics of \FUNC{shmem\_quiet} for a multithreaded
    \openshmem \ac{PE}.
\ChangelogRef{subsec:shmem_quiet}%
%
\item Revised the description of \FUNC{shmem\_barrier\_all} for a multithreaded
    \openshmem \ac{PE}.
\ChangelogRef{subsec:shmem_barrier_all}%
%
\item Revised the description of \FUNC{shmem\_wait} for a multithreaded
    \openshmem \ac{PE}.
\ChangelogRef{subsec:shmem_wait_until}%
%
\item Clarified description for \CONST{SHMEM\_VENDOR\_STRING}.
\ChangelogRef{subsec:library_constants}%
%
\item Clarified description for \CONST{SHMEM\_MAX\_NAME\_LEN}.
\ChangelogRef{subsec:library_constants}%
%
\item Clarified \ac{API} description for \FUNC{shmem\_info\_get\_name}.
\ChangelogRef{subsec:shmem_info_get_name}%
%
\item Expanded the type support for \ac{RMA}, \ac{AMO}, and point-to-point
    synchronization operations.
%% cleveref will compress a list of references by default. It is better to not                                                                                                                             
%% compress this list of *table* references because the clickable hyperref
%% links are useful. You can tell cleveref to not compress the LHS and RHS by
%% inserting an empty item between them; i.e., `,,`.
\ChangelogRef{stdrmatypes,, stdamotypes,, extamotypes,, p2psynctypes}%
%
\item Renamed \ac{AMO} operations to use \FUNC{shmem\_atomic\_*} prefix and
      deprecated old \ac{AMO} routines.
\ChangelogRef{sec:amo, dep:amo_not_shmem_atomic}%
%
\item Added fetching and non-fetching bitwise AND, OR, and XOR atomic
      operations.
\ChangelogRef{sec:amo}%
%
\item Deprecated the entire \Fortran \ac{API}.
\ChangelogRef{dep:fortran}%
%
\item Replaced the \CTYPE{complex} macro in complex-typed reductions with the
      \Cstd[99] (and later) type specifier \CTYPE{\_Complex} to remove an
      implicit dependence on \HEADER{complex.h}.
\ChangelogRef{subsec:shmem_reductions}%
%
\item Clarified that complex-typed reductions in C are optionally supported.
\ChangelogRef{subsec:shmem_reductions}%
%
\end{enumerate}




\section{Version 1.3}
Major changes in \openshmem[1.3] include the addition of
nonblocking \ac{RMA} operations,
atomic \PUT{} and \GET{} operations,
all-to-all collectives,
and \Cstd[11] type-generic interfaces for \ac{RMA} and \ac{AMO} operations.

The following list describes the specific changes in \openshmem[1.3]:
\begin{enumerate}
%
\item Clarified implementation of \acp{PE} as threads.
%
\item Added \CTYPE{const} to every read-only pointer argument.
%
\item Clarified definition of \OPR{Fence}.
\ChangelogRef{subsec:programming_model}%
%
\item Clarified implementation of symmetric memory allocation.
\ChangelogRef{subsec:memory_model}%
%
\item Restricted atomic operation guarantees to other atomic operations with the same datatype.
\ChangelogRef{subsec:amo_guarantees}%
%
\item Deprecation of all constants that start with \CONST{\_SHMEM\_*}.
\ChangelogRef{subsec:library_constants, dep:libconst_undershmem}%
%
\item Added a type-generic interface to \openshmem \ac{RMA} and \ac{AMO}
    operations based on \Cstd[11] Generics.
\ChangelogRef{sec:rma, sec:amo}%
%
\item New nonblocking variants of remote memory access, \FUNC{SHMEM\_PUT\_NBI}
    and \FUNC{SHMEM\_GET\_NBI}.
\ChangelogRef{subsec:shmem_put_nbi, subsec:shmem_get_nbi}%
%
\item New atomic elemental read and write operations, \FUNC{SHMEM\_FETCH} and
    \FUNC{SHMEM\_SET}.
\ChangelogRef{subsec:shmem_atomic_fetch, subsec:shmem_atomic_set}%
%
\item New alltoall data exchange operations, \FUNC{SHMEM\_ALLTOALL}
    and \FUNC{SHMEM\_ALLTOALLS}.
\ChangelogRef{subsec:shmem_alltoall, subsec:shmem_alltoalls}%
%
\item Added \CTYPE{volatile} to remotely accessible pointer argument in
    \FUNC{SHMEM\_WAIT} and \FUNC{SHMEM\_LOCK}.
\ChangelogRef{subsec:shmem_wait_until, subsec:shmem_lock}%
%
\item Deprecation of \FUNC{SHMEM\_CACHE}.
\ChangelogRef{dep:shmem_cache}%
%
\end{enumerate}




\section{Version 1.2}
Major changes in \openshmem[1.2] include
a new initialization routine (\FUNC{shmem\_init}),
improvements to the execution model with an explicit
library-finalization routine (\FUNC{shmem\_finalize}),
an early-exit routine (\FUNC{shmem\_global\_exit}),
namespace standardization,
and clarifications to several \ac{API} descriptions.

The following list describes the specific changes in \openshmem[1.2]:
\begin{enumerate}
%
\item Added specification of \VAR{pSync} initialization for all routines that use it.
%
\item Replaced all placeholder variable names \VAR{target} with \VAR{dest} to
      avoid confusion with \Fortran's \KEYWORD{target} keyword.
%
\item New Execution Model for exiting/finishing \openshmem programs.
\ChangelogRef{subsec:execution_model}%
%
\item New library constants to support \ac{API} that query version and name information.
\ChangelogRef{subsec:library_constants}%
%
\item New \ac{API} \FUNC{shmem\_init} to provide mechanism to start an \openshmem
      program and replace deprecated \FUNC{start\_pes}.
\ChangelogRef{subsec:shmem_init, dep:start_pes}%
%
\item Deprecation of \FUNC{\_my\_pe} and \FUNC{\_num\_pes} routines.
\ChangelogRef{subsec:shmem_my_pe, subsec:shmem_n_pes, dep:func_not_shmemunder}%
%
\item New \ac{API} \FUNC{shmem\_finalize} to provide collective mechanism to cleanly
      exit an \openshmem program and release resources.
\ChangelogRef{subsec:shmem_finalize}%
%
\item New \ac{API} \FUNC{shmem\_global\_exit} to provide mechanism to exit an
    \openshmem program.
\ChangelogRef{subsec:shmem_global_exit}%
%
\item Clarification related to the address of the referenced object in
    \FUNC{shmem\_ptr}.
\ChangelogRef{subsec:shmem_ptr}%
%
\item New \ac{API} to query the version and name information.
\ChangelogRef{subsec:shmem_info_get_version, subsec:shmem_info_get_name}%
%
\item \openshmem library \ac{API} normalization. All \Cstd symmetric memory management
      \ac{API} begins with \FUNC{shmem\_}.
\ChangelogRef{sec:memory_management, dep:func_not_shmemunder}%
%
\item Notes and clarifications added to \FUNC{shmem\_malloc}.
\ChangelogRef{subsec:shmem_malloc}%
%
\item Deprecation of \Fortran \ac{API} routine \FUNC{SHMEM\_PUT}.
\ChangelogRef{dep:fortran_shmem_put}%
See \openshmem[1.4], Section 9.5.1.
%
\item Clarification related to \FUNC{shmem\_wait}.
\ChangelogRef{subsec:shmem_wait_until}%
%
\item Undefined behavior for null pointers without zero counts added.
\ChangelogRef{sec:undefined}%
%
\item Added new Annex for clearly specifying deprecated \ac{API} and its
      support across versions of the \openshmem Specification.
\ChangelogRef{sec:dep}%
%
\end{enumerate}




\section{Version 1.1}
Major changes from \openshmem[1.0] to \openshmem[1.1] include
the introduction of the \HEADER{shmemx.h} header file for nonstandard \ac{API}
extensions,
clarifications to completion semantics and \ac{API} descriptions in agreement with
the \ac{SGI} SHMEM specification,
and general readabilty and usability improvements to the document structure.

The following list describes the specific changes in \openshmem[1.1]:
\begin{enumerate}
%
\item Clarifications of the completion semantics of memory synchronization
      interfaces.
\ChangelogRef{subsec:memory_order}%
%
\item Clarification of the completion semantics of memory load and store
      operations in context of \FUNC{shmem\_barrier\_all} and \FUNC{shmem\_barrier}
      routines.
\ChangelogRef{subsec:shmem_barrier_all, subsec:shmem_barrier}%
%
\item Clarification of the completion and ordering semantics of
      \FUNC{shmem\_quiet} and \FUNC{shmem\_fence}.
\ChangelogRef{subsec:shmem_quiet, subsec:shmem_fence}%
%
\item Clarifications of the completion semantics of \ac{RMA} and \ac{AMO}
      routines.
\ChangelogRef{sec:rma, sec:amo}%
%
\item Clarifications of the memory model and the memory alignment requirements
      for symmetric data objects.
\ChangelogRef{subsec:memory_model}%
%
\item Clarification of the execution model and the definition of a \ac{PE}.
\ChangelogRef{subsec:execution_model}%
%
\item Clarifications of the semantics of \FUNC{shmem\_pe\_accessible} and
      \FUNC{shmem\_addr\_accessible}.
\ChangelogRef{subsec:shmem_pe_accessible, subsec:shmem_addr_accessible}%
%
\item Added an annex on interoperability with \ac{MPI}.
\ChangelogRef{sec:interoperability}%
%
\item Added examples to the different interfaces.
%
\item Clarification of the naming conventions for constant in \Cstd and
      \Fortran.
\ChangelogRef{subsec:library_constants, subsec:shmem_wait_until}%
%
\item Added \ac{API} calls: \FUNC{shmem\_char\_p}, \FUNC{shmem\_char\_g}.
\ChangelogRef{subsec:shmem_p, subsec:shmem_g}%
%
\item Removed \ac{API} calls: \FUNC{shmem\_char\_put},
      \FUNC{shmem\_char\_get}.
\ChangelogRef{subsec:shmem_put, subsec:shmem_get}%
%
\item The usage of \CTYPE{ptrdiff\_t}, \CTYPE{size\_t}, and \CTYPE{int} in the
      interface signature was made consistent with the description.
\ChangelogRef{subsec:coll, subsec:shmem_iput, subsec:shmem_iget}%
%
\item Revised \FUNC{shmem\_barrier} example.
\ChangelogRef{subsec:shmem_barrier}%
%
\item Clarification of the initial value of \VAR{pSync} work arrays for
\FUNC{shmem\_barrier}.
\ChangelogRef{subsec:shmem_barrier}%
%
\item Clarification of the expected behavior when multiple \FUNC{start\_pes}
calls are encountered.
\ChangelogRef{subsec:start_pes}%
%
\item Corrected the definition of atomic increment operation.
\ChangelogRef{subsec:shmem_atomic_inc}%
%
\item Clarification of the size of the symmetric heap and when it is set.
\ChangelogRef{sec:memory_management}%
%
\item Clarification of the integer and real sizes for \Fortran \ac{API}.
\ChangelogRef{
  subsec:shmem_atomic_add,
  subsec:shmem_atomic_compare_swap,
  subsec:shmem_atomic_swap,
  subsec:shmem_atomic_fetch_inc,
  subsec:shmem_atomic_inc,
  subsec:shmem_atomic_fetch_add}%
%
\item Clarification of the expected behavior on program \OPR{exit}.
\ChangelogRef{subsec:execution_model}%
%
\item More detailed description for the progress of \openshmem operations
provided.
\ChangelogRef{subsec:progress}%
%
\item Clarification of naming convention for nonstandard interfaces and their
inclusion in \HEADER{shmemx.h}.
\ChangelogRef{subsec:bindings}%
%
\item Various fixes to \openshmem code examples across the Specification to
include appropriate header files.
%
\item Removing requirement that implementations should detect size mismatch and
return error information for \FUNC{shmalloc} and ensuring consistent
language.
\ChangelogRef{subsec:shmem_malloc, sec:undefined}%
%
\item \Fortran programming fixes for examples.
\ChangelogRef{subsec:shmem_reductions, subsec:shmem_wait_until}%
%
\item Clarifications of the reuse \VAR{pSync} and \VAR{pWork} across
collectives.
\ChangelogRef{
  subsec:coll,
  subsec:shmem_broadcast,
  subsec:shmem_collect,
  subsec:shmem_reductions}%
%
\item Name changes for UV and ICE for \ac{SGI} systems.
\ChangelogRef{sec:openshmem_history}%
%
\end{enumerate}

\chapter{Errata}\label{sec:errata}

Errors or ambiguities in the \openshmem specification may be discovered after
publication.
Errata, or corrections, are included in the the sections below indicating the
version of the OpenSHMEM specification that required the correction or
clarification.
These corrections have been applied to all subsequent versions of the
specification and this section serves as a historical record of the changes
made to assist users and implementers with applying the necessary corrections.
Errata that result in a change to the specifciation are also included in
Annex~\ref{sec:changelog}.
For an implementation to comply with a particular version of \openshmem, it
must account for all errata associated with that version as indicated below.

\section{Version 1.5}

\begin{enumerate}
  \item Removed \openshmem[1.5] Table 9, which was an incomplete duplicate of
      \openshmem[1.5] Table 10, and clarified the types, names, and supporting
      operations for team-based reductions
        (\ref{changelog:v1.6}.\ref{changelog:reduction_table}).
  \item Clarified that \VAR{source} and \VAR{dest} arrays must be the same
      across \acp{PE} in \openshmem reductions
        (\ref{changelog:v1.6}.\ref{changelog:reduction_args}).
  \item Clarified that \OPR{Fence} operations only guarantee ordering for operations
     that are performed on the same context
        (\ref{changelog:v1.6}.\ref{changelog:fence_ctx}).
  \item Clarified that \FUNC{shmem\_test\_all} and
     \FUNC{shmem\_test\_all\_vector} routines return 1 when the test set is empty
        (\ref{changelog:v1.6}.\ref{changelog:test_all}).
  \item Clarified that \FUNC{shmem\_team\_split\_strided} and
     \FUNC{shmem\_team\_split\_2d} return nonzero when the parent team is
     \LibConstRef{SHMEM\_TEAM\_INVALID}
        (\ref{changelog:v1.6}.\ref{changelog:split_strided_2d}).
  \item Corrected the \VAR{level} argument's recommended value in API notes for
     \FUNC{shmem\_pcontrol} to indicate that the value should be greater than 2 to enable
     profiling with profile library defined effects and additional arguments
        (\ref{changelog:v1.6}.\ref{changelog:pcontrol}).
  \item Corrected the requirements regarding associative and commutative binary
      operations used in \openshmem reductions
        (\ref{changelog:v1.6}.\ref{changelog:reduction_associativity}).
\end{enumerate}

%end of setlength command that was started in frontmatter.tex


\clearpage
\phantomsection
\addcontentsline{toc}{chapter}{Index}
\printindex

\end{document}

